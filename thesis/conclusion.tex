% een terugblik op het in je inleiding gestelde doel: welke doelen heb je behaald en welke niet?
% ·       een terugblik: wat heb je geleerd (denk aan kennis, vaardigheden)
% ·       verder onderzoek: welke open vragen zijn er voor vervolgonderzoek. Voor jezelf, waar je niet aan toegekomen bent, of voor een volgende lichting studenten?
The linear time $\mu$-calculus is a logic that can be used to describe properties of concurrent programs. For a logic that describes such properties stutter-invariancy is a natural and desired property according to Lamport \cite{lamport1983whatgood}. For the logic $\mutl$ the stutter-invariant fragment is equal to $\mutl(U,R)$ as was proven by \cite{gheerbrandt2009craiginterpolation}. In this thesis we have given an alternative way of proving this theorem by a translation of $\mutl$ formulas to Nondeterministic Parity Automata and a construction on these parity automata. In Chapter 3 we describe the translation of $\mutl$ to \npa\ via the following chain of translations $\mutl\to\text{ parity formulas }\to \apa\to\npa$. In Chapter 4 we provide a construction for a \npa\ $\A$ that defines the stutter-closed \npa\ $\A^s$ that exactly recognizes $\LL(\A^s)$. Finally we use both constructions to give the sketch of the proof that $\mutl(U,R)$ is the stutter-invariant fragment of $\mutl$. However, the final proof of Chapter 4 is far from finished for the following reasons: First of all we did not completely manage to understand the parity progress measures used in the translation of \apa s to \npa s. Secondly our translation of a $\mutl$ formula to a \npa\ requires $\phi$ to be guarded. When working with the $\deo$-free fragment of $\mutl$ no formula is guarded. So we need to transform $\phi$ to a guarded formula. We know that this equivalent guarded formula exists but did not manage to fully understand this guarded transformation. It may be possible to completely remove the use of fixpoint operators which would result in an alternation-free $\mutl(U,R)$ formula.

Next to the proof that $\mutl(U,R)$ is the stutter-invariant fragment of $\mutl$ we also formalized our stutter-closure construction for \npa s with the formal proof assistant $\lean$. We completely described the construction but did not complete the proof that this construction is correct.

As mentioned in the introduction the original plan for this project was much bigger. We also wanted to describe a translation of parity automata to $\mutl$ formulas and use this construction to describe the stutter-invariant fragment of $\mutl$. This however was not possible in the timespan of this bachelors project. We do however are

\subsection*{Future work}
As mentioned above there are some holes in the proof of the stutter-invariancy proof. Below we summarize some future research objectives that can help finishing the proof of the proof that $\mutl(U,R)$ is the stutter-invariant fragment of $\mutl$.
\begin{itemize}
 \item Describe a direct construction of \apa s to \npa s without the intermediate step of Büchi automata.
 \item Describe the guarded transformation for $\mutl(U,R)$ formulas.
\end{itemize}
We can also finish the $\lean$ formalization of the proof of Theorem \ref{thm:stutterclosednpa}. In addition we can formalize the definition $\mutl$ and translation to \npa\ in $\lean$. It would then be possible to describe a program that checks if a $\mutl$ formula is stutter-invariant.

\subsection*{Personal reflection}
First of all, I must say that despite it being a stressful process, I really enjoyed writing this thesis. I learned a lot of interesting mathematics and computer science and enjoyed the experience of research. As I did not follow the course \textit{Introduction to Modal Logic} this was my first acquaintance with modal logic and that was sometimes quite difficult. It took time to understand modal logic in general and modal $\mu$-calculus in particular. Next to that I started with a vague definition of the linear time $\mu$-calculus and only formally defined the grammar and semantics of $\mutl$ we were going to use in a late stadium of this thesis. That contributed to the difficulties in this thesis but also, more importantly, to the big learning outcomes of this thesis. Since I was not familiar with this specific area of logic I now understand a lot of new concepts, which I really enjoyed. When looking back at this project I could have done things completely different. I lost myself a little in proving the stutter-closure and the $\mutl$ to \npa\ translation and forgot the goal of this thesis: the proof that $\mutl(U,R)$ is the stutter-invariant fragment. If I would have known parity formulas and alternating automata the proof Theorem 3.16 would be proven weeks earlier. But that does not mean this project failed, I did not know these concepts so it was completely expected that I tried the `wrong' proof approaches at first. Luckily Yde pointed me in the right direction at the right moments.

I also learned to write lengthy formal mathematical proofs which was something I found surprisingly hard. Writing your thesis requires a lot more mathematical rigour then just writing the proof of a homework. To also make your proofs readable was even more difficult than I thought. This project was not possible without the tons of feedback I received from my supervisors Yde and Malvin. I had the pleasure to meet with them every week and want to thank them for their supervision and for taking me completely seriously. I am going to miss the printed copies of my work full with red marked suggestions from Yde, and Malvin having the lean-solutions to all of my sorrys.
