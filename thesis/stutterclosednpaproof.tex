Recall the following informal definition of stuttering we gave: Stuttering can be seen as removing duplicate letters and repeating letters. In order to describe these operations in our automaton we will add `shortcut' states and `loop states'. Shortcut states will allow the removing of duplicate letters and the loop states will allow the repeating. In this section we will first formally define these shortcut and loop states in a construction based on~\cite{michaud2015practical}. This will result in an automaton $\A^s$ that exactly recognizes ${\LL(\A)}^s$, which we will prove in Theorem~\ref{thm:stutterclosednpa}.

\begin{definition}\label{def:stutterclosedNPA}
Let \(\A = (\Sigma, A, \delta_\A, a_I, \Omega_\A)\) be a \npa, where without loss of generality we assume that $\Sigma \cap \omega = \emptyset$. We will define its \emph{Stutter-closure} as the automaton \(\A^s = (\Sigma, A^s, \delta_{\A^s}, a_I^s, \Omega_{\A^s})\) as follows:
\begin{itemize}
 \item Define the new set of states as $A^s := (A\times \Omega_\A(A)) \cup (A\times \Sigma)$.
 \item Define the set of initial states $a^s_I:=\{(a, \Omega_\A(a))\mid a\in a_I\}$
 \item Define the parity map $\Omega_{\A^s}: A^s\to\omega$ as following:
 \[
 \Omega_{\A^s}(a) =\left\{\begin{array}{llll}
             n + 2 &\text{ if } a=(a', n)   &\text{with } n\in \Omega_\A(A)\\
             1 &\text{ if } a = (a', v) &\text{with } v\in \Sigma
            \end{array}\right.
\]

\end{itemize}
To define the transition function $\delta_{\A^s}$ let $a\in A^s$. We distinguish two cases:
\begin{description}
 \item[Case $a=(a', v)\in A\times \Sigma$:] Then set the transition function to:
 \[
 \delta_{\A^s}((a', v), v') = \begin{cases}
                       \{(a',v), (a, \Omega_\A(a))\}&\text{if } v'=v\\
                      \emptyset & \text{else}
                      \end{cases}
\]
\item[Case $a=(a', n)\in A\times \Omega_\A(A)$:] Then set the transition function to:
\begin{align*}
 \delta_{\A^s}((a', n), v) &= \{(b, \Omega_\A(b))\in A \times \Omega_\A(A)\mid b\in \delta_\A(a',v)\}\\
                    &\cup \{(b, v)\in Q\times \Sigma\mid  b\in \delta_\A(a',v)\}\\
                    &\cup \{(a_n, m) \in A\times \Omega_\A(A)\mid \exists n, \exists \rho= a'a_1\dots a_n\land a'\twoheadrightarrow_{\rho}^{v^n}a_n\land\\
                    & \phantom{\cup\{} m = \max(\Omega_\A(\{a_i\mid i\in \{1, \dots, n\}\}))
\end{align*}
\end{description}
\end{definition}

Now that we have given this definition we have to proof that this is indeed correct.
\begin{theorem}\label{thm:stutterclosednpa}
Let \(\A = (\Sigma, A, \delta_\A, a_I, \Omega_\A)\) a \npa, and \(\A^s = (\Sigma, A^s \delta_{\A^s}, a_I^s, \Omega_{\A^s})\) its stutter-closure as defined in Definition~\ref{def:stutterclosedNPA}. Then we have
\[
 \LL(\A^s) = {(\LL(\A))}^s
\]
\end{theorem}
\begin{proof}
\subsection*{Left to right inclusion}
We will first prove the $\subseteq$ direction
\begin{equation}
  \LL(\A^s) \subseteq {(\LL(\A))}^s.
\end{equation}\label{eq:lefttorightinclusion}
Suppose we have $w\in \LL(\A^s)$, we now want to show that $w\in(\LL(\A))^s$. For this we need to find a word $w'\in \LL(\A)$ such that $w\sim_sw'$. Or more specifically we need to find a $w_b\in \Sigma^\omega$ and $f, f':\omega\to\omega^+$ such that $w=w_b[f]$ and $w'=w_b[f']$. Firstly we will recursively define the base word $w_b$ and stutter-function $f$ based on the run $\rho_w$ on $w$. Since $w\in\LL(\A^s)$ we know there exists an accepting run on $w$.

\begin{definitiont}\label{def:subsetwbf} To define $w_b$ and $f$ take $i\in \omega$. Let $j := \sum_{m=0}^{i-1}f(m)$. Since $m<=i-1<i$ we have, by the recursion, defined $f(m)$ for all $m$. When $i=0$ we see that the sum $\sum_{m=0}^{i-1}$ is empty so also the base case works. Since $\rho_w$ is an accepting run we know it cannot loop infinitely in a state of the form $(a, v)\in A\times\Sigma$ since $\Omega_{\A^s}(a,v)=1$ and therefore not even. So that means that there exists a smallest $n>0$ such that $\rho_w(j+n)\in A\times\Omega_\A(A)$. Now set $f(i):=n$ and $w_b(i) := w(j)$.
\end{definitiont}
We need to prove that this $w_b$ and $f$ are indeed correct, so that $w=w_b[f]$.
\begin{claim}\label{claim:subsetwwb[f]}
 Let $w_b$ and $f$ as in Definition~\ref{def:subsetwbf}. Then $w=w_b[f]$.
\end{claim}
\begin{proof}
We need to prove that for every $i\in \omega$ we have $w_b(i)=w\left(\sum_{m=0}^{i-1}f(m) + j\right)$ for $0\leq j < f(i)$. We will prove this with induction on $i$.
\begin{description}
 \item[Base case $i=0$] We defined $f(i)$ as the smallest $n$ such that $\rho_w(n)\in A\times \Omega_A(A)$ so that means $\rho_w(j)\in A\times \Omega_A(A)$ for $1\leq j<n$. Since since $\rho_w$ is a run on $w$ we know that $\rho_w(i+1)\in\delta_\A(\rho_w(i), w(i))$ and that leaves us only one possibility for $w(i)$ with $0\leq i < n$ namely $w(0)$ and that is exactly how we defined $w_b(i)$. So that means $w_b(i) = w(j)$ for $0\leq j < n$.
 \item[Inductive case $i>0$] With the inductive hypothesis we know that \[w_b(i-1) =w\left(\sum_{m=0}^{i-2}f(m) + (f(i-1)-1)\right) =  w\left(\sum_{m=0}^{i-1}f(m)-1\right).\]
 From the definition of $f$ we see just as in the proof for $i=0$ that $w\left(\sum_{m=0}^{i-1}f(m) +j\right)=w\left(\sum_{m=0}^{i-1}f(m)\right)$ for $0\leq j < f(i)$ and since $w_b(i)=w\left(\sum_{m=0}^{i-1}f(m)\right)$ we see that $w_b(i) = w\left(\sum_{m=0}^{i-1}f(m) +j\right)$ for $0\leq j< f(i)$.
\end{description}
This proves that for every $i\in\omega$ we have $w_b(i)=w\left(\sum_{m=0}^{i-1}f(m) + j\right)$ for $0\leq j < f(i)$, which means that $w=w_b[f]$.
\end{proof}

\noindent Our second step is to define a $w'$ such that $w\sim_s w'$ and show that $w'\in\LL(\A)$. To define $w'$ we will define a stutter-function $f'$ and let $w:=w_b[f']$. To show that $w'$ is accepting we will define a run $\rho_{w'}$ simultaneously.
\begin{definitiont}\label{def:subsetrhow'_f'}
 To define the stutter-function $f': \omega\to\omega^+$ and run $\rho_{w'}:\omega\to A$, take $i\in \omega$. Let $(a, n)$ be the first state in $\rho_w$ so $(a,n) := \rho_w(0)$ then define the first state in $\rho_{w'}$ as $\rho_{w'}(0) := a$. For the rest of $\rho_{w'}$ take $i\in \omega$ and distinguish cases.
\begin{description}
 \item[Case $f(i)=1$] Let $(a, n):= \rho_{w}\left(\sum_{m=0}^{i-1}f(m)\right)$ and $(a', n'):= \rho_{w}\left(\sum_{m=0}^{i}f(m)\right)=\rho_{w}\left(\sum_{m=0}^{i-1}f(m)+1\right)$. From Claim \ref{claim:subsetwwb[f]} we see that $w\left(\sum_{m=0}^{i-1}f(m)\right) = w_b(i)$. Now since $\rho_{w}$ is a run of $\A^s$ we know that $(a', n')\in \delta_{\A^s}\left((a,n), w\left(\sum_{m=0}^{i-1}f(m)\right)\right)$ so we know that there exists a $n>0$ and a partial run $\rho=aa_1\dots a_{n-1}a'$ such that $a\twoheadrightarrow_{\rho}^{(w_b(i))^n}a'$ and $n'=\max\{\Omega_\A(\rho(j))\mid j\in \{1,\dots,n\}\}$. We will use this $n$ and $\rho$ in the following way.

 Define $f'(i):=n$ and
 \[
  \rho_{w'}\left(\sum_{m=0}^{i-1}f'(m)+j\right) := \rho(j)\text{ for } j\in \{1,\dots,n\}
 \]
 \item[Case $f(i)\neq 1$] Let $(a, n):= \rho_{w}\left(\sum_{m=0}^if(m)\right)$. Now set $f'(i):=1$ and $\rho_{w'}\left(\sum_{m=0}^if'(m)\right):=~a$
\end{description}
\end{definitiont}
Now we need to show that Definition \ref{def:subsetrhow'_f'} is correct so $\rho_{w'}$ is indeed an accepting run on $w'$. We will split this into two claims and first prove that $\rho_{w'}$ defines a run of $\A$ after which we will prove that this run is accepting.
\begin{claim}\label{claim:subsetrhow'run}
 Let $\rho_{w'}$ be the sequence defined in Definition \ref{def:subsetrhow'_f'}. Then $\rho_{w'}$ is a run of $\A$ on the word $w'$. I.e. we have $\rho_{w'}(0)\in q_{I,\A}$ and for all natural numbers $i\in\omega$ it holds that  $\rho_{w'}(i+1)\in\delta_{\A}(\rho_{w'}(i), w(i))$.
\end{claim}

\noindent We will omit the proof here since it only consists of unfolding the right definitions. It can be found in the appendix \ref{appendix:subsetrhow'run}. Now it remains to show that $\rho_w'$ is accepting, or in other words the following claim.
\begin{claim}\label{claim:subsetrhow'pareven}
 Let $\rho_{w'}$ be the run of $\A$ on $w'$ as in Definition \ref{def:subsetrhow'_f'} We have that
 \[
  \max(\inf_{\Omega_\A}(\rho_{w'}))\text{ is even}
 \]
\end{claim}
\begin{proof}
We will first prove that \(\max(\inf_{\Omega_\A}(\rho_{w'}))=\max(\inf_{\Omega_{\A^s}}(\rho_{w}))-2\) by proving two inequalities.
\subsubsection*{Left to right inequality}
We will first prove
\begin{equation}
 \max(\inf_{\Omega_\A}(\rho_{w'}))\leq \max(\inf_{\Omega_{\A^s}}(\rho_{w}))-2
\end{equation}\label{eq:lefttorightpareven}
Let $q\in\inf(\rho_{w'})$ be a state with $\Omega_\A(q)=\max(\inf_{\Omega_\A}(\rho_{w'}))$. That means that we have infinitely many $i's$ such that $\rho_{w'}(i)=q$. Now take $i>\sum_{m=0}^{j} f'(m) > i_d$. Where $i_d$ is the decisive moment (Theorem ... nog uitwerken) and $j$ an arbitrary number. We need to be sure that this $i$ is not part of a partial run.  Now let $i_b$ the smallest integer such that $i\leq \sum_{m=0}^{i_b}f'(m)$ and distinguish cases. We use a case distinction on $f$ which might seem counterintuitive since we could also do cases on $f'$ but this gives a nice direct correspondence to the way we defined $\rho_{w'}$.
\begin{description}
 \item[Case $f(i_b)=1$] Then we know that this $q$ lies on the $\rho_p$ and $\Omega_\A(q)=\max\{\Omega_\A(\rho_p)\}=p$. So that means that $\rho_{w}\left(\sum_{m=0}^{i_b}f(m)\right)=(q',p)$ for a $q\in A$ but most importantly $\Omega_{\A^s}(\rho_{w}\left(\sum_{m=0}^{i_b}f(m)\right))-2=p=\Omega_\A(q)$.
 \item[Case $f(i_b)=n>1$] Now we see that $q=\rho_w\left(\sum_{m=0}^{i_b}f(m)\right)$. Since we see that $n>1$ we have that $\rho_{w}$ passed through a letter state so that means that $\rho_w(\sum_{m=0}^{i_b}f(m))$ is of the form $q, \Omega_\A(q)$ so $\Omega_{\A^s}(\rho_w(\sum_{m=0}^{i_b}f(m)))-2=\Omega_\A(q)$.
%  \textcolor{red}{DIT OVERAL TOEVOEGEN WAAR NODIG!!!!}
\end{description}
This gives us for any $i$ such that $\rho_{w'}(i)=q$ a unique $j$ such that $\Omega_{\A^s}(\rho_{w}(j))-2=\Omega_\A(q)$ since there are finitely many states we know that there is a state in $\inf(\rho_{w})$ with this parity. Which proves equation \ref{eq:lefttorightpareven}
\subsubsection*{Right to left inequality}
We will now prove
\begin{equation}
 \max(\inf_{\Omega_\A}(\rho_{w'}))\geq \max(\inf_{\Omega_{\A^s}}(\rho_{w}))-2
\end{equation}\label{eq:righttoleftpareven}
Let $(q,p)\in\inf(\rho_{w})$ be a state with $\Omega_{\A^s}((q,p))=\max(\inf_{\Omega_{\A^s}}(\rho_{w}))$.
Take $i> i_d$ and define $i_b$ as the smallest integer such that $i\leq \sum_{m=0}^{i_b}f(m)$. We distinguish cases
\begin{description}
 \item[Case $f(i_b)=1$] That means that $\rho_{w}(i)=(q,p)$ with $\Omega_{\A^s}(\rho_w(i))=p+2$ and $p=\max\{\Omega_\A(\rho_p(j))\mid j\in \{1,\dots,n\}\}$ for a $\rho_p$. So therefore there is a $j$ such that $\Omega_\A(\rho_p(j))=\Omega_{\A^s}(\rho_w(i))-2$. Now we have defined $\rho_{w'}\left( \sum_{m=0}^{i_b-1}+j) = \rho_p(j) \right)$ so we have a state in $\rho_{w'}$ with parity $p$.
 \item[Case $f(i_b)=n$] Since we know that all states $\rho_w\left(\sum_{m=0}^{i_b-1}f(m) +l\right)\in A\times \Sigma$ for $0<l<n$ we see that $i=\sum_{m=0}^{i_b}f(m)$ so that means that
 \[
  \rho_{w}(i) = \rho_w\left(\sum_{m=0}^{i_b}f(m)\right) = (q,\Omega(q)) = \left(\rho_{w'}\left(\sum_{m=0}^{i_b}f'(m)\right), \Omega_\A\left(\rho_{w'}\left(\sum_{m=0}^{i_b}f'(m)\right)\right)\right)
 \] and $\Omega_{\A^s}(\rho_{w}(i))-2=\Omega_\A(q)$.
\end{description}
Now we have infinitely many $j's$ such that $\rho_{w'}(j)=\max(\inf_{\Omega_{\A^s}}(\rho_{w}))-2$ and since there are finitely many states we know that there is a state in the inf set with this parity Which proves equation \ref{eq:righttoleftpareven}.

Combining these two inequalities we get that \(\max(\inf_{\Omega_\A}(\rho_{w'}))=\max(\inf_{\Omega_{\A^s}}(\rho_{w}))-2\) and since $\max(\inf_{\Omega_{\A^s}}(\rho_{w}))$ is even we see that  \(
  \max(\inf_{\Omega_\A}(\rho_{w'}))\) is even which proves this claim.
\end{proof}

\noindent Now, by Claims \ref{claim:subsetrhow'run} and \ref{claim:subsetrhow'pareven}, we see that $\rho_{w'}$ is an accepting run of $\A$ on $w'$. Which means that $w'\in\LL(\A)$. From Claim \ref{claim:subsetwwb[f]} we get $w=w_b[f]$ and since we defined $w'=:w_b[f']$ we know that $w\sim_s w'$. Combining these two means that $w\in (\LL(\A))^s$ which proves this equation \ref{eq:lefttorightinclusion}
\subsubsection*{Right to left inclusion}
Now we will prove the $\supseteq$ direction
\begin{equation}
 \mathcal{L}(\mathbb{A}^s) \supseteq (\mathcal{L}(\mathbb{A}))^s.
\end{equation}\label{eq:righttoleftinclusion}
Suppose we have $w\in (\mathcal{L}(\mathbb{A}))^s$ then there exists a word $w'\in\LL(\A)$ such that $w\sim_s w'$. This means that there exists a base word $w_b\in \Sigma$ and stutter-functions $f, f':\omega\to\omega^+$ such that $w = w_b[f]$ and $w'=w_b[f']$

To prove that $w\in \LL(\A^s)$ we will first prove that $w_b\in \LL(\A^s)$ and then define an accepting run $\rho_w$ on $\A$ for $w$ to show that $w\in\LL(\A^s)$. Since $w'\in\LL(\A)$ we know that there exists an accepting run $\rho_{w'}$ on $\A$ of $w'$. We will use this run as the basis for $\rho_{w_b}$ which we will show is an accepting run on $\A^s$ of $w_b$.
\begin{definitiont}\label{def:supsetrhowb}
 To define $\rho_{w_b}:\omega\to A^s$ take $i\in\omega$.\\
 For $i=0$ we set $\rho_{w_b}(0) = (\rho_{w'}(0), \Omega(\rho_{w'}(0)))$.\\
 For $i>0$ let $j:=\sum_{k=0}^{i-2}f'(k)$, $n= f'(i)$ and $m:=\max\{\Omega_\A(\rho_{w'}(k))\mid k\in \{j+1,\dots, j+n\}\}$. Now set $\rho_{w_b}(i) :=(\rho_{w'}(j+n), m)$.
\end{definitiont}

\noindent Now we need to show that this definition is correct so that $\rho_{w_b}$ is indeed an accepting run. We will split this into two claims. Firstly we will show that the $\rho_{w_b}$ defines and run after which we will show that this is an accepting run.
\begin{claim}\label{claim:supsetrhowbrun}
Let $\rho_{w_b}$ be the sequence defined in  Definition \ref{def:supsetrhowb}. Then $\rho_{w_b}$ defines a run of $\A^s$ on $w_b$,i.e. we have for all $i\in\omega$ that $\rho_{w_b}(0)\in q_{I,\A^s}$ and $\rho_{w_b}(i+1)\in \delta_{\A^s}(\rho_{w_b}(i), w_b(i))$.
\end{claim}

\noindent We will omit the proof here, it can be found in the appendix \ref{appendix:supsetrhowbrun} and mainly consists of unfolding the definitions. Now we will show that $\rho_{w_b}$ is accepting, or in other words:
\begin{claim}\label{claim:supsetrhowbpareven}
Let $\rho_{w_b}$ be the run of $\A$ on $w_b$ as defined in Definition \ref{def:supsetrhowb} then we have that
 \[
  \max\left(\inf_{\Omega_{\A^s}}(\rho_{w_b})\right) \text{ is even}.
 \]
\end{claim}
\begin{proof}
We will prove $\max\left(\inf_{\Omega_{\A^s}}(\rho_{w_b})\right)-2 = \max\left(\inf_{\Omega_\A}(\rho_{w'})\right)$ via two inequalities.
\subsubsection*{Left to right inequality}
We will first prove
\begin{equation}
 \max\left(\inf_{\Omega_{\A^s}}(\rho_{w_b})\right)-2 \leq \max\left(\inf_{\Omega_\A}(\rho_{w'})\right)
\end{equation}\label{eq:supsetrhowbparevenleft}
Let $(a,m)\in\inf(\rho_{w_b})$ with $\Omega_{\A^s}((a,m)) = \max\left(\inf_{\Omega_{\A^s}}(\rho_{w_b})\right)$. That means that $m=\max\left(\inf_{\Omega_{\A^s}}(\rho_{w_b})\right)-2$. Now take $i\in\omega$ such that such that $\rho_{w_b}(i)=(a,m)$ and let $j=\sum_{k=0}^{i-1}f'(k)$. Now we see, from Definition \ref{def:supsetrhowb}, that $m=\max\{\rho_{w'}(l)\mid l\in\{j+1,\dots,j+f'(i)\}\}$ so there exists a $l$ such that $\Omega_\A(\rho_{w'}(l))=m$. Since there are infinitely many $i$'s such that $\rho_{w_b}(i)=(a,m)$ and every $i$ gives a unique $l$ there are infinitely many $l$'s such that $\Omega_\A(\rho_{w'}(l))=m$ and since we have finitely many states in $A^s$ we now know that there is a state $a'\in\inf_{\rho_{w'}}$ that occurs infinitely often with $\Omega_{\A}(a')=m=\max\left(\inf_{\Omega_{\A^s}}(\rho_{w_b})\right)-2$ which proves equation \ref{eq:supsetrhowbparevenleft}.
\subsubsection*{Right to left inequality}
Now we will prove
\begin{equation}
 \max\left(\inf_{\Omega_{\A^s}}(\rho_{w_b})\right)-2 \geq \max\left(\inf_{\Omega_\A}(\rho_{w'})\right).
\end{equation}\label{eq:supsetrhowbparevenright}
Let $a\in\inf(\rho_{w'})$ with $\Omega_\A(a)=\max\left(\inf_{\Omega_\A}(\rho_{w'})\right)$. Take $i>\sum_{m=0}^jf'(m)\geq i_d$, where $i_d$ is the decisive moment and $j$ is any natural number, such that $\rho_{w'}(i)=a$. Define $i_b$ as the smallest integer such that $i\leq \sum_{m=0}^{i_b}f'(m)$. We see that \(\rho_{w_b}(i_b+1)=\left(\rho_{w'}\left(\sum_{m=0}^{i_b}f'(m)\right),m\right)\) with \(m=\max\left\{\Omega_\A\left(\rho_{w'}(k)\right)\mid k\in \left\{\sum_{m=0}^{i_b-1}f'(m)+1, \dots, \sum_{m=0}^{i_b}f'(m)\right\}\right\}\). Since we took $i>\sum_{m=0}^jf'(m)\geq i_d$ we see that $\sum_{m=0}^{i_b-1}f'(m)\geq i_d$ so that means that $\rho_{w'}(j)\in\inf(\rho_{w'})$ for all $j\in \{\sum_{m=0}^{i_b-1}f'(m)+1, \dots, \sum_{m=0}^{i_b}f'(m)\}$ which means that the maximum parity on this part of $\rho_{w'}$ is smaller than $\max\left(\inf_{\Omega_\A}(\rho_{w'})\right)$. Since $i\in \{\sum_{m=0}^{i_b-1}f'(m)+1, \dots, \sum_{m=0}^{i_b}f'(m)\}$ we now see that
\[
 m=\Omega_\A(a).\]
Which means that $\Omega_{\A^s}(\rho_{w_b}(i_b+1))-2 = \Omega_\A(a)$ so that gives infinitely many elements of $\rho_{w_b}$ with the desired parity. Since there is a finite amount of states in $A^s$ we know at least one should occur infinitely often which proves equation \ref{eq:supsetrhowbparevenright}.

Combining both inequalities we see that $\max\left(\inf_{\Omega_{\A^s}}(\rho_{w_b})\right)-2 = \max\left(\inf_{\Omega_\A}(\rho_{w'})\right)$ and since $\rho_{w'}$ is an accepting run of $\A$ we have that $\max\left(\inf_{\Omega_\A}(\rho_{w'})\right)$ is even so $\max\left(\inf_{\Omega_{\A^s}}(\rho_{w_b})\right)$ is even as well. And this proves the claim.
\end{proof}

\noindent Following Claim \ref{claim:supsetrhowbrun} and Claim \ref{claim:supsetrhowbpareven} we know that $\rho_{w_b}$ is an accepting run of $\A^s$ on $w_b$ so we see that $w_b\in \LL(\A^s)$ is accepted. The next step is to prove that $w\in\LL(\A^s)$ is accepted. We will follow the same procedure, define a sequence $\rho_{w'}$ and show that this is an accepting run.

\begin{definitiont}\label{def:supsetrhow}
Define the sequence $\rho_w:\omega\to A^s$ as following. For $i=0$ set $\rho_{w}(0):=\rho_{w_b}(0)$. And for $i>0$ let $i_b$ as the smallest integer such that $i\leq\sum_{m=0}^{i_b} f(m)$. This $i_b$ is the index in the base word that corresponds to the index $i$ in $w'$. Now we distinguish cases.
\begin{description}
 \item[Case $f(i_b)=1$] Set $\rho_{w}(i) := \rho_{w_b}(i_b+1)$
 \item [Case $f(i_b)=n>1$] Let $(a_n, m_n):=\rho_{w_b}(i_b+1)$ and $j:= i - \sum_{m=0}^{i_b-1} f(m)$. Now we again distinguish different cases:
 \begin{description}
  \item[Case $(a_n, w(i))\in \delta((\rho_{w}(i-1), w(i-1))$] Now set
  \[
   \rho_w(i) := \begin{cases}
                   (a_n, w(i)) &\text{ if } i < \sum_{m=0}^{i_b} f(m)\\
                   (a_n,\Omega_\A(a_n)) &\text{ if } i = \sum_{m=0}^{i_b} f(m)
                  \end{cases}
  \]
  \item[Case $j < f(i_b)$] then set $\rho_w(i) := \left(\rho_{w'}\left(\sum_{m=0}^{i_b-1}f'(m)+j\right),\Omega_\A\left(\rho_{w'}\left(\sum_{m=0}^{i_b-1}f'(m)+j\right)\right)\right)$. This follows the run on $w'$.
  \item[Case $j=f(i_b)$] Now since we know that $(a_n, w(i))\notin \delta((\rho_{w}(i-1), w(i))$ we see that $j=f(i_b) < f'(i_b)$. So that means we need to take a shortcut transition that starts at $\text{start} = j + \sum_{m=0}^{i_b-1}f'(m)$ and ends at $\text{end} =  \sum_{m=0}^{i_b}f'(m)$ so define
  \[
   \rho_{w}(i) := (\rho_{w'}(\text{end}), \max\left\{\Omega_\A(\rho_{w'}(k))\mid k\in \{\text{start},\dots,\text{end}\}\right\})
  \]
 \end{description}
\end{description}
\end{definitiont}
The following observation will help us in the proofs of the last claims.
\begin{plain}[\textbf{Observation}]
For every $i_b\in\omega$ we see that $\rho_w\left(\sum_{m=0}^{i_b}f(m)\right)=\rho_{w_b}(i_b+1)$.
\end{plain}
Now we need to show that the sequence $\rho_w$ is indeed an accepting run. We will split this into two claims, first we will prove that $\rho_w$ defines a run of $\A^s$ on $w$ and secondly that this is an accepting run.
\begin{claim}\label{claim:supsetrhowrun}
Let $\rho_w$ be the sequence as defined in Definition \ref{def:supsetrhow}. Then $\rho_w$ defines a run of $\A$ on $w$, i.e. we have that $\rho_{w}(0)\in a^s_I$ and for all natural numbers $i\in\omega$ we have that $\rho_{w}(i+1)\in \delta_{\A^s}(\rho_w(i), w(i))$.
\end{claim}
We will omit the proof here since it only consists of writing out the definitions. It can be found in the appendix \ref{appendix:supsetrhowrun}. The next claim is to show that $\rho_w$ is an accepting run.

\begin{claim}\label{claim:supsetrhowpareven}
Let $\rho_w$ the run of $\A$ on $w$ as defined in Definition \ref{def:supsetrhow} then we have that
  \[
  \max\left(\inf_{\Omega_{\A^s}}(\rho_{w})\right) \text{ is even.}
 \]
 Or in other words: $\rho_w$ is accepting.
\end{claim}
\begin{proof}
We will prove that $\max\left(\inf_{\Omega_{\A^s}}(\rho_{w})\right)=\max\left(\inf_{\Omega_{\A^s}}(\rho_{w_b})\right)$ with two inequalities:
\subsubsection*{Left to right inequality}
First we will prove
\begin{equation}
 \max\left(\inf_{\Omega_{\A^s}}(\rho_{w})\right)\leq\max\left(\inf_{\Omega_{\A^s}}(\rho_{w_b})\right)
\end{equation}\label{eq:supsetrhowparevenleft}

% Assert: it cannot be a loop state since the successor of the loop state always has bigger priority. Hier moet ik nog wat mee??

Now take $a\in\inf(\rho_w)$ such that $\Omega_{\A^s}(a) = \max\left(\inf_{\Omega_{\A^s}}(\rho_{w})\right)$ that means that we have infinitely many $i$ such that $\rho_w(i)=a$. Take $i>\sum_{m=0}^jf(m)>i_d$, where $j$ is any natural number and $i_d$ the decisive moment. Then define $i_b$ as the smallest integer such that $i\leq \sum_{m=0}^{i_b}f(m)$. We distinguish cases based on $f(i_b)$:
\begin{description}
 \item[Case $f(i_b)=1$] Then we see that $\rho_{w}(i)=\rho_{w_b}(i_b+1)$
 \item[Case $f(i_b)=n>1$] Firstly we know that $\rho_{w}(i)\notin A\times\Sigma$ since then we would have $\Omega_{\A^s}(a)=1$ but the successors of all these states have bigger parity so it cannot be the maximum. Now let $j:=i - \sum_{m=0}^{i_b}f(m)$, if $j<f(i_b)$ then we see that $\Omega_{\A^s}$ is bigger or equal to all other states on this run after $i_d$. So in particular to
 \[
  \left\{\Omega_{\A^s}(\rho_{w}(k))\mid k\in \left\{\sum_{m=0}^{i_b-1}f(m)+1,\dots, \sum_{m=0}^{i_b}f(m)\right\}\right\}
 \]
 and we see that
 \[
  \max\left\{\Omega_{\A^s}(\rho_{w}(k))\mid k\in \left\{\sum_{m=0}^{i_b-1}f(m)+1,\dots, \sum_{m=0}^{i_b}f(m)\right\}\right\} = \Omega_{\A^s}(\rho_{w_b}(i_b))
 \]
\end{description}
We now have infinitely many $j$ such that $\Omega_{\A^s}(\rho_{w_b}(j))=\max\left(\inf_{\Omega_{\A^s}}(\rho_{w})\right)$ and since there are finitely many states this proves proves equation \ref{eq:supsetrhowparevenleft}.

\subsubsection*{Right to left inequality}
Secondly we will prove
\begin{equation}
 \max\left(\inf_{\Omega_{\A^s}}(\rho_{w})\right)\geq\max\left(\inf_{\Omega_{\A^s}}(\rho_{w_b})\right)
\end{equation}\label{eq:supsetrhowparevenright}
Take $a\in\inf(\rho_{w_b})$ such that $\Omega_{\A^s}(a)=\left(\inf_{\Omega_{\A^s}}(\rho_{w_b})\right)$. Now take a natural number $i\in\omega$ such that $\rho_{w_b}(i)=a$. Now distinguish cases: We know there are infinitely many $i$ such that $\rho_{w_b}(i)=q$. Take such an $i$ and distinguish cases:
\begin{description}
 \item[Case $f(i)=1$] Then we see that $\rho_w\left(\sum_{m=0}^{i}f(m)\right)=a$.
 \item[Case $f(i)=n>1$] That means that $\Omega_{\A^s}(a)$ is the maximum of parities on a partial run. We know that
 \[
  \max\left\{\Omega_{\A^s}(\rho_{w}(k))\mid \left\{\sum_{m=0}^{i-1}f(m)+1,\dots, \sum_{m=0}^{i-1}f(m)\right\}\right\} = \Omega_{\A^s}(\rho_{w_b}(i))
 \]
 so there is a $k$ such that $\rho_w(k)$ has the desired parity.
\end{description}
That gives us infinitely many $j$ such that $\Omega_{\A^s}(\rho_{w}(j))= \Omega_{\A^s}(q)$ and since there are finitely many states we know that there is a state in $\inf_{\rho_w}$ with the desired parity which proves the inequality.

Combining the two inequalities we have that $\max\left(\inf_{\Omega_{\A^s}}(\rho_{w})\right)=\max\left(\inf_{\Omega_{\A^s}}(\rho_{w_b})\right)$ and by applying Claim \ref{claim:supsetrhowbpareven} we know that
\[
 \max\left(\inf_{\Omega_{\A^s}}(\rho_{w})\right) \text{ is even},
\]
as was our proof goal.
\end{proof}
Combining Claim \ref{claim:supsetrhowrun} and Claim \ref{claim:supsetrhowpareven} gives us that $\rho_w$ is an accepting run so we know that $w\in\LL(\A^s)$ is accepted which proves \(\mathcal{L}(\mathbb{A}^s) \supseteq (\mathcal{L}(\mathbb{A}))^s
\) (Equation \ref{eq:righttoleftinclusion}).

Combining the two inclusions we conclude that
\[
 \LL(\A^s)=(\LL(\A))^s,
\]
as was our proof goal. So we see that $\A^s$ correctly recognizes the stutter-closure of $\LL(\A)$.
\end{proof}

Now we have given a definition to determine the stutter-closure of the language of an automaton. But how do we determine for a given automaton if its language is stutter-invariant? To check if the automaton is stutter-invariant want to check if $\LL(\A)=\LL(\A^s)$.

\begin{theorem}\label{thm:automatonstutinvariant}
 Given an \npa\ $\A$ there is an effective procedure to check if the language of $\A$ is stutter-invariant, in other words if $\LL(\A)=\LL(\A)^s$.
\end{theorem}
\begin{proof}
 We can rewrite the equation $\LL(\A)=\LL(\A)^s$ into $\LL(\A)^s\cap \overline{\LL(\A)}=\emptyset$ and apply Theorem \ref{thm:stutterclosednpa} to obtain the following equation: $\LL(\A^s)\cap \overline{\LL(\A)}=\emptyset$. Where $\overline{\LL(\A)}$ denotes the complement of $\LL(\A)$. We can use the standard boolean operations intersection and complementation and a non-emptiness check to perform this procedure. However, these procedures impose a possibly exponential size blowup so this might not be super efficient \cite{boker2019automatatranslatoins}.
\end{proof}

% Hier toevoegen dat je de loop states weg kan halen, maar dat is misschien niet echt nodig op dit moment alleen als het $pUq$ bewijs toegevoegd wordt.
% \begin{theorem}
%  We can remove the states of the form
%  \((q, l)\) if $q\in \delta_\A(q, l)$.
% \end{theorem}

