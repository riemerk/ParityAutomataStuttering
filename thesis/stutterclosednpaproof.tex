In this section I will prove that the construction presented in definition \ref{def:stutterclosed} is correct.
\begin{theorem}
Let $\A$ a NPA and $\A^s$ it's stutter closed automaton as defined in definition \ref{def:stutterclosed}. The language of $\A^s$ is the stuttterclosure of the language of $\A$:
\[
 \mathcal{L}(\mathbb{A}^s) = (\mathcal{L}(\mathbb{A}))^s
\]
\end{theorem}
\begin{proof}
First we will prove the direction \( \mathcal{L}(\mathbb{A}^s) \subseteq (\mathcal{L}(\mathbb{A}))^s\).\\
Syppose we have $w\in \LL(\A^s)$ we need to show that there exists $w'\in \LL(\A)$ such that $w\sim_sw'$. In other words we need to find a $w_b\in \Sigma^\omega$ and $f, f':\N\to\N^+$ such that $w=w_b[f]$ and $w'=w_b[f']$. Firstly I will define $w_b$ and $f$.\\
\begin{definitiont}\label{def:subsetrhow}
From $\rho_w$ and $w$ start to define $w_b$ and $f$. To do this take $i\in \omega$. First let $l := \sum_{m=0}^{i-1}f(m)$. Now we look at $\rho_w$. Since $\rho_w$ is an accepting run we know it cannot loop infinitely in a state of the form $(q,a)\in Q_\A\times\Sigma$ since $\Omega_{\A^s}(q,a)=1$ and not even. So therefore there exists a smallest $n$ such that $\rho_w(l+n)\in Q_\A\times\Omega_\A(Q_\A)$. \\ Now set $f(i):=n$ and $w_b(i) := w(l)$
\end{definitiont}
\begin{claim}\label{claim:wwb[f]}
 Let $w_b$ and $f$ from Definition \ref{def:subsetrhow} then $w=w_b[f]$.
\end{claim}
\begin{proof}
We need to prove that $w_b(k)=w(\sum_m=0^{k-1}f(m) + j)$ for $0\leq j < f(k)$ for every $k\in \omega$. We will prove this with induction on $k$.
\begin{description}
 \item[Base case $k=0$] We defined $f(k)$ as the smallest $n$ such that $\rho_w(n)\in Q_\A\times \Omega_A(Q_\A)$ so that $\rho_w(j)\in Q_\A\times \Omega_A(Q_\A)$ for $1\leq j<n$. Since since $\rho_w$ is a run on $w$ we know that $\rho_w(i+1)\in\delta_\A(\rho_w(i), w(i))$ and that leaves us only one possibility for $w(i)$ with $0\leq i < n$ namely the $w(0)$ and that is exactly how we defined $w_b(k)$. So that means $w_b(k) = w(i)$ for $0\leq i < n$.
 \item[Inductive case $k>0$] With the inductive hypothesis we know that $w_b(k-1) = w(\sum_m=0^{k-1}f(m) -1)$. Then from the definition of $f$ we again see that $w(\sum_m=0^{k-1}f(m) +i)=w(\sum_m=0^{k-1}f(m))$ for $0\leq i < n$ and since $w_b(k)=w(\sum_m=0^{k-1}f(m))$ we see that $w_b(k) = w(\sum_m=0^{k-1}f(m) +i)$ for $0\leq i< n$.
\end{description}
This proves that $w=w_b[f]$.
\end{proof}
Define $w'$ as $w_b[f']$. Nextly we will define $f'$ and $\rho_{w'}$.
\begin{definitiont}\label{def:subsetrhow'}
 To define $f'$ and $\rho_{w'}$. Let $(q, p) = \rho_w(0)$ then set $\rho_{w'} = q$. Secondly take $i\in\omega$ and distingish cases
\begin{description}
 \item[Case $f(i)=1$] Let $(q, p):= \rho_{w}(\sum_{m=0}^{i-1}f(m))$ and $(q', p'):= \rho_{w}(\sum_{m=0}^{i}f(m))=\rho_{w}(\sum_{m=0}^{i-1}f(m)+1)$. Now since $(q', p')\in \delta_{\A^s}((q,p), w(\sum_{m=0}^{i-1}f(m)))$. From Claim 1 we see that $w(\sum_{m=0}^{i-1}f(m)) = w_b(i)$ so we know that $\exists n>0$ and $\rho_s=qq_1\dots q_{n-1}q'$ such that $q\twoheadrightarrow_{\rho_s}^{(w_b(i))^n}q'$ and $p'=\max\{\Omega_\A(\rho_s(j))\mid j\in \{1,\dots,n\}\}$. Set $f'(i)=n$ and
 \[
  \rho_{w'}\left(j +\sum_{m=0}^{i-1}f'(m)\right) = \rho_s(j)\text{ for } j\in \{1,\dots,n\}
 \]
 \item[Case $f(i)\neq 1$] Now set $f'(i):=1$  let $(q, p):= \rho_{w}\left(\sum_{m=0}^if(m)\right)$ and set $\rho_{w'}\left(\sum_{m=0}^if'(m)\right)~=~q$
\end{description}
\end{definitiont}


Now Now we need to show that $\rho_{w'}$ is indeed an accepting run on $w'$, in other words:
\begin{claim}
 Let $\rho_{w'}$ from Definition \ref{def:subsetrhow'} then $\rho_{w'}(i+1)\in\delta_{\A}(\rho_{w'}(i), w(i))$ and $\rho_{w'}(0)\in q_{I,\A}$
\end{claim}
\textcolor{red}{Dit is ook een saai bewijs misschien, voor in de appendix??}
\begin{proof}
First $\rho_{w'}(0)\in q_{I,\A}$. Let $\rho_{w}(0)=(q,p)$ we know that $\rho_{w'}(0)=q$. Since $\rho_{w'}$ is a run of $\A^s$ we know that $\rho_{w'}(0)=(q,p)\in q_{I, \A^s} = \{(q, \Omega(q))\mid q\in q_{I, \A}\}$ so $\rho_{w'}=q\in q_{I,\A}$. Now we will prove that $\rho_{w'}(i+1)\in\delta_{\A}(\rho_{w'}(i), w(i))$.\\
Let $i\in \omega$. Let $i_b$ as the smallest integer such that $i+1\leq\sum_{m=0}^{i_b} f'(m)$. Let $(q,p)= \rho_w(\sum_{m=0}^{i_b}f(m)$, from the definition we know that $\rho_{w'}(\sum_{m=0}^{i_b}f'(m))=q$. Distinguish cases (the same as in the definition):
\begin{description}
 \item[Case $f(i_b)=1$] Let $j$ such that $i = \sum_{m=0}^{i_b} f'(m) + j -1$ $(j>1)$ then we see that $\rho_{w'}\left(j +\sum_{m=0}^{i-1}f'(m)\right) = \rho_s(j)$ and since $q\twoheadrightarrow_{\rho_s}^{(w_b(i_b))^n}q'$ we see that $\rho_s(j)\in \delta_\A(\rho_s(j-1), w_b(i_b))$. So that means that
 \begin{align*}
  \rho_{w'}(i+1) &= \rho_{w'}(j +\sum_{m=0}^{i-1}f'(m)) = \rho_s(j)\\
  &\in \delta_\A(\rho_s(j-1), w_b(i_b))\\
  &= \delta_\A(\rho_{w'}(\sum_{m=0}^{i_b} f'(m) + j -1), w_b(i_b)) \\
  &= \delta_\A(\rho_{w'}(i), w'(i))
 \end{align*}
 \item[Case $f(i_b)=n\neq 1$]. That means that $f'(i)=1$ (we know that from Definition \ref{def:subsetrhow'}). Let $\rho_{w}(\sum_{m=0}^{i_b-1}f(m))=(q, p)$ From Definition \ref{def:subsetrhow} we see that $\rho_{w}(l+\sum_{m=0}^{i_b-1}f(m))=(q', w_b(i_b))$ for $1\leq l <n$ and $\rho_{w}(\sum_{m=0}^{i_b}f(m))=(q', \Omega(q'))$ since $\rho_w$ is a run we know then that $q'\in \delta_\A(q, w_b(i_b))$. Since $f'(i_b)=1$  we see that $i =  \sum_{m=0}^{i_b-1}f'(m)$ and $i+1 = \sum_{m=0}^{i_b}f'(m)$ therefore we see that
 \begin{align*}
  \rho_{w'}(i+1) &= \rho_{w'}(\sum_{m=0}^{i_b}f'(m))=\rho_{w}(\sum_{m=0}^{i_b}f(m))=q'\in \delta_\A(q, w_b(i_b)) \\
  &= \delta_\A(\rho_{w}(\sum_{m=0}^{i_b-1}f(m)), w_b(i_b)) = \delta_\A(\rho_{w'}(\sum_{m=0}^{i_b-1}f'(m)), w_b(i_b))\\
  &= \delta_\A(\rho_{w'}(i+1), w_b(i_b))
 \end{align*}
\end{description}
Combinining these cases proves the claim.
\textcolor{red}{Verwijzingen naar definities en claims checken en de haakjes goed doen. }
\end{proof}
\begin{claim}
 We have that
 \[
  \max(\inf_{\Omega_\A}(\rho_{w'}))\text{ is even}
 \]
\end{claim}
\begin{proof}
We will first prove that \(\max(\inf_{\Omega_\A}(\rho_{w'}))=\max(\inf_{\Omega_{\A^s}}(\rho_{w}))-2\) via the two inequalities. First \\
\(\leq\) Let $q\in\inf(\rho_{w'})$ be a state with $\Omega_\A(q)=\max(\inf_{\Omega_\A}(\rho_{w'}))$. That means that we have infinitely many $i's$ such that $\rho_{w'}(i)=q$. Now take $i>\sum_{m=0}^{j} f'(m) > i_d$. Dan definieer nu $i_b$ dan ofwel $f(i_b)=1$ dan ligt deze $q$ op $\rho_s$ en gelijk aan max dus $p=\Omega(q)$. Als $f(i_b)=n$ dan zie je dat $q=\rho_w(\sum_{m=0}^{i_b}f(m))$ dus $\Omega_{\A^s}(\rho_w(\sum_{m=0}^{i_b}f(m)))-2=\Omega_\A(q)$ dus oneindig veel $i$tjes dus $\max(\inf_{\Omega_\A}(\rho_{w'}))\leq \max(\inf_{\Omega_{\A^s}}(\rho_{w}))-2$.\\
\(\geq\) Let $(q,p)\in\inf(\rho_{w})$ be a state with $\Omega_{\A^s}((q,p))=\max(\inf_{\Omega_{\A^s}}(\rho_{w}))$. Now take $i> i_d$. Nu neem $i_b$ dan cases $f(i_b)=n$ dan zie je dat $\rho_{w}(i) = \rho_w(\sum_{m=0}^{i_b}f(m)) = (q,\Omega(q)) = (\rho_{w'}(\sum_{m=0}^{i_b}f'(m)), \Omega(\rho_{w'}(\sum_{m=0}^{i_b}f'(m))))$ and $\Omega(\rho_{w}(i))-2=\Omega(q)$ so we have that $\max(\inf_{\Omega_\A}(\rho_{w'}))\geq \max(\inf_{\Omega_{\A^s}}(\rho_{w}))-2$ since we have infinitely many $q$ with $\Omega_\A(q)=...$.\\

Combining these two inequalities we get \(\max(\inf_{\Omega_\A}(\rho_{w'}))=\max(\inf_{\Omega_{\A^s}}(\rho_{w}))-2\) and since $\max(\inf_{\Omega_{\A^s}}(\rho_{w}))$ is even we have that  \(
  \max(\inf_{\Omega_\A}(\rho_{w'}))\) is even.
\end{proof}
Combining claims 2 and 3 we know that $w'\in \LL(\A)$ and from claim 1 and the fact that $w'=w_b[f']$ we know that $w\sim_s w'$ so that means that $w\in (\LL(\A))^s$ which proves the inclusion..\\\\
Secondly the other direction \(\mathcal{L}(\mathbb{A}^s) \supseteq (\mathcal{L}(\mathbb{A}))^s
\)

Suppose we have $w\in (\mathcal{L}(\mathbb{A}))^s$ then there exists a $w'\in\LL(\A)$ such that $w\sim_s w'$. Recall definition about language of $\A$... Following Definition \ref{def:stutequiv} we know there exists $w_b\in \Sigma$ and $f, f':\N\to\N^+$ such that $w = w_b[f]$ and $w'=w_b[f']$ \\
To prove that $w\in \LL(\A^s)$. Since $w'\in\LL(\A)$ we know that there exists an accepting run $\rho_{w'}$ in $\A$. I will first prove that $w_b\in \LL(\A^s)$ and then give an accpeting run for $w$. To show that $w_b\in\LL(\A^s)$ I will give an run and show that this is accepting.\\
\textbf{Definition} (2) To define $\rho_{w_b}:\omega\to Q_{\A^s}$ take $i\in\omega$. Since $w'\in\LL(\A)$ we know that there is an accepting run $\rho_{w'}:\omega\to Q_\A$ on $w'$. We will use this as the basis for $\rho_{w_b}$. Set $\rho_{w_b}(0) = (\rho_{w'}(0), \Omega(\rho_{w'}(0)))$. To define $\rho_{w_b}(i)$ for $i>0$ Let $j=\sum_{m=0}^{i-2}f'(m)$ and $n= f'(i)$ and let $p=\max\{\rho_{w'}(k)\mid k\in \{j+1,\dots, j+n\}\}$. Now set $\rho_{w_b}(i)=(\rho_{w'}(j+n), p)$.
\setcounter{claim}{0}
\textcolor{red}{Misschien dit bewijs in een appendix doen, is niet zo heel interessant namelijk}
\begin{claim}
Let $\rho_{w_b}$ from the definition above, then for all $i$ we have $\rho_{w_b}(i+1)\in \delta_{\A^s}(\rho_{w_b}(i), w_b(i))$ and $\rho_{w_b}(0)\in q_{I,\A^s}$.
\end{claim}
\begin{proof}
First we prove that $\rho_{w_b}(i+1)\in \delta_{\A^s}(\rho_{w_b}(i), w_b(i))$ for all $i\in\omega$.

Let $j=\sum_{m=0}^{i-1}f'(m)$ and $n=f'(i)$
 \begin{align*}
  \rho_{w_b}(i+1)&=\left(\rho_{w'}\left(j+f'(i), \max\{\rho_{w'}(k)\mid k\in \{j+1,\dots, j+n\}\}\right)\right)\text{ Definition ...}\\
  &=\left(\rho_{w'}\left(j+f'(i), \max\{\rho_{w'}(k)\mid k\in \{j+1,\dots, j+n\}\}\right)\right)\\
  &\in \{(q_n, p) \in Q\times \Omega(Q)\mid \exists n, \exists \rho= qq_1\dots q_n \land q\twoheadrightarrow_\rho^{w_b(i)}q_n\land\\
  & \phantom{\in\{}p = \max(\Omega(\{q_i\mid i\in \{1, \dots, n\}\}))\land q = \rho_{w'}(j)\}
  \intertext{To understand this inclusion we set $n=f'(i)$ and $\rho=\rho_{w'}(j)\dots\rho_{w'}(j+n)$ and since $w_b(i)=w'(j+k)$ for $k<f'(i)$ we see that $\rho$ is a partial run.}
  &\subseteq \delta_{\A^s}({\rho_{w_b}(i), w_b(i))})\text{ Since $\rho_{w_b}(i)=(\rho_{w'}(j), p)$ for a $p\in\Omega_\A(Q_\A)$}
 \end{align*}
which proves the first part of the  claim and now the second part about the initial state. We see that $\rho_{w_b}(0)=(\rho_{w'}(0),\Omega_\A(\rho_{w'}(0)))$ and since $\rho_{w'}$ is a run we know that $\rho_{w'}(0)\in q_{I,\A}$ so that means that $\rho_{w_b}(0)\in\{(q, \Omega_\A(q))\mid q\in q_{I,\A}\}=q_{I,\A^s}$.
\end{proof}
\begin{claim}
 \[
  \max\left(\inf_{\Omega_{\A^s}}(\rho_{w_b})\right) \text{ is even}
 \]
\end{claim}
\begin{proof}
I will fist prove $\max\left(\inf_{\Omega_{\A^s}}(\rho_{w_b})\right)-2 = \max\left(\inf_{\Omega_\A}(\rho_{w'})\right)$ via two inequalities.\\
First $\max\left(\inf_{\Omega_{\A^s}}(\rho_{w_b})\right)-2 \leq \max\left(\inf_{\Omega_\A}(\rho_{w'})\right)$\\
We know there exists $q\in\inf(\rho_{w_b})$ with $\Omega_{\A^s}(q) = \max\left(\inf_{\Omega_{\A^s}}(\rho_{w_b})\right)$. This occurs infinitely and we will show that there exists a state in $\rho_{w'}$ with parity $\Omega_{\A^s}(q)-2$ that also occurs infinitely. Take a $i$ such that $\rho_{w_b}(i)=q$. Let $j=\sum_{m=0}^{i-1}$ Now we see that $\Omega_{\A^s}(q)=\max\{\rho_{w'}(k)\mid k\in\{j+1,\dots,j+f'(i)\}\}+2$ so there exists a $k$ such that $\Omega_\A(\rho_{w'}(k))=\Omega_{\A^s}(q)-2$. Since there are infinitely many $i$'s such that $\rho_{w_b}(i)=q$ and every $i$ gives a unique $k$ there are infinitely many $k$'s such that $\Omega_\A(\rho_{w'}(k))=\Omega_{\A^s}(q)-2$ and since we have finitely many  states in $Q$ we now know that there is a state $q'\in\inf_{\rho_{w'}}$ with $\Omega_{\A}(q')=\max\left(\inf_{\Omega_{\A^s}}(\rho_{w_b})\right)-2$ which proves this inequality.\\
Now $\max\left(\inf_{\Omega_{\A^s}}(\rho_{w_b})\right)-2 \geq \max\left(\inf_{\Omega_\A}(\rho_{w'})\right)$\\
We know there exists $q\in\inf(\rho_{w'})$ with $\Omega_\A(q)=\max\left(\inf_{\Omega_\A}(\rho_{w'})\right)$. Now define $i_d$ as the decisive moment and take $i>\sum_{m=0}^jf'(m)\geq i_d$ that means that every state $\rho_{w'}(i)$ has smaller or equal parity then $q$. Now define $i_b$ as the base index and we see that $\rho_{w_b}(i_b+1)=\{\rho_{w'}(\sum_{m=0}^{i_b}f'(m)), \max\{\Omega_\A(\rho_{w'}(k))\mid k\in \{\sum_{m=0}^{i_b-1}f'(m)+1, \dots, \sum_{m=0}^{i_b}f'(m)\}\}\}$ where $\max\{\Omega_\A(\rho_{w'}(k))\mid k\in \{\sum_{m=0}^{i_b-1}f'(m)+1, \dots, \sum_{m=0}^{i_b}f'(m)\}\}=\Omega_\A(q)$ since $\sum_{m=0}^{i_b-1}f'(m)\geq i_d$. That means that $\Omega_{\A^s}(\rho_{w_b}(i_b+1))-2 = \Omega_\A(q)$ so that proves the inequality.\\
Combining the two inequalities we have that $\max\left(\inf_{\Omega_{\A^s}}(\rho_{w_b})\right)-2 = \max\left(\inf_{\Omega_\A}(\rho_{w'})\right)$ and since $\rho_{w'}$ is a run we have that $\max\left(\inf_{\Omega_\A}(\rho_{w'})\right)$ is even so $\max\left(\inf_{\Omega_{\A^s}}(\rho_{w_b})\right)$ as well.
\end{proof}
Following Claim 1 and 2 we know that $\rho_{w_b}$ is $w_b\in \LL(\A^s)$. Now we want to proof that $w\in\LL(\A^s)$ to do that we give an accepting run on $w$.\\
\textbf{Definition} Define $\rho_w:\omega\to Q_{\A^s}$ as following. Set $\rho_{w}(0)=\rho_{w_b}(0)$. For $i>0$ let $i_b$ as the smallest integer such that $i\leq\sum_{m=0}^{i_b} f(m)$. This $i_b$ is the index in the base word that corresponds to the index $i$ in $w'$. Now we distinguish cases.
\begin{description}
 \item[Case $f(i_b)=1$] Set $\rho_{w}(i) = \rho_{w_b}(i_b+1)$
 \item [Case $f(i_b)=n>1$] Let $(q_n, p_n):=\rho_{w_b}(i_b+1)$ and $j= i - \sum_{m=0}^{i_b-1} f(m)$ Now we again have to distinguish cases:
 \begin{description}
  \item[Case $(q_n, w(i))\in \delta((\rho_{w}(i-1), w(i-1))$] Now set
  \[
   \rho_w(i) = \begin{cases}
                   (q_n, w(i)) &\text{ if } i < \sum_{m=0}^{i_b} f(m)\\
                   (q_n,\Omega_\A(q_n)) &\text{ if } i = \sum_{m=0}^{i_b} f(m)
                  \end{cases}
  \]
  \item[Case $j < f(i_b)$] then set $\rho_w(i) = \left(\rho_{w'}\left(\sum_{m=0}^{i_b-1}f'(m)+j\right),\Omega_\A\left(\rho_{w'}\left(\sum_{m=0}^{i_b-1}f'(m)+j\right)\right)\right)$. This is the run in $w'$.
  \item[Case $j=f(i_b)$] Now since we know that $(q_n, w(i))\notin \delta((\rho_{w}(i-1), w(i))$ we see that $j=f(i_b) < f'(i_b)$. So that means we need to take a shortcut transition in the following form: Let $\text{start} = j + \sum_{m=0}^{i_b-1}f'(m)$ and $\text{end} =  \sum_{m=0}^{i_b}f'(m)$
  \[
   \rho_{w}(i) = (\rho_{w'}(\text{end}), \max\{\Omega_\A(\rho_{w'}(k))\mid k\in \{\text{start},\dots,\text{end}\}\})
  \]
 \end{description}
\end{description}
observation: We see that $\rho_w(\sum_{m=0}^{i_b}f(m))=\rho_{w_b}(i_b+1)$

\begin{claim}\label{claim:rhowbwelldefined}
Let $\rho_w$ from definition... then for all $i$ we have $\rho_{w}(i+1)\in \delta_{\A^s}(\rho_w(i), w(i))$ and $\rho_{w}(0)\in q_{I,\A^s}$
\end{claim}
Take $i\in\omega$ and we look at $\rho_{w}(i+1)$. Let $i_b$ smallest integer such that $i+1\leq \sum_{m=0}^{i_b}f(m)$. Then distinguish cases
\begin{description}
 \item[Case $f(i_b)=1$] So $\sum_{m=0}^{i_b}f(m)=\sum_{m=0}^{i_b-1}f(m)+1$ That means that $i=\sum_{m=0}^{i_b-1}f(m)$ so $\rho_{w}(i)=\rho_{w}(\sum_{m=0}^{i_b-1}f(m)) = (\rho_{w_b}(i_b))$. That easily gives
 \begin{align*}
  \rho_{w}(i+1) &= \rho_{w_b}(i_b+1)\text{ From Definition ...}\\
                &\in\delta_{\A^s}((\rho_{w_b}(i_b), w_b(i_b)))\text{Since this is a run}\\
                &=\delta_{\A^s}\left(\left(\rho_{w}\left(\sum_{m=0}^{i_b-1}f(m)\right), w\left(\sum_{m=0}^{i_b-1}f(m)\right)\right)\right)\text{ Lemma \ref{lemma:stutbaseword} and observation}\\
                &=\delta_{\A^s}(\rho_{w}(i), w(i))
 \end{align*}
 \item [Case $f(i_b)=n>1$] Let $(q_n, p_n):=\rho_{w_b}(i_b+1)$ and $j= i + 1 - \sum_{m=0}^{i_b-1} f(m)$ Now we again have to distinguish cases:
 \begin{description}
  \item[Case $(q_n, w(i))\in \delta((\rho_{w}(i-1), w(i-1))$] We immediately see $\rho_w(i+1)\in\delta(\rho_w(i), w(i)$.
  \item[Case $j=1$] That means that $\rho_{w}(i) = \rho_{w}(\sum_{m=0}^{i_b-1}f(m))=\rho_{w_b}(i_b) = (\rho_{w'}(\sum_{m=0}^{i_b-1}f'(m), p)$ for a $p\in\omega$. Since we have
  \begin{align*}
   \rho_w(i+1) &= \left(\rho_{w'}\left(\sum_{m=0}^{i_b-1}f'(m)+1\right),\Omega_\A\left(\rho_{w'}\left(\sum_{m=0}^{i_b-1}f'(m)+1\right)\right)\right)\\
   &\in \left\{(q', \Omega(q'))\mid q'\in\delta\left(\rho_{w'}\left(\sum_{m=0}^{i_b-1}f'(m)\right), w'\left(\sum_{m=0}^{i_b-1}f'(m)\right)\right)\right\}\\
   &=\left\{(q', \Omega(q'))\mid q'\in\delta\left(\rho_{w'}\left(\sum_{m=0}^{i_b-1}f'(m)\right), w\left(\sum_{m=0}^{i_b-1}f(m)\right)\right)\right\}
   \intertext{Since we have $w\left(\sum_{m=0}^{i_b-1}f(m)\right) = w_b(i_b) = w'\left(\sum_{m=0}^{i_b-1}f'(m)\right)$ with Lemma \ref{lemma:stutbaseword}}
  &\subseteq \delta_{\A^s}(\rho_w(i), w(i))
  \end{align*}
  which proves the claim
  \item[Case $1<j<f(i_b)$] We use the fact that $w\left(\sum_{m=0}^{i_b-1}f(m)+j\right) = w_b(i_b) = w'\left(\sum_{m=0}^{i_b-1}f'(m)+j\right)$, so that means that we can use the copied transitions everywhere.
  \item[Case $j=f(i_b)$] We took a shortcut transition which is there sinc $\rho=\rho_{w'}(j + \sum_{m=0}^{i_b-1}f'(m)) \dots \rho_{w'}( \sum_{m=0}^{i_b}f'(m))$ is a partial run.
 \end{description}
\end{description}
\begin{proof}
Saaaaaai moet nog eventjes checken.
\end{proof}
\begin{claim}
  \[
  \max\left(\inf_{\Omega_{\A^s}}(\rho_{w})\right) \text{ is even}
 \]
\end{claim}
\begin{proof}
We will prove that $\max\left(\inf_{\Omega_{\A^s}}(\rho_{w})\right)=\max\left(\inf_{\Omega_{\A^s}}(\rho_{w_b})\right)$ with two inequalities:\\
Firstly: $\max\left(\inf_{\Omega_{\A^s}}(\rho_{w})\right)\leq\max\left(\inf_{\Omega_{\A^s}}(\rho_{w_b})\right)$\\
Assert: it cannot be a loop state since the successor of the loop state always has bigger priority.

Now take $q\in\inf_(\rho_w)$ such that $\Omega_{\A^s}(q) = \max\left(\inf_{\Omega_{\A^s}}(\rho_{w})\right)$ that means that we have infinitely many $i$ such that $\rho_w(i)=q$. Take $i>\sum_{m=0}^jf(m)>i_d$ then we look at $i_b$ if $f(i_b)=1$ we immediately see that this state is also in $\rho_{w_b}$. Else we see that if it is a normal state then it is part of the partial run in $w'$ and in particular it is bigger so the parity is equal to max. If it is shortcut than also bigger so max of partial run.
Secondly $\max\left(\inf_{\Omega_{\A^s}}(\rho_{w})\right)\geq\max\left(\inf_{\Omega_{\A^s}}(\rho_{w_b})\right)$\\
Now if $f(i_b)=1$ then we see immediately. Now else it is the max of the partial run. If the max is before $f(i_b)$ then it there is a normal state with this parity. Else it is part of the shortcut state.
\end{proof}
Now we know that $w\in\LL(\A^s)$ which proves that \(\mathcal{L}(\mathbb{A}^s) \supseteq (\mathcal{L}(\mathbb{A}))^s
\).
\end{proof}

