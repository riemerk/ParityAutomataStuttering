First some introduction here.
\begin{definition}\label{def:stutterclosedNPA}
Let \(\A = (\Sigma, Q_\A, \delta_\A, q_{I,\A}, \Omega_\A)\) be some NPA, where without loss of generality we assume that $\Sigma \cap \omega = \emptyset$. We will define its \textbf{Stutter closure} as the automaton \(\A^s = (\Sigma, Q_{\A^s} \delta_{\A^s}, q_{I,\A^s}, \Omega_{\A^s})\) as follows:
\begin{itemize}
 \item Define the new set of states as $Q_{\A^s} := Q_\A\times \Omega_\A(Q_\A) \cup Q_\A\times \Sigma$.
 \item Define the set of initial states $q_{I,\A^s}:=\{(q, \Omega(q))\mid q\in q_{I,\A}\}$
 \item Define the parity map $\Omega_{\A^s}$ as following:
 \[
 \Omega_{\A^s}(q) =\left\{\begin{array}{llll}
             p + 2 &\text{ if } q=(q', p)   &\text{with } p\in \Omega_\A(Q_\A)\\
             1 &\text{ if } q = (q', l) &\text{with } l\in \Sigma
            \end{array}\right.
\]

\end{itemize}
To define the transition function $\delta_{\A^s}$ let $(q,a)\in Q_{\A^s}$. We distinguish cases:
\begin{description}
 \item[Case $a\in \Sigma$] Then set the transition function to:
 \[
 \delta_{\A^s}((q, a), b) = \begin{cases}
                       \{(q, a), (q, \Omega_\A(q))\}&\text{if } a = b\\
                      \emptyset & \text{else}
                      \end{cases}
\]
\item[Case $a\in \Omega_\A(Q_\A)$] Then set the transition function to:
\begin{align*}
 \delta_{\A^s}((q, a), b) &= \{(q', \Omega_\A(q'))\in Q_\A \times \Omega_\A(Q_\A)\mid q'\in \delta_\A(q, b)\}\\
                    &\cup \{(q', b)\in Q\times \Sigma\mid  q'\in \delta_\A(q, b)\}\\
                    &\cup \{(q_n, p) \in Q_\A\times \Omega_\A(Q_\A)\mid \exists n, \exists \rho_p= qq_1\dots q_n \land q\twoheadrightarrow_{\rho_p}^{b^n}q_n\land\\
                    & \phantom{\cup\{} p = \max(\Omega_\A(\{q_i\mid i\in \{1, \dots, n\}\}))
\end{align*}
\end{description}
\end{definition}

Now that we have given this definition we have to proof that this is indeed correct.
\begin{theorem}\label{thm:stutterclosednpa}
Let \(\A = (\Sigma, Q_\A, \delta_\A, q_{I,\A}, \Omega_\A)\) a NPA and \(\A^s = (\Sigma, Q_{\A^s} \delta_{\A^s}, q_{I,\A^s}, \Omega_{\A^s})\) its stutter closure as defined in Definition \ref{def:stutterclosedNPA}. Then we have
\[
 \LL(\A^s) = (\LL(\A))^s
\]
\end{theorem}
\begin{proof} I will prove this equality by proving the $\subset$ and $\supset$ inclusions. Firstly I will prove the direction \( \mathcal{L}(\mathbb{A}^s) \subseteq (\mathcal{L}(\mathbb{A}))^s\).

Syppose we have $w\in \LL(\A^s)$. If we want to show that $w\in(\LL(\A))^s$ we need to find a word $w'\in \LL(\A)$ such that $w\sim_sw'$. Or more specifically we need to find a $w_b\in \Sigma^\omega$ and $f, f':\omega\to\omega^+$ such that $w=w_b[f]$ and $w'=w_b[f']$. Firstly we will recursively define $w_b$ and $f$. We take as input $\rho_w$ and $w$
\begin{definitiont}\label{def:subsetwbf} To define $w_b$ and $f$ take $i\in \omega$. Let $l := \sum_{m=0}^{i-1}f(m)$. Since $m<=i-1<i$ we have defined $f(m)$ so this is a well-defined sum. Since $\rho_w$ is an accepting run we know it cannot loop infinitely in a state of the form $(q,a)\in Q_\A\times\Sigma$ since $\Omega_{\A^s}(q,a)=1$ and not even. So therefore there exists a smallest $n>0$ such that $\rho_w(l+n)\in Q_\A\times\Omega_\A(Q_\A)$. Now set $f(i):=n$ and $w_b(i) := w(l)$.
\end{definitiont}
We need to prove that this $w_b$ and $f$ are indeed correct.
\begin{claim}\label{claim:subsetwwb[f]}
 Let $w_b$ and $f$ as in Definition \ref{def:subsetwbf}. Then $w=w_b[f]$.
\end{claim}
\begin{proof}
We need to prove that $w_b(i)=w\left(\sum_{m=0}^{i-1}f(m) + j\right)$ for $0\leq j < f(i)$ for every $k\in \omega$. We will prove this with induction on $i$.
\begin{description}
 \item[Base case $i=0$] We defined $f(i)$ as the smallest $n$ such that $\rho_w(n)\in Q_\A\times \Omega_A(Q_\A)$ so that means $\rho_w(j)\in Q_\A\times \Omega_A(Q_\A)$ for $1\leq j<n$. Since since $\rho_w$ is a run on $w$ we know that $\rho_w(i+1)\in\delta_\A(\rho_w(i), w(i))$ and that leaves us only one possibility for $w(i)$ with $0\leq i < n$ namely $w(0)$ and that is exactly how we defined $w_b(i)$. So that means $w_b(i) = w(j)$ for $0\leq j < n$.
 \item[Inductive case $i>0$] With the inductive hypothesis we know that $w_b(i-1) = w(\sum_{m=0}^{i-1}f(m)~-~1)$. Then from the definition of $f$ we again see that $w\left(\sum_{m=0}^{i-1}f(m) +j\right)=w\left(\sum_{m=0}^{i-1}f(m)\right)$ for $0\leq j < n$ and since $w_b(i)=w\left(\sum_{m=0}^{i-1}f(m)\right)$ we see that $w_b(i) = w\left(\sum_{m=0}^{i-1}f(m) +j\right)$ for $0\leq j< n$.
\end{description}
This proves that $w=w_b[f]$.
\end{proof}
Our second step is to define a $w'$ such that $w\sim_s w'$ and show that $w'\in\LL(\A)$. For $w'$ take $w_b[f']$ with $f'$ as in the next definition. To show that $w'\in\LL(\A)$ we will define $\rho_{w'}$.
\begin{definitiont}\label{def:subsetrhow'_f'}
 To define $f'$ and $\rho_{w'}$, take $i\in \omega$. Firstly for $i=0$ let $(q, p) = \rho_w(0)$ then set $\rho_{w'}(0) := q$. For $i>0$ distnguish cases.
\begin{description}[font=\normalfont\itshape]
 \item[Case $f(i)=1$] Let $(q, p):= \rho_{w}\left(\sum_{m=0}^{i-1}f(m)\right)$ and $(q', p'):= \rho_{w}\left(\sum_{m=0}^{i}f(m)\right)=\rho_{w}\left(\sum_{m=0}^{i-1}f(m)+1\right)$. From Claim \ref{claim:subsetwwb[f]} we see that $w\left(\sum_{m=0}^{i-1}f(m)\right) = w_b(i)$. Now since $\rho_{w}$ is a run in $\A^s$ we know $(q', p')\in \delta_{\A^s}\left((q,p), w\left(\sum_{m=0}^{i-1}f(m)\right)\right)$ so we know that $\exists n>0$ and $\rho_p=qq_1\dots q_{n-1}q'$ such that $q\twoheadrightarrow_{\rho_p}^{(w_b(i))^n}q'$ and $p'=\max\{\Omega_\A(\rho_p(j))\mid j\in \{1,\dots,n\}\}$. Set $f'(i):=n$ and
 \[
  \rho_{w'}\left(\sum_{m=0}^{i-1}f'(m)+j\right) := \rho_s(j)\text{ for } j\in \{1,\dots,n\}
 \]
 \item[Case $f(i)\neq 1$] Let $(q, p):= \rho_{w}\left(\sum_{m=0}^if(m)\right)$. Now set $f'(i):=1$ set $\rho_{w'}\left(\sum_{m=0}^if'(m)\right):=~q$
\end{description}
\end{definitiont}
Now we need to show that Definition \ref{def:subsetrhow'_f'} is correct so $\rho_{w'}$ is indeed an accepting run on $w'$. We will split this into two claims
\begin{claim}\label{claim:subsetrhow'run}
 Let $\rho_{w'}$ from Definition \ref{def:subsetrhow'_f'} then $\rho_{w'}(0)\in q_{I,\A}$ and $\rho_{w'}(i+1)\in\delta_{\A}(\rho_{w'}(i), w(i))$ for any $i\in\omega$
\end{claim}
\textcolor{red}{Dit bewijs is heel saai, misschien aan het einde plaatsen. Is namelijk gewoon definitie uitschrijven.}
\begin{proof}
To prove the first part of the claim let $\rho_{w}(0)=(q,p)$. From Definition \ref{def:subsetrhow'_f'} we know that $\rho_{w'}(0)=q$. Since $\rho_{w'}$ is a run of $\A^s$ we know that $\rho_{w'}(0)=(q,p)\in q_{I, \A^s} = \{(q, \Omega_\A(q))\mid q\in q_{I, \A}\}$ so $\rho_{w'}=q\in q_{I,\A}$. \\
Tho prove the second claim ($\rho_{w'}(i+1)\in\delta_{\A}(\rho_{w'}(i), w(i))$) let $i\in \omega$. Define $i_b$ as the smallest integer such that $i+1\leq\sum_{m=0}^{i_b} f'(m)$. Let $(q,p)= \rho_w\left(\sum_{m=0}^{i_b}f(m)\right)$. From Definition \ref{def:subsetrhow'_f'} we know that $\rho_{w'}\left(\sum_{m=0}^{i_b}f'(m)\right)=q$. Now we distinguish cases, these are almost the same as in Definition \ref{def:subsetrhow'_f'} since we want to check if this definition is correct.
\begin{description}
 \item[Case $f(i_b)=1$] Let $j:=i+1 - \sum_{m=0}^{i_b-1} f'(m)$ and $n:=f'(i_b)$. From Defintion
\ref{def:subsetrhow'_f'} we see that $\rho_{w'}\left(\sum_{m=0}^{i_b-1}f'(m)+j\right) = \rho_s(j)$ and since $q\twoheadrightarrow_{\rho_s}^{(w_b(i_b))^n}q'$ we see that $\rho_s(j)\in \delta_\A(\rho_s(j-1), w_b(i_b))$. So that means that
 \begin{align*}
  \rho_{w'}(i+1) &= \rho_{w'}\left(\sum_{m=0}^{i_b-1}f'(m)+j\right) = \rho_s(j)\\
  &\in \delta_\A(\rho_s(j-1), w_b(i_b))\\
  &\phantom{\in}= \delta_\A\left(\rho_{w'}\left(\sum_{m=0}^{i_b-1}f'(m)+j-1\right), w_b(i_b)\right) \\
  &\phantom{\in}= \delta_\A(\rho_{w'}(i), w'(i)),
  \intertext{ Where in the last step we use Lemma \ref{lemma:stutbaseword} to rewrite $w'(i)=w_b(i_b)$}
 \end{align*}
 \item[Case $f(i_b)=n\neq 1$]. That means that $f'(i_b)=1$ (we know that from Definition \ref{def:subsetrhow'_f'}). Let $\rho_{w}\left(\sum_{m=0}^{i_b-1}f(m)\right)=(q, p)$. From Definition \ref{def:subsetwbf} we see that $\rho_{w}\left(l+\sum_{m=0}^{i_b-1}f(m)\right)=(q', w_b(i_b))$ for $1\leq l <n$ and $\rho_{w}\left(\sum_{m=0}^{i_b}f(m)\right)=(q', \Omega(q'))$. Since $\rho_w$ is a run we know then that $q'\in \delta_\A(q, w_b(i_b))$. Since $f'(i_b)=1$  we see that $i = \sum_{m=0}^{i_b-1}f'(m)$ and $i+1 = \sum_{m=0}^{i_b}f'(m)$ therefore we see that:
 \begin{align*}
  \rho_{w'}(i+1) &= \rho_{w'}\left(\sum_{m=0}^{i_b}f'(m)\right)=\rho_{w}\left(\sum_{m=0}^{i_b}f(m)\right)=q'\\
  &\in \delta_\A(q, w_b(i_b)) = \delta_\A\left(\rho_{w}\left(\sum_{m=0}^{i_b-1}f(m)\right), w_b(i_b)\right) \\
  &= \delta_\A\left(\rho_{w'}\left(\sum_{m=0}^{i_b-1}f'(m)\right), w'\left(\sum_{m=0}^{i_b-1}f'(m)\right)\right)
  = \delta_\A(\rho_{w'}(i), w'(i))
 \end{align*}
\end{description}
Combinining these cases proves the claim.
\textcolor{red}{Verwijzingen naar definities en claims checken en de haakjes goed doen. }
\end{proof}
\begin{claim}\label{claim:subsetrhow'pareven}
 We have that
 \[
  \max(\inf_{\Omega_\A}(\rho_{w'}))\text{ is even}
 \]
\end{claim}
\begin{proof}
We will first prove that \(\max(\inf_{\Omega_\A}(\rho_{w'}))=\max(\inf_{\Omega_{\A^s}}(\rho_{w}))-2\) via two inequalities. \textbf{First}:
\begin{equation}
 \max(\inf_{\Omega_\A}(\rho_{w'}))\leq \max(\inf_{\Omega_{\A^s}}(\rho_{w}))-2
\end{equation}
\(\max(\inf_{\Omega_\A}(\rho_{w'}))\leq \max(\inf_{\Omega_{\A^s}}(\rho_{w}))-2\):\\
Let $q\in\inf(\rho_{w'})$ be a state with $\Omega_\A(q)=\max(\inf_{\Omega_\A}(\rho_{w'}))$. That means that we have infinitely many $i's$ such that $\rho_{w'}(i)=q$. Now take $i>\sum_{m=0}^{j} f'(m) > i_d$. Where $i_d$ is the decisive moment (Theorem ... nog uitwerken) and $j$ an arbitrary number. We need to be sure that this $i$ is not part of a partial run.  Now let $i_b$ the smallest integer such that $i\leq \sum_{m=0}^{i_b}f'(m)$ and distinguish cases. We use a case distinction on $f$ which might seem counterintuitive since we could also do cases on $f'$ but this gives a nice direct correspondence to the way we defined $\rho_{w'}$.
\begin{description}
 \item[Case $f(i_b)=1$] Then we know that this $q$ lies on the $\rho_p$ and $\Omega_\A(q)=\max\{\Omega_\A(\rho_p)\}=p$. So that means that $\rho_{w}\left(\sum_{m=0}^{i_b}f(m)\right)=(q',p)$ for a $q\in Q_\A$ but most importantly $\Omega_{\A^s}(\rho_{w}\left(\sum_{m=0}^{i_b}f(m)\right))-2=p=\Omega_\A(q)$.
 \item[Case $f(i_b)=n>1$] Now we see that $q=\rho_w\left(\sum_{m=0}^{i_b}f(m)\right)$. Since we see that $n>1$ we have that $\rho_{w}$ passed through a letter state so that means that $\rho_w(\sum_{m=0}^{i_b}f(m))$ is of the form $q, \Omega_\A(q)$ so $\Omega_{\A^s}(\rho_w(\sum_{m=0}^{i_b}f(m)))-2=\Omega_\A(q)$.
 \textcolor{red}{DIT OVERAL TOEVOEGEN WAAR NODIG!!!!}
\end{description}
This gives us for any $i$ such that $\rho_{w'}(i)=q$ a unique $j$ such that $\Omega_{\A^s}(\rho_{w}(j))-2=\Omega_\A(q)$ since there are finitely many states we know that there is a state in $\inf(\rho_{w})$ with this parity. So we see $\max(\inf_{\Omega_\A}(\rho_{w'}))\leq \max(\inf_{\Omega_{\A^s}}(\rho_{w}))-2$.\\
Now to prove $\max(\inf_{\Omega_\A}(\rho_{w'}))\geq \max(\inf_{\Omega_{\A^s}}(\rho_{w}))-2$\\ Let $(q,p)\in\inf(\rho_{w})$ be a state with $\Omega_{\A^s}((q,p))=\max(\inf_{\Omega_{\A^s}}(\rho_{w}))$.
Take $i> i_d$ and define $i_b$ as the smallest integer such that $i\leq \sum_{m=0}^{i_b}f(m)$. We distinguish cases
\begin{description}
 \item[Case $f(i_b)=1$] That means that $\rho_{w}(i)=(q,p)$ with $\Omega_{\A^s}(\rho_w(i))=p+2$ and $p=\max\{\Omega_\A(\rho_p(j))\mid j\in \{1,\dots,n\}\}$ for a $\rho_p$. So therefore there is a $j$ such that $\Omega_\A(\rho_p(j))=\Omega_{\A^s}(\rho_w(i))-2$. Now we have defined $\rho_{w'}\left( \sum_{m=0}^{i_b-1}+j) = \rho_p(j) \right)$ so we have a state in $\rho_{w'}$ with parity $p$.
 \item[Case $f(i_b)=n$] Since we know that all states $\rho_w\left(\sum_{m=0}^{i_b-1}f(m) +l\right)\in Q_\A\times \Sigma$ for $0<l<n$ we see that $i=\sum_{m=0}^{i_b}f(m)$ so that means that
 \[
  \rho_{w}(i) = \rho_w\left(\sum_{m=0}^{i_b}f(m)\right) = (q,\Omega(q)) = \left(\rho_{w'}\left(\sum_{m=0}^{i_b}f'(m)\right), \Omega_\A\left(\rho_{w'}\left(\sum_{m=0}^{i_b}f'(m)\right)\right)\right)
 \] and $\Omega_{\A^s}(\rho_{w}(i))-2=\Omega_\A(q)$.
\end{description}
Now we have infinitely many $j's$ such that $\rho_{w'}(j)=\max(\inf_{\Omega_{\A^s}}(\rho_{w}))-2$ and since there are finitely many states we have that there is a state in the inf set with this parity.\\

Combining these two inequalities we get \(\max(\inf_{\Omega_\A}(\rho_{w'}))=\max(\inf_{\Omega_{\A^s}}(\rho_{w}))-2\) and since $\max(\inf_{\Omega_{\A^s}}(\rho_{w}))$ is even we have that  \(
  \max(\inf_{\Omega_\A}(\rho_{w'}))\) is even.
\end{proof}
Combining claims \ref{claim:subsetrhow'run} and \ref{claim:subsetrhow'pareven} we know that $w'\in \LL(\A)$. From Claim \ref{claim:subsetwwb[f]} we get $w=w_b[f]$ and since we defined $w'=:w_b[f']$ we know that $w\sim_s w'$. Combining these two means that $w\in (\LL(\A))^s$ which proves the inclusion.\\\\
Secondly we will prove the other direction \(\mathcal{L}(\mathbb{A}^s) \supseteq (\mathcal{L}(\mathbb{A}))^s
\) Suppose we have $w\in (\mathcal{L}(\mathbb{A}))^s$ then there exists a $w'\in\LL(\A)$ such that $w\sim_s w'$. Recall definition about language of $\A$... Following Definition \ref{def:stutequiv} we know there exists $w_b\in \Sigma$ and $f, f':\omega\to\omega^+$ such that $w = w_b[f]$ and $w'=w_b[f']$

To prove that $w\in \LL(\A^s)$ we will provide an accepting run in $\A^s$ for $w$. I will first prove that $w_b\in \LL(\A^s)$ and then define an accepting run for $w$. To show that $w_b\in\LL(\A^s)$ I will give an run and show that this is accepting. Since $w'\in\LL(\A)$ we know that there exists an accepting run $\rho_{w'}$ in $\A$ we will use this run as the basis.
\begin{definitiont}\label{def:supsetrhowb}
 To define $\rho_{w_b}:\omega\to Q_{\A^s}$ take $i\in\omega$.\\
 Set $\rho_{w_b}(0) = (\rho_{w'}(0), \Omega(\rho_{w'}(0)))$.\\
 For $i>0$: Let $j=\sum_{m=0}^{i-2}f'(m)$, $n= f'(i)$ and $p=\max\{\Omega_\A(\rho_{w'}(k))\mid k\in \{j+1,\dots, j+n\}\}$. Now set $\rho_{w_b}(i)=(\rho_{w'}(j+n), p)$.
\end{definitiont}
Now we need to check that this defined run is indeed a welldefined run.
\textcolor{red}{Misschien dit bewijs in een appendix doen, is niet zo heel interessant namelijk}
\begin{claim}\label{claim:supsetrhowbrun}
Let $\rho_{w_b}$ from Definition \ref{def:supsetrhowb}, then we have $\rho_{w_b}(0)\in q_{I,\A^s}$ and $\rho_{w_b}(i+1)\in \delta_{\A^s}(\rho_{w_b}(i), w_b(i))$ for all  $i$.
\end{claim}
\begin{proof}
First to prove the first part of the claim. We see that $\rho_{w_b}(0)=(\rho_{w'}(0),\Omega_\A(\rho_{w'}(0)))$ and since $\rho_{w'}$ is a run we know that $\rho_{w'}(0)\in q_{I,\A}$ so that means that $\rho_{w_b}(0)\in\{(q, \Omega_\A(q))\mid q\in q_{I,\A}\}=q_{I,\A^s}$.
Now to prove the second part ($\rho_{w_b}(i+1)\in \delta_{\A^s}(\rho_{w_b}(i), w_b(i))$ for all $i\in\omega$) of the claim. Let $j=\sum_{m=0}^{i-1}f'(m)$ and $n=f'(i)$
 \begin{align*}
  \rho_{w_b}(i+1)&=\left(\rho_{w'}\left(j+f'(i), \max\{\Omega_\A(\rho_{w'}(k))\mid k\in \{j+1,\dots, j+n\}\}\right)\right)\text{ Definition \ref{def:supsetrhowb}}\\
  &\in \{(q_n, p) \in Q\times \Omega(Q)\mid \exists n, \exists \rho_p= qq_1\dots q_n \land q\twoheadrightarrow_{\rho_p}^{w_b(i)}q_n\land\\
  & \phantom{\in\{}p = \max(\Omega(\{q_i\mid i\in \{1, \dots, n\}\}))\land q = \rho_{w'}(j)\}
  \intertext{To understand this inclusion we set $n=f'(i)$ and $\rho_p=\rho_{w'}(j)\dots\rho_{w'}(j+n)$ and since $w_b(i)=w'(j+k)$ for $k<f'(i)$ and $\rho_{w'}$ is a run on $w$ we see that $\rho_p$ is a partial run.}
  &\subseteq \delta_{\A^s}({\rho_{w_b}(i), w_b(i))})\text{ Since $\rho_{w_b}(i)=(\rho_{w'}(j), p)$ for a $p\in\Omega_\A(Q_\A)$}
 \end{align*}
 \end{proof}
 In order to show that $w_b\in\LL(\A^s)$ we need to check if this run is accepting, i.e. if the maximum parity that occurs infinitely is even.
\begin{claim}\label{claim:supsetrhowbpareven}
 \[
  \max\left(\inf_{\Omega_{\A^s}}(\rho_{w_b})\right) \text{ is even}
 \]
\end{claim}
\begin{proof}
I will first prove $\max\left(\inf_{\Omega_{\A^s}}(\rho_{w_b})\right)-2 = \max\left(\inf_{\Omega_\A}(\rho_{w'})\right)$ via two inequalities.\\
First to prove $\max\left(\inf_{\Omega_{\A^s}}(\rho_{w_b})\right)-2 \leq \max\left(\inf_{\Omega_\A}(\rho_{w'})\right)$\\
Let $q\in\inf(\rho_{w_b})$ with $\Omega_{\A^s}(q) = \max\left(\inf_{\Omega_{\A^s}}(\rho_{w_b})\right)$. Now take $i>i_d$ such that $\rho_{w_b}(i)=q$ and let $j=\sum_{m=0}^{i-1}$. Now we see that $\Omega_{\A^s}(q)=\max\{\rho_{w'}(k)\mid k\in\{j+1,\dots,j+f'(i)\}\}+2$ so there exists a $k$ such that $\Omega_\A(\rho_{w'}(k))=\Omega_{\A^s}(q)-2$. Since there are infinitely many $i$'s such that $\rho_{w_b}(i)=q$ and every $i$ gives a unique $k$ there are infinitely many $k$'s such that $\Omega_\A(\rho_{w'}(k))=\Omega_{\A^s}(q)-2$ and since we have finitely many  states in $Q$ we now know that there is a state $q'\in\inf_{\rho_{w'}}$ with $\Omega_{\A}(q')=\max\left(\inf_{\Omega_{\A^s}}(\rho_{w_b})\right)-2$ which proves that $\max\left(\inf_{\Omega_{\A^s}}(\rho_{w_b})\right)-2 \leq \max\left(\inf_{\Omega_\A}(\rho_{w'})\right)$.\\
Now to prove $\max\left(\inf_{\Omega_{\A^s}}(\rho_{w_b})\right)-2 \geq \max\left(\inf_{\Omega_\A}(\rho_{w'})\right)$\\
Let $q\in\inf(\rho_{w'})$ with $\Omega_\A(q)=\max\left(\inf_{\Omega_\A}(\rho_{w'})\right)$. Take $i>\sum_{m=0}^jf'(m)\geq i_d$. Define $i_b$ as the smallest integer such that $i\leq \sum_{m=0}^{i_b}f'(m)$. We see that
\[\rho_{w_b}(i_b+1)=\left(\rho_{w'}\left(\sum_{m=0}^{i_b}f'(m)\right), \max\left\{\Omega_\A\left(\rho_{w'}(k)\right)\mid k\in \left\{\sum_{m=0}^{i_b-1}f'(m)+1, \dots, \sum_{m=0}^{i_b}f'(m)\right\}\right\}\right)                                                                                                                                                                                                                                                                                                      \]
                                                                                                                                                              where
                                                                                                                                                              \[\max\left\{\Omega_\A(\rho_{w'}(k))\mid k\in \left\{\sum_{m=0}^{i_b-1}f'(m)+1, \dots, \sum_{m=0}^{i_b}f'(m)\right\}\right\}=\Omega_\A(q)\]
                                                                                                                                                              since $\sum_{m=0}^{i_b-1}f'(m)\geq i_d$. That means that $\Omega_{\A^s}(\rho_{w_b}(i_b+1))-2 = \Omega_\A(q)$ so that gives infinitely many states in $\rho_{w_b}$ with the desired parity. Since there is a finite amount of states we know at least one should be in the inf set which proves the inequality.

Combining the two inequalities we have that $\max\left(\inf_{\Omega_{\A^s}}(\rho_{w_b})\right)-2 = \max\left(\inf_{\Omega_\A}(\rho_{w'})\right)$ and since $\rho_{w'}$ is a run we have that $\max\left(\inf_{\Omega_\A}(\rho_{w'})\right)$ is even so $\max\left(\inf_{\Omega_{\A^s}}(\rho_{w_b})\right)$ as well.
\end{proof}
Following Claim \ref{claim:supsetrhowbrun} and Claim \ref{claim:supsetrhowbpareven} we know that $\rho_{w_b}$ is an accepting run so $w_b\in \LL(\A^s)$. Now we want to prove that $w\in\LL(\A^s)$ to do that we give an accepting run on $w$.
\begin{definitiont}\label{def:supsetrhow}
Define $\rho_w:\omega\to Q_{\A^s}$ as following. Set $\rho_{w}(0)=\rho_{w_b}(0)$. For $i>0$ let $i_b$ as the smallest integer such that $i\leq\sum_{m=0}^{i_b} f(m)$. This $i_b$ is the index in the base word that corresponds to the index $i$ in $w'$. Now we distinguish cases.
\begin{description}
 \item[Case $f(i_b)=1$] Set $\rho_{w}(i) = \rho_{w_b}(i_b+1)$
 \item [Case $f(i_b)=n>1$] Let $(q_n, p_n):=\rho_{w_b}(i_b+1)$ and $j= i - \sum_{m=0}^{i_b-1} f(m)$ Now we again have to distinguish cases:
 \begin{description}
  \item[Case $(q_n, w(i))\in \delta((\rho_{w}(i-1), w(i-1))$] Now set
  \[
   \rho_w(i) = \begin{cases}
                   (q_n, w(i)) &\text{ if } i < \sum_{m=0}^{i_b} f(m)\\
                   (q_n,\Omega_\A(q_n)) &\text{ if } i = \sum_{m=0}^{i_b} f(m)
                  \end{cases}
  \]
  \item[Case $j < f(i_b)$] then set $\rho_w(i) = \left(\rho_{w'}\left(\sum_{m=0}^{i_b-1}f'(m)+j\right),\Omega_\A\left(\rho_{w'}\left(\sum_{m=0}^{i_b-1}f'(m)+j\right)\right)\right)$. This is the run in $w'$.
  \item[Case $j=f(i_b)$] Now since we know that $(q_n, w(i))\notin \delta((\rho_{w}(i-1), w(i))$ we see that $j=f(i_b) < f'(i_b)$. So that means we need to take a shortcut transition in the following form: Let $\text{start} = j + \sum_{m=0}^{i_b-1}f'(m)$ and $\text{end} =  \sum_{m=0}^{i_b}f'(m)$
  \[
   \rho_{w}(i) = (\rho_{w'}(\text{end}), \max\left\{\Omega_\A(\rho_{w'}(k))\mid k\in \{\text{start},\dots,\text{end}\}\right\})
  \]
 \end{description}
\end{description}
\end{definitiont}

\begin{plain}[\textbf{Observation}]
We see that $\rho_w\left(\sum_{m=0}^{i_b}f(m)\right)=\rho_{w_b}(i_b+1)$
\end{plain}

\begin{claim}\label{claim:supsetrhowrun}
Let $\rho_w$ from Definition \ref{def:supsetrhow} then we have $\rho_{w}(0)\in q_{I,\A^s}$ and $\rho_{w}(i+1)\in \delta_{\A^s}(\rho_w(i), w(i))$ for all $i$.
\end{claim}
\begin{proof}
To prove the first part of the claim we see that
\(
 \rho_w(0)=\rho_{w_b}(0)\in q_{I,\A^s}\text{ since } \rho_{w_b} \text{ is a run.}
\)\\
To prove the second part of the claim take $i\in\omega$ and we look at $\rho_{w}(i+1)$. Let $i_b$ smallest integer such that $i+1\leq \sum_{m=0}^{i_b}f(m)$. Then distinguish cases:
\begin{description}
 \item[Case $f(i_b)=1$] So $\sum_{m=0}^{i_b}f(m)=\sum_{m=0}^{i_b-1}f(m)+1$ That means that $i=\sum_{m=0}^{i_b-1}f(m)$ so $\rho_{w}(i)=\rho_{w}\left(\sum_{m=0}^{i_b-1}f(m)\right) = (\rho_{w_b}(i_b))$. That easily gives
 \begin{align*}
  \rho_{w}(i+1) &= \rho_{w_b}(i_b+1)\text{ From Definition \ref{def:supsetrhow}.}\\
                &\in\delta_{\A^s}((\rho_{w_b}(i_b), w_b(i_b)))\text{ Since $\rho_{w_b}$ is a run.}\\
                &=\delta_{\A^s}\left(\left(\rho_{w}\left(\sum_{m=0}^{i_b-1}f(m)\right), w\left(\sum_{m=0}^{i_b-1}f(m)\right)\right)\right)\text{ With Lemma \ref{lemma:stutbaseword} and the observation.}\\
                &=\delta_{\A^s}(\rho_{w}(i), w(i))
 \end{align*}
 \item [Case $f(i_b)=n>1$] Let $(q_n, p_n):=\rho_{w_b}(i_b+1)$ and $j= i + 1 - \sum_{m=0}^{i_b-1} f(m)$ Now we again have to distinguish cases:
 \begin{description}
  \item[Case $(q_n, w(i))\in \delta((\rho_{w}(i-1), w(i-1))$] We immediately see $\rho_w(i+1)\in\delta(\rho_w(i), w(i)$.
  \item[Case $j<f(i_b)$] That means that $\rho_{w}(i) = \rho_{w}\left(\sum_{m=0}^{i_b-1}f(m)+j-1\right)$ now we see:
  \begin{align*}
   \rho_w(i+1) &= \left(\rho_{w'}\left(\sum_{m=0}^{i_b-1}f'(m)+j\right),\Omega_\A\left(\rho_{w'}\left(\sum_{m=0}^{i_b-1}f'(m)+j\right)\right)\right)\\
   &\in \left\{(q', \Omega(q'))\mid q'\in\delta_\A\left(\rho_{w'}\left(\sum_{m=0}^{i_b-1}f'(m) + j -1\right), w'\left(\sum_{m=0}^{i_b-1}f'(m)+j-1\right)\right)\right\}\\
   &=\left\{(q', \Omega(q'))\mid q'\in\delta_\A\left(\rho_{w'}\left(\sum_{m=0}^{i_b-1}f'(m)+j-1\right), w\left(\sum_{m=0}^{i_b-1}f(m)+j-1\right)\right)\right\}
   \intertext{Since we have $w\left(\sum_{m=0}^{i_b-1}f(m)+j-1\right) = w_b(i_b) = w'\left(\sum_{m=0}^{i_b-1}f'(m)+j-1\right)$ with Lemma \ref{lemma:stutbaseword}}
  &\subseteq \delta_{\A^s}(\rho_w(i), w(i))
  \end{align*}
whic proves the claim
\item[Case $j=f(i_b)$] We took a shortcut transition which is there since \\$\rho=\rho_{w'}\left(j + \sum_{m=0}^{i_b-1}f'(m)\right) \dots \rho_{w'}\left(\sum_{m=0}^{i_b}f'(m)\right)$ is a partial run.
 \textcolor{red}{Nog even netjes met verzamelingen etc.}
 \end{description}
\end{description}
\end{proof}
\begin{claim}\label{claim:supsetrhowpareven}
  \[
  \max\left(\inf_{\Omega_{\A^s}}(\rho_{w})\right) \text{ is even}
 \]
\end{claim}
\begin{proof}
We will prove that $\max\left(\inf_{\Omega_{\A^s}}(\rho_{w})\right)=\max\left(\inf_{\Omega_{\A^s}}(\rho_{w_b})\right)$ with two inequalities:\\
Firstly: $\max\left(\inf_{\Omega_{\A^s}}(\rho_{w})\right)\leq\max\left(\inf_{\Omega_{\A^s}}(\rho_{w_b})\right)$\\
Assert: it cannot be a loop state since the successor of the loop state always has bigger priority.

Now take $q\in\inf_(\rho_w)$ such that $\Omega_{\A^s}(q) = \max\left(\inf_{\Omega_{\A^s}}(\rho_{w})\right)$ that means that we have infinitely many $i$ such that $\rho_w(i)=q$. Take $i>\sum_{m=0}^jf(m)>i_d$. Then define $i_b$ as the smallest integer such that $i\leq \sum_{m=0}^{i_b}f(m)$. We distinguish cases
\begin{description}
 \item[Case $f(i_b)=1$] That gives $\rho_{w}(i)=\rho_{w_b}(i+1)$
 \item[Case $f(i_b)=n>1$] Firstly we know that $\rho_{w}(i)\notin Q_\A\times\Sigma$ since the successors of all these states have bigger parity. Now let $j=i - \sum_{m=0}^{i_b}f(m)$, if $j<f(i_b)$ then we see that $\Omega_{\A^s}$ is bigger or equal to all other states on this run after $i_d$. So in particular to
 \[
  \left\{\Omega_{\A^s}(\rho_{w}(k))\mid k\in \left\{\sum_{m=0}^{i_b-1}f(m)+1,\dots, \sum_{m=0}^{i_b}f(m)\right\}\right\}
 \]
 and we see that
 \[
  \max\left\{\Omega_{\A^s}(\rho_{w}(k))\mid k\in \left\{\sum_{m=0}^{i_b-1}f(m)+1,\dots, \sum_{m=0}^{i_b}f(m)\right\}\right\} = \Omega_{\A^s}(\rho_{w_b}(i_b))
 \]
\end{description}
Secondly $\max\left(\inf_{\Omega_{\A^s}}(\rho_{w})\right)\geq\max\left(\inf_{\Omega_{\A^s}}(\rho_{w_b})\right)$\\
Take $q\in\inf(\rho_{w_b})$ such that $\Omega_{\A^s}(q)=\left(\inf_{\Omega_{\A^s}}(\rho_{w_b})\right)$. We know there are infinitely many $i$ such that $\rho_{w_b}(i)=q$. Take such an $i$ and distinguish cases:
\begin{description}
 \item[Case $f(i)=1$] Then we see that $\rho\left(\sum_{m=0}^{i}f(m)\right)=q$.
 \item[Case $f(i)=n>1$] That means that $\Omega_{\A^s}(q)$ is the maximum of paritys on a partial run. We know that
 \[
  \max\left\{\Omega_{\A^s}(\rho_{w}(k))\mid \left\{\sum_{m=0}^{i-1}f(m)+1,\dots, \sum_{m=0}^{i-1}f(m)\right\}\right\} = \Omega_{\A^s}(\rho_{w_b}(i))
 \]
 so there is a $k$ such that $\rho_w(k)$ has the desired parity.
\end{description}
That gives us infinitely many $j$ such that $\Omega_{\A^s}(\rho_{w}(j))= \Omega_{\A^s}(q)$ and since there are finitely many states we know that there is a state in $\inf_{\rho_w}$ with the desired parity which proves the inequality.\\
Combining the two inequalities we have that $\max\left(\inf_{\Omega_{\A^s}}(\rho_{w})\right)=\max\left(\inf_{\Omega_{\A^s}}(\rho_{w_b})\right)$ and by applying Claim \ref{claim:supsetrhowbpareven} we know that
\[
 \max\left(\inf_{\Omega_{\A^s}}(\rho_{w})\right) \text{ is even},
\]
as claimed.
\end{proof}
Combining Claim \ref{claim:supsetrhowrun} and Claim \ref{claim:supsetrhowpareven} gives us that  $w\in\LL(\A^s)$ which proves that \(\mathcal{L}(\mathbb{A}^s) \supseteq (\mathcal{L}(\mathbb{A}))^s
\).\\
Now combining the two inclusions we have
\[
 \LL(\A^s)=(\LL(\A))^s,
\]
as was our proof goal.
\end{proof}

Now we have given a definition to determine the stutter-closure of the language of an automaton. But how do determine for a given automaton if it's language is stutter-invariant. We want to check if $\LL(\A)=\LL(\A^s)$. Determining wheter two (infinite) languages are the same is dificult but we can use a standard emptines check (source) to determine if $\LL(\A^s)\cap \overline{\LL(\A)}=\emptyset$. That gives us a decidable procedure to determine if the language of an automaton is stutter-invariant. Hier wat toevoegen over complexiteit...

Now some lemmas
\begin{proposition}
 \[
 \LL((\A\cup \A')^s)
 \]
\end{proposition}

Hier toevoegen dat je de loop states weg kan halen
\begin{theorem}
 We can remove the states of the form
 \((q, l)\) if $q\in \delta_\A(q, l)$.
\end{theorem}

