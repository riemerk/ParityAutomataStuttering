Definitions about stuttering are adapted from \cite{etessami1999stutter}.
\begin{definition}\label{def:stutequiv}
Two $\omega-$words $w$ and $w'$ over the same alphabet $\Sigma$ are called \textbf{stutter-equivalent} if there exists $f, f': \mathbb{N} \to \mathbb{N}^+$ and a sequence of letters $a_i\in\Sigma$:  $a_0a_1a_2\dots$ such that $w = a_0^{f(0)}a_1^{f(1)}a_2^{f(2)}\dots$ and $w' = a_0^{f'(0)}a_1^{f'(1)}a_2^{f'(2)}\dots$. We denote this as $w\sim_s w'$.
\end{definition}
Nog iets toevoegen $w[f]$ als notatie (makkelijker).

\begin{lemma}\label{lemma:stutbaseword}
Let $w, w_b : \omega\to\Sigma$ and $f:\omega\to\omega^+$ and $w=w_b[f]$ then we have
\[
 w_b(i) = w((\sum_{m=0}^{i-1}f(m)+j)\text{ for } j < f(i) \text{ for all } i\in\omega
\]
\end{lemma}
\begin{definition}\label{def:stutinvariant}
We call a language \textbf{stutter-invariant} if it holds that if $w\sim_sw'$ then $w\in L\Leftrightarrow w'\in L$
\end{definition}
\begin{definition}\label{def:stutclosure}
 For a language we define the \textbf{stutter-closure} as the following set
 \[
 L^s = \{w\in \Sigma^\omega\mid w\sim_s w' \text{ for a } w'\in L\}
 \]
\end{definition}



