\begin{definition}\cite{demri2016temporal}
Let $\mathbb{G}$ a parity formula over $\Psf$. Define the \textbf{language} of a parity formula as:
 \[
  \Mod_P(\mathbb{G}) := \{\sigma \in (\PP(P))^\omega\mid \sigma, 0 \Vdash \mathbb{G}\}
 \]
\end{definition}

\textcolor{red}{Ik weet nog niet zo goed hoe ik deze strategy tree precies ga gebruiken, kan ook kijken naar de gewone strategy tree zoals jij hem definieert maar dan met de opmerking dat je alleen maar naar de laatste toestand hoeft te kijken. Alleen op deze manier kan je hem wel heel makkelijk een op een vertalen naar een DAG (run)}
\begin{proposition}
 Knowledge about strategy tree generalize to knowledge about positional strategy tree. In particular the fact that a branch in the tree is a match in the game.
\end{proposition}
\begin{theorem}\label{thm:parformtoapa}
 There is an algorithm that constructs, for a strongly guarded and simple parity formula $G$ over $\Psf$ into an equivalent APA $\A$ such that $\Mod_\Psf(\mathbb{G}) = \LL(\A)$
\end{theorem}
\begin{proof}
 Let $\mathbb{G}=(V, E, L, \Omega_\mathbb{G}, v_I)$ a strongly guarded and simple parity formula. We will define $\A=(\Sigma, Q, q_0, \delta, \Omega')$ as follows:
 \begin{itemize}
  \item Set the alphabet $\Sigma := \PP(\Psf)$
  \item Define the the set of states $Q$ as $Q:= V$
  \item Define $q_0:=v_I$
 \end{itemize}
 To define the transition function $\delta$ let $q\in Q$ and $a\in\Sigma (\PP(\Psf))$ and set:
\begin{align*}
   \delta(v, a) &:= \top        \text{ if } L(v)=p\text{ and } p\in a & \delta(v,a)&:= \delta(v_1,a)\land\delta(v_2, a)\text{ if } L(v) = \land\\
   \delta(v, a) &:= \bot        \text{ if } L(v)=p\text{ and } p\notin a&\delta(v,a)&:= \delta(v_1,a)\lor\delta(v_2, a)\text{ if } L(v) = \lor\\
   \delta(v, a) &:= \top \text{ if } L(v)=\overline{p}\text{ and } p\notin a&\delta(v, a)&:= v'\text{ if } L(v) = \deo\\
   \delta(v, a) &:= \bot \text{ if } L(v)=\overline{p}\text{ and } p\in a &\delta(v, a) &:= \top        \text{ if } L(v)=\top\\
   \delta(v, a) &:= \delta(v', a)\text{ if } L(v)=\epsilon&\delta(v, a) &:= \bot        \text{ if } L(v)=\bot
  \end{align*}
  For the priority map define:
  \[
  \Omega_\mathbb{A}(v) = \begin{cases}
                \Omega(v) &\text{if }v\in \Dom(\Omega)\\
                0&\text{else}
               \end{cases}
  \]
  \textbf{Observation:} if $(v, l)$ is a dead end for $\forall$ then $\delta(v, a_l)=\top$\\
\textbf{Claim:} $\Mod_\Psf(\mathbb{G}) = \LL(\A)$\\
\textit{Proof}: First $\Mod_\Psf(\mathbb{G}) \subseteq \LL(\A)$. Suppose $w=a_0a_1a_2\dots \in \Mod_\Psf(\mathbb{G})$ then we know that $(v_I, 0)$ is a winning position for $\exists$. This means there is a winning strategy $f$ for $\exists$ and since $\mathbb{G}$ is a parity game we even know that there exists a positional winning strategy $f$ for $\exists$ in $\mathcal{B}@(v_I,0)$. Now take the strategy graph $\mathbb{B}_a^f$, we are going to construct a succesfull run in $\A$ based on this positional strategy tree. Recall that a run in $\A$ is a DAG $\rho=(V_\rho,E_\rho)$ where $\rho$ satisfies four conditions R1-R4. Define the nodes
\[
 V_\rho = \{(v_I, 0)\}\cup\bigcup_{i\geq 1}\left\{(v, i)\mid \exists v'\in V, L(v')=\deo, \{v\}=E[v'], (v', i-1)\in \mathbb{B}_{(v_I,0)}^f \right\}
\]
and the edges
\[
 E_\rho = \bigcup_{i\geq 0}\left\{((v, i), (v', i+1))\in V_\rho\times V_\rho\mid (v', i+1)\in\mathbb{B}_{(v,i)}^f \right\}
\]
Now we have to check that this is indeed a run. Satisfaction of R1-R3 is an easy verification since $\mathbb{G}_{(v_I,0)}^f$ is already a graph. \textcolor{red}{Werk dit even netjes uit}. For the satisfaction of R4 we need to dive in the definition of $\delta$. \\
\textcolor{red}{Naam van $E_q$ moet anders!}
\textbf{Claim:} For every $(q, l)\in V_\rho$ we have $E_q=\{q'\mid ((q, l), (q', l+1))\in E\} \models \delta(q, a_l)$\\
\textit{Proof}: We are going to look at the level $\mathbb{B}_{(q, l)}^f[l]$ of the game graph. Since $\mathbb{G}$ is (strongly) guarded we know that this level is a DAG where the leafs (hebben we leafs?) correspond to literal or modal nodes. Now I am going to proof this with an iteration on the topological ordering of $\mathbb{B}_{(q, l)}^f[l]$, call this ordering $w_0w_1\dots w_n$. \textcolor{red}{Hoe nu inductiehypothese benoemen??}.\\
\textbf{Base case: $w_n$}. We know that $w_n$ has no outgoing edges, so it must correspond to a dead end. Since $f$ is a winning strategy for $\exists$ we know that $\forall$ should be at move. Therefore we now that
$\delta(w_n,a_l)=\top$ so clearly $E_q\models \delta(w_n, a_l)$. \\
\textbf{Induction step:} Take $w_i, i<n$. Here we distinguish four different cases:
\begin{itemize}
 \item[Case:] \(E[w_i]=0\) In this case $w_i$ is a dead end for $\forall$ in the game graph so we know that $\delta(w_i, a_l)=\top$.
 \item[Case:] $L(w_i)=\land$. In this case we have $\delta(w_i, a_l) = \delta(w_j, a_l) \land \delta(w_k, a_l)$, since we have a topological ordering we know that $i<j$ and $i<k$ so per the induction (iteration) hypothesis we know that $E_q\models \delta(w_j, a_l)$ and $E_q\models\delta(w_k, a_l)$ so that means $E_q\models \delta(w_i, a_l)$.
 \item[Case:] $L(w_i)=\lor$. We see that $\delta(w_i, a_l)= \delta(v_1, a_l)\lor \delta(v_2, a_l)$ Since $\mathbb{B}_{(q, l)}^f[l]$ is a level of the strategy graph we know that $E_f[w_i] = f(w_i)=v_m, m\in {1, 2}$. Since we have a topological ordering we know that $v_m=w_j$ for a $j>i$. Therefore we know that $E_q\models \delta(v_j, a_l)$ so also $E_q\models \delta(w_i, a_l)$ \textcolor{red}{Oei de indices zijn hier wel heel erg verwarrend}.
 \item[Case:] $L(w_i)=\deo$. Here we see that $\delta(w_i, a_l) = v'$ where $E[w_i]=\{v'\}$. Since $(w_i, l)\in \mathbb{B}_{(v_I, 0)}^f$ we know that $(v', l+1)\in V_\rho$. Also since $(w_i, l)\in \mathbb{B}_{(q, l)}^f$ we see that $(v', l+1)\in \mathbb{B}_{(q, l)}^f$ so therefore $v'\in E_q$ so we see that $E_q\models\delta(w_i, a_l)$.
\end{itemize}
Now we have iterated over the the whole level $\mathbb{B}_{(q, l)}^f[l]$ so we see that $E_q \models \delta(q, a_l)$.
\qed.\\

% I will prove this via induction on the complexity on $\delta(q,a_l)$. Since $\mathbb{G}$ is guarded there are no loops so we cannot loop infinitely over $\delta$ and get a fixpoint. \textcolor{red}{IH goed formuleren, iets met $\delta$ en $E[v]$, en hier heb je dus $v\leq q$ eigenlijk in de parityformuleboom}: Induction hypo: for every $v\in \mathbb{T}^f_{(q, l)}$ we have $E_v\models (v, a_l)$ The base cases are: $L(v)=\deo$ or $L(v)\in\mathtt{At}(\Psf)$. Now if $L(v
% )=\deo$ we see that $\delta(v, a_l)= v'$. Also we see that $(v', l+1)\in V$ since $L(v)=\deo$, $(v, l)\in \mathbb{T}_{(v_I, 0)}$ and $\{v'\} =E[v]$. Secondly $((q,l), (v', l+1))\in E$ since $(v, l)\leq (q,l)$ and $E[v]=v'$ so we see that $(v', l+1)\in \mathbb{T}_{(q, l)}^f$ since $v\in \mathbb{T}^f_{(q, l)}$. So $\{q'\mid ((q, l), (q', l+1))\in E\} \models \delta(v, a_l)$. If $L(v)\in \mathtt{At}(\Psf)$ we know that this should be a dead end where it is $\forall$'s turn since $f$ is a winning strategy. So $\delta(v, a_l)=\top$ so every set satisfies this transition function. \\
% Now for induction assume $\delta(v, a_l) = \delta(v_1, a_l)\land \delta(v_2,a_l)$, we see that both $v_1$ and $v_2$ are in $\mathbb{T}^f_{(q, l)}$ so we can apply the induction hypothesis. If we have $\delta(v, a_l) = \delta(v_1, a_l)\lor \delta(v_2,a_l)$ choose the node that is in $\mathbb{T}^f_{(q, l)}$ and apply the induction hypothesis. If we have $\delta(v, a_l)=\delta(v', a_l)$ we have $v'\in \mathbb{T}^f_{(q, l)}$ so we can apply the induction hypothesis. Now we see that $E_q\models \delta(q, a_l)$.\qed

Now we have to proof that $\rho=(V,E)$ is indeed an accepting run. That means that for every infinite path $w$ through $\rho$ we have that $\max\{\Omega(q)\mid q\in \inf(u)\}$ is even. To to that we will prove a correspondence between infinite paths in $\rho$ and infinite matches in $\mathcal{E}(\mathbb{G}, \sigma)@(v_I, 0)$\\
\textbf{Claim:} Every infinite path $u$ in $\rho$ corresponds \textcolor{red}{Wat bedoel je met correspondence precies} to a $f$-guided infinite match $\pi$ in $\mathcal{E}(\mathbb{G}, \sigma)@(v_I, 0)$ where we have $\{\Omega_\mathbb{A}(v)\mid v\in u\}\cup\{0\}= \{\Omega_{\mathcal{E}}(p)\mid p\in \pi\}$, \textcolor{red}{En benoem eerste state in $\pi$}. \\
\textit{Proof:}\\
An infinite path $u$ consists of states $q_0q_1q_2\dots$ where $q_0 = v_I$ and we have $q_{i+1}\in E_{q_i}$ for every $i$. Now construct the infinite match $\pi = (q_0,0)\dots (q_1,1)\dots (q_2, 2)\dots (q_3,3)\dots$, we see that this is an infinite match since when $q_{i+1}\in E_{q_i}$ we know that $q_{i+1}\in\mathbb{B}_{(q,i)}^f$ so there is a finite $f$-guided path between $(q_i,i)$ and $(q_{i+1},i+1)$. Now we have to show that $\{\Omega_\mathbb{A}(v)\mid v\in u\}\cup\{0\}= \{\Omega_{\mathcal{E}}(p)\mid p\in \pi\}$. First observe that $\Omega_\mathcal{E}((q_i,i)) = \Omega_\A (q_i)$ since they both correspond to \(
  \Omega_\mathbb{A}(v) = \begin{cases}
                \Omega(v) &\text{if }v\in \Dom(\Omega)\\
                0&\text{else}
               \end{cases}
  \).
  Now take a node $(\pi_j, i)$ in $\pi$ inbetween $(q_i, i)$ and $(q_{i+1}, i+1)$. My claim is that $\Omega_{\mathcal{E}}(\pi_j, i) = 0$. Suppose not, then we would have $v\in \Dom (\Omega_\mathbb{G})$. But since $\mathbb{G}$ is strongly guarded and simple we know that $v$ should be in the scope of a modal operator and that the modal operator is directly preceding this node. That would mean that $L(\pi_{j-1})=\deo$ and that would mean that $\pi_j$ should be in level $i+1$ which contradicts the fact that $(\pi_j, i)$ is chosen inbetween $(q_i, i)$ and $(q_{i+1}, i+1)$. That proves that $\{\Omega_\mathbb{A}(v)\mid v\in u\}\cup\{0\}= \{\Omega_{\mathcal{E}}(\pi_i)\mid \pi_i\in \pi\}$ since the only non-zero nodes in $\pi$ corresponds to states in $\mathbb{G}$.
\qed \\ Since every infinite path in $\rho$ corresponds to an infinite $f$-guided match and since $(v_I, 0)$ is a winning position we know that $\max\{\Omega_\mathcal{E}(\pi_i)\mid \pi_i\in\inf(\pi)\}$ is even and since $\{\Omega_\mathbb{A}(v)\mid v\in u\}\cup\{0\}= \{\Omega_{\mathcal{E}}(p)\mid p\in \pi\}$ we also know that $\max\{\Omega(q)\mid q\in \inf(u)\}$ is even. So $\rho$ is an accepting run for $w$ in $\A$.  So now we know that if $w\in \Mod_\Psf(\mathbb{G})$ then $w\in \LL(\A)$\\
%
% That means that every infinite path in $\rho$ is accepting. We are working with the parity condition so it means that for every $w\in Q^\omega$ that defines a path through $\rho$ we have $\max\{\Omega(q)\mid q\in w\}$ is even. We see that an infinite path through the strategy tree corresponds to an infinite word through $\rho$. We also know that since $\mathbb{G}$ is guarded and simple that every state is immediately preceded by a modal node. So that means that if we have an infinite word through $\rho$ that $\Omega'(\inf(w_\rho))=\Omega'(\inf(\pi))\cup\{0\}$ where $\pi$ is the infinite match that corresponds to $w_\rho$ \textcolor{red}{Uitleggen correspondentie??}. And since $\max(\Omega'(\inf(\pi)))$ is even we know that $\max(\Omega'(\inf(w_\rho)))$ is even as well so that means that $\rho$ is accepting so that $w\in \LL(\A)$\\


Now the other direction $\Mod_\Psf(\mathbb{G}) \supseteq \LL(\A)$: Suppose we have $w\in \\(\A)$ then there exists an accepting run $\rho=(V_\rho, E_\rho)$ on $w$, without loss of generality assume this run is minimal \textcolor{red}{Hier dan verwijzen naar stelling?}. We have to create a winning strategy for $\exists$ in $\mathcal{B}@(v_I,0)$. We are going to create a non-positional strategy $f$ such every finite path in the strategy tre $\mathcal{B}$ ends in a dead end for $\forall$ and every infinite path has an even maximum parity. \\
First we are going to define $f$, to define $f$ we are inductively going to build op the strategy tree $\mathbb{T}_{(v_I, 0)}^f$. \\
For every $(q, l)\in V_\rho$ we are going to prove with induction on the depth of $v$ that for every $(v,l)\in \mathbb{T}^f_{(q,l)}[l]$ we have $E_q\models \delta(v, a_l)$, and that in ever dead end it is $\forall$'s turn to play. We define $E_q:=\{q'\mid ((q, l), (q', l+1))\in E\} \models \delta(q, a_l)$ as before. \\
Induction hypothesis: for every every $\pi$ with depth $k$ we have if $last(\pi)=(v,l)$ $E_q\models \delta(v, a_l)$.\\
Base case: $k=0$ since $\mathbb{T}^f_{(q,l)}[l]$ is a tree with one root we now that the only $\pi$ with depth $0$ has $last(\pi)=(q,l)$. Since $\rho$ is a run we know by R4 that $E_q\models \delta(q, a_l)$. If $L(q)=\lor$ we have to choose between $v_1$ and $v_2$. We know that $\delta(q, a_l)=\delta(v_1,a_l)\lor \delta(v_2, a_l)$, so $E_q\models \delta(v_1, a_l)$ or $E_q\models\delta(v_2, a_l)$ since we know that $\rho$ is minimal either we know that exclusively one is true, in that case choose $f(\pi) = v_i$ where $v_i$ is true or that $||\delta(v_1, a_l)||=||\delta(v_2, a_l)||$, in that case choose $v_1$.\\
Inductive case: $k>0$. Take a $\pi\in \mathbb{T}^f_{(q, l)}[l]$ with $last(\pi)=(v, l)$. First look at the parent of $\pi$, call this $\pi_p$ (it has depth $k-1$ so we can apply the induction hypothesis) if $last(\pi_p)=(v', l)$ we know that $E_q\models\delta(v, a_l)$. Now cases:\\
Case $L(v')=\lor$, in this case we know $\delta(v, a_l) =\delta(v_1, a_l)\lor \delta(v_2, a_l)$ (IH) so exactly one of $E_q\models\delta(v_1, a_l)$ and $E_q\models\delta(v_2, a_l)$ is true of the \textcolor{red}{valuations} are the same. We know that $f(\pi_p)=v_i$ where $\delta(v_i, a_l)$ is true so that means $v=v_i$ so $E_q\models\delta(v, a_l)$\\
Case $L(v') = \land$, in this case we know $\delta(v, a_l) =\delta(v_1, a_l)\land \delta(v_2, a_l)$ (IH) where $v=v_1\lor v=v_2$ so $E_q\models\delta(v, a_l)$\\
The case $L(v')=\deo$ is not possible since then $\pi$ would be in the next level. This is a leaf that we have to connect to another level.\\
Now if $L(v)=\lor$ we have to make a choice for $f(\pi)$, make the same choice as above in the base case.\\
If $L(v)\in Lit(\Psf)$ we know that $\delta(v, a_l)\in \{\top, \bot\}$ since $E_q\models \delta(v, a_l)$ we know that $\delta(v, a_l)=\top$. Then that means that this is a dead end where $\forall$ is at move.\\
Conclusion: For every $\pi \in \mathbb{T}_{(q, l)}^f[l]$ if $last(\pi)=(v,l$ we have $E_q\models \delta(v, l)$ and every dead end in $\mathbb{T}_{(q, l)}^f[l]$ belongs to $B_\forall$.

So now we have created a lot of levels in the strategy tree but we have to make sure that these connect to each other in a well defined strategy tree. We have to check for every leaf in every $\mathbb{T}_{(q, l)}^f[l]$ that corresponds to $L(v)=\deo$ that there is a strategy tree $\mathbb{T}_{(v', l)}^f[l+1]$.\\
\Claim: we can connect all strategy trees to form a well defined strategy tree for $\mathcal{E}(\mathbb{G}, \sigma)@(v_I, 0)$.\\
\Proof: Fist of all we know that $(q, 0)= (q_0, 0) = (v_I, 0)$ so we have $\mathbb{T}_{(v_I, 0)}^f[0]$. Now take an arbitrary $\mathbb{T}_{(q, l)}^f[l]$ and an arbitrary leaf $\pi$ with $last(\pi)=(v, l)$ in $\mathbb{T}_{(q, l)}^f[l]$. We know that $L(v)=\deo$, $E[v]=\{v'\}$ and $\delta(v, a_l)=v'$. By the above claim we know that $E_q\models \delta(v, a_l)=v'$ so $v'\in E_q$ so therefore we know that $(v', l+1)\in V_\rho$. That means that we can take $\mathbb{T}_{(v', l+1)}^f[l+1]$ from the above statement so this leaf is connected.\\
Conclusion: we can connect every leaf that is not a dead ind in every level to another level that we created so $\mathbb{T}_{(q, l)}^f$ is well-defined.\\
Now we have to make sure that this strategy tree indeed defines a winning strategy for $\exists$ in $\mathcal{E}(\mathbb{G}, \sigma)@(v_I, 0)$. This comes in two parts: first every finite match ends in a dead end for $\forall$ and for every infinite match we have that the maximum parity is even.\\
\Claim: Every finite match in $\mathbb{T}_{(v_I,0)}^f$ ends in a dead end for $\forall$\\
\Proof: We see that every finite match in $\mathbb{T}_{(v_I,0)}^f$ ends in a leaf $\pi$ in some level $\mathbb{T}_{(q,l)}^f[l]$, if $last(\pi) = (v,l)$ we know that $L(v)\in \Lit(\Psf)$ so from the induction claim we know that this is indeed a dead end where it is $\forall$'s turn.\\
\Claim: For every inifinite match $\pi$ in $\mathbb{T}_{(v_I,0)}^f$ we have $\max(\inf_{\Omega_\mathcal{E}} (\pi))$ is even.\\
\Proof: First we are going to correspond matches to paths in the run. \\
\textbf{Claim:} Every infinite $f$-guided infinite match $\pi$ in $\mathcal{E}(\mathbb{G}, \sigma)@(v_I, 0)$ corresponds to an infinte path $u$ in $\rho$ where we have $\{\Omega_\mathbb{A}(v)\mid v\in u\}\cup\{0\} = \{\Omega_{\mathcal{E}}(p)\mid p\in \pi\}$, \textcolor{red}{En benoem eerste state in $\pi$}. \\
\Proof: Take a $f$-guided match $\pi=(v_0, l_0)(v_1, l_1)(v_2, l_2)\dots$ in $\mathcal{E}(\mathbb{G}, \sigma)@(v_I, 0)$. Now mark $q_0=(v_0, l_0)$ and mark $q_i$ as the $i$'th node $(v_j, l_j$ in $\pi$ where $L(v_{j-1})=\deo$. Construct the infinite path $u=q_0q_1q_2\dots$ in $\rho$.\\
\Claim: $u$ is an infinite path in $\rho$.\\
\Proof: We have to show that for every $q_i=(v_j, l_j)$ we have $q_{i+1}=(v_{j'}, l_{j'})$ and $v_{j'}\in E_{q_i}$. We know that $(v_{j'-1}, l_{j'-1})=\last(\pi)$ for a $\pi\in \mathbb{T}_{(v_j,l_j)}^f$ and $L(v_{j'-1})=\deo$ so $\delta(v, a_l)=v_{j'}$. From the induction claim we have $E_q\models\delta(v_{j'-1})$ so $v_{j'}\in E_{q_i}$ which proofs this claim. \qed\\
Now we easily see that for $q_i=(v_j, l_j)$ we have  $\Omega_\mathcal{E}((v_j, l_j)) = \Omega_\A (v_j)$ since they both correspond to \(
  \Omega_\mathbb{A}(v_j) = \begin{cases}
                \Omega_{\mathbb{G}}(v_j) &\text{if }v\in \Dom(\Omega)\\
                0&\text{else}
               \end{cases}
\). Now take a node $(v_j, l_j)$ in $\pi$ strictlt inbetween $q_i$ and $q_{i+1}$. My claim is that $\Omega_{\mathcal{E}}(v_j, l_j) = 0$. Suppose not, then we would have $v\in \Dom (\Omega_\mathbb{G})$. But since $\mathbb{G}$ is strongly guarded and simple we know that $v$ should be in the scope of a modal operator and that the modal operator is directly preceding this node. That would mean that $L(v_{j-1})=\deo$ and that would mean that $(v_j, l_j)$ should be in level $l_j+1$ which contradicts the fact that $q_{i+1}$ is chosen as the first node after $q_i$ that follows a modal node. That proves that $\{\Omega_\mathbb{A}(v)\mid v\in u\}\cup\{0\}= \{\Omega_{\mathcal{E}}(\pi_i)\mid \pi_i\in \pi\}$ since the only non-zero nodes in $\pi$ corresponds to states in $\mathbb{G}$.\qed\\
Now we know that for every infinite $f$-guided infinite match $\pi$ there is a infinite path $u$ in$\rho$. Since $\rho$ is an accepting run we know that $\max(\inf_{\Omega_\A}(u))$ is even and since $\{\Omega_\mathbb{A}(v)\mid v\in u\}\cup\{0\} = \{\Omega_{\mathcal{E}}(p)\mid p\in \pi\}$ we then also know that $\max(\inf_{\Omega_\mathcal{E}}(\pi))$ is even, which proofs the claim.\qed\\

Now we know that $f$ is a winning strategy for $\exists$ in $\mathcal{E}(\mathbb{G}, \sigma)@(v_I, 0)$ so we know that $w\in \Mod_\Psf(\mathbb{G})$) which proofs that indeed

% Now take a node $(v_j, l_j)$ in $\pi$ inbetween a $q_i$ and $q_{i+1}$.
%
% to do this we will make a strategy tree. Take the game tree, we have to prune it so that we have a strategy for $\exists$ essentially we have to make a choice for every $v$ where $L(v)=\lor$. Inductively from the top go downward. If there is a place where $\delta(v, a_l)=\delta(v_1, a_l)\lor \delta(v_2, a_l)$ where $v\leq q$ so if $E_q\models \delta(q, a_l)$ it should either models $\delta(v_1, a_l)$ or $\delta(v_2, a_l)$. Now set $f((v,l))=v_i$ where $E_q\models \delta(v_i, a_l)$. What we now see is that the levels in the run correspond to states $(v', i)$ where we have $L(v)=\deo$ and $(v, i-1)\in \mathbb{T}_{(v_I,0)}^f$. Now we have to proof that this gives a winning strategy tree. We have to check that every finite match ends in a dead end for $\forall$ and that every infinite match hase even infinite priority. For the finite case we look at every level. So we have $E_q\models \delta(q, a_l)$ now we look inductively at this $\delta$ function. For every $(v, i)\in \mathbb{T}^{temp}_{(q, i)}$ If it has the form $\delta(v, a_l)=\delta(v_1, a_l)\land \delta(v_2, a_l)$ we know that both will be true. And if it has the form of $\delta(v, a_l)=\delta(v_1, a_l)\lor \delta(v_2, a_l)$ that at least one of them is true and that is the path added to $\mathbb{T}$. Now if we have $\delta(v, a_l) = \top$ we know that this corresponds to a dead end for $\forall$, $\bot$ is not allowed. Now if we have $\delta(v, a_l)=v'$ we are in a next level. From there on we will check for finiteness and if this path does not stop somewhere it is infinte. We also see that infinite matches correspond to infinite paths in $\rho$. Following the reasoning above we see that the max parity is even iff. So now we have shown that
\[
 \Mod_\Psf(\mathbb{G}) = \LL(\A)
\]
so $\A$ is equivalent to $\mathbb{G}$.
% \textcolor{red}{Allemaal nog iets netter, dat is echt best moeilijk zeg}
\end{proof}
