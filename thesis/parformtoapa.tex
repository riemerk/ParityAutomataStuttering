In this section we will present a translation from a simply guarded  linear parity formula to an
\apa. Furthermore we will prove that this construction is correct.
\begin{definition}\label{def:apafromparform}
 Let $\G=(A, E, L, \Omega_\G, a_I)$ be a simple guarded parity formula over the set of propositional letters $\Psf$. Define the \apa\ $\A_\G=(\Sigma, A, a_I, \delta, \Omega_{\mathbb{A}_\G})$ as follows:
 \begin{itemize}
  \item Set the alphabet to $\Sigma := \PP(\Psf)$;
  \item Define the set of states as $A$;
  \item Define the initial state $a_I$;
 \end{itemize}
 To define the transition function $\delta$ let $a\in A$ and $v\in\Sigma$. Recall that if $a\in A$ is modal or silent we use $e(a)$ as a notation for $E[a]=\{e(a)\}$ and if $a\in A$ is boolean we use $e_L(a)$ and $e_R(a)$ as a notation for $E[a]=\{e_L(a),e_R(a)\}$.
\begin{align*}
   \delta(a, v) &:= \top        \text{ if } L(a)=p\text{ and } p\in v        &\delta(a, v)  &:= \delta(e_L(a),v)\land\delta(e_R(a), v)\text{ if } L(a) = \land\\
   \delta(a, v) &:= \bot        \text{ if } L(a)=p\text{ and } p\notin v     &\delta(a, v)  &:= \delta(e_L(a),v)\lor \delta(e_R(a), v)\text{ if } L(a) = \lor\\
   \delta(a, v) &:= \top \text{ if } L(a)=\overline{p}\text{ and } p\notin v &\delta(a, v)  &:= e(a)\text{ if } L(a) = \deo\\
   \delta(a, v) &:= \bot \text{ if } L(a)=\overline{p}\text{ and } p\in v    &\delta(a, v)  &:= \top        \text{ if } L(a)=\top\\
   \delta(a, v) &:= \delta(e(a), v)\text{ if } L(a)=\epsilon                 &\delta(a, v)  &:= \bot        \text{ if } L(a)=\bot
  \end{align*}
  For the priority map define:
  \[
  \Omega_{\mathbb{A}_\G}(a) = \begin{cases}
                \Omega_\G(a) &\text{if }a\in \Dom(\Omega_{\G})\\
                0&\text{else}
               \end{cases}
  \]
\end{definition}

\begin{lemma}\label{lemma:deadendequivtop}
Let $\G=(A, E, L, \Omega_\G, a_I)$ be a parity formula and $\A_\G=(\Sigma, A, a_I, \delta, \Omega_{\mathbb{A}_\G})$ be the \apa\ as in Definition \ref{def:apafromparform}. Then we have that $(a, l)$ is a dead end for $\forall$ in $\mathcal{E}(\G, w)@(a_I, 0)$ if and only if $\delta(a, w(l))=\top$.
\end{lemma}
\begin{proof}
This follows directly from the definition of the board of $\mathcal{E}(\G, w)$ and the definition of $\delta$ in $\A_\G$
\end{proof}

\textbf{Note voor Yde}: Ik ben een heel klein beetje onzeker of ik de levels nu goed heb gebruikt. Ik bedoel dus alle nodes met de vorm $(v,l)$. Als je in een strategy  $\mathbb{T}_{(q,l)}^f$ zit zijn dat dus vanaf dat punt gezien eigenlijk het eerste level.

\begin{theorem}\label{thm:parformtoapa}
 Let $\G=(A, E, L, \Omega_\G, a_I)$ be a simple guarded parity formula and $\A_\G=(\Sigma, A, a_I, \delta, \Omega_{\mathbb{A}_\G})$ be the automaton from Definition \ref{def:apafromparform}. Then $\G$ and $\A_\G$ are equivalent, i.e.
 \[\Mod_\Psf(\G) = \LL(\A_\G).\]
\end{theorem}
\begin{proof}
\subsection*{Left to right direction}
We will first prove
\begin{equation}\label{eq:lefttoright}
 \Mod_\Psf(\G) \subseteq \LL(\A_\G).
\end{equation}
Suppose $w \in \Mod_\Psf(\G)$ then we know that $(a_I, 0)$ is a winning position for $\exists$ in $\mathcal{E}(\G, w)$. This means there is a winning strategy $f$ for $\exists$ in $\E(\G,w)@(a_I,0)$ and since $\G$ is a parity game we can without loss of generality assume that $f$ is positional. To prove that $w\in \LL(\A_\G)$ we are going to construct a successful run in $\A_\G$ based on the strategy graph $\mathbb{B}_{(a_I,0)}^f$ (Definition \ref{def:stratgraph}). Recall that a run in $\A$ is a directed acyclic graph (DAG) $\rho=(V_\rho,E_\rho)$ where $\rho$ satisfies the four conditions R1-R4 (Definition \ref{def:aparun}). We will define this run $\rho$ as following:
\begin{definitiont}
Define the run $\rho:=(V,E)$ where the nodes $V$ are defined as:
 \[
 V_\rho := \{(a_I, 0)\}\bigcup_{i\geq 1}\left\{(a, i)\in A\times \omega \mid \exists a'\in A\text{ such that } L(a')=\deo\land e(a')=a\land (a', i-1)\in \mathbb{B}_{(a_I,0)}^f \right\}.
\]
And the edges $E$ are defined as:
\[
 E_\rho := \bigcup_{l\geq 0}\left\{((a, l), (a', l+1))\in V_\rho\times V_\rho\mid (a', l+1)\in\mathbb{B}_{(a,l)}^f \right\}.
\]
\end{definitiont}

Firstly we have to check that this is indeed a run. Satisfaction of R1-R3 follows with an easy verification:
\begin{description}
 \item[R1)] By definition of $V_\rho$ we see $V_\rho\subseteq A\times \omega$ and that $(a_I, 0) \in V_\rho$.
 \item[R2)] By definition of $E_\rho$ we see that $E\subseteq \bigcup_{l\geq 0}(A\times \{l\})\times (A\times \{l+1\})$.
 \item[R3)] From the definition of $V_\rho$ we see that for every $(a,l)\in V_\rho\setminus\{(a_I,0)\}$ there exists $a'$ with $L(a')=\deo$, $e(a')=q$ and $(a', l-1)\in \mathbb{B}^f_{(a_I, 0)}$. Now let $(a_{\text{top}}, l-1)$ the node with smallest depth above $(a', l-1)$ in $\mathbb{B}^f_{(a_I, 0), l-1}$. That means that there exists $(a_p, l-2)$ with $L(a_p)=\deo$ and $E(a_p)=a_{\text{top}}$ so we see that $(a_{\text{top}}, l-1)\in V_\rho$ and clearly $(a,l)\in \mathbb{B}_{(a_{\text{top}}, l-1)}^f$ so we see that $((a_{\text{top}}, l-1), (a, l))\in E$.
\end{description}

For the condition R4 we need  dive in the definition of $\delta$.
\begin{claim}
The run $\rho$ as defined above satisfies condition R4. In other words for every vertex  $(a, l)\in V_\rho$ we have that $E_l[a]:=\{a'\mid ((a, l), (a', l+1))\in E_\rho\} \models \delta(a, w(l))$.
\end{claim}
\begin{proof} We are going to look at the level $\mathbb{B}_{(a, l), 0}^f$ of the game graph. Since $\G$ is (strongly) guarded we know that this level is a directed acyclic graph where the nodes $a$ with $|E[a]|=0$ correspond to literals and the nodes $v$ where $E[a]\not\subseteq \mathbb{B}_{(a, l), 0}^f$ correspond to modal nodes. We are going to prove the claim with strong reverse induction on the topological ordering of $\mathbb{B}_{(a, l), 0}^f$ (definition \ref{def:topoorder} and theorem \ref{thm:daghastopoorder}), call this ordering $(a_0,l)(a_1,l)\dots (a_n,l)$ (so we start with the induction at $v_n$).
\textcolor{red}{Deze inductie??? Hoe benoemen}
% This gives the following induction hypothesis: on $k$: For every $n\geq j\geq k$ we have $E_l[a]\models\delta(a_j, w(l))$ \\
We have the following induction hypothesis: for $j$ we have $E_l[a]\models\delta(a_j, w(l))$ \\
\textit{Induction step:} Take $a_i$ with $i\leq n$. We distinguish four different cases:
\begin{description}
 \item[Case:] \(|E[a_i]|=0\). That means $a_i\in \At(\Psf)$ and since $f$ is a winning strategy we have that $av_i, l)$ is a dead end for $\forall$ in the game graph. So by Lemma \ref{lemma:deadendequivtop}  we know that $\delta(a_i, w(l))=\top$ so clearly $E_l[a]\models\delta(a_i, w(l))$.
 \item[Case:] $L(a_i)=\land$. In this case we have $\delta(a_i, w(l)) = \delta(e_L(a_i), w(l)) \land \delta(e_R(a_i), w(l))$, since we have a topological ordering we know that $e_L(a_i)=a_j$ with $j>i$ and $e_R(a_i)=v_k$ with $k>i$. Since we are doing strong induction we assumed that our induction hypothesis is true for all $j>i$ so we see that $E_l[a]\models \delta(a_j, w(l))$ and $E_l[a]\models\delta(a_k, w(l))$ so that means $E_l[a]\models \delta(a_i, w(l))$.
 \item[Case:] $L(a_i)=\lor$. We see that $\delta(a_i, w(l))= \delta(e_L(a_i), w(l)) \lor \delta(e_R(a_i), w(l))$. Since $\mathbb{B}_{(a, l), 0}^f$ is a level of the strategy graph we know that $E_f[a_i] = f(a_i)=a_m$ for a $a_m\in {e_L(a_i), e_R(a_i)}$. Since we have a topological ordering we know that $m>i$. Therefore with the induction hypothesis know that $E_l[a]\models \delta(a_m, w(l))$ so also $E_l[a]\models \delta(a_i, w(l))$.
 \item[Case:] $L(a_i)=\deo$. Here we see that $\delta(a_i, w(l)) = e(a_i)$. Since $(a_i, l)\in \mathbb{B}_{(a_I, 0)}^f$ we know that $(e(a_i), l+1)\in V_\rho$ (from the definition of $V_\rho$). Then from the fact that $(a_i, l)\in \mathbb{B}_{(a, l)}^f$ we see that $(e(a_i), l+1)\in \mathbb{B}_{(a, l)}^f$ so therefore $e(a_i)\in E_l[a]$. Now we see that $E_l[a]\models\delta(a_i, w(l))$.
\end{description}
Since $a=a_0$ this now proves that that $E_l[a] \models \delta(a, w(l))$, as desired.
\end{proof}
\noindent Secondly we have to proof that $\rho$ is indeed an accepting run (definition \ref{def:aparun}). That means that $\max(\inf_{\Omega_{\mathbb{A}_\G}}(u))$ is even for every infinite path $u$ in $\rho$. To prove this fact we will establish a correspondence between infinite paths in $\rho$ and infinite matches in $\mathcal{E}(\G, w)@(v_I, 0)$

\begin{claim}
For every infinite path $u$ in $\rho$ there exists a $f$-guided infinite match $\pi$ in $\mathcal{E}(\G, w)@(a_I, 0)$ where we have $\inf_{\Omega_{\A_\G}}(u) \cup \{0\}= \inf_{\Omega_{\mathcal{E}}}(\pi)$.
% For every infinite path $u$ in $\rho$ there exists a $f$-guided infinite match $\pi$ in $\mathcal{E}(\G, w)@(v_I, 0)$ where we have $\{\Omega_{\mathbb{A}_\G})(v)\mid v\in u\}\cup\{0\}\cup\{\Omega_{\mathcal{E}}(\pi_s)\}= \{\Omega_{\mathcal{E}}(p)\mid p\in \pi\}$ (where denotes $\pi_s = (v, l)$ the first node in $\pi$ with $v\in \Dom(\Omega_\G)$).

\end{claim}

\begin{proof} An infinite path $u$ in $\rho$ consists of states $a_0a_1a_2\dots$ where $a_0 = a_I$ and we have $a_{i+1}\in E_i[a]$ for every $i$. Now construct the infinite match $\pi = (a_0,0)\dots (a_1,1)\dots (a_2, 2)\dots$. In order to make this an infinite match we need to fill in the gaps between $(a_i,i)$ and $(a_{i+1},i+1)$ with finite $f$-guided paths. We see that we can do that since when $a_{i+1}\in E_{(a_i, i)}$ we know that $(a_{i+1}, i+1)\in\mathbb{B}_{(a,i)}^f$ so there is a finite $f$-guided path between $(a_i,i)$ and $(a_{i+1},i+1)$ in $\mathbb{B}_{(a_I, 0)}^f$.

Now we have to show that $\inf_{\Omega_{\A_\G}}(u) \cup \{0\}= \inf_{\Omega_{\mathcal{E}}}(\pi)$. First observe that $\Omega_\mathcal{E}((q_i,i)) = \Omega_{\A_\G} (q_i)$ since they both correspond to
\[
  \Omega_{\A_\G}(q_i)  =\begin{cases}
                \Omega(q_i) &\text{if }q_i\in \Dom(\Omega)\\
                0&\text{else}
               \end{cases} =  \Omega_\mathcal{E}((q_i,i)).
  \]
That immediately gives us that $\inf_{\Omega_{\A_\G}}(u) \cup \{0\}\subseteq \inf_{\Omega_{\mathcal{E}}}(\pi)$, since every state in $u$ corresponds with a node in $\pi$. Now to show that $\inf_{\Omega_{\A_\G}}(u) \cup \{0\}\supseteq \inf_{\Omega_{\mathcal{E}}}(\pi)$ take a node $\pi_j = (a_j, i)$ in $\pi$ in between $(a_i, i)$ and $(a_{i+1}, i+1)$. My claim is that $\Omega_{\mathcal{E}}(\pi_j, i) = 0$. Suppose that $\Omega_{\mathcal{E}}(a_j, i) \neq 0$, then we would have $a_j\in \Dom (\Omega_\G)$. But since $\G$ is simple guarded that means that $L(a_{j-1})=\deo$. We also know that the node before $(a_{i+1},i+1)$ in $\pi$ is a modal node. Now we have a contradiction since we took a finite $f$-guided path between $(a_i,i)$ and $(a_{i+1},i+1)$ in $\mathbb{B}_{(a_I, 0)}^f$, this should have now passed two modal nodes which means it should end in level $i+2$.
This proves that indeed $\inf_{\Omega_{\A_\G}}(u) \cup \{0\}= \inf_{\Omega_{\mathcal{E}}}(\pi)$.
\end{proof}
% . Now assume that $v_j$ is in the scope of a modal operator, since $\G$ is modal simple we know that the modal operator is directly preceding this node. That would mean that $L(\pi_{j-1})=\deo$ and that would mean that $\pi_j$ should be in level $i+1$ which contradicts the fact that $(\pi_j, i)$ is chosen inbetween $(q_i, i)$ and $(q_{i+1}, i+1)$.
%
%
% Now we have to show that $\{\Omega_{\mathbb{A}_\G}(v)\mid v\in u\}\cup\{0\}\cup\{\Omega_{\mathcal{E}}(\pi_s)\}= \{\Omega_{\mathcal{E}}(p)\mid p\in \pi\}$. First observe that $\Omega_\mathcal{E}((q_i,i)) = \Omega_{\A_\G} (q_i)$ since they both correspond to
% \[
%   \Omega_{\A_\G}(q_i)  =\begin{cases}
%                 \Omega(q_i) &\text{if }q_i\in \Dom(\Omega)\\
%                 0&\text{else}
%                \end{cases} =  \Omega_\mathcal{E}((q_i,i))
%   \].
%
%   Now take a node $\pi_j = (v_j, i)$ in $\pi$ in between $(q_i, i)$ and $(q_{i+1}, i+1)$. My claim is that $\Omega_{\mathcal{E}}(\pi_j, i) = 0$ or $\pi_j=\pi_s$. Suppose that $\Omega_{\mathcal{E}}(v_j, i) \neq 0$, then we would have $v\in \Dom (\Omega_\G)$. But since $\G$ is strongly guarded we know that $v_j$ should be in the scope of a modal operator unless it is the first state encountered from the start so that means $\pi_j=\pi_s$. Now assume that $v_j$ is in the scope of a modal operator, since $\G$ is modal simple we know that the modal operator is directly preceding this node. That would mean that $L(\pi_{j-1})=\deo$ and that would mean that $\pi_j$ should be in level $i+1$ which contradicts the fact that $(\pi_j, i)$ is chosen inbetween $(q_i, i)$ and $(q_{i+1}, i+1)$.

% That proves that $\{\Omega_\mathbb{A}(v)\mid v\in u\}\cup\{0\}\cup\{\Omega_{\mathcal{E}}(\pi_s)\}= \{\Omega_{\mathcal{E}}(\pi_i)\mid \pi_i\in \pi\}$ since the only non-zero nodes in $\{\Omega_{\mathcal{E}}(\pi_i)\mid \pi_i\in \pi\}$ corresponds to states in $\G$ or to $\pi_s$.

\noindent To show that $\rho$ is an accepting run we will examine every infinite path $u\in\rho$ and show that $\max(\inf_{\Omega_{\A_\G}}(u))$ is even. From the above claim we see that $u$ corresponds to an infinite $f$-guided match $\pi$ with $\inf_{\Omega_{\A_\G}}(u) \cup \{0\}= \inf_{\Omega_{\mathcal{E}}}(\pi)$. We know that $\max(\inf_{\Omega_\mathcal{E}}(\pi))$ is even since $(a_I, 0)$ is a winning position in $\mathcal{E}(\G, w)@(v_I, 0)$ so we also know that $\max(\inf_{\Omega_{\A_\G}}(u))$ is even. Therefore $\rho$ is an accepting run of $\A_\G$ for $w$ which proves that \( \Mod_\Psf(\G) \subseteq \LL(\A_\G)\) (Equation \ref{eq:lefttoright}).
\subsection*{Right to left direction}
Secondly we will prove
\begin{equation}
 \Mod_\Psf(\G) \supseteq \LL(\A_\G).
\end{equation}\label{eq:righttoleft}
Suppose we have $w\in \LL(\A_\G)$, that means  there exists an accepting run $\rho=(V_\rho, E_\rho)$ on $w$. To show that $w\in \Mod_\Psf(\G)$ we have to show that $(a_I,0)$ is a winning position in $\mathcal{E}(\G, w)@(v_I,0)$ for $\exists$. First we will define a strategy $f$ for $\exists$ and nextly show that this strategy is indeed winning.
\begin{definitiont}\label{def:parformtoapastrat}
To define the strategy $f:\PM_\exists\to V\times \omega$, take a partial match $\pi \in \PM_\exists$ and the last node of this match $(a, l):=\last(\pi)$. Recall from Definition \ref{def:evalgameparform} there are two cases where $\pi$ belongs to $\PM_\exists$:
\begin{description}
 \item[Case $L(a)=\lor$:]
 To define $f$ we need to choose between $e_L(a)$ and $e_R(a)$. Let $(a_p, l-1)$ the first node with $L(a_p)=\deo$ in $\mathbb{T}_{(a_I, 0)}^\mathcal{E}$ above $(a, l)$. We see that this node has one child and call this $(b,l)$. Now set $f$ to
\[f(\pi)=
 \begin{cases}
  (e_L(a), l)&\text{if } (b, l)\in V_\rho \text{ and } E_{(b, l)}\models \delta(e_L(a), w(l))\\
  (e_R(a), l)&\text{if } (b, l)\in V_\rho \text{ and } E_{(b, l)}\models \delta(e_R(a), w(l))\\
  \text{random}&\text{else.}
 \end{cases}
\]
\item[Case: $L(a)\in \At(\Psf)$]
In this case there are no admissible moves for $\exists$ so choose $f(\pi)$ to be a random position. This may sound confusing but later on we will prove that this case actually never occurs.
\end{description}
\end{definitiont}
\noindent We now want to show that this $f$ always defines a winning strategy for $\exists$. We will prove this via several claims. Firstly we show that all the transition functions are satisfied which gives us that every dead end belongs to $\forall$ and that $f$ never chooses randomly. This will be proven per level and we will connect all the levels to form a complete strategy tree. Finally we will prove a correspondence between matches in the strategy tree and infinite paths in the run of $\A$ to prove that $f$ is a winning strategy. Recall the definition $E_l[a]:=\{a'\mid ((a, l), (a', l+1))\in E_\rho\}$ as before.

\begin{claim}\label{claim:satisfies}
% $E_l[a]\models \delta(b,w(l))$ for every $(b,l)\in \mathbb{T}^f_{(a,l),0}$ for every $(a,l)\in V_\rho$.
For every vertex $(a,l)\in V_\rho$ and for every node $b\in A$ such that $(b,l)\in \mathbb{T}^f_{(a,l),0}$ we have that $E_l[a]\models\delta(b, w(l))$
\end{claim}
\begin{proof} Take an arbitrary node $(a,l)\in V_\rho$. We are going to prove the claim via root-to-leaf induction (on depth $k$) in $\mathbb{T}^f_{(a,l),0}$
\begin{description}
 \item[Induction hypothesis:] For every every $\pi\in\mathbb{T}^f_{(a,l),0}$ with depth $k$ and $(b,l):=\last(\pi)$ we have that $E_l[a]\models \delta(b, w(l))$.
 \item[Base case ($k=0$):] Since $\mathbb{T}^f_{(a,l),0}$ is a tree with one root we now that the only $\pi$ with depth $0$ has $\last(\pi)=(a,l)$. Since $\rho$ is a run we know by R4 that $E_l[a]\models \delta(a, w(l))$.
 \item[Inductive case ($k>0$):] Take a $\pi\in \mathbb{T}^f_{(a, l),0}$ with $\last(\pi)=(b, l)$. First look at the parent of $\pi$, call this $\pi_p$ and let $\last(\pi_p)=(b_p, l)$. This node has depth $k-1$ so with the induction hypothesis we know that $E_l[a]\models\delta(b_p, w(l))$. Now we distinguish cases:
 \begin{description}
  \item[Case $L(b_p)=\lor$:] In this case we know $\delta(b_p, w(l)) = \delta(e_L(b_p), w(l))\lor \delta(e_R(b_p), w(l))$ and $f(\pi_p)=(b,l)$ with $b\in \{e_L(b_p), e_R(b_p)\}$. From the definition of $f$ we know that $E_l[a]\models\delta(f(\pi_p), w(l))$ so that proves that $E_l[a]\models \delta(e_L(b_p), w(l))\lor \delta(e_R(b_p), w(l))$ since it satisfies at least one of the parts.
\item[Case $L(b_p) = \land$:] In this case we know $\delta(b_p, w(l)) =\delta(e_L(b_p), w(l))\land \delta(e_R(b_p), w(l))$  where $b\in \{e_L(v_p), e_R(v_p)\}$. With the induction hypothesis we know that $E_l[a]\models\delta(b_p, w(l))$ so also $E_l[a]\models\delta(b, w(l))$.
\item[Case $L(b_p)=\deo$:] This case is not possible since then $\pi$ would be in the next level.
% The \textbf{case}
 \end{description}
\end{description}
So we can conclude that for every $(a,l)\in V_\rho$ we have that for $b\in A$ such that $(b,l)\in \mathbb{T}^f_{(a,l),0}$ it holds that $E_l[a]\models \delta(b,w(l))$.
\end{proof}

\begin{claim}\label{claim:norandom}
For every vertex $(a,l)\in V_\rho$ the strategy $f$, as defined in Definition \ref{def:parformtoapastrat}, never chooses randomly in the level of the strategy tree $\mathbb{T}^f_{(a,l),0}$.
\end{claim}
\begin{proof} Let $(a,l)\in V_\rho$ be a vertex in the run and $\pi\in \mathbb{T}^f_{(a,l),0}$ be a partial match in this level of the strategy tree with $(b,l):=\last(\pi)$. We see that the strategy $f$ chooses randomly in two cases:
\begin{description}
 \item[Case:] $L(b)=\lor$ and $E_l[a]\not\models\delta(e_L(b), w(l))$ and $E_l[a]\not\models\delta(e_R(b), w(l))$\\
 From Claim 1 we know that $E_l[a]\models \delta(b, w(l))$. Since $\delta(b, w(l))=\delta(e_L(b), w(l))\lor\delta(e_R(b), w(l))$ we see that $E_l[a]\models\delta(e_L(b),w(l))$ or $E_l[a]\models\delta(e_R(b),w(l))$ so that means that $f$ does not does not choose randomly in this case
 \item[Case:] $L(b)\in \Lit(\Psf)$. \\
 Since this $\pi$ belongs to $\PM_\exists$ we know that this node should be a dead end for $\exists$. That would mean that $\delta(b,w(l))=\bot$. With claim 1 we know that $E_l[a]\models\delta(b,w(l))$ so clearly that gives a contradiction.\qedhere
\end{description}
\end{proof}

\begin{claim}\label{claim:deadendabel}
For every vertex $(a,l)\in V_\rho$ we have that every dead end in $\mathbb{T}^f_{(a,l),0}$ belongs to $\forall$.
\end{claim}
\begin{proof}
If there would be a dead end  that belongs to $\exists$ then $f$ would choose randomly and that never happens according to Claim \ref{claim:norandom}.
\end{proof}

Now we have proven three crucial claims locally, but we want to make these claims about the global strategy tree $\mathbb{T}^f_{(v_I,0)}$. We will prove that for every leaf in every $\mathbb{T}_{(a, l),0}^f$ that corresponds to $L(a)=\deo$ the claims (3.11.4-6) also hold for the strategy tree $\mathbb{T}_{(e(v), l),0}^f$. From here it follows that Claims 3.11.4, 5 and 6 generalize to the whole  strategy tree $\mathbb{T}_{(v_I, 0)}^f$.

%  We can generalize claims 1, 2 and 3 to $\mathbb{T}_{(v_I, 0)}^f$.
\begin{claim}
For every vertex $(a,l)\in V_\rho$ and every node $b\in A$ such that $L(b)=\deo$ and $(b,l)\in \mathbb{T}_{(a,l),0}$ we have that Claims \ref{claim:satisfies} up to \ref{claim:deadendabel} hold for $\mathbb{T}_{(e(b), l+1),0}^f$. And these three claims also hold for $\mathbb{T}_{(v_I, 0),0}^f$.
%  In other words: we have checked $\mathbb{T}_{(v_I, 0),0}^f$ and for every $(q,l)\in V_\rho$ we have that for every modal leaf in $\mathbb{T}_{(q, l),0}^f$ we also checked the strategy tree $\mathbb{T}_{(e(v), l),0}^f$.
\end{claim}
\begin{proof} Firstly we see that $(v_I,0)\in V_\rho$ and since Claims \ref{claim:satisfies} up to \ref{claim:deadendabel} hold for every $(a,l)\in V_\rho$ they also hold for $(v_I,0)$.

Secondly take an arbitrary $(a,l)\in V_\rho$ and its corresponding level of the strategy tree  $\mathbb{T}_{(a, l),0}^f$. Now take an arbitrary leaf of this level $\pi$ with $(b,l):=\last(\pi)$ and $L(b)=\deo$. We know that $\delta(b, w(l))=e(b)$. By Claim \ref{claim:satisfies} we know that $E_l[a]\models \delta(b, w(l))=e(b)$ so that means $e(b)\in E_l[a]$. Therefore we know that $(e(b), l+1)\in V_\rho$. That means that Claims \ref{claim:satisfies} up to \ref{claim:deadendabel} hold for $\mathbb{T}_{(e(v), l+1),0}^f$.\\
We conclude that we can connect every leaf that is not a dead end in every level to another level that we created. So we can generalize Claims \ref{claim:satisfies} up to \ref{claim:deadendabel} to the whole $\mathbb{T}_{(a_I, l)}^f$.
\end{proof}

This allows us to generalize these claims so we know that:
\begin{enumerate}
 \item For every node $(a,l)\in\mathbb{T}_{(a_I,0)}^f $ such that there exists a node $(a',l-1)\in \mathbb{T}_{(a_I),0}^f$ with $L(a')=\deo$ and $e(a')=a$. We have that for every state $b\in A$ such that $(b, l)\in \mathbb{T}_{(a,l),0}^f$ that $E_l[a]\models\delta(b, w(l))$.
 \item The strategy $f$ never chooses randomly in $\mathbb{T}_{(a_I,0)}^f$.
 \item Every dead end in $\mathbb{T}_{(a_I,0)}^f$ belongs to $\forall$.
\end{enumerate}

Now we have to make sure that this strategy tree indeed defines a winning strategy for $\exists$ in $\mathcal{E}(\G, w)@(v_I, 0)$. The fact that every finite match ends in a dead end for $\forall$ has already been proven. Nextly we will show that for every infinite match $\pi$ the maximum parity that occurs infinitely often $\max(\inf_{\Omega_\mathcal{E}} (\pi))$ is even.

\begin{claim}\label{claim:infpathmatch}
Every infinite $f$-guided infinite match $\pi$ in $\mathcal{E}(\G, w)@(v_I, 0)$ corresponds to an infinite path $u$ in $\rho$ where we have
$\inf_{\Omega_\mathcal{E}} (\pi) = \inf_{\Omega_{\A_\G}}(u)\cup\{0\}$.
\end{claim}
\begin{proof}
Take an $f$-guided match $\pi=(a_0, l_0)(a_1, l_1)(a_2, l_2)\dots$ in $\mathcal{E}(\G, w)@(a_I, 0)$. Now mark $b_0=(a_0, l_0)$ and mark $b_i$ as the $i$'th node $(a_j, l_j)$ in $\pi$ where $L(a_{j-1})=\deo$. Now construct the infinite $A$-sequence $u=b_0b_1b_2\dots$..


First we have to check that the sequence $u$ defines an infinite path in $\rho$, that means that for every $b_i=(a_j, l_j)$ and $b_{i+1}=(a_{j'}, l_j + 1)$ we have $a_{j'}\in E_{l_j}[a_j]$. We know that $(a_{j'-1}, l_j)=\last(\pi')$ for a partial match $\pi'\in \mathbb{T}_{(a_j,l_j),0}^f$ and $L(a_{j'-1})=\deo$ so $\delta(a, w(l_j))=a_{j'}$. From claim \ref{claim:satisfies} we have $E_{l_j}[a_j]\models\delta(a_{j'-1})$ so $a_{j'}\in E_{l_j}[a_j]$ which proves that $u$ is an infinite path in $\rho$.
Nextly we will look at the parities that occur infinitely often. We easily see that for $b_i=(a_j, l_j)$ we have $\Omega_\mathcal{E}((a_j, l_j)) = \Omega_{\A_\G} (b_i)$ since they both correspond to
\[
  \Omega_\mathcal{E}((a_j, l_j))=
  \begin{cases}
    \Omega_{\G}(a_j) &\text{if }a_j\in \Dom(\Omega_\G)\\
    0&\text{else}
  \end{cases} = \Omega_{\A_\G}(b_i).
\]
With this we immediately see that $\inf_{\Omega_\mathcal{E}} (\pi) \supseteq \inf_{\Omega_{\A_\G}}(u)\cup\{0\}$. Now to show the reverse inclusion we show that every element in $\inf_{\Omega_\mathcal{E}} (\pi)$ that is not in $\inf_{\Omega_{\A_\G}}(u)$ is equal to zero. To show this take a node $(a_j, l_j)\in \pi$ strictly in between $b_i$ and $b_{i+1}$. Now we claim that $\Omega_{\mathcal{E}}(a_j, l_j) = 0$. Suppose $\Omega_{\mathcal{E}}(a_j, l_j) \neq 0$, then we would have $a_j\in \Dom (\Omega_\G)$. However since $\G$ is simple guarded we know that $L(a_{j-1})=\deo$ which contradicts to the fact that $b_{i+1}$ is chosen as the first node after $b_{i}$ that follows a modal node. So we see that $\inf_{\Omega_\mathcal{E}} (\pi) \subseteq \inf_{\Omega_{\A_\G}}(u)\cup\{0\}$ which proves that $\inf_{\Omega_\mathcal{E}} (\pi) = \inf_{\Omega_{\A_\G}}(u)\cup\{0\}$.
\end{proof}

Now to see that $f$ defines a winning strategy we consider an arbitrary infinite $f$-guided match $\pi$. With claim \ref{claim:infpathmatch} we see that there exists an infinite path $u$ in $\rho$ with $\inf_{\Omega_\mathcal{E}} (\pi) = \inf_{\Omega_{\A_\G}}(u)\cup\{0\}$. Since $\rho$ is accepting we know that $\max(\inf_{\Omega_{\A_\G}}(u))$ is even and therefore $\max(\inf_{\Omega_\mathcal{E}}(\pi))$ is even as well. This proves that $f$ is a winning strategy so that means that $w\in \Mod_\Psf(\G)$ and we have proven $\Mod_\Psf(\G)\supseteq \LL(\A_\G)$ (Equation \ref{eq:righttoleft})

Combining both inclusions we see that \(
 \Mod_\Psf(\G) = \LL(\A_\G),
\)
so $\A_\G$ is equivalent to $\G$.
\end{proof}
%%%%% OUDE TEKST::

% \textcolor{red}{Hier moet je aanpassen dat de inf sets gelijk zijn (easy)}
% \begin{claim} Every infinite $f$-guided infinite match $\pi$ in $\mathcal{E}(\G, w)@(v_I, 0)$ corresponds to an infinte path $u$ in $\rho$ where we have $\{\Omega_{\A_\G}(v)\mid v\in u\}\cup\{0\}\cup\{\Omega_{\mathcal{E}}(\pi_s)\} = \{\Omega_{\mathcal{E}}(p)\mid p\in \pi\}$. Where $\pi_s$ denotes the first state in $\pi$.
% \end{claim}
% \begin{proof}Take a $f$-guided match $\pi=(v_0, l_0)(v_1, l_1)(v_2, l_2)\dots$ in $\mathcal{E}(\G, w)@(v_I, 0)$. Now mark $q_0=(v_0, l_0)$ and mark $q_i$ as the $i$'th node $(v_j, l_j)$ in $\pi$ where $L(v_{j-1})=\deo$. Construct the infinite path $u=q_0q_1q_2\dots$ in $\rho$.\\
% First we have to check that $u$ is an infinite path in $\rho$, that means that for every $q_i=(v_j, l_j)$ let $q_{i+1}=(v_{j'}, l_j + 1)$ we have $v_{j'}\in E_{(v_j, l_j)}$. We know that $(v_{j'-1}, l_{j'-1})=\last(\pi)$ for a $\pi\in \mathbb{T}_{(v_j,l_j)}^f[l]$ and $L(v_{j'-1})=\deo$ so $\delta(v, w(l))=v_{j'}$. From the claim 1 we have $E_l[q]\models\delta(v_{j'-1})$ so $v_{j'}\in E_{(v_j, l_j)}$ which proves that $u$ is an infinite path.\\
%
% Now we are going to look at the parities. We easily see that for $q_i=(v_j, l_j)$ we have $\Omega_\mathcal{E}((v_j, l_j)) = \Omega_{\A_\G} (v_j)$ since they both correspond to
% \[
%   \Omega_\mathcal{E}((v_j, l_j))=
%   \begin{cases}
%     \Omega_{\G}(v_j) &\text{if }v_j\in \Dom(\Omega_\G)\\
%     0&\text{else}
%   \end{cases} = \Omega_{\A_\G}(v_j)
% \].
%
% Now take a node $(v_j, l_j)$ in $\pi$ strictly in between $q_i$ and $q_{i+1}$. My claim is that $\Omega_{\mathcal{E}}(v_j, l_j) = 0$ or $(v_j, l_j)=\pi_s$. Suppose $\Omega_{\mathcal{E}}(v_j, l_j) \neq 0$, then we would have $v_j\in \Dom (\Omega_\G)$. But since $\G$ is strongly guarded we know that $v_j$ should be in the scope of a modal operator unless it is the first state encountered from the start so $\pi_j=\pi_s$. Now assume that $v_j$ is in the scope of a modal operrator. Since $\G$ is modal simple we know that the modal operator is directly preceding this node. That would mean that $L(v_{j-1})=\deo$ and that would mean that $(v_j, l_j)$ should be in level $l_j+1$ which contradicts the fact that $q_{i+1}$ is chosen as the first node after $q_i$ that follows a modal node. That proves that $\{\Omega_\mathbb{A}(v)\mid v\in u\}\cup\{0\}\cup\{\Omega_{\mathcal{E}}(\pi_s)\}= \{\Omega_{\mathcal{E}}(\pi_i)\mid \pi_i\in \pi\}$ since the only non-zero nodes in $\pi$ corresponds to states in $\G$.
% \end{proof}
%
% \begin{claim} For every inifinite match $\pi$ in $\mathbb{T}_{(v_I,0)}^f$ we have $\max(\inf_{\Omega_\mathcal{E}} (\pi))$ is even.
% \end{claim}
% \begin{proof}From claim 6 we know that for every infinite $f$-guided infinite match $\pi$ there is a infinite path $u$ in $\rho$. Since $\rho$ is an accepting run we know that $\max(\inf_{\Omega_{\A_\G}}(u))$ is even and since $\{\Omega_{\A_\G}(v)\mid v\in u\}\cup\{0\}\cup\{\Omega_\E(\pi_s)\} = \{\Omega_{\mathcal{E}}(p)\mid p\in \pi\}$ we then also know that $\max(\inf_{\Omega_\mathcal{E}}(\pi))$ is even, which proofs the claim.
% \end{proof}

% Combining claims 5 and 7 gives us that $f$ is a winning strategy for $\exists$ in $\mathcal{E}(\G, w)@(v_I, 0)$ so we know that $w\in \Mod_\Psf(\G)$) which proofs that indeed
% \[
%  \Mod_\Psf(\G) = \LL(\A_\G),
% \]
% so $\A_\G$ is equivalent to $\G$.
