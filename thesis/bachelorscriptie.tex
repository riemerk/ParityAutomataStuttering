\documentclass[dubaWI, english]{uvamath}%use option "english" for an English thesis
%Alle mogelijke opties zijn: complatex, tweedejaarsproject, bachelorscriptie, dubaWI, dubaWN

%preamble
\usepackage[english]{babel} %Remark: use the same language for uvamath and babel
\usepackage{graphicx}
\usepackage[pdfborder={0 0 0}]{hyperref}
\usepackage{lipsum}
\usepackage{amsfonts}
\usepackage{amsthm}
\usepackage{amsmath,amssymb, mathrsfs}
\usepackage{mathtools}
\usepackage{modalops}
\usepackage{hyperref}
\usepackage{stackrel}
% \usepackage{logix}
% \setmainfont{STIX Two Text}
% \setmathfont{STIX Two Math}

% Required packages and libraries
\usepackage{tikz}
\usetikzlibrary{automata, arrows.meta, positioning}

\theoremstyle{plain}

\theoremstyle{definition}
\newtheorem{definition}{Definition}[chapter]
\newtheorem{example}[definition]{Example}
\theoremstyle{plain}
\newtheorem{theorem}[definition]{Theorem}
\newtheorem{fact}[definition]{Fact}
\newtheorem{proposition}[definition]{Proposition}


\renewcommand{\phi}{\varphi}
\renewcommand{\epsilon}{\varepsilon}
\DeclareMathOperator{\Mod}{Mod}
\DeclareMathOperator{\Dom}{Dom}
\DeclareMathOperator{\last}{last}

\DeclareMathOperator{\Lit}{Lit}


\newcommand{\A}{\mathbb{A}}
\newcommand{\N}{\mathbb{N}}
\newcommand{\Claim}{\textbf{Claim}}
\newcommand{\Proof}{\textit{Proof}}


\newcommand{\LL}{\mathcal{L}}
\newcommand{\PP}{\mathcal{P}}
\newcommand{\Psf}{\mathsf{P}}

%definitions
\newcommand{\F}{\mathcal{F}}




\title{Stutter-invariance and equivalences in logic and automata}
\author[riemer.kerkstra@student.uva.nl, 13283529]{Riemer Kerkstra}

\supervisors{prof.\ dr.\ Yde Venema}
\supervisors{dr.\ Malvin Gattinger}
\secondgrader{dr.\ Alexi Block Gorman}
\secondgrader{prof. dr.\ Balder ten Cate}
% \coverimage{\includegraphics[scale=0.8]{figuur.pdf}}

\begin{document}
\maketitle

\begin{abstract}
Schrijf een samenvatting van hoogstens een halve bladzijde waarin je kort uitlegt wat je hebt gedaan. De samenvatting schrijf je als laatste. Je mag er vanuit gaan dat je docent de lezer is.
\end{abstract}

\tableofcontents

\chapter{Inleiding}
Vertel iets over je artikel, noem je vraagstelling. Richt je tekst op medestudenten. (Richtlijn 2 bladzijden).hh
\chapter{Preliminaries}
In this section we will provide the reader with some preliminary defintions and theorom which will be helpful and nescsecary to understand the thesis.
\section{$\mu$-temporal logic}
\begin{definition}\label{def:mutempform}
 We define the collection $\mu$-TL of \textbf{temporal fixpoint formulas} as follows:
\[
\phi ::= \top \mid \bot \mid p\mid\bar{p}\mid (\phi\land\phi)\mid(\phi\lor\phi)\mid\deo\phi\mid\nu.x\phi\mid \mu x.\phi\mid \phi U\phi\mid \phi R\phi
\]
\end{definition}
\begin{definition}\cite[Section 6.1]{demri2016temporal}\label{def:linearmodel}
 A \textbf{linear transition system} or linear (Kripke) model of type $\Psf$ is an infinite sequence
 \[
  \sigma: \omega \to \PP(\Psf)
 \]
 it can (and will) be convenient to see this as an $\omega$-word of $(\PP(\Psf))^\omega$.
\end{definition}
\begin{definition}Extension of\cite[Definition 2.20]{venema2024modalmucalculus}
 Give a clean modal fixpoint formula $\xi$ and a linear transition system $\sigma$ we define the evaluation game $\mathcal{E}(\xi, \sigma)$ as a parity game with players $\exists$ and $\forall$ moving along a board consisting of positions $\phi, s)\in Sf(\xi)\times \omega$. The admissable moves are\\
 \begin{tabular}{|c|c|c|}
  \hline
  Position & Player & Admissable moves\\
  \hline
  $(\phi_1\lor\phi_2,s)$& $\exists$ & $\{(\phi_1, s),(\phi_2, s)\}$\\
  $(\phi_1\land\phi_2,s)$& $\forall$ & $\{(\phi_1, s),(\phi_2, s)\}$\\
  $(\deo\phi,s)$&-&$\{(\phi, s+1)\}$\\
  $(\eta_xx.\delta_x, s)$&-&$\{(\delta_x, s)\}$\\
  $(x, s)$ with $x\in BV(\xi)$&-&$\{(\delta_x, s)\}$\\
  $(\phi U\psi,s)$& $\exists$ & $\{(\psi, s),(\phi\land\deo\phi U \psi, s)\}$\\
  $(\phi R\psi,s)$& $\exists$ & $\{(\phi \land \psi, s),(\phi\land\deo\phi U \psi, s)\}$\\
  $(\bot,s)$& $\exists$ & $\emptyset$\\
  $(\top,s)$& $\forall$ & $\emptyset$\\
  $(p,s)$ with $p\in FV(\xi)$ and $p\in \sigma(s)$& $\forall$ & $\emptyset$\\
  $(p,s)$ with $p\in FV(\xi)$ and $p\notin \sigma(s)$& $\exists$ & $\emptyset$\\
  $(\bar{p},s)$ with $p\in FV(\xi)$ and $p\notin \sigma(s)$& $\forall$ & $\emptyset$\\
  $(\bar{p},s)$ with $p\in FV(\xi)$ and $p\in \sigma(s)$& $\exists$ & $\emptyset$\\
\hline
 \end{tabular}\\
 Where we allow infinite unfolding of $\nu$ and $R$ operators and only finite unfoldings of $\mu$ and $U$ operators. We say
 \[
  \sigma, s\Vdash_g\xi \text{ iff } (\xi, s)\in \text{Win}_\exists(\mathcal{E}(\xi, \sigma))
 \]
\end{definition}
\begin{definition}
 We can also define the semantics in the following way:
 \begin{align*}
  \sigma, i &\Vdash p & \text{ iff } p\in \sigma(i)\\
  \sigma, i &\Vdash \bar{p} & \text{ iff } p\notin \sigma(i)\\
  \sigma, i&\Vdash \bot &\text{ never}\\
  \sigma, i&\Vdash \top &\text{ always}\\
  \sigma, i&\Vdash \phi_1\land\phi_2&\text{ iff }\sigma, i\Vdash \phi_1 \text{ and }\sigma, i\Vdash \phi_2\\
  \sigma, i&\Vdash \phi_1\lor\phi_2&\text{ iff }\sigma, i\Vdash \phi_1 \text{ or }\sigma, i\Vdash \phi_2\\
  \sigma, i&\Vdash \deo \phi&\text{ iff }\sigma, i+1\Vdash \phi\\
  \sigma, i &\Vdash \phi_1U\phi_2 &\text{ iff there is }j\geq i\text{ such that }\sigma, j\Vdash \phi_1 \text{ and} \\
                                 &&\sigma, k\Vdash \phi_2 \text{ for all } k \text{ with }i\leq k<j\\
 \sigma, i &\Vdash \phi_1R\phi_2 &\text{ iff } \sigma, j \Vdash \phi_1 \text{ for all } j\geq i\text{ or}\\
                                &&\text{there is }j\geq i\text{ such that }\sigma, j\Vdash \phi_1 \land \phi_2 \text{ and } \\
                                 &&\sigma, k\Vdash \phi_2 \text{ for all } k \text{ with }i\leq k<j
 \end{align*}
 \textcolor{red}{And for fixpoints zelfde maar dan met verz.}
\end{definition}
\begin{theorem}
 The game theoretic semantics are equivalent to the algebraic semantics (specifically for the Until and release)
\end{theorem}
\begin{proof}
 Most of this has been done by Yde. Bot TODO for the Until and Release (can rewrite the Release but rewriting until is lame)
\end{proof}

\section{Linear parity formulas}
\begin{definition}\label{def:linparform}
 We will define a linear parity formula over $\Psf$ almost similar to parity formulas according to definition 6.1 from Venema in \cite{venema2024modalmucalculus}. Let $\mathbb{G}=(V, E, L, \Omega, v_I)$ the same but $L:V\to\mathtt{At}(\Psf)\cup\{\land,\lor,\deo,\epsilon\}$. Still $|E[v]|=1$ if $L(v)=\deo$ and nodes labeled $\deo$ are called modal. Write $E[v]=\{v_1,v_2\}$ if $|E[v]|=2$ and $E[v]=\{v'\}$ if $|E[v]|=1$. The evaluation game $(\mathcal{E}, \sigma)$ is the parity game where the board consists of the set $V\times \omega$, the priority map $\Omega':V\times \omega\to\omega$ is given by
 \[
  \Omega'(v, x):=\begin{cases}
                  \Omega(v)&\text{if }v\in \Dom(\Omega)\\
                  0&\text{else}
                 \end{cases}
 \]
 with the following game graph:\\
 \begin{tabular}{|c|c|c|}
  \hline
  Position & Player & Admissable moves\\
  \hline
  $(v, s)$ with $L(v)=\lor$& $\exists$ & $\{(v_1, s),(v_2, s)\}$\\
  $(v, s)$ with $L(v)=\land$& $\forall$ & $\{(v_1, s),(v_2, s)\}$\\
  $(v, s)$ with $L(v)=\deo$&-&$\{(v', s+1)\}$\\
  $(v, s)$ with $L(v)=\epsilon$&-&$\{(v', s)\}$\\
  $(v, s)$ with $L(v)=p$ and $p\in\sigma(s)$& $\forall$ & $\emptyset$\\
  $(v, s)$ with $L(v)=p$ and $p\notin\sigma(s)$& $\exists$ & $\emptyset$\\
  $(v, s)$ with $L(v)=\bar{p}$ and $p\notin\sigma(s)$& $\forall$ & $\emptyset$\\
  $(v, s)$ with $L(v)=\bar{p}$ and $p\in\sigma(s)$& $\exists$ & $\emptyset$\\
  $(v, s)$ with $L(v)=\top$& $\forall$ & $\emptyset$\\
  $(v, s)$ with $L(v)=\bot$& $\exists$ & $\emptyset$\\
\hline
 \end{tabular}\\
 In this thesis we will refer to these as just parity formulas.
\end{definition}

\begin{definition}\label{def:evalgameparform}
 Let $\sigma: \omega \to \PP(\Psf)$ a linear model and let $\mathbb{G}=(V, E, L, \Omega, v_I)$ a linear parity formula. The \emph{evaluation game} $\mathcal{E}(\mathbb{G}, \sigma)$ is the parity game $G, E', \Sigma'$ with players $\exists$ and $\forall$ of which the board consists of the set $V\times \omega$, the priority map $\Omega': V\times \omega \to \omega$ is given by
 \[
  \Omega'(v, n) := \begin{cases}
                    \Omega(v) & \text{if }v \in \Dom (\Omega)\\
                    0 &\text{else}
                   \end{cases}
 \]
and the game graph is given in definition \ref{def:linparform}.
\end{definition}

\begin{definition}
 We say \(\sigma, n\Vdash \mathbb{G}\) if the pair $(v_I, n)$ is a winning position for $\exists$ in $\mathcal{E}(\mathbb{G},\sigma)$
\end{definition}
\begin{theorem}\cite[theorem]{venema2024modalmucalculus}
 Par game has pos strat
\end{theorem}
\begin{definition}
 Where $f$ is a surviving positional strategy tree for player $\exists$ in $\mathcal{B}@a$, we may represent $f$ as the pruned subgraph of the game graph $\mathbb{B}_a^\mathcal{B}$ that is based on those nodes that correspond to $f$-guided matches. In this so called \emph{strategy graph} $\mathbb{B}_a^f$ we have:
 \[
  E_f[p] := \begin{cases}
                                     E[p] &\text{if } p \in B_\forall\\
                                     f(p) &\text{if } p \in B_\exists
                                    \end{cases}
 \]
 Define the level $l$ as the subgraph of $\mathbb{B}_a^f$ consisting of the nodes of the form $V\times \{l\}$ denote this with $\mathbb{B}_a^f[l]$.
\end{definition}
\section{Automata}
\subsection{Nondeterministic parity automata (NPA)}
\begin{definition} (\cite{venema2024modalmucalculus})\label{def:NPA}
A non-deterministic parity automaton (NPA) is a quintuple \(\A = (Q, \Sigma, \delta, q_I, \Omega)\) where $Q$ is a finite set of states, $\Sigma$ a finite alphabet, $\delta \subset Q\times \Sigma \to \mathcal{P}(Q)$ the transition function, $q_I\subset Q$ the set of initial states and $\Omega: Q\to \omega$ the parity map.
\end{definition}
For the definition of $\hat{\delta}$, $q\stackrel{a}{\rightarrow}q'$, $q\stackrel{w}{\twoheadrightarrow}q'$, runs and accepted languages ($\LL(\A)$) I refer to \cite{venema2024modalmucalculus}. A run is acepting iff \(\max\{\Omega(q)|q\in \inf(\rho)\}\) is even.

\begin{definition}\label{def:partialrun}
 Given a NPA \(\A = (Q, \Sigma, \delta, q_I, \Omega)\) we can extend the definition of $q\stackrel{w}{\twoheadrightarrow}q'$ to include the states visited. Let $w=a_0\dots a_{n-1}$ a $\Sigma$-word and $\rho=q_0\dots q_n$ a finite list of states. We will write $q_1\stackrel{w}{\twoheadrightarrow^\rho}q_n$ if $q_i\stackrel{a_i}{\longrightarrow}q_{i+1}$ for all $0\leq i\leq n-1$ and call $\rho$ a \textbf{partial run} in this context.
\end{definition}

\subsection{Alternating Parity Automata (APA)}

\begin{definition}(\cite[Section 14.3.1]{demri2016temporal})\label{def:posboolform}
Define $\mathbb{B}^+(X)$ as the set of \textbf{positive boolean formulas} over $X$. They respect the following grammar
\[
 \mathcal{F}::= \bot\mid\top\mid x\mid (\mathcal{F}\lor\mathcal{F})\mid (\mathcal{F}\land\mathcal{F})
\]
And we say that: $Y\models x$ iff $x\in Y$, $Y\models \mathcal{F}\lor \mathcal{F}'$ iff $Y\models \mathcal{F}$ or $Y\models \mathcal{F}'$, $Y\models \mathcal{F}\land\mathcal{F}'$ iff $Y\models \mathcal{F}$ and $Y\models \mathcal{F}'$ and we have if $Y\subset Y'$ and $Y\models \mathcal{F}$ then also $Y'\models \mathcal{F}$.
\end{definition}
\begin{definition}(\cite[Definition 14.3.2]{demri2016temporal})\label{def:APA}
 A \textbf{alternating parity tree automaton} (ATPA) $\mathcal{A}=(\Sigma, Q, q_0,\delta, \Omega)$ is defined as:
 \begin{itemize}
  \item $\Sigma$: alphabet
  \item $Q$ finite set of states
  \item $q_0\in Q$ initial state
  \item $\delta: Q\times \Sigma \to \mathbb{B}^+(Q)$ transition function
  \item $\Omega: Q\to\mathbb{N}$ the parity map
 \end{itemize}
\end{definition}
For the definition of runs and acceptance criteria I refer to \cite[Section 14.3.1]{demri2016temporal}.
\textcolor{red}{Voor de tijd stuur ik je nu even dit hoofdstuk, maar in de eindversie ga ik dit wel zelf neertypen denk ik}

\section{Stuttering}
Definitions about stuttering are adapted from \cite{etessami1999stutter}.
\begin{definition}
Two $\omega-$words $w$ and $w'$ over the same alphabet $\Sigma$ are called \textbf{stutter-equivalent} if there exists $f, f': \mathbb{N} \to \mathbb{N}^+$ and a sequence of letters $a_i\in\Sigma$:  $a_0a_1a_2\dots$ such that $w = a_0^{f(0)}a_1^{f(1)}a_2^{f(2)}\dots$ and $w' = a_0^{f'(0)}a_1^{f'(1)}a_2^{f'(2)}\dots$. We denote this as $w\sim_s w'$
\end{definition}
Nog iets toevoegen $w[f]$ als notatie (makkelijker).

\begin{definition}
We call a language \textbf{stutter-invariant} if it holds that if $w\sim_sw'$ then $w\in L\Leftrightarrow w'\in L$
\end{definition}
\begin{definition}
 For a language we define the \textbf{stutter-closure} as the following set
 \[
 L^s = \{w\in \Sigma^\omega\mid w\sim_s w' \text{ for a } w'\in L\}
 \]
\end{definition}


\chapter{Translating a $\mu$-TL formula to an Alternating Parity Automata}
\section{Guardedness and modally simple}
Recall the following definitions about guardedness for $\mu$-ML formulas and parity formulas
\begin{definition}\label{def:guardedmuform}\cite[Definition 2.19]{venema2024modalmucalculus}
 A variable $x$ is guarded in a $\mu$-TL formula $\phi$ if every free occurence of $x$ in $\phi$ is in the scope of a modal operator. A formula $\xi\in\mu$-TL is \emph{guarded} if for every subformula of $\xi$ of the form $\eta x.\delta$, $x$ is guarded in $\delta$.
\end{definition}

\begin{theorem}\label{thm:muformequivguarded}\cite[Proposition 3.27]{venema2024modalmucalculus}
 Every formula $\phi$ is equivalent to a guarded formula $\phi_g$
\end{theorem}
\begin{theorem}\label{thm:nextdistributive}
The $\deo$-operator is distributive. In other words the following equivalences hold:
\begin{enumerate}
%  \item $\deo\neg \phi\equiv \neg \deo\phi$
 \item $\deo (\phi_1\land\phi_2)\equiv \deo\phi_1\land \deo\phi_2$
 \item $\deo (\phi_1\lor\phi_2)\equiv \deo\phi_1\lor \deo\phi_2$
 \item $\deo (\phi_1 U\phi_2)\equiv (\deo\phi_1)U(\deo\phi_2)$
 \item $\deo (\phi_1 R\phi_2)\equiv (\deo\phi_1)R(\deo\phi_2)$
\end{enumerate}
\end{theorem}
\begin{proof}
\begin{enumerate}
 \item We see that
 \begin{align*}
  \sigma, i\Vdash \deo (\phi_1\land\phi_2)&\iff \sigma, i+1\Vdash \phi_1\land\phi_2\\
  &\iff \sigma, i+1\Vdash \phi_1 \text{ and }\sigma, i+1\Vdash \phi_2\\
  &\iff \sigma, i\Vdash\deo \phi_1 \text{ and }\sigma, i\Vdash\deo \phi_2\\
  &\iff \sigma, i\Vdash\deo (\phi_1 \land \phi_2 )\\
 \end{align*}
 \item Observe:
 \begin{align*}
  \sigma, i\Vdash \deo (\phi_1\lor\phi_2)&\iff \sigma, i+1\Vdash \phi_1\land\phi_2\\
  &\iff \sigma, i+1\Vdash \phi_1 \text{ or }\sigma, i+1\Vdash \phi_2\\
  &\iff \sigma, i\Vdash\deo \phi_1 \text{ or }\sigma, i\Vdash\deo \phi_2\\
  &\iff \sigma, i\Vdash\deo (\phi_1 \lor \phi_2 )\\
 \end{align*}
 \item
 \begin{align*}
  \sigma, i\Vdash \deo (\phi_1U\phi_2)&\iff \sigma, i+1 \Vdash (\phi_1U\phi_2)\\
  &\iff \exists j\geq i+1\text{ such that }\sigma, j\Vdash \phi_1 \text{ and } \\
&\phantom{\iff \exists}\sigma, k\Vdash \phi_2 \text{ for all } k \text{ with }i+1\leq k<j\\
&\iff \exists j\geq i\text{ such that }\sigma, j+1\Vdash \phi_1 \text{ and } \\
&\phantom{\iff \exists}\sigma, k+1\Vdash \phi_2 \text{ for all } k \text{ with }i\leq k<j\\
&\iff \exists j\geq i\text{ such that }\sigma, j\Vdash \deo \phi_1 \text{ and } \\
&\phantom{\iff \exists}\sigma, k\Vdash \deo\phi_2 \text{ for all } k \text{ with }i\leq k<j\\
&\iff \sigma, i\Vdash (\deo\phi_1)U(\deo\phi_2)
 \end{align*}
 \item
 \begin{align*}
  \sigma, i\Vdash \deo (\phi_1R\phi_2)&\iff \sigma, i+1 \Vdash (\phi_1R\phi_2)\\
  &\iff \sigma, j \Vdash \phi_1\text{ for all }j\geq i+1\text{ or }\\
  &\phantom{\iff}\exists j\geq i+1\text{ such that }\sigma, j\Vdash \phi_1 \land \phi_2 \text{ and } \\
&\phantom{\iff \exists}\sigma, k\Vdash \phi_2 \text{ for all } k \text{ with }i+1\leq k<j\\
&\iff \sigma, j +1\Vdash \phi_1\text{ for all }j\geq i\text{ or }\\
&\phantom{\iff}\exists j\geq i\text{ such that }\sigma, j+1\Vdash \phi_1 \land \phi_2 \text{ and } \\
&\phantom{\iff \exists}\sigma, k+1\Vdash \phi_2 \text{ for all } k \text{ with }i\leq k<j\\
&\iff \sigma, j \Vdash \deo \phi_1\text{ for all }j\geq i\text{ or }\\
&\phantom{\iff} \exists j\geq i\text{ such that }\sigma, j\Vdash \deo \phi_1 \land \phi_2 \text{ and } \\
&\phantom{\iff \exists}\sigma, k\Vdash \deo\phi_2 \text{ for all } k \text{ with }i\leq k<j\\
&\iff \sigma, i\Vdash (\deo\phi_1)R(\deo\phi_2)
 \end{align*}
\end{enumerate}

\end{proof}
\begin{definition}\label{def:modalsimple}
 A formula $\xi\in\mu$-TL is called modal simple if every occurence of $\deo$ is immediately followed by a either another $\deo$-operator, a bound variable, a propositional letter or a fixpoint operator.
\end{definition}
\begin{theorem}\label{thm:muformequivsimple}
 Every formula $\xi\in\mu$-TL is equivalent to a modal simple formula $\xi_s$, if $x$ is guarded in $\xi$ then $x$ is guarded in $\xi_s$ as well.
\end{theorem}
\begin{proof}
For this translation define $S : \omega \times Sub(\xi)$ where $Sub(\xi)$ is the set of subformulas of $\xi$.
\begin{align*}
 S(n,\deo \phi):=S(n+1, \phi)& S(n, l):= \deo^n l \text{ where } l\in Lit(\Psf)\\
  S(n, \phi_1\lor\phi_2) := S(n, \phi_1)\lor S(n, \phi_2)& S(n, \phi_1\land\phi_2) := S(n, \phi_1)\land S(n, \phi_2)\\
  S(n, \phi_1U \phi_2) := (S(n, \phi_1))U(S(n, \phi_2))& S(n, \phi_1R \phi_2) := (S(n, \phi_1))R(S(n, \phi_2))\\
  S(n, \mu x.\phi) := \deo^n\eta x.S(0, \phi)& S(n, \nu x.\phi) := \deo^n\nu x.S(0, \phi)\\
\end{align*}
Now we have to prove that $S(0, \xi) \equiv \xi$. Assume for induction that $S(n,\phi)\equiv \deo^n\phi$ for $\phi<\xi$ strict subformula. Now induction step. Cases:
\begin{itemize}
 \item $S(k,\phi_1\land\phi_2)\equiv \deo^k(\phi_1\land\phi_2)$: We see that $S(k, \phi_1)\equiv\deo^k\phi_1$ and $S(k, \phi_2)\equiv\deo^k\phi_2$ so
 \begin{align*}
  S(k,\phi_1\land\phi_2) &= S(k, \phi_1)\land S(k, \phi_2)\\
                         &\equiv (\deo^k\phi_1)\land ()\deo^k\phi_2)\text{ by IH}\\
                         &\equiv \deo^k(\phi_1\land\phi_2)\text{ by distributivity of $\deo$}
 \end{align*}
 \item $S(k,\phi_1\lor\phi_2)\equiv \deo^k(\phi_1\lor\phi_2)$: We see that $S(k, \phi_1)\equiv\deo^k\phi_1$ and $S(k, \phi_2)\equiv\deo^k\phi_2$ so
 \begin{align*}
  S(k,\phi_1\lor\phi_2) &= S(k, \phi_1)\lor S(k, \phi_2)\\
                         &\equiv (\deo^k\phi_1)\lor (\deo^k\phi_2)\text{ by IH}\\
                         &\equiv \deo^k(\phi_1\lor\phi_2)\text{ by distributivity of $\deo$}
 \end{align*}
 \item $S(k,\phi_1U\phi_2)\equiv \deo^k(\phi_1U\phi_2)$: We see that $S(k, \phi_1)\equiv\deo^k\phi_1$ and $S(k, \phi_2)\equiv\deo^k\phi_2$ so
 \begin{align*}
  S(k,\phi_1U\phi_2) &= S(k, \phi_1)US(k, \phi_2)\\
                         &\equiv (\deo^k\phi_1)U (\deo^k\phi_2)\text{ by IH}\\
                         &\equiv \deo^k(\phi_1U\phi_2)\text{ by distributivity of $\deo$}
 \end{align*}
 \item $S(k,\phi_1R\phi_2)\equiv \deo^k(\phi_1R\phi_2)$: We see that $S(k, \phi_1)\equiv\deo^k\phi_1$ and $S(k, \phi_2)\equiv\deo^k\phi_2$ so
 \begin{align*}
  S(k,\phi_1R\phi_2) &= S(k, \phi_1)R S(k, \phi_2)\\
                         &\equiv (\deo^k\phi_1)R (\deo^k\phi_2)\text{ by IH}\\
                         &\equiv \deo^k(\phi_1R\phi_2)\text{ by distributivity of $\deo$}
 \end{align*}
\end{itemize}
\textcolor{red}{Oei, misschien moet ik hier gewoon zeggen dat elke stap volgt uit distributiviteit (moeilijk woord zeg) van $\deo$ maar het is altijd moeilijk in te schatten wat je uit moet schrijven en wat niet..., moet nog even uitgeschreven worden}.
It is easy to see that $S(0, \xi)$ is simple since the only cases where there are $\deo$-operators are in front of fixpoint formulas of in front of literals.
\end{proof}
\begin{definition}\label{def:guardedparform}\cite[Definition 6.62]{venema2024modalmucalculus}
 A path $\pi=v_0v_1v_2\dots v_n$ through a parity formula is \textbf{unguarded} if $n\geq 1$, $v_0, v_n\in \Dom(\Omega)$ while there is no $i$ with $0<i\leq n$, such that $v_i$ is a modal node. A parity formula is \textbf{guarded} if it has no unguarded cycles, and \textbf{strongly guarded} if it has no unguarded paths.
\end{definition}

\begin{definition}
 A parity formula $\mathbb{G}$ is called simple if we have
 \[
  L(v) = \deo \implies L(v')\in \mathtt{At}(\Psf) \text{ or } v'\in \Dom(\Omega) \text{ or }L(v')=\deo
  \]

\end{definition}
\begin{theorem}\label{thm:mutlformURfree}
 Every $\mu$-TL formula $\phi$ can be written without the use of $U$ and $R$ using the following equivalences
 \[
  \phi U\psi \equiv \mu x. (\psi \lor (\phi \land \deo x)) \text { and } \phi R\psi \equiv \nu x. ((\phi\land \psi) \lor (\phi \land \deo x))
 \]
\end{theorem}
\begin{proof}
First $\phi U\psi \equiv \mu x. (\psi \lor (\phi \land \deo x))$ and first the case $\implies$. Now suppose $\exists$ has a winning strategy for $(\phi U\psi, 0)$. Then she chooses $\phi \land \deo \phi U\psi$ untill at some point she chooses $\psi$ since she can only unfold finitely often. At all the intermediate points $\forall$ has no strategy of spoiling. He always chooses $\deo \phi U\psi$. Now call the final point $s$, here $\exists$ chooses $\psi$ and then we know that $\psi$ is true. $\exists$ now has a winning strategy for $\mu x. (\psi \lor (\phi \land \deo x)), 0)$ she always chooses $(\phi \land \deo x)$ upto $s$ where she chooses $\psi$ and she wins since $\sigma, s \Vdash \psi$. Now $\forall$ has to choose $\deo x$ since $\phi$ is true at every point upto $s$. So $\exists$ wins. The proof of the other direction is analogous. For the $R$ case either $\exists$ never chooses $(\phi\land \psi)$ this gives the strategy for $\nu$ to also always choose $(\phi\land \psi)$ or the same argument for an $s$ applies. For the other direction the proof is analogous.
\end{proof}
\begin{theorem}
 Let $\phi$ be a $\mu$-TL formula. Without loss of generality assume $\phi$ is guarded, simple and $U/R$ free. Let $\mathbb{H}_\phi$ the parity formula from theorem \ref{thm:muformtoparform}. This $\mathbb{H}_\phi$ is strongly guarded and simple
\end{theorem}
\begin{proof}
 I will prove this with contradiction. Assume $\mathbb{H}_\phi$ is not strongly guarded. Then there exists a unguarded path $\pi=v_0v_1v_2\dots v_n$ with $n\geq 1$ and $v_0, v_n\in \Dom(\Omega)$. From the construction of $\mathbb{H}_\phi$ we know that the states in $\mathbb{H}_\phi$ correspond to the bound variables in $\phi$ so $v_n$ corresponds to a bounded variable in $\phi$. Since $\phi$ is guarded we know that every bound variable is in the scope of a $\deo$-operator and because $\phi$ is simple we know that this $\deo$-operator is immediately in front of the bound variable. Therefore we know that $v_{n-1}$ corresponds to a $\deo$-operator so $v_{n-1}$ is a modal node which contradicts the fact that $\pi$ is an unguarded path. \\
 $\mathbb{G}$ is simple since we know that $V$ is based on the subformula graph and $\phi$ is simple. The only states are the nodes that correspond with the bound variables and since every bound variable is preceded by a modal node we know that $\mathbb{G}$ is simple.
\end{proof}

\section{From $\mu$-TL to parity formulas}
\begin{theorem}\label{thm:muformtoparform}\cite[Theorem 6.35]{venema2024modalmucalculus}
 There is an algorithm that constructs, for a clean formula $\xi\in \mu$-ML$(\Psf)$, an equivalent parity formula $\mathbb{H}_\xi = (V_\xi, E_\xi, L_\xi, \Omega_\xi, \xi)$, based on the subformula graph of $\xi$
\end{theorem}
\textbf{Discussion}: Theorem 6.35 from \cite{venema2024modalmucalculus} works with $\mu$-ML formulas. Now we can adjust this definition to work with $\mu$-TL formulas that are $U/R$ free and transform this to linear parity formulas. The proof will follow the proof of Venema but has to be worked out in a later stadium (also discussing how much I can refer to Venema and how much I have to show myself in the thesis).

\section{Translation of parity formulas to alternating parity automata}
\begin{definition}\cite{demri2016temporal}
Let $\mathbb{G}$ a parity formula over $\Psf$. Define the \textbf{language} of a parity formula as:
 \[
  \Mod_P(\mathbb{G}) := \{\sigma \in (\PP(P))^\omega\mid \sigma, 0 \Vdash \mathbb{G}\}
 \]
\end{definition}

\textcolor{red}{Ik weet nog niet zo goed hoe ik deze strategy tree precies ga gebruiken, kan ook kijken naar de gewone strategy tree zoals jij hem definieert maar dan met de opmerking dat je alleen maar naar de laatste toestand hoeft te kijken. Alleen op deze manier kan je hem wel heel makkelijk een op een vertalen naar een DAG (run)}
\begin{proposition}
 Knowledge about strategy tree generalize to knowledge about positional strategy tree. In particular the fact that a branch in the tree is a match in the game.
\end{proposition}
\begin{theorem}\label{thm:parformtoapa}
 There is an algorithm that constructs, for a strongly guarded and simple parity formula $G$ over $\Psf$ into an equivalent APA $\A$ such that $\Mod_\Psf(\mathbb{G}) = \LL(\A)$
\end{theorem}
\begin{proof}
 Let $\mathbb{G}=(V, E, L, \Omega_\mathbb{G}, v_I)$ a strongly guarded and simple parity formula. We will define $\A=(\Sigma, Q, q_0, \delta, \Omega')$ as follows:
 \begin{itemize}
  \item Set the alphabet $\Sigma := \PP(\Psf)$
  \item Define the the set of states $Q$ as $Q:= V$
  \item Define $q_0:=v_I$
 \end{itemize}
 To define the transition function $\delta$ let $q\in Q$ and $a\in\Sigma (\PP(\Psf))$ and set:
\begin{align*}
   \delta(v, a) &:= \top        \text{ if } L(v)=p\text{ and } p\in a & \delta(v,a)&:= \delta(v_1,a)\land\delta(v_2, a)\text{ if } L(v) = \land\\
   \delta(v, a) &:= \bot        \text{ if } L(v)=p\text{ and } p\notin a&\delta(v,a)&:= \delta(v_1,a)\lor\delta(v_2, a)\text{ if } L(v) = \lor\\
   \delta(v, a) &:= \top \text{ if } L(v)=\overline{p}\text{ and } p\notin a&\delta(v, a)&:= v'\text{ if } L(v) = \deo\\
   \delta(v, a) &:= \bot \text{ if } L(v)=\overline{p}\text{ and } p\in a &\delta(v, a) &:= \top        \text{ if } L(v)=\top\\
   \delta(v, a) &:= \delta(v', a)\text{ if } L(v)=\epsilon&\delta(v, a) &:= \bot        \text{ if } L(v)=\bot
  \end{align*}
  For the priority map define:
  \[
  \Omega_\mathbb{A}(v) = \begin{cases}
                \Omega(v) &\text{if }v\in \Dom(\Omega)\\
                0&\text{else}
               \end{cases}
  \]
  \textbf{Observation:} if $(v, l)$ is a dead end for $\forall$ then $\delta(v, a_l)=\top$\\
\textbf{Claim:} $\Mod_\Psf(\mathbb{G}) = \LL(\A)$\\
\textit{Proof}: First $\Mod_\Psf(\mathbb{G}) \subseteq \LL(\A)$. Suppose $w=a_0a_1a_2\dots \in \Mod_\Psf(\mathbb{G})$ then we know that $(v_I, 0)$ is a winning position for $\exists$. This means there is a winning strategy $f$ for $\exists$ and since $\mathbb{G}$ is a parity game we even know that there exists a positional winning strategy $f$ for $\exists$ in $\mathcal{B}@(v_I,0)$. Now take the strategy graph $\mathbb{B}_a^f$, we are going to construct a succesfull run in $\A$ based on this positional strategy tree. Recall that a run in $\A$ is a DAG $\rho=(V_\rho,E_\rho)$ where $\rho$ satisfies four conditions R1-R4. Define the nodes
\[
 V_\rho = \{(v_I, 0)\}\cup\bigcup_{i\geq 1}\left\{(v, i)\mid \exists v'\in V, L(v')=\deo, \{v\}=E[v'], (v', i-1)\in \mathbb{B}_{(v_I,0)}^f \right\}
\]
and the edges
\[
 E_\rho = \bigcup_{i\geq 0}\left\{((v, i), (v', i+1))\in V_\rho\times V_\rho\mid (v', i+1)\in\mathbb{B}_{(v,i)}^f \right\}
\]
Now we have to check that this is indeed a run. Satisfaction of R1-R3 is an easy verification since $\mathbb{G}_{(v_I,0)}^f$ is already a graph. \textcolor{red}{Werk dit even netjes uit}. For the satisfaction of R4 we need to dive in the definition of $\delta$. \\
\textcolor{red}{Naam van $E_q$ moet anders!}
\textbf{Claim:} For every $(q, l)\in V_\rho$ we have $E_q=\{q'\mid ((q, l), (q', l+1))\in E\} \models \delta(q, a_l)$\\
\textit{Proof}: We are going to look at the level $\mathbb{B}_{(q, l)}^f[l]$ of the game graph. Since $\mathbb{G}$ is (strongly) guarded we know that this level is a DAG where the leafs (hebben we leafs?) correspond to literal or modal nodes. Now I am going to proof this with an iteration on the topological ordering of $\mathbb{B}_{(q, l)}^f[l]$, call this ordering $w_0w_1\dots w_n$. \textcolor{red}{Hoe nu inductiehypothese benoemen??}.\\
\textbf{Base case: $w_n$}. We know that $w_n$ has no outgoing edges, so it must correspond to a dead end. Since $f$ is a winning strategy for $\exists$ we know that $\forall$ should be at move. Therefore we now that
$\delta(w_n,a_l)=\top$ so clearly $E_q\models \delta(w_n, a_l)$. \\
\textbf{Induction step:} Take $w_i, i<n$. Here we distinguish four different cases:
\begin{itemize}
 \item[Case:] \(E[w_i]=0\) In this case $w_i$ is a dead end for $\forall$ in the game graph so we know that $\delta(w_i, a_l)=\top$.
 \item[Case:] $L(w_i)=\land$. In this case we have $\delta(w_i, a_l) = \delta(w_j, a_l) \land \delta(w_k, a_l)$, since we have a topological ordering we know that $i<j$ and $i<k$ so per the induction (iteration) hypothesis we know that $E_q\models \delta(w_j, a_l)$ and $E_q\models\delta(w_k, a_l)$ so that means $E_q\models \delta(w_i, a_l)$.
 \item[Case:] $L(w_i)=\lor$. We see that $\delta(w_i, a_l)= \delta(v_1, a_l)\lor \delta(v_2, a_l)$ Since $\mathbb{B}_{(q, l)}^f[l]$ is a level of the strategy graph we know that $E_f[w_i] = f(w_i)=v_m, m\in {1, 2}$. Since we have a topological ordering we know that $v_m=w_j$ for a $j>i$. Therefore we know that $E_q\models \delta(v_j, a_l)$ so also $E_q\models \delta(w_i, a_l)$ \textcolor{red}{Oei de indices zijn hier wel heel erg verwarrend}.
 \item[Case:] $L(w_i)=\deo$. Here we see that $\delta(w_i, a_l) = v'$ where $E[w_i]=\{v'\}$. Since $(w_i, l)\in \mathbb{B}_{(v_I, 0)}^f$ we know that $(v', l+1)\in V_\rho$. Also since $(w_i, l)\in \mathbb{B}_{(q, l)}^f$ we see that $(v', l+1)\in \mathbb{B}_{(q, l)}^f$ so therefore $v'\in E_q$ so we see that $E_q\models\delta(w_i, a_l)$.
\end{itemize}
Now we have iterated over the the whole level $\mathbb{B}_{(q, l)}^f[l]$ so we see that $E_q \models \delta(q, a_l)$.
\qed.\\

% I will prove this via induction on the complexity on $\delta(q,a_l)$. Since $\mathbb{G}$ is guarded there are no loops so we cannot loop infinitely over $\delta$ and get a fixpoint. \textcolor{red}{IH goed formuleren, iets met $\delta$ en $E[v]$, en hier heb je dus $v\leq q$ eigenlijk in de parityformuleboom}: Induction hypo: for every $v\in \mathbb{T}^f_{(q, l)}$ we have $E_v\models (v, a_l)$ The base cases are: $L(v)=\deo$ or $L(v)\in\mathtt{At}(\Psf)$. Now if $L(v
% )=\deo$ we see that $\delta(v, a_l)= v'$. Also we see that $(v', l+1)\in V$ since $L(v)=\deo$, $(v, l)\in \mathbb{T}_{(v_I, 0)}$ and $\{v'\} =E[v]$. Secondly $((q,l), (v', l+1))\in E$ since $(v, l)\leq (q,l)$ and $E[v]=v'$ so we see that $(v', l+1)\in \mathbb{T}_{(q, l)}^f$ since $v\in \mathbb{T}^f_{(q, l)}$. So $\{q'\mid ((q, l), (q', l+1))\in E\} \models \delta(v, a_l)$. If $L(v)\in \mathtt{At}(\Psf)$ we know that this should be a dead end where it is $\forall$'s turn since $f$ is a winning strategy. So $\delta(v, a_l)=\top$ so every set satisfies this transition function. \\
% Now for induction assume $\delta(v, a_l) = \delta(v_1, a_l)\land \delta(v_2,a_l)$, we see that both $v_1$ and $v_2$ are in $\mathbb{T}^f_{(q, l)}$ so we can apply the induction hypothesis. If we have $\delta(v, a_l) = \delta(v_1, a_l)\lor \delta(v_2,a_l)$ choose the node that is in $\mathbb{T}^f_{(q, l)}$ and apply the induction hypothesis. If we have $\delta(v, a_l)=\delta(v', a_l)$ we have $v'\in \mathbb{T}^f_{(q, l)}$ so we can apply the induction hypothesis. Now we see that $E_q\models \delta(q, a_l)$.\qed

Now we have to proof that $\rho=(V,E)$ is indeed an accepting run. That means that for every infinite path $w$ through $\rho$ we have that $\max\{\Omega(q)\mid q\in \inf(u)\}$ is even. To to that we will prove a correspondence between infinite paths in $\rho$ and infinite matches in $\mathcal{E}(\mathbb{G}, \sigma)@(v_I, 0)$\\
\textbf{Claim:} Every infinite path $u$ in $\rho$ corresponds \textcolor{red}{Wat bedoel je met correspondence precies} to a $f$-guided infinite match $\pi$ in $\mathcal{E}(\mathbb{G}, \sigma)@(v_I, 0)$ where we have $\{\Omega_\mathbb{A}(v)\mid v\in u\}\cup\{0\}= \{\Omega_{\mathcal{E}}(p)\mid p\in \pi\}$, \textcolor{red}{En benoem eerste state in $\pi$}. \\
\textit{Proof:}\\
An infinite path $u$ consists of states $q_0q_1q_2\dots$ where $q_0 = v_I$ and we have $q_{i+1}\in E_{q_i}$ for every $i$. Now construct the infinite match $\pi = (q_0,0)\dots (q_1,1)\dots (q_2, 2)\dots (q_3,3)\dots$, we see that this is an infinite match since when $q_{i+1}\in E_{q_i}$ we know that $q_{i+1}\in\mathbb{B}_{(q,i)}^f$ so there is a finite $f$-guided path between $(q_i,i)$ and $(q_{i+1},i+1)$. Now we have to show that $\{\Omega_\mathbb{A}(v)\mid v\in u\}\cup\{0\}= \{\Omega_{\mathcal{E}}(p)\mid p\in \pi\}$. First observe that $\Omega_\mathcal{E}((q_i,i)) = \Omega_\A (q_i)$ since they both correspond to \(
  \Omega_\mathbb{A}(v) = \begin{cases}
                \Omega(v) &\text{if }v\in \Dom(\Omega)\\
                0&\text{else}
               \end{cases}
  \).
  Now take a node $(\pi_j, i)$ in $\pi$ inbetween $(q_i, i)$ and $(q_{i+1}, i+1)$. My claim is that $\Omega_{\mathcal{E}}(\pi_j, i) = 0$. Suppose not, then we would have $v\in \Dom (\Omega_\mathbb{G})$. But since $\mathbb{G}$ is strongly guarded and simple we know that $v$ should be in the scope of a modal operator and that the modal operator is directly preceding this node. That would mean that $L(\pi_{j-1})=\deo$ and that would mean that $\pi_j$ should be in level $i+1$ which contradicts the fact that $(\pi_j, i)$ is chosen inbetween $(q_i, i)$ and $(q_{i+1}, i+1)$. That proves that $\{\Omega_\mathbb{A}(v)\mid v\in u\}\cup\{0\}= \{\Omega_{\mathcal{E}}(\pi_i)\mid \pi_i\in \pi\}$ since the only non-zero nodes in $\pi$ corresponds to states in $\mathbb{G}$.
\qed \\ Since every infinite path in $\rho$ corresponds to an infinite $f$-guided match and since $(v_I, 0)$ is a winning position we know that $\max\{\Omega_\mathcal{E}(\pi_i)\mid \pi_i\in\inf(\pi)\}$ is even and since $\{\Omega_\mathbb{A}(v)\mid v\in u\}\cup\{0\}= \{\Omega_{\mathcal{E}}(p)\mid p\in \pi\}$ we also know that $\max\{\Omega(q)\mid q\in \inf(u)\}$ is even. So $\rho$ is an accepting run for $w$ in $\A$.  So now we know that if $w\in \Mod_\Psf(\mathbb{G})$ then $w\in \LL(\A)$\\
%
% That means that every infinite path in $\rho$ is accepting. We are working with the parity condition so it means that for every $w\in Q^\omega$ that defines a path through $\rho$ we have $\max\{\Omega(q)\mid q\in w\}$ is even. We see that an infinite path through the strategy tree corresponds to an infinite word through $\rho$. We also know that since $\mathbb{G}$ is guarded and simple that every state is immediately preceded by a modal node. So that means that if we have an infinite word through $\rho$ that $\Omega'(\inf(w_\rho))=\Omega'(\inf(\pi))\cup\{0\}$ where $\pi$ is the infinite match that corresponds to $w_\rho$ \textcolor{red}{Uitleggen correspondentie??}. And since $\max(\Omega'(\inf(\pi)))$ is even we know that $\max(\Omega'(\inf(w_\rho)))$ is even as well so that means that $\rho$ is accepting so that $w\in \LL(\A)$\\


Now the other direction $\Mod_\Psf(\mathbb{G}) \supseteq \LL(\A)$: Suppose we have $w\in \\(\A)$ then there exists an accepting run $\rho=(V_\rho, E_\rho)$ on $w$, without loss of generality assume this run is minimal \textcolor{red}{Hier dan verwijzen naar stelling?}. We have to create a winning strategy for $\exists$ in $\mathcal{B}@(v_I,0)$. We are going to create a non-positional strategy $f$ such every finite path in the strategy tre $\mathcal{B}$ ends in a dead end for $\forall$ and every infinite path has an even maximum parity. \\
First we are going to define $f$, to define $f$ we are inductively going to build op the strategy tree $\mathbb{T}_{(v_I, 0)}^f$. \\
For every $(q, l)\in V_\rho$ we are going to prove with induction on the depth of $v$ that for every $(v,l)\in \mathbb{T}^f_{(q,l)}[l]$ we have $E_q\models \delta(v, a_l)$, and that in ever dead end it is $\forall$'s turn to play. We define $E_q:=\{q'\mid ((q, l), (q', l+1))\in E\} \models \delta(q, a_l)$ as before. \\
Induction hypothesis: for every every $\pi$ with depth $k$ we have if $last(\pi)=(v,l)$ $E_q\models \delta(v, a_l)$.\\
Base case: $k=0$ since $\mathbb{T}^f_{(q,l)}[l]$ is a tree with one root we now that the only $\pi$ with depth $0$ has $last(\pi)=(q,l)$. Since $\rho$ is a run we know by R4 that $E_q\models \delta(q, a_l)$. If $L(q)=\lor$ we have to choose between $v_1$ and $v_2$. We know that $\delta(q, a_l)=\delta(v_1,a_l)\lor \delta(v_2, a_l)$, so $E_q\models \delta(v_1, a_l)$ or $E_q\models\delta(v_2, a_l)$ since we know that $\rho$ is minimal either we know that exclusively one is true, in that case choose $f(\pi) = v_i$ where $v_i$ is true or that $||\delta(v_1, a_l)||=||\delta(v_2, a_l)||$, in that case choose $v_1$.\\
Inductive case: $k>0$. Take a $\pi\in \mathbb{T}^f_{(q, l)}[l]$ with $last(\pi)=(v, l)$. First look at the parent of $\pi$, call this $\pi_p$ (it has depth $k-1$ so we can apply the induction hypothesis) if $last(\pi_p)=(v', l)$ we know that $E_q\models\delta(v, a_l)$. Now cases:\\
Case $L(v')=\lor$, in this case we know $\delta(v, a_l) =\delta(v_1, a_l)\lor \delta(v_2, a_l)$ (IH) so exactly one of $E_q\models\delta(v_1, a_l)$ and $E_q\models\delta(v_2, a_l)$ is true of the \textcolor{red}{valuations} are the same. We know that $f(\pi_p)=v_i$ where $\delta(v_i, a_l)$ is true so that means $v=v_i$ so $E_q\models\delta(v, a_l)$\\
Case $L(v') = \land$, in this case we know $\delta(v, a_l) =\delta(v_1, a_l)\land \delta(v_2, a_l)$ (IH) where $v=v_1\lor v=v_2$ so $E_q\models\delta(v, a_l)$\\
The case $L(v')=\deo$ is not possible since then $\pi$ would be in the next level. This is a leaf that we have to connect to another level.\\
Now if $L(v)=\lor$ we have to make a choice for $f(\pi)$, make the same choice as above in the base case.\\
If $L(v)\in Lit(\Psf)$ we know that $\delta(v, a_l)\in \{\top, \bot\}$ since $E_q\models \delta(v, a_l)$ we know that $\delta(v, a_l)=\top$. Then that means that this is a dead end where $\forall$ is at move.\\
Conclusion: For every $\pi \in \mathbb{T}_{(q, l)}^f[l]$ if $last(\pi)=(v,l$ we have $E_q\models \delta(v, l)$ and every dead end in $\mathbb{T}_{(q, l)}^f[l]$ belongs to $B_\forall$.

So now we have created a lot of levels in the strategy tree but we have to make sure that these connect to each other in a well defined strategy tree. We have to check for every leaf in every $\mathbb{T}_{(q, l)}^f[l]$ that corresponds to $L(v)=\deo$ that there is a strategy tree $\mathbb{T}_{(v', l)}^f[l+1]$.\\
\Claim: we can connect all strategy trees to form a well defined strategy tree for $\mathcal{E}(\mathbb{G}, \sigma)@(v_I, 0)$.\\
\Proof: Fist of all we know that $(q, 0)= (q_0, 0) = (v_I, 0)$ so we have $\mathbb{T}_{(v_I, 0)}^f[0]$. Now take an arbitrary $\mathbb{T}_{(q, l)}^f[l]$ and an arbitrary leaf $\pi$ with $last(\pi)=(v, l)$ in $\mathbb{T}_{(q, l)}^f[l]$. We know that $L(v)=\deo$, $E[v]=\{v'\}$ and $\delta(v, a_l)=v'$. By the above claim we know that $E_q\models \delta(v, a_l)=v'$ so $v'\in E_q$ so therefore we know that $(v', l+1)\in V_\rho$. That means that we can take $\mathbb{T}_{(v', l+1)}^f[l+1]$ from the above statement so this leaf is connected.\\
Conclusion: we can connect every leaf that is not a dead ind in every level to another level that we created so $\mathbb{T}_{(q, l)}^f$ is well-defined.\\
Now we have to make sure that this strategy tree indeed defines a winning strategy for $\exists$ in $\mathcal{E}(\mathbb{G}, \sigma)@(v_I, 0)$. This comes in two parts: first every finite match ends in a dead end for $\forall$ and for every infinite match we have that the maximum parity is even.\\
\Claim: Every finite match in $\mathbb{T}_{(v_I,0)}^f$ ends in a dead end for $\forall$\\
\Proof: We see that every finite match in $\mathbb{T}_{(v_I,0)}^f$ ends in a leaf $\pi$ in some level $\mathbb{T}_{(q,l)}^f[l]$, if $last(\pi) = (v,l)$ we know that $L(v)\in \Lit(\Psf)$ so from the induction claim we know that this is indeed a dead end where it is $\forall$'s turn.\\
\Claim: For every inifinite match $\pi$ in $\mathbb{T}_{(v_I,0)}^f$ we have $\max(\inf_{\Omega_\mathcal{E}} (\pi))$ is even.\\
\Proof: First we are going to correspond matches to paths in the run. \\
\textbf{Claim:} Every infinite $f$-guided infinite match $\pi$ in $\mathcal{E}(\mathbb{G}, \sigma)@(v_I, 0)$ corresponds to an infinte path $u$ in $\rho$ where we have $\{\Omega_\mathbb{A}(v)\mid v\in u\}\cup\{0\} = \{\Omega_{\mathcal{E}}(p)\mid p\in \pi\}$, \textcolor{red}{En benoem eerste state in $\pi$}. \\
\Proof: Take a $f$-guided match $\pi=(v_0, l_0)(v_1, l_1)(v_2, l_2)\dots$ in $\mathcal{E}(\mathbb{G}, \sigma)@(v_I, 0)$. Now mark $q_0=(v_0, l_0)$ and mark $q_i$ as the $i$'th node $(v_j, l_j$ in $\pi$ where $L(v_{j-1})=\deo$. Construct the infinite path $u=q_0q_1q_2\dots$ in $\rho$.\\
\Claim: $u$ is an infinite path in $\rho$.\\
\Proof: We have to show that for every $q_i=(v_j, l_j)$ we have $q_{i+1}=(v_{j'}, l_{j'})$ and $v_{j'}\in E_{q_i}$. We know that $(v_{j'-1}, l_{j'-1})=\last(\pi)$ for a $\pi\in \mathbb{T}_{(v_j,l_j)}^f$ and $L(v_{j'-1})=\deo$ so $\delta(v, a_l)=v_{j'}$. From the induction claim we have $E_q\models\delta(v_{j'-1})$ so $v_{j'}\in E_{q_i}$ which proofs this claim. \qed\\
Now we easily see that for $q_i=(v_j, l_j)$ we have  $\Omega_\mathcal{E}((v_j, l_j)) = \Omega_\A (v_j)$ since they both correspond to \(
  \Omega_\mathbb{A}(v_j) = \begin{cases}
                \Omega_{\mathbb{G}}(v_j) &\text{if }v\in \Dom(\Omega)\\
                0&\text{else}
               \end{cases}
\). Now take a node $(v_j, l_j)$ in $\pi$ strictlt inbetween $q_i$ and $q_{i+1}$. My claim is that $\Omega_{\mathcal{E}}(v_j, l_j) = 0$. Suppose not, then we would have $v\in \Dom (\Omega_\mathbb{G})$. But since $\mathbb{G}$ is strongly guarded and simple we know that $v$ should be in the scope of a modal operator and that the modal operator is directly preceding this node. That would mean that $L(v_{j-1})=\deo$ and that would mean that $(v_j, l_j)$ should be in level $l_j+1$ which contradicts the fact that $q_{i+1}$ is chosen as the first node after $q_i$ that follows a modal node. That proves that $\{\Omega_\mathbb{A}(v)\mid v\in u\}\cup\{0\}= \{\Omega_{\mathcal{E}}(\pi_i)\mid \pi_i\in \pi\}$ since the only non-zero nodes in $\pi$ corresponds to states in $\mathbb{G}$.\qed\\
Now we know that for every infinite $f$-guided infinite match $\pi$ there is a infinite path $u$ in$\rho$. Since $\rho$ is an accepting run we know that $\max(\inf_{\Omega_\A}(u))$ is even and since $\{\Omega_\mathbb{A}(v)\mid v\in u\}\cup\{0\} = \{\Omega_{\mathcal{E}}(p)\mid p\in \pi\}$ we then also know that $\max(\inf_{\Omega_\mathcal{E}}(\pi))$ is even, which proofs the claim.\qed\\

Now we know that $f$ is a winning strategy for $\exists$ in $\mathcal{E}(\mathbb{G}, \sigma)@(v_I, 0)$ so we know that $w\in \Mod_\Psf(\mathbb{G})$) which proofs that indeed

% Now take a node $(v_j, l_j)$ in $\pi$ inbetween a $q_i$ and $q_{i+1}$.
%
% to do this we will make a strategy tree. Take the game tree, we have to prune it so that we have a strategy for $\exists$ essentially we have to make a choice for every $v$ where $L(v)=\lor$. Inductively from the top go downward. If there is a place where $\delta(v, a_l)=\delta(v_1, a_l)\lor \delta(v_2, a_l)$ where $v\leq q$ so if $E_q\models \delta(q, a_l)$ it should either models $\delta(v_1, a_l)$ or $\delta(v_2, a_l)$. Now set $f((v,l))=v_i$ where $E_q\models \delta(v_i, a_l)$. What we now see is that the levels in the run correspond to states $(v', i)$ where we have $L(v)=\deo$ and $(v, i-1)\in \mathbb{T}_{(v_I,0)}^f$. Now we have to proof that this gives a winning strategy tree. We have to check that every finite match ends in a dead end for $\forall$ and that every infinite match hase even infinite priority. For the finite case we look at every level. So we have $E_q\models \delta(q, a_l)$ now we look inductively at this $\delta$ function. For every $(v, i)\in \mathbb{T}^{temp}_{(q, i)}$ If it has the form $\delta(v, a_l)=\delta(v_1, a_l)\land \delta(v_2, a_l)$ we know that both will be true. And if it has the form of $\delta(v, a_l)=\delta(v_1, a_l)\lor \delta(v_2, a_l)$ that at least one of them is true and that is the path added to $\mathbb{T}$. Now if we have $\delta(v, a_l) = \top$ we know that this corresponds to a dead end for $\forall$, $\bot$ is not allowed. Now if we have $\delta(v, a_l)=v'$ we are in a next level. From there on we will check for finiteness and if this path does not stop somewhere it is infinte. We also see that infinite matches correspond to infinite paths in $\rho$. Following the reasoning above we see that the max parity is even iff. So now we have shown that
\[
 \Mod_\Psf(\mathbb{G}) = \LL(\A)
\]
so $\A$ is equivalent to $\mathbb{G}$.
% \textcolor{red}{Allemaal nog iets netter, dat is echt best moeilijk zeg}
\end{proof}

\section{Translation of APA to NPA}
\begin{definition}\label{def:APAtoNPA}
 Given an (ATPA) $\mathbb{A}=(\Sigma, Q, q_0,\delta, \Omega)$ define the (equivalent) NPA $\mathbb{A}'=(\Sigma, Q', q_0',\delta', \Omega')$ as follows:
 \begin{itemize}
  \item $Q' = \mathcal{P}(Q\times \Omega(Q)) \times \{0, 1\}$
  \item $q_0' \subseteq Q' = \begin{cases}
               \{(\{(q_0, \Omega(q_0)\}, 0)\} &= \text{if } \Omega(q_0) \text{ is even}\\
               \{(\{(q_0, \Omega(q_0))\}, 0), (\{(q_0, 0)\}, 1)\}&= \text{else}
               \end{cases}$
  \item $\delta' : Q'\times \Sigma \to \mathcal{P}(Q')$ The transition function $\delta'((X, p), a)$ has two cases:\\
  \textbf{Final} When for all $(q, p)\in X$ it holds that $p$ is even then all paths have visited an even parity (and that is also maximal) so we reset. Now set \\
  \begin{multline*}
   \delta'((X, p), a) = \biggl\{ (Y, 0) |Z\subseteq Q, \text{ such that for every } (q, p_q)\in X, \text{ holds } Z\models \delta(q, a)\land \\
   \{q|(q, p_q)\in Y\}=Z
   \land \text{ for every } (q, p_q)\in Y \text{ holds }
   p_q=\Omega(q)\biggr\}\cup\\
   \Biggl\{
   (Y, 1) |Z\subseteq Q, \text{ such that for every } (q, p_q)\in X, \text{ holds } Z\models \delta(q, a)\land \\
   \left\{q|(q, p_q)\in Y\right\}=Z
   \land \text{ for every } (q, p_q)\in Y \text{ holds }
   p_q=\begin{cases}0&\text {if }\Omega(q) \text{ odd } \\ \Omega(q)&\text{else}\end{cases}
   \Biggr\}
  \end{multline*}
  \textbf{Non-final} When for some $(q, p)\in X$ it holds that $p$ is odd then not all paths have visited a maximum even parity so we go further. Now set\\
\begin{multline*}
   \delta'((X, p), a) = \Biggl\{ (Y, 0) |Z\subseteq Q, \text{ such that for every } (q, p_q)\in X, \text{ holds } Z\models \delta(q, a)\land \\
   \{q|(q, p_q)\in Y\}=Z
   \land \text{ for every } (q, p_q)\in Y \text{ holds } \\
   p_q=\min\{\max(p_{q'}, \Omega(q))|\exists (q', p_{q'})\in X, \exists Z_{q'}\subseteq Z, \text{ With $Z_{q'}$ minimal }, q\in Z_{q'}\land Z_{q'}\models \delta(q', a)\}\Biggr\}\cup\\
   \Biggl\{ (Y, 1) |Z\subseteq Q, \text{ such that for every } (q, p_q)\in X, \text{ holds } Z\models \delta(q, a)\land \\
   \{q|(q, p_q)\in Y\}=Z\text{ for every } (q, p_q)\in Y \text{ holds } \\
   p_q=\min\Biggl\{\max\Biggl(p_{q'}, \begin{cases}0&\text {if }\Omega(q) \text{ odd } \\
   \Omega(q)&\text{else}\end{cases}\Biggr)|\exists (q', p_{q'})\in X, \exists Z_{q'}\subseteq Z\\
   \text{ With $Z_{q'}$ minimal }, q\in Z_{q'}\land Z_{q'}\models \delta(q', a)\Biggr\}\Biggr\}
  \end{multline*}
  \textcolor{red}{ do define minimal (easy but has to be done)}
  \item Now for $\Omega' : Q'\to \mathbb{N}$ define
  \[
   \Omega'(X, p) =\begin{cases}
                   3 &\text{if } p = 1\\
                   2 &\text{if for all } (q, p_q)\in X \text{ holds } p_q \text{ even }\\
                   1&\text{else}
                  \end{cases}
  \]

 \end{itemize}
\end{definition}
\begin{theorem}\label{thm:APAtoNPA}
The NPA indeed recognizes the same language as the APTA in other words: $\mathcal{L}(\mathbb{A})=\mathcal{L}(\mathbb{A}')$
\end{theorem}

\begin{proof}
First we prove \(\mathcal{L}(\mathbb{A})\subseteq \mathcal{L}(\mathbb{A}')\)\\
Suppose we have a minimal run $\rho=(V, E)$ on $w$ in the APTA $\mathbb{A}$ now we have to construct a run $\rho'=q'_0q'_1q'_2\dots$ in the NPA $\mathbb{A}$. Define
\[q'_0 = \begin{cases}
               (\{(q_0, \Omega(q_0))\}, 0) &\text{ if } \Omega(q_0)\leq \min\max \Omega \text{ of all infinite paths in }\rho\\
               (\{(q_0, 0)\}, 1) &\text{ if } \Omega(q_0)> \min\max \Omega \text{ of all infinite paths in }\rho\\
              \end{cases}
\]
For $i>0$ we have two cases. If the previous state was final (that means that every state had a even parity attached to it):
\[
 q'_i = \begin{cases}
            (\{(q, \Omega(q))|(q, i)\in V\}, 0) \text{ if all }\Omega(q) \leq \min\max \Omega \text{ of all infinite paths in }\rho\\
            \left(\left\{(q, p_q)|(q, i)\in V, \text{ where } p_q=\begin{cases}
            0 & \text{ if } \Omega(q) > \max \Omega \text{ on any infite path passing }q\\
            \Omega(q) & \text{ else }
            \end{cases}\right\}, 1\right)&\text{ else }
            \end{cases}
\]

Otherwise if the previous state is non-final define:
\[
 q'_i = \begin{cases}
            (\{(q, p_q)|(q, i)\in V, p_q=\min\{\max(\Omega(q), p_{q'})|((q', i-1), (q,i))\in E\}\}, 0)&h\\\text{ if all }\Omega(q) \leq \min\max \Omega \text{ of all infinite paths in }\rho\\
            \Biggl(\Biggl\{(q, p_q)|(q, i)\in V, \text{ where }\\
            p_q=\begin{cases}
            \min\{p_{q'}|((q', i-1), (q,i))\in E\} & \text{ if } \Omega(q) > \max \Omega \text{ on any infite path passing }q\\
            \min\{\max(\Omega(q), p_{q'})|((q', i-1), (q,i))\in E\} & \text{ else }
            \end{cases}
            \Biggr\}\\, 1\Biggr) \text{ else }

            \end{cases}
\]
\textcolor{red}{To do: prove that this is indeed a run on $w$ in $\mathbb{A}'$ (this is per definition of $\delta'$).}\\
Now if $\rho$ is an accpeting run on $w$ in $\mathbb{A}$ that means that on any infinite path the maximum parity is even. So therefore there will be an infinite amount of times that alle states have an even $p_q$ so therefore we see that the max of the $\omega$ inf of $\rho'$ is at least 2. It cannot be $3$ since then there would be an infinite number of times a state with a bigger odd parity then all it's infinite paths. So that means that there will be an infinite path without an even maximum parity which contradicts the fact that $\rho$ is accepting.
Now we see that if $\rho$ is accepting that $\rho'$ is also accepting\qed.\\
Now we prove \(\mathcal{L}(\mathbb{A}')\subseteq \mathcal{L}(\mathbb{A})\)\\
Suppose \(w\in \mathcal{L}(\mathbb{A}')\) then there is a succesfull run \(\rho=q_0q_1q_2\dots\) on $w$ in the NPA $\mathbb{A}'$. Now we are going to make a run $\rho'$ in the APTA $\mathbb{A}$, this should be a DAG.
Define \begin{itemize}
        \item \(\rho'=(V,E)\)
        \item \(V = \{(q, i)|i\in \mathbb{N}, (q, p)\in X, (X, j)\in q_i\}\)
        \item \(E=\bigcup_{l\geq 0} \{((q, l), (q', l+1))|(q, l)\in V, \exists \text{ minimal } Z_q=\{q'|(q', l+1)\in V\} \text{ such that} Z_q\models \delta(q, a_l)\land q'\in Z_q\}\)
        \item Now this fulfills the conditions R1-R4 per definition (nog uitwerken dit)
       \end{itemize}
  Since we know that $\rho$ is accepting we know that the $\max(\Omega'(\inf(\rho))$ is even. By definition of $\Omega'$ we know that this then should equal 2. Therefore there are an infinite number states where all $p_q$ are even and that means that every path that visited that state has an even maximum parity. Also there is only a finite number of times that there is a bigger odd parity so that is not in the inf set of an infinite path. That means that every infinite path in $\rho'$ has an even maximum parity and therefore $\rho'$ is accepting on $w$. \textcolor{red}{Misschien moet dit hier en daar wat rigoreuzer}
\end{proof}
\section{Final theorem}
\begin{theorem}
 Let $\phi$ be a $\mu$-TL formula. Without loss of generality assume assume $\phi$ is guarded and simple. There exists an equivalent NPA $\A$ where
 \[
  \Mod_\Psf(\phi) = \LL(\A)
 \]
\end{theorem}
\begin{proof}
This proof will follow a chain of theorems in this week's booklet. First of all transform $\phi$ to a $U/R$ free formula using theorem \ref{thm:mutlformURfree}. Then use theorem \ref{thm:muformtoparform} to transform this to a strongly guarded and simple parity formula. Use theorem \ref{thm:parformtoapa} to transform this to an APA. Finally use definition \ref{def:APAtoNPA} to transform this to a NPA. This transformation is correct using theorem \ref{thm:APAtoNPA}. Now we have
\[
  \Mod_\Psf(\phi) = \LL(\A)
\]
\end{proof}
\chapter{Recognizing the stutterclosure of a nondeterministic parity automaton}
\section{Formal definition of stutter closed automata}
As seen in the examples of last week we have a construction to obtain an automaton that recognizes the stuttter closure of a language. I informally introduced the procedures but now I will give a formal definition of the automaton $\A^s$. Later on I will prove that this indeed recognizes the stutter-closure.
\begin{definition}\label{def:stutterclosed}
Let \(\A = (Q, \Sigma, \delta, q_I, \Omega)\) be some NPA, where without loss of generality we assume that $\Sigma \cap \omega = \emptyset$. We will define it's \textbf{Stutter-closed automaton} \(\mathbb{A}^s = (Q', \Sigma, \delta', q_I', \Omega')\) as follows:\\
First define the new set of states $Q'$ as $Q' := Q\times \Omega(Q) \cup Q\times \Sigma$ and the set of initial states $q_I'$ as $q_I':=\{(q, \Omega(q))|q\in q_I\}$. To define the transition function $\delta'$ let $(q,a)\in Q'$\\
\textbf{Case 1: $a\in \Sigma$}:
\[
 \delta'((q, a), b) = \begin{cases}
                       \{(q, a), (q, \Omega(q))\}&\text{if } a = b\\
                      \emptyset & \text{else}
                      \end{cases}
\]
\textbf{Case 2: $a\in \Omega(Q)$}:
\begin{align*}
 \delta'((q, a), b) &= \{(q', \Omega(q'))\in Q \times \Omega(Q)\mid q'\in \delta(q, b)\}\\ &\text{\textcolor{red}{dit is eigenlijk overbodig want wordt ook gevangen in laatste case met $n=1$}}\\
                    &\cup \{(q', b)\in Q\times \Sigma\mid  q'\in \delta(q, b)\}\\
                    &\cup \{(q', p) \in Q\times \Omega(Q)\mid \exists n, \exists \rho (qq_1\dots q_{n-1}q') \land q\stackrel{b^n}{\twoheadrightarrow^\rho}q'\land\\
                    & p = \max\{\max(\Omega(q_i), q')\mid i\in \{1, \dots, n-1\}\} \}
\end{align*}
Lastly we define the parity map $\Omega'$ as following:
\[
 \Omega'(q) =\left\{\begin{array}{llll}
             p + 2 &\text{ if } q=(q', p)   &\text{with } p\in \omega\\
             1 &\text{ if } q = (q', l) &\text{with } l\in \Sigma
            \end{array}\right.
\]
\end{definition}
\section{Proof that construction is correct}
In this section I will prove that the construction presented in definition \ref{def:stutterclosed} is correct.
\begin{theorem}
Let $\A$ a NPA and $\A^s$ it's stutter closed automaton as defined in definition \ref{def:stutterclosed}. The language of $\A^s$ is the stuttterclosure of the language of $\A$:
\[
 \mathcal{L}(\mathbb{A}^s) = (\mathcal{L}(\mathbb{A}))^s
\]
\end{theorem}
\begin{proof}
First $\supseteq$. \\
Suppose we have $w\in (\mathcal{L}(\mathbb{A}))^s$ then there exists a $w'\in\mathcal{L}(\mathbb{A})$ such that $w\sim_s w'$. Following definition 1 we know there exists $w_b=a_0a_1a_2\dots$ such that $w = a_0^{f(0)}a_1^{f(1)}a_2^{f(2)}\cdots$ and $w' = a_0^{f'(0)}a_1^{f'(1)}a_2^{f'(2)}\cdots$. \\h
\textbf{Claim 1:} $w_b\in \mathcal{L}(\mathbb{A}^s)$.\\
\textit{Proof of claim:} I need to construct a succesfull run in $\mathbb{A}^s$ for $a_0a_1a_2\dots$. Since we know that $w'\in \mathcal{L}(\mathbb{A})$ there exists a successful run $\rho = q_0q_1q_2\dots$ on $w'$ in $\mathbb{A}$, we will use this as the basis to construct the run of $w_b$. We will construct this run $\rho' = q'_0q'_1\dots$ in $\A^s$ with the same maximum parity as $\rho$ inductively.\\
Define $q'_0=(q_0, \Omega(q_0))$. Now for $q'_i$ (assume that the run $\rho'$ upto $q'_{i-1}$ is a run with the same behaviour (maximum parity) sa $\rho$ the word $w'$ upto $\dots a_{i-2}^{f'(i-2)}$):\\
\textbf{Case  $f'(i-1)=1$:} Let $(q_j, p) = q'_{i-1}$ now define $q'_i = (q_{j+1}, \Omega(q_{j+1}))$. We see that $q'_{i-1}\stackrel{a_{i-1}}{\longrightarrow}q'_i$ since $(q_{j+1}, \Omega(q_{j+1}))\in \{(q', \Omega(q'))\in Q \times \Omega(Q)\mid q'\in \delta(q, a_i)\}$ since $q_j\stackrel{a_{i-1}}{\longrightarrow}q_{j+1}$. \textcolor{red}{(Aangezien we hebben dat $w[j]$ gelijk is aan die $a_{i-1}$ blablabla en $\rho$ een run op $w$)} On this part the maximum parity stays the same as in $\rho$. \\

\textbf{Case $f'(i-1)=n>1$:} Let $(q_j, p) = q'{i-1}$ and define $q'_i = (q_{j+n}, \max\{\Omega(q_k)|k\in \{j+1, \dots, j+n\}\})$. We see that $q'_{i-1}\stackrel{a_{i-1}}{\longrightarrow}q'_i$ since $(q_{j+n}, \max\{\Omega(q_k)|k\in \{j+1, \dots, j+n\}\})\in \{(q', p) \in Q\times \Omega(Q)\mid \exists n, q\stackrel{b^n}{\twoheadrightarrow}q' (qq_1\dots q_{n-1}q') \text{ is a partial run } \land p = \max\{\Omega(q_i), q'\mid i\in \{1, \dots, n-1\}\} \}$ since $q_j\stackrel{a_i^{n}}{\longrightarrow}q_{j+n}$. On this part the maximum parity stays the same as in $\rho$
%
% Assumption: suppose we have $\rho'=\cdots q_j$ that runs up to $a_i$ in $w_b$.\\
% \textbf{Case  $f'(i)=1$:} Now we can go via the transition $((q_j, \Omega(q_j)), a_i, (q_{j+1}, \Omega(q_{j+1}))\in \delta'$. This transition is there since $(q_j, a_i, q_{j+1})\in \delta$ so we can add $(q_{j+1}, \Omega(q_{j+1})$ to $\rho'$. On this part the maximum parity stays the same since $\Omega'((q_{j+1}, \Omega(q_{j+1})) = \Omega(q_{j+1})$\\
% \textbf{Case $f'(i)=n>1$:} We know that $\rho = q_jq_{j+1}\dots q_{j+n}\dots$ and there are transitions $(q_k, a_i, q_{k+1})$ for $k\in \{j, \dots, j+n-1\}$. So that means that there is a (shortcut) transition $((q_j, \Omega(q_j)), a_i, (q_{j+n}, p))\in \delta'$ where $p=\max\{\Omega(q_k)\mid k\in \{j, \dots, j+n\}\}$. So we can add $(q_{j+n}, p)$ to $\rho'$. On this part the maximum parity stays the same since $\Omega'((q_{j+n}, p)) =p = \max\{\Omega(q_k)\mid k\in \{j, \dots, j+n\}\}$\\
% We see that the maximum parity of the inf set of $\rho'$ is the same as the maximum parity of the inf set of $\rho$ since the maximum parity is preserved on every stretch of $\rho'$. So that means that $\rho'$ is accepting. Which completes the proof of claim 1. \qed

\textbf{Claim 2: $w = a_0^{f(0)}a_1^{f(1)}a_2^{f(2)}\dots \in \LL(\A^s)$}:\\

\textit{Proof of claim}: We are going to create a successful run $\rho'' = q''_0q''_1\dots$ for $w$ in $\mathbb{A}^s$ based on the run $\rho'=q'_0q'_1q'_2\dots$ of $w_b$ in $\mathbb{A}^s$. On the same inductive way:
First define $q''_0 := q'_0$. For $q''_i$Let $k\in \N$ the number so that $i-1=\sum_{m=0}^{k-1}f(m)$ \textcolor{red}{Edge case met $i-1=0$}. This is well defined since we know that we added exactly $f(k-1)$ states in the previous step. Now we see two cases:\\
\textbf{Case $f(k)=1$:}. Let $q'_j=q''_{i-1}$. Now set $q''_i:=q'_{j+1}$. Now claim: $w[i] = a_{j+1}$  \textcolor{red}{Waarom is deze transitie daar, ik vind het moeilijk om hier nu formeel te bewijzen dat dit klopt.}
\textbf{Case $f(k)=n>1$:} Let $(q_j, p_j) = q'_{k}$ and $(q_l, p_l) = q'_{k+1}$. If $(q_l, a_k)\in \delta((q_j, p_j), a_k)$ set $q''_m = (q_j, a_k)\text{ for } m\in \{i, \dots, i+n-2\}$ and $q''_{i+n-1} = (q_j, \Omega(q_j)$. If not, set $q''_{i+r} = (q_{j+1+r}, \Omega(q_{j+1+r}))$ for $0\leq r \leq n-1$ or as long as $(q_l, a_k)\in \delta((q_{j+1+r_{last}}, \Omega(q_{j+1+r_{last}})), a_k)$, if this is the case then use self loop for $n-r_{last}-2$ times. If you do not encounter the self loop once go to the shortcut state.
Also here the maximum parity stays the same along every section.
% Suppose we have a run $q''$ that mimics $\rho'$ up to $q_j$. \\
% \textbf{Case $f(j) =1$:} Now we have $(q_j, a_j, q_{j+1})\in\delta'$ so add $q_{j+1}$ to $\rho''$. On this section the maximum parity stays the same since the runs are the same on this section. \\
% \textbf{Case $f(j) = n > 1$:} Now we have to use a self loop. If $q_{j+1}=(q, \Omega(q))$ pass through the self loop $(q, a_j)$ $n-1$ times and after that go to $q_{j+1} = (q, \Omega(q))$ (so add $(q,a_j)^{n-1}(q,\Omega(q))$ to $\rho''$. If not, so $q_{j+1}=(q,p)$ with $p\neq \Omega(q)$ then first add $(q,p)$ to $\rho''$ if $n=2$ directly add $(q, \Omega(q)$ to $\rho''$ otherwise go to $(q, a_j)$ (the self-loop state) and pass through it $n-2$ times (the first repeat is already in the step to $(q,p)$ (add $(q,a_j)^{n-2}(q,\Omega(q))$ to $\rho''$). Now this has absorbed all $a_j^{f(j)}$. Now we are at $a_{j+1}^{f(j+1)}$ and repeat the process. In both cases , the maximum parity stays the same since the $\Omega'(q,l)=0$.\\
This creates a successful run in $\mathbb{A}^s$ since the maximum parity in the inf set stays the same since it is equal on every section. This completes the proof of claim 2. And proves the inclusion$\mathcal{L}(\mathbb{A}^s) \supseteq (\mathcal{L}(\mathbb{A}))^s$.\qed\\
For the other inclusion $\subseteq$ observe the following proof:\\
Suppose we have $w=a_0a_1a_2\dots\in \mathcal{L}(\mathbb{A}^s)$ then there is a successful run $\rho = q_0q_1q_2\dots$ on $w$ in $\mathbb{A}^s$. Now we are looking for $w'\in \mathcal{L}(\mathbb{A})$ such that $w\sim_sw'$. I will construct $w_b, f, f'$ and also already $\rho'$ the run for $w'$ in $\A$ such that $w=wb[f]$ and $w'=wb[f']$. For $w_b=b_0b_1b_2\dots$, we will define:\\
Let $(q'_0, p)=q_0$. For $b_j$ (with $i$ passed, start at $j=0$ and $i=1$).  Now there are two cases:\textcolor{red}{Ik wil hier een iteratie (inductie??) doen waar ik meegeef waar in de run $\rho$ je bent }\\
\textbf{Case 1: $q_i=(q'_k, p)\in Q\times \Omega(Q)$}: Set $b_j = a_{i-1}$ and $f(j) = 1$. Let $(q'_{k-1}, p)=q_{i-1}$ From the definition we know that there are two possibilities here: Either there is a connection in $\A$ from $q'_{k-1}$ to $q'_{k}$, then we know that $p=\Omega(q'_k)$ in this case set $f'(j)=1$ and add $q'_k$ to $\rho'$. If not then we know $\exists n, q\stackrel{{a_{i-1}}^n}{\twoheadrightarrow}q' (qq_1\dots q_{n-1}q') \text{ is a partial run } \land p = \max\{\Omega(q_i), q_i\mid i\in \{1, \dots, n-1\}\}$. Now set $f(j) = n$ and add $q_1\dots q_{n-1}q'$ to $\rho'$ .  Finally pass $i+1$\\
\textbf{Case 2: $q_i=(q'_k, a_{i-1})\in Q\times \Sigma$}: Determine $l\in \N$ such that all $q_{i+m} = (q'_k, a_{i-1})\text{ for } m\in \{0, \dots, l-2\}$. Now set $b_j=a_{i-1}$ and $f(j) = l$ and $f'(j)=1$ and add $q'_k$ to $\rho'$. Now pass $i+l+2$\\
Now we have $w'$ and the run $\rho'$ but is this accepted in $\A$? Yes since on every section the maximum parity stays the same per definition. This proves the second inclusion which proves the theorem.

\end{proof}

\chapter{Using the $\mu$-TL to NPA translation to describe stutterinvariance}

\clearpage%to fix page numbering in ToC
\addcontentsline{toc}{chapter}{Bibliography}
\bibliographystyle{plain}
\bibliography{sources}


\clearpage%to fix page numbering in ToC
\chapter*{Populaire samenvatting}
\addcontentsline{toc}{chapter}{Populaire samenvatting}

\appendix
\chapter{Graph theory}
Hier iets zeggen over DAG's, topological orderings enzo.


\end{document}
