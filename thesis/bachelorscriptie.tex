\documentclass[dubaWI, english]{uvamath}%use option "english" for an English thesis
%Alle mogelijke opties zijn: complatex, tweedejaarsproject, bachelorscriptie, dubaWI, dubaWN

\usepackage[english]{babel} %Remark: use the same language for uvamath and babel
\usepackage{graphicx}
\usepackage[pdfborder={0 0 0}]{hyperref}
\usepackage{lipsum}
\usepackage{amsfonts}
\usepackage{amsthm}
\usepackage{amsmath,amssymb, mathrsfs}
\usepackage{mathtools}
\usepackage{modalops}
\usepackage{hyperref}
\usepackage{stackrel}
% \usepackage{logix}
% \setmainfont{STIX Two Text}
% \setmathfont{STIX Two Math}

% Required packages and libraries
\usepackage{tikz}
\usetikzlibrary{automata, arrows.meta, positioning}

\theoremstyle{plain}

\theoremstyle{definition}
\newtheorem{definition}{Definition}[chapter]
\newtheorem{example}[definition]{Example}
\theoremstyle{plain}
\newtheorem{theorem}[definition]{Theorem}
\newtheorem{claim}{Claim}

\newtheorem{lemma}[definition]{Lemma}

\newtheorem{fact}[definition]{Fact}
\newtheorem{proposition}[definition]{Proposition}


\renewcommand{\phi}{\varphi}
\renewcommand{\epsilon}{\varepsilon}
\DeclareMathOperator{\Mod}{Mod}
\DeclareMathOperator{\Dom}{Dom}
\DeclareMathOperator{\last}{last}
\DeclareMathOperator{\Sub}{Sub}

\newcommand{\Lit}{\mathtt{Lit}}
\newcommand{\At}{\mathtt{At}}

\DeclareMathOperator{\PM}{PM}



\newcommand{\A}{\mathbb{A}}
\newcommand{\G}{\mathbb{G}}
\newcommand{\E}{\mathcal{E}}

\newcommand{\N}{\mathbb{N}}
\newcommand{\Claim}{\textbf{Claim}}
\newcommand{\Proof}{\textit{Proof}}


\newcommand{\LL}{\mathcal{L}}
\newcommand{\PP}{\mathcal{P}}
\newcommand{\Psf}{\mathsf{P}}

%definitions
\newcommand{\F}{\mathcal{F}}




\title{Stutter-invariance and equivalences in logic and automata}
\author[riemer.kerkstra@student.uva.nl, 13283529]{Riemer Kerkstra}

\supervisors{prof.\ dr.\ Yde Venema}
\supervisors{dr.\ Malvin Gattinger}
\secondgrader{dr.\ Alexi Block Gorman}
\secondgrader{dr.\ Balder ten Cate}

% \coverimage{\includegraphics[scale=0.8]{figuur.pdf}}

\begin{document}
\maketitle

\begin{abstract}
Schrijf een samenvatting van hoogstens een halve bladzijde waarin je kort uitlegt wat je hebt gedaan. De samenvatting schrijf je als laatste. Je mag er vanuit gaan dat je docent de lezer is.
\end{abstract}

\tableofcontents

\chapter{Introduction}
In this thesis we will study linear temporal logic and its connection to automata theory. Linear temporal logic (LTL) was, among other people, introduced by Pnueli in 1977. Pnueli introduced LTL in order to specify and verify \textit{reactive and concurrent systems}. Temporal logic formulas can describe a property of these systems but it is also important to decide if this property holds in a certain model. That is where automata theory is helpful. In order to solve the decision problem for temporal logics, a formula is translated into an automata on infinite words. Then satisfiability of a formula is equivalent to non-emptiness of the language of the automaton \cite{sep-logic-temporal}. Next to the general linear temporal logic there are also numerous extensions and varieties such as branching time temporal logic and the linear time $\mu$-calculus which we will study in this thesis.

Linear temporal logic uses infinite models on the natural numbers. Such model can be viewed as describing the subset of propositional letters that are true at the time step given by a natural number. Visually this can be represented in figure \ref{fig:modelexample}.
\begin{figure}[h!]
 \centering
\begin{tikzpicture}
  % Define the nodes
  \node[circle, draw] (0) at (6,0) {$0$};
  \node[circle, draw] (1) at (8,0) {$1$};
  \node[circle, draw] (2) at (10,0) {$2$};
  \node[circle, draw] (3) at (12,0) {$3$};
  \node (4) at (14,0) {$...$};

  % Draw arrows between nodes to represent the accessibility relation
  \draw[->] (0) -- (1);
  \draw[->] (1) -- (2);
  \draw[->] (2) -- (3);
  \draw[->] (3) -- (4);

  % Optional: Add labels or other decorations
  \node[above] at (0.north) {$\{p, q\}$};
  \node[above] at (1.north) {$\{q\}$};
  \node[above] at (2.north) {$\{p, q, r\}$};
  \node[above] at (3.north) {$\{p, q,r\}$};
\end{tikzpicture}
\caption{Example of a linear model, where we for example say that $p$ and $q$ hold at time step $0$.}
\label{fig:modelexample}
\end{figure}
% \vspace{-3mm}

\noindent In the context of automata these models are usually viewed as infinite $\omega$-words over the alphabet $\PP(\Psf)$ where $\Psf$ is the set of propositional letters. When using Linear Temporal Logic to describe the behaviour of concurrent programs Lamport \cite{lamport1983whatgood} introduced the concept of stuttering in 1983. He argues that a formula should not describe the number of steps needed to reach the next state. He argues that this is a desired and natural property for a temporal logic to have. We can more formally define this notion of \emph{stuttering} as follows. Informally stuttering means repeating letters and removing consecutive occurences from the word. We say that two words are stutter-equivalent if they can be transformed into each other by stuttering.
For example the words $w=aabbcc$ and $w'=abbbccc$ are stutter-equivalent (notation $w\sim_s w'$) since we can remove one $a$ from $w$ add a $b$ and add a $c$ to $w$ to get $w'$. But $w=aabbcc$ and $w=aadcc$ are not stutter-equivalent since you cannot add new letters or remove all consecutive occurrences of a letter.
A language is \emph{stutter-invariant} if for every $w, w'\in \Sigma^\omega$ (where $\Sigma^\omega$ is the set of infinite words over alphabet $\Sigma$) with $w\sim_s w'$ stutter equivalent we have

\[
 w\in\LL \text{ if and only if } w'\in\LL.
\]

Lamport argues that any temporal logic should only have stutter-invariant languages. In this thesis we will study the Linear Time $\mu$-calculus ($\mutl$), an extension of LTL where the fixpoint operators $\mu$ and $\nu$ are added. It is shown by  Amélie Gheerbrandt and Balder ten Cate \cite{gheerbrandt2009craiginterpolation} that $\mutl(U,R)$ is the stutter-invariant fragment of the Linear Time $\mu$-calculus. We also know that $\mutl$ is $\omega$-regular which means that for every $\phi\in\mutl$ there exists an $\omega$-automaton that defines exactly the language of $\phi$ \cite{demri2016temporal}. In this thesis we will work with Non-Deterministic Parity Automaton (\npa) which are a specific type of $\omega$-automata. We specifically choose the parity condition since this allows for a natural translation of the fixpoint formulas. We also know that there exists a procedure to check if a $\omega$-regular language is stutter-invariant. For example Thibaud Michaud and Alexandre Duret-Lutz \cite{michaud2015practical} give a procedure to determine if a Nondeterministic Transition Büchi automaton has a stutter-invariant language.

In this thesis we will alter the construction of \cite{michaud2015practical} to work with \npa s and use this to prove, just like Gheerbrandt and Ten Cate, that $\mutl(U,R)$ is the stutter-invariant fragment of $\mutl$. The original plan of this project was to provide a proof that for every $\mutl$ formula $\phi$ there exists an equivalent \npa\, and vice versa. Then to provide a construction to determine the stutter-closure of an \npa\, and use this construction to prove that $\mutl(U,R)$ is the stutter-invariant fragment of $\mutl$. This plan was too ambitious for a Bachelor project so the scope of this thesis is smaller. We will first provide a translation for a $\mutl$ formula a a \npa. For a \npa\, $\A$ we will provide a construction to obtain an \npa\, $\A^s$ that recognizes the stutter-closure of the language of $\A$. With this construction we will give the sketch of the proof that $\mutl(U,R)$ is the stutter-invariant fragment of $\mutl$. We will also explain the difficulties why we did not find a full proof in Section \ref{section:mutlstutinvariant}.

In addition we will define parity automata and prove the stutter-closure construction on these automata in the formal proof assistant $\lean$. Due to time-issues we will only formalize part of the proof of Theorem \ref{thm:stutterclosednpa} in $\lean$. Obviously this computer verified proof of our proof will help our belief of the correctness of this proof. But formalization in $\lean$ is also a useful skill to have in the current-day mathematics world. More motivation about formalizing in general and in $\lean$ specifically will be given in Chapter 5.

For this thesis we used the lecture notes \textit{Lectures on the Modal $mu$-calculus} of Yde Venema \cite{venema2024modalmucalculus} as the basis for our definition of the linear time $\mu$-calculus. We will refer to these in the specific sections. For the definition of parity and alternating parity automaton we used the textbook \textit{Temporal Logics in Computer Science} of Demri, Goranko and Lange \cite{demri2016temporal} and will also refer to these in the specific sections. We also used the textbook \textit{Automata, Logics, and Infinite Games} \cite{gradel2003automata} as background material. In addition to the textbooks as sources for our definitions we explored the existing literature on this topic. We directly used theorems from the literature but also used ideas from existing articles for the theorems in this thesis.

The thesis is structured as follows. Chapter 2 provides the necessary background about $\mutl$, parity-automata and stuttering. In Chapter 3 we provide a translation of $\mutl$ to parity automata and prove that this translation is correct. In Chapter 4 we develop a construction to determine whether a $\omega$-automaton is stutter-invariant and give an overview of the way to prove that every formula in $\mutl(U,R)$ is stutter-invariant. Lastly in Chapter 5 we will describe the formalization process in $\lean$. As the subjects of this thesis: logic, automata and formalization in $\lean$ not only lie on the boundary of Mathematics and Computer science but actually in the intersection of both fields it is less meaningful to define a Mathematical and a Computer Science part of this thesis. Formally the supervision of the Mathematics part of this thesis was done by Yde Venema. He supervised the theory and proofs of logic and automata. Malvin Gattinger, as Computer Science supervisor, supervised the formalization in $\lean$.


% Oude versie:::
% In this thesis we will study Linear time temporal fixpoint logic or Linear Time $\mu$-calculus ($\mutl$) and it's stutter-invariant fragment. Linear Time $\mu$-calculus is a logic that describes properties on a linear model with both temporal and fixpoint logic. \textcolor{red}{Wil ik hier meer uitleggen over mutl, of is dit goed?} It is an extension of the normal linear time logic with added fixpoint operators. Such a logic describes languages of infinite $\omega$-words, such that these words satisfy a formula. \textcolor{red}{Dit is vaag, moet iets duidelijker}.
%
% On these $\omega$-words we can define the notion of stuttering.Intuitively stuttering means repeating letters and removing consecutive occurences from the word. We say that two words are stutter-equivalent if they can be transformed into each other by stuttering.
% For example the words $w=aabbcc$ and $w'=abbbccc$ are stutter-equivalent (notation $w\sim_s w'$) since we can remove one $a$ from $w$ add a $b$ and add a $c$ to $w$ to get $w'$. But $w=aabbcc$ and $w=aadcc$ are not stutter-equivalent since you cannot add new letters or remove all consecutive occurences of a letter.
% A language is \emph{stutter-invariant} if for every $w, w'\in \Sigma^\omega$ (where $\Sigma^\omega$ is the set of infinite words over alphabet $\Sigma$) with $w\sim_s w'$ stutter equivalent we have
%
% \[
%  w\in\LL \text{ if and only if } w'\in\LL.
% \]
% According to Lamport \cite{lamport1983whatgood} stutter-invariancy is a desired and natural property of a temporal logic. He introduced temporal logic to study concurrent programs and he argues that when one wants to talk about the next state of the program you actually mean the next state where a significant change occurs. Therefore there should be no difference between $aaac$ and $aac$ since in both cases the next significant change after $a$ is $c$. That means we have to use a logic that is stutter-invariant, and for this we want to check if a specific logic is stutter-invariant.
%
% It is shown by  Amelie Gheerbrandt and Balder ten Cate \cite{gheerbrandt2009craiginterpolation} that $\mutl(U,R)$ is the stutter-invariant fragment of the Linear Time $\mu$-calculus. We also know that  $\mutl$ is $\omega$-regular which means that for every $\phi\in\mutl$ there exists an $\omega$-automaton that defines exactly the language of $\phi$, [Bron]. In this thesis we will work with Non-Deterministic Parity Automaton (NPA) which are a specific type of $\omega$-automata. We also know that there exists a procedure to check if a $\omega$-regular language is stutter-invariant. For example Thibaud Michaud and Alexandre Duret-Lutz \cite{michaud2015practical} give a procedure to determine if a Nondeterministic Transition Büchi automaton has a stutter-invariant language.
%
% In this thesis we will alter the construction of \cite{michaud2015practical} to work with NPA's and use this to prove, just like Gheerbrandt and Ten Cate, that $\mutl(U,R)$ is the stutter-invariant fragment of $\mutl$. The original plan of this project was to provide a proof that for every $\mutl$ formula $\phi$ there exists an equivalent NPA and vice versa. Then to provide a construction to determine the stutter-closure of an NPA and use this construction to prove that $\mutl(U,R)$ is the stutter-invariant fragment of $\mutl$. Unfortunately we had to change the scope of this project. We did not provide a translation of NPA to $\mutl$ formulas and also did not complete the proof of the stutter-invariancy of $\mutl(U,R)$. We will explain the difficulties in the latter proof in Section \ref{section:mutlstutinvariant}. In addition we will define parity automata and prove the stutter-closure construction on these automata in the formal proof assistant $\lean$.
%
% The thesis is structured as follows. Chapter 2 provides the necessary background about $\mutl$, parity-automata and stuttering. In Chapter 3 we provide a translation of $\mutl$ to parity automata and prove that this translation is correct. In Chapter 4 we develop a construction to determine whether a $\omega$-automaton is stutter-invariant and give an overview of the way to prove that every formula in $\mutl(U,R)$ is stutter-invariant. Lastly in Chapter 5 we describe the formalization process in $\lean$.
%
% For this thesis we used the lecture notes \textit{Lectures on the Modal $mu$-calculus} of Venema \cite{venema2024modalmucalculus} as the basis for our definition of the linear time $\mu$-calculus. We will refer to these in the specific sections. For the definition of parity and alternating parity automaton we used the textbook \textit{Temporal Logics in Computer Science} of Demri, Goranko and Lange \cite{demri2016temporal} and will also refer to these in the specific sections. We also used the textbook \textit{Automata, Logics, and Infinite Games} \cite{gradel2003automata} as background material. Furthermore if a theorem is adopted from an article or textbook we will only cite the source. If we adapted the theorem to our specific case (for example from the general $\mu$-calculus to the linear time $\mu$-calculus) we will also mention that. In addition to the textbooks as sources for our definitions we used theorems from several articles or as a basis for our own theorems.???
%
%
% Now we will move away from the third person perspective and describe the process of the bachelor \emph{project} that lies at the basis of this thesis. I started this project by making myself familiar with the theory of $\omega$-automata and the linear time $\mu$-calculus. I quickly started to work on the theorems and had weekly meetings with my mathematics supervisor Yde Venema to discuss these proofs. He (heavily) commented on my writing style and helped me in the part of the proofs where I got stuck. In the last one and a half month I worked on the final proof and started to finalize all definitions and proofs to form this thesis. For the Computer Science part of this project I started by learning some $\lean$ and quickly started to define and prove the problem. My computer science supervisor Malvin Gattinger helped me with my $\lean$ style and knowledge and to eliminate sorrys in my proof.

\chapter{Preliminaries}
In this section we will provide the reader with the basic concepts and definitions that are neccesary to understand this thesis. We will start with a section about Parity games since those are the basis of the semantics for $\mutl$ formulas and for linear parity formulas. In Section \ref{section:mutl} we will provide the basic syntax and semantics of $\mu$-TL and some basic notions about formulas. In Section \ref{section:parform} we will introduce Linear Parity Formulas which are needed as a intermediate structure in the translation of a $\mu$-TL formula to an $\omega$-automaton. Nextly in Section \ref{section:automata} we will introduce both Alternating and Nondeterministic $\omega$-automata. Lastly in Section \ref{section:stuttering} we will formally define the notion of stuttering.
\section{Parity games}\label{section:pargame}
In this section we will define parity games as they will form the basis of the acceptance of $\mutl$ formulas and of linear parity formulas. The definitions are, with some minor changes, adapted from  \cite[Chapter 5]{venema2024modalmucalculus}.
The games in this thesis are played by two players: Eloise whom we denote by $\exists$ and Abelard whom we denote by $\forall$. An arbitrary player is denoted by $\tau$ and the opponent by $\overline{\tau}$. In general Eloise wants to show that a structure (formula or game) is satisfied and Abelard wants to show the contrary.

\begin{definition}\label{def:pargame}
 A parity game is a quadruple $\mathcal{B}=(B_\exists, B_\forall, E, \Omega)$ such that $B_\exists$ and $B_\forall$ are disjoint sets, and $E\subseteq B^2$, where $B:=B_\exists\cup B_\forall$ and $\Omega : B\to\omega$ is the \emph{priority map}. .  We will call $\mathbb{B}=(B_0, B_1, E)$ the board or game graph of this parity game. An \emph{initialized parity game} is a parity game with a position $a$ attached to it as initial position. We denote this as $\mathcal{B}@a$.
 We will denote $E[a]$ as the set of \emph{admissible} moves for $a\in B$, that is $E[a]:=\{b\in B\mid (a,b)\in E\}$. A position $a\in B$ is a \emph{dead end} if $E[a]=\emptyset$. If $a\in B$ we let $\tau_a$ denote the (unique) player such that $a\in B_{\tau_a}$, and say that $a$ \emph{belongs to} $\tau_a$, or that it is $\tau_a$'s \emph{turn} to move at $a$.
\end{definition}
In order to explain the working of the priority map, and with it the winning condition of a parity game we first need to define matches in a parity game.
\begin{definition}\label{def:match}
 A \emph{path} through the board $\mathbb{B}=(B_\exists,B_\forall,E)$ of a parity game $\mathcal{B}=(B_\exists, B_\forall, E, \Omega)$ is a nonempty (finite or infinite) sequence $\pi\in B^\infty$ such that $E\pi_i\pi_{i+1}$ for every $i$. A \emph{full} or {complete match} through $\mathbb{B}$ is either an infinite $\mathbb{B}$-path, or a finite $\mathbb{B}$-path ending in a dead end. A \emph{partial match} is a finite path through $\mathbb{B}$ that does not end in a dead end. We let $\PM$ denote the set of all partial matches and $\PM_\tau$ denote the set of all partial matches such that it is $\tau$'s turn in the last position of the match.
\end{definition}

\noindent In order to determine which player has won the game we will look at the set of positions that occur infinitely often and the maximum parity of this set.

\begin{definition}
 Let $\alpha:\omega\to B$ be a $B$-sequence. Given an element $a\in B$ we define the \emph{frequency} of $a$ in $\alpha$ as $\#_a(\alpha):=|\{i\in\omega\mid\alpha(i)=a\}|$. Based on this frequency the set of elements that occur infinitely often as $\inf(\alpha):=\{a\in \alpha\mid \#_a(\alpha)=\omega\}$ and the set of parities that occur infinitely often as $\inf_\Omega(\alpha):=\{\Omega(a)\mid a\in \inf(\alpha)\}$
\end{definition}

\begin{definition}
 Let $\mathcal{B}=(B_\exists, B_\forall, E, \Omega)$ be a parity game and $\pi$ a match through $\mathbb{B}$. If $\pi$ is a finite match we say that that $\tau$ has won this match if $\last(\pi)$ is a dead end for $\overline{\tau}$. If $\pi$ is an infinite match we say that $\exists$ has won this match if \(\max (\inf_\Omega(\pi))\) is even and that $\forall$ has won if this is odd.
\end{definition}

\noindent In order to define the winning positions in a parity game we will define the notion of a strategy.
\begin{definition}\label{def:winning}
Give a parity game $\mathcal{B}=(B_\exists, B_\forall, E, \Omega)$ and a player $\tau$, a $\tau$-\emph{strategy}, is a map $f: \PM_\tau\to B$ that prescribes the moves $\tau$ should make. In case we are dealing with an initialized parity game $\mathcal{B}@a$ we can take the strategy as a map $f:\PM_\tau(a)\to B$. A match $\pi$ is \emph{guided by} a $\tau$-strategy if for any partial match $\pi'<\pi$ with $\last(\pi')\in B_\tau$, the next position on $\pi$ (after $\pi'$) is equal to $f(\pi')$. A $\tau$-strategy $f$ is \emph{surviving} in $\mathcal{B}@a$ if every move that it prescribes to $f$-guided partial matches in $\PM_\tau(a)$ is admissible and \emph{winning} for $\tau$ if in addition all $f$-guided full matches starting at $a$ are won by $\tau$. A position $a\in B$ is winning for $\tau$ if $\tau$ has a winning strategy for the game $\mathcal{B}@a$. The collection of all winning positions for $\tau$ in $\mathcal{B}$ is called the \emph{winning region} for $\tau$ in $\mathcal{B}$, and denoted as $\Win_\tau(\mathcal{B})$.
\end{definition}
% Note that we defined $f$ as a full map from all partial matches to ... Geen idee hier eventjes.

In general strategies can depend on all positions that lie before the last, but in some cases we want the strategy to only take into account the last position of a partial match. That is what we call a positional strategy.
\begin{definition}
 A strategy $f$ is \emph{positional} if $f(\pi)=f(\pi')$ for any $\pi,\pi'$ with $\last(\pi)=\last(\pi')$. In this case a positional $\tau$-strategy may be represented as a map $f:B_\tau\to B$.
\end{definition}

\noindent The following result will allow us to always use a positional strategy when working with parity games. It
\begin{theorem}[Positional Determinacy of Parity games]\cite[Theorem 5.28]{venema2024modalmucalculus}
 Let $\mathcal{B} = (B_\exists, B_\forall, E, \Omega)$ be a parity game. Then $B = \Win_\exists(\mathcal{B})\cup \Win_\forall(\mathcal{B})$ and there are positional strategies $f_\exists$ and $f_\forall$ such that for each player $\tau$ and every $a\in\Win_\tau(\mathcal{B})$, the strategy $f_\tau$ is winning for $\tau$ in $\mathcal{B}@a$.
 \end{theorem}
\noindent The proof of this theorem can be found in \cite{venema2024modalmucalculus}.
As last definitions in this section we will define the strategy tree and strategy graph.
\begin{definition}\label{def:strattree}
 If $f$ is a surviving player for $\tau$ in $\mathcal{B}@a$, we may represent $f$ as the pruned subtree of the gametree $\mathbb{T}_a^\mathcal{B}$ that is based on those nodes that correspond to $f$-guided matches of $\mathcal{B}@a$. In this so called \emph{strategy tree} $\mathbb{T}_a^f$ we have
 \[
  \stackrel{\rightarrow}{E}_f[\pi] := \begin{cases}
                                     \stackrel{\rightarrow}{E}[\pi] &\text{if } \pi \in \PM_{\bar{\tau}}\\
                                     \{\pi \cdot f(\pi)\} &\text{if } \pi \in \PM_{\tau}.
                                    \end{cases}
 \]
\end{definition}
\begin{definition}\label{def:stratgraph}
 If $f$ is a surviving, positional strategy for player $\tau$ in $\mathcal{B}@a$, we may represent $f$ as the pruned subgraph of the game graph (instead of the game tree) of $\mathbb{B}_a^\mathcal{B}$ that is based on those nodes that correspond to $f$-guided matches. In this so called \emph{strategy graph} $\mathbb{B}_a^f$ we have:
 \[
  E_f[b] := \begin{cases}
                                     E[b] &\text{if } b \in B_{\bar{\tau}}\\
                                     f(b) &\text{if } b \in B_\tau.
                                    \end{cases}
 \]
\end{definition}

\section{Linear time $\mu$-calculus}\label{section:mutl}
\begin{definition}\label{def:mutempform}
 Let $\Psf$ be a set of propositional letters. We define the collection $\mu$-TL of \textbf{temporal fixpoint formulas} over $\Psf$ as follows:
\[
\phi ::= \top \mid \bot \mid p\mid\bar{p}\mid (\phi\land\phi)\mid(\phi\lor\phi)\mid\deo\phi\mid\nu x.\phi\mid \mu x.\phi\mid \phi U\phi\mid \phi R\phi
\]
for $p\in\Psf$.
\end{definition}

\begin{definition}\cite[Section 6.1]{demri2016temporal}\label{def:linearmodel}
 An\emph{$\omega$}-word of alphabet $\Sigma$ is an infinite sequence
 \[
  w: \omega \to \Sigma.
 \]. If $\Sigma=\PP(\Psf)$ we call $w$ a \emph{linear transition system} or \emph{linear (Kripke) model}
\end{definition}
% \begin{definition}\cite[Section 6.1]{demri2016temporal}\label{def:linearmodel}
%  A \textbf{linear transition system} or linear (Kripke) model of type $\Psf$ is an infinite sequence
%  \[
%   w: \omega \to \PP(\Psf).
%  \]
%  We will also refer to $w$ as an $\omega$-word in $(\PP(\Psf))^\omega$.
% \end{definition}
\textcolor{red}{Definities over bound en free toevoegen}
\begin{definition}
 Clean and tidy
\end{definition}

\begin{definition}(Extension of \cite[Definition 2.20]{venema2024modalmucalculus})
 Given a clean modal fixpoint formula $\xi$ and a linear transition system $w$ we define the evaluation game $\mathcal{E}(\xi, w)$ as a parity game with players $\exists$ and $\forall$ moving along a board consisting of positions $(\phi, s)\in Sf(\xi)\times \omega$. The addmissible moves are\\
 \begin{tabular}{|c|c|c|}
  \hline
  Position & Player & Admissible moves\\
  \hline
  $(\phi_1\lor\phi_2,i)$& $\exists$ & $\{(\phi_1, i),(\phi_2, i)\}$\\
  $(\phi_1\land\phi_2,i)$& $\forall$ & $\{(\phi_1, i),(\phi_2, i)\}$\\
  $(\deo\phi,i)$&-&$\{(\phi, i+1)\}$\\
  $(\eta_xx.\delta_x, i)$&-&$\{(\delta_x, i)\}$\\
  $(x, i)$ with $x\in BV(\xi)$&-&$\{(\delta_x, i)\}$\\
  $(\phi U\psi,s)$& $\exists$ & $\{(\psi, s),(\phi\land\deo\phi U \psi, s)\}$\\
  $(\phi R\psi,s)$& $\exists$ & $\{(\phi \land \psi, s),(\phi\land\deo\phi U \psi, s)\}$\\
  $(\bot,i)$& $\exists$ & $\emptyset$\\
  $(\top,i)$& $\forall$ & $\emptyset$\\
  $(p,i)$ with $p\in FV(\xi)$ and $p\in w(i)$& $\forall$ & $\emptyset$\\
  $(p,i)$ with $p\in FV(\xi)$ and $p\notin w(i)$& $\exists$ & $\emptyset$\\
  $(\bar{p},i)$ with $p\in FV(\xi)$ and $p\notin w(i)$& $\forall$ & $\emptyset$\\
  $(\bar{p},i)$ with $p\in FV(\xi)$ and $p\in w(i)$& $\exists$ & $\emptyset$\\
\hline
 \end{tabular}\\
 Where we allow infinite unfolding of $\nu$ and $R$ operators and only finite unfoldings of $\mu$ and $U$ operators. We say
 \[
  w, i\Vdash_g\xi \text{ iff } (\xi, i)\in \text{Win}_\exists(\mathcal{E}(\xi, w))
 \]
\end{definition}
\begin{definition}
 We can also define the semantics in the following algebraic way:
 \begin{align*}
  w, i &\Vdash p & \text{ iff } p\in w(i)\\
  w, i &\Vdash \bar{p} & \text{ iff } p\notin w(i)\\
  w, i&\Vdash \bot &\text{ never}\\
  w, i&\Vdash \top &\text{ always}\\
  w, i&\Vdash \phi_1\land\phi_2&\text{ iff }w, i\Vdash \phi_1 \text{ and }w, i\Vdash \phi_2\\
  w, i&\Vdash \phi_1\lor\phi_2&\text{ iff }w, i\Vdash \phi_1 \text{ or }w, i\Vdash \phi_2\\
  w, i&\Vdash \deo \phi&\text{ iff }w, i+1\Vdash \phi\\
  w, i &\Vdash \phi_1U\phi_2 &\text{ iff there is }j\geq i\text{ such that }w, j\Vdash \phi_1 \text{ and} \\
                                 &&w, k\Vdash \phi_2 \text{ for all } k \text{ with }i\leq k<j\\
 w, i &\Vdash \phi_1R\phi_2 &\text{ iff } w, j \Vdash \phi_1 \text{ for all } j\geq i\text{ or}\\
                                &&\text{there is }j\geq i\text{ such that }w, j\Vdash \phi_1 \land \phi_2 \text{ and } \\
                                 &&w, k\Vdash \phi_2 \text{ for all } k \text{ with }i\leq k<j
 \end{align*}
 \textcolor{red}{And for fixpoints zelfde maar dan met verz.}
\end{definition}
\begin{theorem}
 The game theoretic semantics are equivalent to the algebraic semantics (specifically for the Until and release)
\end{theorem}
\begin{proof}
 Most of this has been done by Yde. Bot TODO for the Until and Release (can rewrite the Release but rewriting until is lame)
\end{proof}


\section{Linear parity formulas}\label{section:parform}
\begin{definition}\cite[Section 5.1]{venema2024modalmucalculus}\label{def:pargame}
 A parity game is a quadruple $\mathbb{B}=(B_0, B_1, E, \Omega)$ such that $B_0$ and $B_1$ are disjoint sets, and $E\subseteq B^2$, where $B:=B_0\cup B_1$. We will denote $E[p]$ as the set of \emph{admissible} moves for $p\in B$, that is $E[p]:=\{q\in B\mid (p, q)\in E\}$. A position $p\in B$ is a \emph{dead end} if $E[p]=\empty$. If $p\in B$ we let $\tau_p$ denote the (unique) player such that $p\in B_{\sigma_p}$, and say that $p$ \emph{belongs to} $\tau_p$, or that it is $\tau_p$'s \emph{turn} to move at $p$. Initialized board game:
\end{definition}

\begin{definition}\cite[Definition 5.4]{venema2024modalmucalculus}\label{def:match}
 A \emph{path} through a parity game $\mathbb{B}=(B_0, B_1, E, \Omega)$ is a nonempty (finite or infinte) sequence $\pi\in B^\infty$ such tath $E\pi_i\pi_{i+1}$ for every $i$.
\end{definition}
\begin{definition}
 Partial matches
\end{definition}
\begin{definition}\cite[Definition 5.8]{venema2024modalmucalculus}\label{def:winning}
Give a parity game $\mathbb{B}=(B_0, B_1, E, \Omega)$ and a player $\tau$, a $\tau$-\emph{strategy}, is a map $f: \PM_\tau\to B$, and in case we have an initialized board game $\mathbb{B}@q$ then can take the strategy as a map $f:\PM_\sigma(q)\to B$. A match $\pi$ is \emph{guided by} a $\tau$-strategy if for any partial match $\pi'<\pi$ with $\last(\pi')\in B_\tau$, the next position on $\pi$ (after $\pi'$ is equal to $f(\pi')$. \\
A strategy
\end{definition}
\begin{theorem}\cite[Theorem 5.38]{venema2024modalmucalculus}
 Let $\mathcal{B}$ be a parity game. Then $\mathcal{B}$
\end{theorem}
\begin{definition}\label{def:strattree}
 Where $f$ is a surviving player for $\tau$ in $\mathcal{B}@a$, we may represent $f$ as the prunted subtree of the gametree $\mathbb{T}_a^\mathcal{B}$ that is based on those nodes that correspond to $f$-guided matches of $\mathcal{B}@a$. In this so called \emph{strategy tree} $\mathbb{T}_a^f$ we have
 \[
  \stackrel{\rightarrow}{E}_f[p] := \begin{cases}
                                     \stackrel{\rightarrow}{E}[\pi] &\text{if } \pi \in \PM_{\bar{\tau}}\\
                                     \{\pi \cdot f(\pi)\} &\text{if } p \in \PM_{\tau}
                                    \end{cases}
 \]
\end{definition}


\begin{definition}\label{def:stratgraph}
 If $f$ is a surviving positional strategy tree for player $\exists$ in $\mathcal{B}@a$, we may represent $f$ as the pruned subgraph of the game graph (instead of the game tree) of $\mathbb{B}_a^\mathcal{B}$ that is based on those nodes that correspond to $f$-guided matches. In this so called \emph{strategy graph} $\mathbb{B}_a^f$ we have:
 \[
  E_f[p] := \begin{cases}
                                     E[p] &\text{if } p \in B_\forall\\
                                     f(p) &\text{if } p \in B_\exists
                                    \end{cases}
 \]
\end{definition}

\begin{definition}\label{def:linparform}
 We will define a linear parity formula over $\Psf$ almost similar to parity formulas according to definition 6.1 from Venema in \cite{venema2024modalmucalculus}. Let $\mathbb{G}=(V, E, L, \Omega, v_I)$ the same but $L:V\to\mathtt{At}(\Psf)\cup\{\land,\lor,\deo,\epsilon\}$. Still $|E[v]|=1$ if $L(v)=\deo$ and nodes labeled $\deo$ are called modal. Write $E[v]=\{v_1,v_2\}$ if $|E[v]|=2$ and $E[v]=\{v'\}$ if $|E[v]|=1$. The evaluation game $(\mathcal{E}, \sigma)$ is the parity game where the board consists of the set $V\times \omega$, the priority map $\Omega':V\times \omega\to\omega$ is given by
 \[
  \Omega'(v, x):=\begin{cases}
                  \Omega(v)&\text{if }v\in \Dom(\Omega)\\
                  0&\text{else}
                 \end{cases}
 \]
 with the following game graph:\\
 \begin{tabular}{|c|c|c|}
  \hline
  Position & Player & Admissable moves\\
  \hline
  $(v, s)$ with $L(v)=\lor$& $\exists$ & $\{(v_1, s),(v_2, s)\}$\\
  $(v, s)$ with $L(v)=\land$& $\forall$ & $\{(v_1, s),(v_2, s)\}$\\
  $(v, s)$ with $L(v)=\deo$&-&$\{(v', s+1)\}$\\
  $(v, s)$ with $L(v)=\epsilon$&-&$\{(v', s)\}$\\
  $(v, s)$ with $L(v)=p$ and $p\in\sigma(s)$& $\forall$ & $\emptyset$\\
  $(v, s)$ with $L(v)=p$ and $p\notin\sigma(s)$& $\exists$ & $\emptyset$\\
  $(v, s)$ with $L(v)=\bar{p}$ and $p\notin\sigma(s)$& $\forall$ & $\emptyset$\\
  $(v, s)$ with $L(v)=\bar{p}$ and $p\in\sigma(s)$& $\exists$ & $\emptyset$\\
  $(v, s)$ with $L(v)=\top$& $\forall$ & $\emptyset$\\
  $(v, s)$ with $L(v)=\bot$& $\exists$ & $\emptyset$\\
\hline
 \end{tabular}\\
 In this thesis we will refer to these as just parity formulas.
\end{definition}

\begin{definition}\label{def:evalgameparform}
 Let $\sigma: \omega \to \PP(\Psf)$ a linear model and let $\mathbb{G}=(V, E, L, \Omega, v_I)$ a linear parity formula. The \emph{evaluation game} $\mathcal{E}(\mathbb{G}, \sigma)$ is the parity game $(G, E', \Omega')$ with players $\exists$ and $\forall$ of which the board consists of the set $V\times \omega$, the priority map $\Omega': V\times \omega \to \omega$ is given by
 \[
  \Omega'(v, n) := \begin{cases}
                    \Omega(v) & \text{if }v \in \Dom (\Omega)\\
                    0 &\text{else}
                   \end{cases}
 \]
and the game graph is given in definition \ref{def:linparform}.
\end{definition}

\begin{definition}
 We say \(\sigma, n\Vdash \mathbb{G}\) if the pair $(v_I, n)$ is a winning position for $\exists$ in $\mathcal{E}(\mathbb{G},\sigma)$.
 We define the \textbf{language} of a parity formula as:
 \[
  \Mod_\Psf(\mathbb{G}) := \{\sigma \in (\PP(\Psf))^\omega\mid \sigma, 0 \Vdash \mathbb{G}\}
 \]
\end{definition}

\textcolor{red}{Alle definities overnemen over verwijzen naar Yde?}
\begin{definition}
 Let $\mathcal{E}(\mathbb{G}, \sigma)$ the evaluation game of a parity formula $\mathbb{G}=(V, E, L, \Omega, v_I)$. That means $B=V\times \omega$.

 Define the \emph{level} $l$ of the strategy graph as the subgraph of $\mathbb{B}_a^f$ consisting of the nodes of the form $V\times \{l\}$ denote this with $\mathbb{B}_a^f[l]$.\\
 Define the \emph{level} $l$ of the strategy tree as the subtree of $\mathbb{B}_a^f$ consisting of the nodes $\pi\in \PM(a)$ where we have $\last(\pi)\in V\times \{l\}$ denote this with $\mathbb{B}_a^f[l]$.\\
\end{definition}


\section{Automata}\label{section:automata}
\subsection{Nondeterministic parity automata (NPA)}
\begin{definition} (\cite{venema2024modalmucalculus})\label{def:NPA}
A \emph{non-deterministic parity automaton} (NPA) is a quintuple \(\A = (\Sigma,Q, q_I, \delta, \Omega)\) where
\begin{itemize}
 \item $\Sigma$ denotes the alphabet.
 \item $Q$ is a finite set of states.
 \item $q_I\subseteq Q$ the set of initial states.
 \item $\delta: Q\times \Sigma \to \mathcal{P}(Q)$ the transition function.
 \item $\Omega: Q\to \omega$ is the parity map.
\end{itemize}
\end{definition}


For the definition of $\hat{\delta}$, $q\stackrel{a}{\rightarrow}q'$, $q\stackrel{w}{\twoheadrightarrow}q'$, runs and accepted languages ($\LL(\A)$) I refer to \cite{venema2024modalmucalculus}. A run is acepting if \(\max(\inf_\Omega(\rho))\) is even.

\begin{definition}\label{def:partialrun}
 Given a NPA \(\A = (Q, \Sigma, \delta, q_I, \Omega)\) we can extend the definition of $q\stackrel{w}{\twoheadrightarrow}q'$ to include the states visited. Let $w=a_0\dots a_{n-1}$ be a $\Sigma$-word and $\rho=q_0\dots q_n$ a finite list of states. We will write $q_1\twoheadrightarrow_w^\rho q_n$. if $q_i\stackrel{a_i}{\longrightarrow}q_{i+1}$ for all $0\leq i\leq n-1$ and call $\rho$ a \textbf{partial run} in this context.
\end{definition}

\subsection{Alternating Parity Automata (APA)}

\begin{definition}(\cite[Section 14.3.1]{demri2016temporal})\label{def:posboolform}
Define $\mathrm{BF}(X)$ as the set of \emph{positive boolean formulas} over $X$. Given by the following grammar
\[
 \mathcal{F}::= \bot\mid\top\mid x\mid (\mathcal{F}\lor\mathcal{F})\mid (\mathcal{F}\land\mathcal{F})
\]
And we say that: $Y\models x$ iff $x\in Y$, $Y\models \mathcal{F}\lor \mathcal{F}'$ iff $Y\models \mathcal{F}$ or $Y\models \mathcal{F}'$, $Y\models \mathcal{F}\land\mathcal{F}'$ iff $Y\models \mathcal{F}$ and $Y\models \mathcal{F}'$.
\end{definition}
Statement: and we have if $Y\subset Y'$ and $Y\models \mathcal{F}$ then also $Y'\models \mathcal{F}$.
\begin{definition}(\cite[Definition 14.3.2]{demri2016temporal})\label{def:APA}
 An \emph{alternating parity automaton} (APA) is a quintuple $\A=(\Sigma, Q, q_0, \delta, \Omega)$ where
 \begin{itemize}
  \item $\Sigma$ denotes the alphabet.
  \item $Q$ is a finite set of states.
  \item $q_0\in Q$ is the initial state.
  \item $\delta: Q\times \Sigma \to \mathrm{BF}(Q)$ is the transition function.
  \item $\Omega: Q\to\omega$ is the parity map.
 \end{itemize}
\end{definition}

Runs in APA's are defined as directed acyclic graphs that follow the transition function.
\begin{definition}\label{def:aparun}\cite[Definition 14.3.4]{demri2016temporal}
 Let $\A=(\Sigma, Q, q_0,\delta, \Omega)$ be an APA. A \emph{run} on the $\omega$-word $w: \omega\to\Sigma$ is a (possibly infinite) directed acyclic graph $\rho = (V, E)$ satisfying the following conditions.
 \begin{itemize}
  \item[(R1)] $V\subseteq Q\times \omega$ and $(q_0,0)\in V$.
  \item[(R2)] $E\subseteq \bigcup_{l\geq 0} (Q\times \{l\})\times (Q\times\{l+1\})$.
  \item[(R3)] For every $(q,l)\in V\setminus\{(q_0,0)\}$, there is $q'\in Q$ such that $((q',l-1), (q,l))\in E$.
  \item[(R4)] For every $(q, l)\in V$ we have $E_l[q]:=\{q'\mid ((q,l),(q', l+1))\in E\}\models\delta(q,w(l))$.
 \end{itemize}
A node $(q,l)$ is said to be of \emph{level} $l$.
 A run is accepting if for every infinite path $u$ through $\rho$ we have $\max(\inf_\Omega(u))$ is even.
 \textcolor{red}{Definieer die max verzameling}

We call $\rho$ minimal if it satisfies R4' instead of R4.
\begin{itemize}
   \item[(R4')] For every $(q, l)\in V$ we have $E_l[q]:=\{q'\mid ((q,l),(q', l+1))\in E\}\models\delta(q,a_l)$ \emph{and} for no strict subset $Y$ of $E_l[q]$ we have $Y\models\delta(q,w(l))$.
 \end{itemize}
\end{definition}


\section{Stuttering}\label{section:stuttering}
Definitions about stuttering are adapted from \cite{etessami1999stutter}.
\begin{definition}
Two $\omega-$words $w$ and $w'$ over the same alphabet $\Sigma$ are called \textbf{stutter-equivalent} if there exists $f, f': \mathbb{N} \to \mathbb{N}^+$ and a sequence of letters $a_i\in\Sigma$:  $a_0a_1a_2\dots$ such that $w = a_0^{f(0)}a_1^{f(1)}a_2^{f(2)}\dots$ and $w' = a_0^{f'(0)}a_1^{f'(1)}a_2^{f'(2)}\dots$. We denote this as $w\sim_s w'$
\end{definition}
Nog iets toevoegen $w[f]$ als notatie (makkelijker).

\begin{definition}
We call a language \textbf{stutter-invariant} if it holds that if $w\sim_sw'$ then $w\in L\Leftrightarrow w'\in L$
\end{definition}
\begin{definition}
 For a language we define the \textbf{stutter-closure} as the following set
 \[
 L^s = \{w\in \Sigma^\omega\mid w\sim_s w' \text{ for a } w'\in L\}
 \]
\end{definition}


\chapter{Translating a $\mu$-TL formula to a Non Deterministic Parity automaton}
\section{Guardedness and modally simple}
Recall the following definitions about guardedness for $\mu$-ML formulas and parity formulas
\begin{definition}\label{def:guardedmuform}\cite[Definition 2.19]{venema2024modalmucalculus}
 A variable $x$ is guarded in a $\mu$-TL formula $\phi$ if every free occurence of $x$ in $\phi$ is in the scope of a modal operator. A formula $\xi\in\mu$-TL is \emph{guarded} if for every subformula of $\xi$ of the form $\eta x.\delta$, $x$ is guarded in $\delta$.
\end{definition}

\begin{theorem}\label{thm:muformequivguarded}\cite[Proposition 3.27]{venema2024modalmucalculus}
 Every formula $\phi$ is equivalent to a guarded formula $\phi_g$
\end{theorem}
\begin{theorem}\label{thm:nextdistributive}
The $\deo$-operator is distributive. In other words the following equivalences hold:
\begin{enumerate}
%  \item $\deo\neg \phi\equiv \neg \deo\phi$
 \item $\deo (\phi_1\land\phi_2)\equiv \deo\phi_1\land \deo\phi_2$
 \item $\deo (\phi_1\lor\phi_2)\equiv \deo\phi_1\lor \deo\phi_2$
 \item $\deo (\phi_1 U\phi_2)\equiv (\deo\phi_1)U(\deo\phi_2)$
 \item $\deo (\phi_1 R\phi_2)\equiv (\deo\phi_1)R(\deo\phi_2)$
\end{enumerate}
\end{theorem}
\begin{proof}
\begin{enumerate}
 \item We see that
 \begin{align*}
  \sigma, i\Vdash \deo (\phi_1\land\phi_2)&\iff \sigma, i+1\Vdash \phi_1\land\phi_2\\
  &\iff \sigma, i+1\Vdash \phi_1 \text{ and }\sigma, i+1\Vdash \phi_2\\
  &\iff \sigma, i\Vdash\deo \phi_1 \text{ and }\sigma, i\Vdash\deo \phi_2\\
  &\iff \sigma, i\Vdash\deo (\phi_1 \land \phi_2 )\\
 \end{align*}
 \item Observe:
 \begin{align*}
  \sigma, i\Vdash \deo (\phi_1\lor\phi_2)&\iff \sigma, i+1\Vdash \phi_1\land\phi_2\\
  &\iff \sigma, i+1\Vdash \phi_1 \text{ or }\sigma, i+1\Vdash \phi_2\\
  &\iff \sigma, i\Vdash\deo \phi_1 \text{ or }\sigma, i\Vdash\deo \phi_2\\
  &\iff \sigma, i\Vdash\deo (\phi_1 \lor \phi_2 )\\
 \end{align*}
 \item
 \begin{align*}
  \sigma, i\Vdash \deo (\phi_1U\phi_2)&\iff \sigma, i+1 \Vdash (\phi_1U\phi_2)\\
  &\iff \exists j\geq i+1\text{ such that }\sigma, j\Vdash \phi_1 \text{ and } \\
&\phantom{\iff \exists}\sigma, k\Vdash \phi_2 \text{ for all } k \text{ with }i+1\leq k<j\\
&\iff \exists j\geq i\text{ such that }\sigma, j+1\Vdash \phi_1 \text{ and } \\
&\phantom{\iff \exists}\sigma, k+1\Vdash \phi_2 \text{ for all } k \text{ with }i\leq k<j\\
&\iff \exists j\geq i\text{ such that }\sigma, j\Vdash \deo \phi_1 \text{ and } \\
&\phantom{\iff \exists}\sigma, k\Vdash \deo\phi_2 \text{ for all } k \text{ with }i\leq k<j\\
&\iff \sigma, i\Vdash (\deo\phi_1)U(\deo\phi_2)
 \end{align*}
 \item
 \begin{align*}
  \sigma, i\Vdash \deo (\phi_1R\phi_2)&\iff \sigma, i+1 \Vdash (\phi_1R\phi_2)\\
  &\iff \sigma, j \Vdash \phi_1\text{ for all }j\geq i+1\text{ or }\\
  &\phantom{\iff}\exists j\geq i+1\text{ such that }\sigma, j\Vdash \phi_1 \land \phi_2 \text{ and } \\
&\phantom{\iff \exists}\sigma, k\Vdash \phi_2 \text{ for all } k \text{ with }i+1\leq k<j\\
&\iff \sigma, j +1\Vdash \phi_1\text{ for all }j\geq i\text{ or }\\
&\phantom{\iff}\exists j\geq i\text{ such that }\sigma, j+1\Vdash \phi_1 \land \phi_2 \text{ and } \\
&\phantom{\iff \exists}\sigma, k+1\Vdash \phi_2 \text{ for all } k \text{ with }i\leq k<j\\
&\iff \sigma, j \Vdash \deo \phi_1\text{ for all }j\geq i\text{ or }\\
&\phantom{\iff} \exists j\geq i\text{ such that }\sigma, j\Vdash \deo \phi_1 \land \phi_2 \text{ and } \\
&\phantom{\iff \exists}\sigma, k\Vdash \deo\phi_2 \text{ for all } k \text{ with }i\leq k<j\\
&\iff \sigma, i\Vdash (\deo\phi_1)R(\deo\phi_2)
 \end{align*}
\end{enumerate}

\end{proof}
\begin{definition}\label{def:modalsimple}
 A formula $\xi\in\mu$-TL is called modally simple if every occurrence of $\deo$ is immediately followed by another $\deo$-operator, a bound variable, a propositional letter or a fixpoint operator.
\end{definition}
\begin{theorem}\label{thm:muformequivsimple}
 There is an effective procedure rewriting every formula $\xi\in\mu$-TL into an equivelnt simple formula $\xi_s$. Furthermore if $x$ is guarded in $\xi$ then $x$ is guarded in $\xi_s$ as well.
\end{theorem}
\begin{proof}
For this translation define $S : \omega \times \Sub(\xi)$ where $\Sub(\xi)$ is the set of subformulas of $\xi$.
\begin{align*}
 S(n,\deo \phi)          &:=S(n+1, \phi)                  & S(n, l)                 &:= \deo^n l \text{ if } l\in \Lit(\Psf)\\
  S(n, \phi_1\lor\phi_2) &:= S(n, \phi_1)\lor S(n, \phi_2)& S(n, \phi_1\land\phi_2) &:= S(n, \phi_1)\land S(n, \phi_2)\\
  S(n, \phi_1U \phi_2)   &:= (S(n, \phi_1))U(S(n, \phi_2))& S(n, \phi_1R \phi_2)    &:= (S(n, \phi_1))R(S(n, \phi_2))\\
  S(n, \mu x.\phi)       &:= \deo^n\eta x.S(0, \phi)      & S(n, \nu x.\phi)        &:= \deo^n\nu x.S(0, \phi)\\
\end{align*}
By induction on the complexity of $\phi$ we will shot that $S(n,\phi)\equiv \deo^n\phi$ for all $n\in\omega$.

Now we have to prove that $S(0, \xi) \equiv \xi$. Assume for induction that $S(n,\phi)\equiv \deo^n\phi$ for $\phi<\xi$ strict subformula. Now induction step. Cases:
\begin{itemize}
 \item $S(k,\phi_1\land\phi_2)\equiv \deo^k(\phi_1\land\phi_2)$: We see that $S(k, \phi_1)\equiv\deo^k\phi_1$ and $S(k, \phi_2)\equiv\deo^k\phi_2$ so
 \begin{align*}
  S(k,\phi_1\land\phi_2) &= S(k, \phi_1)\land S(k, \phi_2)\\
                         &\equiv (\deo^k\phi_1)\land ()\deo^k\phi_2)\text{ by IH}\\
                         &\equiv \deo^k(\phi_1\land\phi_2)\text{ by distributivity of $\deo$}
 \end{align*}
 \item $S(k,\phi_1\lor\phi_2)\equiv \deo^k(\phi_1\lor\phi_2)$: We see that $S(k, \phi_1)\equiv\deo^k\phi_1$ and $S(k, \phi_2)\equiv\deo^k\phi_2$ so
 \begin{align*}
  S(k,\phi_1\lor\phi_2) &= S(k, \phi_1)\lor S(k, \phi_2)\\
                         &\equiv (\deo^k\phi_1)\lor (\deo^k\phi_2)\text{ by IH}\\
                         &\equiv \deo^k(\phi_1\lor\phi_2)\text{ by distributivity of $\deo$}
 \end{align*}
 \item $S(k,\phi_1U\phi_2)\equiv \deo^k(\phi_1U\phi_2)$: We see that $S(k, \phi_1)\equiv\deo^k\phi_1$ and $S(k, \phi_2)\equiv\deo^k\phi_2$ so
 \begin{align*}
  S(k,\phi_1U\phi_2) &= S(k, \phi_1)US(k, \phi_2)\\
                         &\equiv (\deo^k\phi_1)U (\deo^k\phi_2)\text{ by IH}\\
                         &\equiv \deo^k(\phi_1U\phi_2)\text{ by distributivity of $\deo$}
 \end{align*}
 \item $S(k,\phi_1R\phi_2)\equiv \deo^k(\phi_1R\phi_2)$: We see that $S(k, \phi_1)\equiv\deo^k\phi_1$ and $S(k, \phi_2)\equiv\deo^k\phi_2$ so
 \begin{align*}
  S(k,\phi_1R\phi_2) &= S(k, \phi_1)R S(k, \phi_2)\\
                         &\equiv (\deo^k\phi_1)R (\deo^k\phi_2)\text{ by IH}\\
                         &\equiv \deo^k(\phi_1R\phi_2)\text{ by distributivity of $\deo$}
 \end{align*}
\end{itemize}
\textcolor{red}{Oei, misschien moet ik hier gewoon zeggen dat elke stap volgt uit distributiviteit (moeilijk woord zeg) van $\deo$ maar het is altijd moeilijk in te schatten wat je uit moet schrijven en wat niet..., moet nog even uitgeschreven worden}.
It is easy to see that $S(0, \xi)$ is simple since the only cases where there are $\deo$-operators are in front of fixpoint formulas of in front of literals.
\end{proof}
\begin{definition}\label{def:guardedparform}\cite[Definition 6.62]{venema2024modalmucalculus}
 A path $\pi=v_0v_1v_2\dots v_n$ through a parity formula is \textbf{unguarded} if $n\geq 1$, $v_0, v_n\in \Dom(\Omega)$ while there is no $i$ with $0<i\leq n$, such that $v_i$ is a modal node. A parity formula is \textbf{guarded} if it has no unguarded cycles, and \textbf{strongly guarded} if it has no unguarded paths.
\end{definition}

\begin{definition}
 A parity formula $\mathbb{G}$ is called simple if we have
 \[
  L(v) = \deo \implies L(v')\in \mathtt{At}(\Psf) \text{ or } v'\in \Dom(\Omega) \text{ or }L(v')=\deo
  \]

\end{definition}
\begin{theorem}\label{thm:mutlformURfree}
 Every $\mu$-TL formula $\phi$ can be written without the use of $U$ and $R$ using the following equivalences
 \[
  \phi U\psi \equiv \mu x. (\psi \lor (\phi \land \deo x)) \text { and } \phi R\psi \equiv \nu x. ((\phi\land \psi) \lor (\phi \land \deo x))
 \]
\end{theorem}
\begin{proof}
First $\phi U\psi \equiv \mu x. (\psi \lor (\phi \land \deo x))$ and first the case $\implies$. Now suppose $\exists$ has a winning strategy for $(\phi U\psi, 0)$. Then she chooses $\phi \land \deo \phi U\psi$ untill at some point she chooses $\psi$ since she can only unfold finitely often. At all the intermediate points $\forall$ has no strategy of spoiling. He always chooses $\deo \phi U\psi$. Now call the final point $s$, here $\exists$ chooses $\psi$ and then we know that $\psi$ is true. $\exists$ now has a winning strategy for $\mu x. (\psi \lor (\phi \land \deo x)), 0)$ she always chooses $(\phi \land \deo x)$ upto $s$ where she chooses $\psi$ and she wins since $\sigma, s \Vdash \psi$. Now $\forall$ has to choose $\deo x$ since $\phi$ is true at every point upto $s$. So $\exists$ wins. The proof of the other direction is analogous. For the $R$ case either $\exists$ never chooses $(\phi\land \psi)$ this gives the strategy for $\nu$ to also always choose $(\phi\land \psi)$ or the same argument for an $s$ applies. For the other direction the proof is analogous.
\end{proof}
\begin{theorem}
 Let $\phi$ be a $\mu$-TL formula. Without loss of generality assume $\phi$ is guarded, simple and $U/R$ free. Let $\mathbb{H}_\phi$ the parity formula from theorem \ref{thm:muformtoparform}. This $\mathbb{H}_\phi$ is strongly guarded and simple
\end{theorem}
\begin{proof}
 I will prove this with contradiction. Assume $\mathbb{H}_\phi$ is not strongly guarded. Then there exists a unguarded path $\pi=v_0v_1v_2\dots v_n$ with $n\geq 1$ and $v_0, v_n\in \Dom(\Omega)$. From the construction of $\mathbb{H}_\phi$ we know that the states in $\mathbb{H}_\phi$ correspond to the bound variables in $\phi$ so $v_n$ corresponds to a bounded variable in $\phi$. Since $\phi$ is guarded we know that every bound variable is in the scope of a $\deo$-operator and because $\phi$ is simple we know that this $\deo$-operator is immediately in front of the bound variable. Therefore we know that $v_{n-1}$ corresponds to a $\deo$-operator so $v_{n-1}$ is a modal node which contradicts the fact that $\pi$ is an unguarded path. \\
 $\mathbb{G}$ is simple since we know that $V$ is based on the subformula graph and $\phi$ is simple. The only states are the nodes that correspond with the bound variables and since every bound variable is preceded by a modal node we know that $\mathbb{G}$ is simple.
\end{proof}


\section{From $\mu$-TL to parity formulas}
\begin{theorem}
 There is an algorithm that constructs, for a clean formula $\xi\in \mu$-ML$(\Psf)$, an equivalent parity formula $\mathbb{H}_\xi = (V_\xi, E_\xi, L_\xi, \Omega_\xi, \xi)$, based on the subformula graph of $\xi$
\end{theorem}
\begin{proof}
 Eerst constructie beschrijven. Dan zeggen we dat de boards van beide games gelijk zijn en ook de parities dus equivalent. Dan kan je ook citeren naar evt Yde en Wilke.
\end{proof}
This will be done
\textbf{Discussion}: Theorem 6.35 from \cite{venema2024modalmucalculus} works with $\mu$-ML formulas. Now we can adjust this definition to work with $\mu$-TL formulas that are $U/R$ free and transform this to linear parity formulas. The proof will follow the proof of Venema but has to be worked out in a later stadium (also discussing how much I can refer to Venema and how much I have to show myself in the thesis).

\section{Translation of parity formulas to alternating parity automata}
\begin{definition}\cite{demri2016temporal}
Let $\mathbb{G}$ a parity formula over $\Psf$. Define the \textbf{language} of a parity formula as:
 \[
  \Mod_P(\mathbb{G}) := \{\sigma \in (\PP(P))^\omega\mid \sigma, 0 \Vdash \mathbb{G}\}
 \]
\end{definition}

\textcolor{red}{Ik weet nog niet zo goed hoe ik deze strategy tree precies ga gebruiken, kan ook kijken naar de gewone strategy tree zoals jij hem definieert maar dan met de opmerking dat je alleen maar naar de laatste toestand hoeft te kijken. Alleen op deze manier kan je hem wel heel makkelijk een op een vertalen naar een DAG (run)}
\begin{proposition}
 Knowledge about strategy tree generalize to knowledge about positional strategy tree. In particular the fact that a branch in the tree is a match in the game.
\end{proposition}
\begin{theorem}\label{thm:parformtoapa}
 There is an algorithm that constructs, for a strongly guarded and simple parity formula $G$ over $\Psf$ into an equivalent APA $\A$ such that $\Mod_\Psf(\mathbb{G}) = \LL(\A)$
\end{theorem}
\begin{proof}
 Let $\mathbb{G}=(V, E, L, \Omega_\mathbb{G}, v_I)$ a strongly guarded and simple parity formula. We will define $\A=(\Sigma, Q, q_0, \delta, \Omega')$ as follows:
 \begin{itemize}
  \item Set the alphabet $\Sigma := \PP(\Psf)$
  \item Define the the set of states $Q$ as $Q:= V$
  \item Define $q_0:=v_I$
 \end{itemize}
 To define the transition function $\delta$ let $q\in Q$ and $a\in\Sigma (\PP(\Psf))$ and set:
\begin{align*}
   \delta(v, a) &:= \top        \text{ if } L(v)=p\text{ and } p\in a & \delta(v,a)&:= \delta(v_1,a)\land\delta(v_2, a)\text{ if } L(v) = \land\\
   \delta(v, a) &:= \bot        \text{ if } L(v)=p\text{ and } p\notin a&\delta(v,a)&:= \delta(v_1,a)\lor\delta(v_2, a)\text{ if } L(v) = \lor\\
   \delta(v, a) &:= \top \text{ if } L(v)=\overline{p}\text{ and } p\notin a&\delta(v, a)&:= v'\text{ if } L(v) = \deo\\
   \delta(v, a) &:= \bot \text{ if } L(v)=\overline{p}\text{ and } p\in a &\delta(v, a) &:= \top        \text{ if } L(v)=\top\\
   \delta(v, a) &:= \delta(v', a)\text{ if } L(v)=\epsilon&\delta(v, a) &:= \bot        \text{ if } L(v)=\bot
  \end{align*}
  For the priority map define:
  \[
  \Omega_\mathbb{A}(v) = \begin{cases}
                \Omega(v) &\text{if }v\in \Dom(\Omega)\\
                0&\text{else}
               \end{cases}
  \]
  \textbf{Observation:} if $(v, l)$ is a dead end for $\forall$ then $\delta(v, a_l)=\top$\\
\textbf{Claim:} $\Mod_\Psf(\mathbb{G}) = \LL(\A)$\\
\textit{Proof}: First $\Mod_\Psf(\mathbb{G}) \subseteq \LL(\A)$. Suppose $w=a_0a_1a_2\dots \in \Mod_\Psf(\mathbb{G})$ then we know that $(v_I, 0)$ is a winning position for $\exists$. This means there is a winning strategy $f$ for $\exists$ and since $\mathbb{G}$ is a parity game we even know that there exists a positional winning strategy $f$ for $\exists$ in $\mathcal{B}@(v_I,0)$. Now take the strategy graph $\mathbb{B}_a^f$, we are going to construct a succesfull run in $\A$ based on this positional strategy tree. Recall that a run in $\A$ is a DAG $\rho=(V_\rho,E_\rho)$ where $\rho$ satisfies four conditions R1-R4. Define the nodes
\[
 V_\rho = \{(v_I, 0)\}\cup\bigcup_{i\geq 1}\left\{(v, i)\mid \exists v'\in V, L(v')=\deo, \{v\}=E[v'], (v', i-1)\in \mathbb{B}_{(v_I,0)}^f \right\}
\]
and the edges
\[
 E_\rho = \bigcup_{i\geq 0}\left\{((v, i), (v', i+1))\in V_\rho\times V_\rho\mid (v', i+1)\in\mathbb{B}_{(v,i)}^f \right\}
\]
Now we have to check that this is indeed a run. Satisfaction of R1-R3 is an easy verification since $\mathbb{G}_{(v_I,0)}^f$ is already a graph. \textcolor{red}{Werk dit even netjes uit}. For the satisfaction of R4 we need to dive in the definition of $\delta$. \\
\textcolor{red}{Naam van $E_q$ moet anders!}
\textbf{Claim:} For every $(q, l)\in V_\rho$ we have $E_q=\{q'\mid ((q, l), (q', l+1))\in E\} \models \delta(q, a_l)$\\
\textit{Proof}: We are going to look at the level $\mathbb{B}_{(q, l)}^f[l]$ of the game graph. Since $\mathbb{G}$ is (strongly) guarded we know that this level is a DAG where the leafs (hebben we leafs?) correspond to literal or modal nodes. Now I am going to proof this with an iteration on the topological ordering of $\mathbb{B}_{(q, l)}^f[l]$, call this ordering $w_0w_1\dots w_n$. \textcolor{red}{Hoe nu inductiehypothese benoemen??}.\\
\textbf{Base case: $w_n$}. We know that $w_n$ has no outgoing edges, so it must correspond to a dead end. Since $f$ is a winning strategy for $\exists$ we know that $\forall$ should be at move. Therefore we now that
$\delta(w_n,a_l)=\top$ so clearly $E_q\models \delta(w_n, a_l)$. \\
\textbf{Induction step:} Take $w_i, i<n$. Here we distinguish four different cases:
\begin{itemize}
 \item[Case:] \(E[w_i]=0\) In this case $w_i$ is a dead end for $\forall$ in the game graph so we know that $\delta(w_i, a_l)=\top$.
 \item[Case:] $L(w_i)=\land$. In this case we have $\delta(w_i, a_l) = \delta(w_j, a_l) \land \delta(w_k, a_l)$, since we have a topological ordering we know that $i<j$ and $i<k$ so per the induction (iteration) hypothesis we know that $E_q\models \delta(w_j, a_l)$ and $E_q\models\delta(w_k, a_l)$ so that means $E_q\models \delta(w_i, a_l)$.
 \item[Case:] $L(w_i)=\lor$. We see that $\delta(w_i, a_l)= \delta(v_1, a_l)\lor \delta(v_2, a_l)$ Since $\mathbb{B}_{(q, l)}^f[l]$ is a level of the strategy graph we know that $E_f[w_i] = f(w_i)=v_m, m\in {1, 2}$. Since we have a topological ordering we know that $v_m=w_j$ for a $j>i$. Therefore we know that $E_q\models \delta(v_j, a_l)$ so also $E_q\models \delta(w_i, a_l)$ \textcolor{red}{Oei de indices zijn hier wel heel erg verwarrend}.
 \item[Case:] $L(w_i)=\deo$. Here we see that $\delta(w_i, a_l) = v'$ where $E[w_i]=\{v'\}$. Since $(w_i, l)\in \mathbb{B}_{(v_I, 0)}^f$ we know that $(v', l+1)\in V_\rho$. Also since $(w_i, l)\in \mathbb{B}_{(q, l)}^f$ we see that $(v', l+1)\in \mathbb{B}_{(q, l)}^f$ so therefore $v'\in E_q$ so we see that $E_q\models\delta(w_i, a_l)$.
\end{itemize}
Now we have iterated over the the whole level $\mathbb{B}_{(q, l)}^f[l]$ so we see that $E_q \models \delta(q, a_l)$.
\qed.\\

% I will prove this via induction on the complexity on $\delta(q,a_l)$. Since $\mathbb{G}$ is guarded there are no loops so we cannot loop infinitely over $\delta$ and get a fixpoint. \textcolor{red}{IH goed formuleren, iets met $\delta$ en $E[v]$, en hier heb je dus $v\leq q$ eigenlijk in de parityformuleboom}: Induction hypo: for every $v\in \mathbb{T}^f_{(q, l)}$ we have $E_v\models (v, a_l)$ The base cases are: $L(v)=\deo$ or $L(v)\in\mathtt{At}(\Psf)$. Now if $L(v
% )=\deo$ we see that $\delta(v, a_l)= v'$. Also we see that $(v', l+1)\in V$ since $L(v)=\deo$, $(v, l)\in \mathbb{T}_{(v_I, 0)}$ and $\{v'\} =E[v]$. Secondly $((q,l), (v', l+1))\in E$ since $(v, l)\leq (q,l)$ and $E[v]=v'$ so we see that $(v', l+1)\in \mathbb{T}_{(q, l)}^f$ since $v\in \mathbb{T}^f_{(q, l)}$. So $\{q'\mid ((q, l), (q', l+1))\in E\} \models \delta(v, a_l)$. If $L(v)\in \mathtt{At}(\Psf)$ we know that this should be a dead end where it is $\forall$'s turn since $f$ is a winning strategy. So $\delta(v, a_l)=\top$ so every set satisfies this transition function. \\
% Now for induction assume $\delta(v, a_l) = \delta(v_1, a_l)\land \delta(v_2,a_l)$, we see that both $v_1$ and $v_2$ are in $\mathbb{T}^f_{(q, l)}$ so we can apply the induction hypothesis. If we have $\delta(v, a_l) = \delta(v_1, a_l)\lor \delta(v_2,a_l)$ choose the node that is in $\mathbb{T}^f_{(q, l)}$ and apply the induction hypothesis. If we have $\delta(v, a_l)=\delta(v', a_l)$ we have $v'\in \mathbb{T}^f_{(q, l)}$ so we can apply the induction hypothesis. Now we see that $E_q\models \delta(q, a_l)$.\qed

Now we have to proof that $\rho=(V,E)$ is indeed an accepting run. That means that for every infinite path $w$ through $\rho$ we have that $\max\{\Omega(q)\mid q\in \inf(u)\}$ is even. To to that we will prove a correspondence between infinite paths in $\rho$ and infinite matches in $\mathcal{E}(\mathbb{G}, \sigma)@(v_I, 0)$\\
\textbf{Claim:} Every infinite path $u$ in $\rho$ corresponds \textcolor{red}{Wat bedoel je met correspondence precies} to a $f$-guided infinite match $\pi$ in $\mathcal{E}(\mathbb{G}, \sigma)@(v_I, 0)$ where we have $\{\Omega_\mathbb{A}(v)\mid v\in u\}\cup\{0\}= \{\Omega_{\mathcal{E}}(p)\mid p\in \pi\}$, \textcolor{red}{En benoem eerste state in $\pi$}. \\
\textit{Proof:}\\
An infinite path $u$ consists of states $q_0q_1q_2\dots$ where $q_0 = v_I$ and we have $q_{i+1}\in E_{q_i}$ for every $i$. Now construct the infinite match $\pi = (q_0,0)\dots (q_1,1)\dots (q_2, 2)\dots (q_3,3)\dots$, we see that this is an infinite match since when $q_{i+1}\in E_{q_i}$ we know that $q_{i+1}\in\mathbb{B}_{(q,i)}^f$ so there is a finite $f$-guided path between $(q_i,i)$ and $(q_{i+1},i+1)$. Now we have to show that $\{\Omega_\mathbb{A}(v)\mid v\in u\}\cup\{0\}= \{\Omega_{\mathcal{E}}(p)\mid p\in \pi\}$. First observe that $\Omega_\mathcal{E}((q_i,i)) = \Omega_\A (q_i)$ since they both correspond to \(
  \Omega_\mathbb{A}(v) = \begin{cases}
                \Omega(v) &\text{if }v\in \Dom(\Omega)\\
                0&\text{else}
               \end{cases}
  \).
  Now take a node $(\pi_j, i)$ in $\pi$ inbetween $(q_i, i)$ and $(q_{i+1}, i+1)$. My claim is that $\Omega_{\mathcal{E}}(\pi_j, i) = 0$. Suppose not, then we would have $v\in \Dom (\Omega_\mathbb{G})$. But since $\mathbb{G}$ is strongly guarded and simple we know that $v$ should be in the scope of a modal operator and that the modal operator is directly preceding this node. That would mean that $L(\pi_{j-1})=\deo$ and that would mean that $\pi_j$ should be in level $i+1$ which contradicts the fact that $(\pi_j, i)$ is chosen inbetween $(q_i, i)$ and $(q_{i+1}, i+1)$. That proves that $\{\Omega_\mathbb{A}(v)\mid v\in u\}\cup\{0\}= \{\Omega_{\mathcal{E}}(\pi_i)\mid \pi_i\in \pi\}$ since the only non-zero nodes in $\pi$ corresponds to states in $\mathbb{G}$.
\qed \\ Since every infinite path in $\rho$ corresponds to an infinite $f$-guided match and since $(v_I, 0)$ is a winning position we know that $\max\{\Omega_\mathcal{E}(\pi_i)\mid \pi_i\in\inf(\pi)\}$ is even and since $\{\Omega_\mathbb{A}(v)\mid v\in u\}\cup\{0\}= \{\Omega_{\mathcal{E}}(p)\mid p\in \pi\}$ we also know that $\max\{\Omega(q)\mid q\in \inf(u)\}$ is even. So $\rho$ is an accepting run for $w$ in $\A$.  So now we know that if $w\in \Mod_\Psf(\mathbb{G})$ then $w\in \LL(\A)$\\
%
% That means that every infinite path in $\rho$ is accepting. We are working with the parity condition so it means that for every $w\in Q^\omega$ that defines a path through $\rho$ we have $\max\{\Omega(q)\mid q\in w\}$ is even. We see that an infinite path through the strategy tree corresponds to an infinite word through $\rho$. We also know that since $\mathbb{G}$ is guarded and simple that every state is immediately preceded by a modal node. So that means that if we have an infinite word through $\rho$ that $\Omega'(\inf(w_\rho))=\Omega'(\inf(\pi))\cup\{0\}$ where $\pi$ is the infinite match that corresponds to $w_\rho$ \textcolor{red}{Uitleggen correspondentie??}. And since $\max(\Omega'(\inf(\pi)))$ is even we know that $\max(\Omega'(\inf(w_\rho)))$ is even as well so that means that $\rho$ is accepting so that $w\in \LL(\A)$\\


Now the other direction $\Mod_\Psf(\mathbb{G}) \supseteq \LL(\A)$: Suppose we have $w\in \\(\A)$ then there exists an accepting run $\rho=(V_\rho, E_\rho)$ on $w$, without loss of generality assume this run is minimal \textcolor{red}{Hier dan verwijzen naar stelling?}. We have to create a winning strategy for $\exists$ in $\mathcal{B}@(v_I,0)$. We are going to create a non-positional strategy $f$ such every finite path in the strategy tre $\mathcal{B}$ ends in a dead end for $\forall$ and every infinite path has an even maximum parity. \\
First we are going to define $f$, to define $f$ we are inductively going to build op the strategy tree $\mathbb{T}_{(v_I, 0)}^f$. \\
For every $(q, l)\in V_\rho$ we are going to prove with induction on the depth of $v$ that for every $(v,l)\in \mathbb{T}^f_{(q,l)}[l]$ we have $E_q\models \delta(v, a_l)$, and that in ever dead end it is $\forall$'s turn to play. We define $E_q:=\{q'\mid ((q, l), (q', l+1))\in E\} \models \delta(q, a_l)$ as before. \\
Induction hypothesis: for every every $\pi$ with depth $k$ we have if $last(\pi)=(v,l)$ $E_q\models \delta(v, a_l)$.\\
Base case: $k=0$ since $\mathbb{T}^f_{(q,l)}[l]$ is a tree with one root we now that the only $\pi$ with depth $0$ has $last(\pi)=(q,l)$. Since $\rho$ is a run we know by R4 that $E_q\models \delta(q, a_l)$. If $L(q)=\lor$ we have to choose between $v_1$ and $v_2$. We know that $\delta(q, a_l)=\delta(v_1,a_l)\lor \delta(v_2, a_l)$, so $E_q\models \delta(v_1, a_l)$ or $E_q\models\delta(v_2, a_l)$ since we know that $\rho$ is minimal either we know that exclusively one is true, in that case choose $f(\pi) = v_i$ where $v_i$ is true or that $||\delta(v_1, a_l)||=||\delta(v_2, a_l)||$, in that case choose $v_1$.\\
Inductive case: $k>0$. Take a $\pi\in \mathbb{T}^f_{(q, l)}[l]$ with $last(\pi)=(v, l)$. First look at the parent of $\pi$, call this $\pi_p$ (it has depth $k-1$ so we can apply the induction hypothesis) if $last(\pi_p)=(v', l)$ we know that $E_q\models\delta(v, a_l)$. Now cases:\\
Case $L(v')=\lor$, in this case we know $\delta(v, a_l) =\delta(v_1, a_l)\lor \delta(v_2, a_l)$ (IH) so exactly one of $E_q\models\delta(v_1, a_l)$ and $E_q\models\delta(v_2, a_l)$ is true of the \textcolor{red}{valuations} are the same. We know that $f(\pi_p)=v_i$ where $\delta(v_i, a_l)$ is true so that means $v=v_i$ so $E_q\models\delta(v, a_l)$\\
Case $L(v') = \land$, in this case we know $\delta(v, a_l) =\delta(v_1, a_l)\land \delta(v_2, a_l)$ (IH) where $v=v_1\lor v=v_2$ so $E_q\models\delta(v, a_l)$\\
The case $L(v')=\deo$ is not possible since then $\pi$ would be in the next level. This is a leaf that we have to connect to another level.\\
Now if $L(v)=\lor$ we have to make a choice for $f(\pi)$, make the same choice as above in the base case.\\
If $L(v)\in Lit(\Psf)$ we know that $\delta(v, a_l)\in \{\top, \bot\}$ since $E_q\models \delta(v, a_l)$ we know that $\delta(v, a_l)=\top$. Then that means that this is a dead end where $\forall$ is at move.\\
Conclusion: For every $\pi \in \mathbb{T}_{(q, l)}^f[l]$ if $last(\pi)=(v,l$ we have $E_q\models \delta(v, l)$ and every dead end in $\mathbb{T}_{(q, l)}^f[l]$ belongs to $B_\forall$.

So now we have created a lot of levels in the strategy tree but we have to make sure that these connect to each other in a well defined strategy tree. We have to check for every leaf in every $\mathbb{T}_{(q, l)}^f[l]$ that corresponds to $L(v)=\deo$ that there is a strategy tree $\mathbb{T}_{(v', l)}^f[l+1]$.\\
\Claim: we can connect all strategy trees to form a well defined strategy tree for $\mathcal{E}(\mathbb{G}, \sigma)@(v_I, 0)$.\\
\Proof: Fist of all we know that $(q, 0)= (q_0, 0) = (v_I, 0)$ so we have $\mathbb{T}_{(v_I, 0)}^f[0]$. Now take an arbitrary $\mathbb{T}_{(q, l)}^f[l]$ and an arbitrary leaf $\pi$ with $last(\pi)=(v, l)$ in $\mathbb{T}_{(q, l)}^f[l]$. We know that $L(v)=\deo$, $E[v]=\{v'\}$ and $\delta(v, a_l)=v'$. By the above claim we know that $E_q\models \delta(v, a_l)=v'$ so $v'\in E_q$ so therefore we know that $(v', l+1)\in V_\rho$. That means that we can take $\mathbb{T}_{(v', l+1)}^f[l+1]$ from the above statement so this leaf is connected.\\
Conclusion: we can connect every leaf that is not a dead ind in every level to another level that we created so $\mathbb{T}_{(q, l)}^f$ is well-defined.\\
Now we have to make sure that this strategy tree indeed defines a winning strategy for $\exists$ in $\mathcal{E}(\mathbb{G}, \sigma)@(v_I, 0)$. This comes in two parts: first every finite match ends in a dead end for $\forall$ and for every infinite match we have that the maximum parity is even.\\
\Claim: Every finite match in $\mathbb{T}_{(v_I,0)}^f$ ends in a dead end for $\forall$\\
\Proof: We see that every finite match in $\mathbb{T}_{(v_I,0)}^f$ ends in a leaf $\pi$ in some level $\mathbb{T}_{(q,l)}^f[l]$, if $last(\pi) = (v,l)$ we know that $L(v)\in \Lit(\Psf)$ so from the induction claim we know that this is indeed a dead end where it is $\forall$'s turn.\\
\Claim: For every inifinite match $\pi$ in $\mathbb{T}_{(v_I,0)}^f$ we have $\max(\inf_{\Omega_\mathcal{E}} (\pi))$ is even.\\
\Proof: First we are going to correspond matches to paths in the run. \\
\textbf{Claim:} Every infinite $f$-guided infinite match $\pi$ in $\mathcal{E}(\mathbb{G}, \sigma)@(v_I, 0)$ corresponds to an infinte path $u$ in $\rho$ where we have $\{\Omega_\mathbb{A}(v)\mid v\in u\}\cup\{0\} = \{\Omega_{\mathcal{E}}(p)\mid p\in \pi\}$, \textcolor{red}{En benoem eerste state in $\pi$}. \\
\Proof: Take a $f$-guided match $\pi=(v_0, l_0)(v_1, l_1)(v_2, l_2)\dots$ in $\mathcal{E}(\mathbb{G}, \sigma)@(v_I, 0)$. Now mark $q_0=(v_0, l_0)$ and mark $q_i$ as the $i$'th node $(v_j, l_j$ in $\pi$ where $L(v_{j-1})=\deo$. Construct the infinite path $u=q_0q_1q_2\dots$ in $\rho$.\\
\Claim: $u$ is an infinite path in $\rho$.\\
\Proof: We have to show that for every $q_i=(v_j, l_j)$ we have $q_{i+1}=(v_{j'}, l_{j'})$ and $v_{j'}\in E_{q_i}$. We know that $(v_{j'-1}, l_{j'-1})=\last(\pi)$ for a $\pi\in \mathbb{T}_{(v_j,l_j)}^f$ and $L(v_{j'-1})=\deo$ so $\delta(v, a_l)=v_{j'}$. From the induction claim we have $E_q\models\delta(v_{j'-1})$ so $v_{j'}\in E_{q_i}$ which proofs this claim. \qed\\
Now we easily see that for $q_i=(v_j, l_j)$ we have  $\Omega_\mathcal{E}((v_j, l_j)) = \Omega_\A (v_j)$ since they both correspond to \(
  \Omega_\mathbb{A}(v_j) = \begin{cases}
                \Omega_{\mathbb{G}}(v_j) &\text{if }v\in \Dom(\Omega)\\
                0&\text{else}
               \end{cases}
\). Now take a node $(v_j, l_j)$ in $\pi$ strictlt inbetween $q_i$ and $q_{i+1}$. My claim is that $\Omega_{\mathcal{E}}(v_j, l_j) = 0$. Suppose not, then we would have $v\in \Dom (\Omega_\mathbb{G})$. But since $\mathbb{G}$ is strongly guarded and simple we know that $v$ should be in the scope of a modal operator and that the modal operator is directly preceding this node. That would mean that $L(v_{j-1})=\deo$ and that would mean that $(v_j, l_j)$ should be in level $l_j+1$ which contradicts the fact that $q_{i+1}$ is chosen as the first node after $q_i$ that follows a modal node. That proves that $\{\Omega_\mathbb{A}(v)\mid v\in u\}\cup\{0\}= \{\Omega_{\mathcal{E}}(\pi_i)\mid \pi_i\in \pi\}$ since the only non-zero nodes in $\pi$ corresponds to states in $\mathbb{G}$.\qed\\
Now we know that for every infinite $f$-guided infinite match $\pi$ there is a infinite path $u$ in$\rho$. Since $\rho$ is an accepting run we know that $\max(\inf_{\Omega_\A}(u))$ is even and since $\{\Omega_\mathbb{A}(v)\mid v\in u\}\cup\{0\} = \{\Omega_{\mathcal{E}}(p)\mid p\in \pi\}$ we then also know that $\max(\inf_{\Omega_\mathcal{E}}(\pi))$ is even, which proofs the claim.\qed\\

Now we know that $f$ is a winning strategy for $\exists$ in $\mathcal{E}(\mathbb{G}, \sigma)@(v_I, 0)$ so we know that $w\in \Mod_\Psf(\mathbb{G})$) which proofs that indeed

% Now take a node $(v_j, l_j)$ in $\pi$ inbetween a $q_i$ and $q_{i+1}$.
%
% to do this we will make a strategy tree. Take the game tree, we have to prune it so that we have a strategy for $\exists$ essentially we have to make a choice for every $v$ where $L(v)=\lor$. Inductively from the top go downward. If there is a place where $\delta(v, a_l)=\delta(v_1, a_l)\lor \delta(v_2, a_l)$ where $v\leq q$ so if $E_q\models \delta(q, a_l)$ it should either models $\delta(v_1, a_l)$ or $\delta(v_2, a_l)$. Now set $f((v,l))=v_i$ where $E_q\models \delta(v_i, a_l)$. What we now see is that the levels in the run correspond to states $(v', i)$ where we have $L(v)=\deo$ and $(v, i-1)\in \mathbb{T}_{(v_I,0)}^f$. Now we have to proof that this gives a winning strategy tree. We have to check that every finite match ends in a dead end for $\forall$ and that every infinite match hase even infinite priority. For the finite case we look at every level. So we have $E_q\models \delta(q, a_l)$ now we look inductively at this $\delta$ function. For every $(v, i)\in \mathbb{T}^{temp}_{(q, i)}$ If it has the form $\delta(v, a_l)=\delta(v_1, a_l)\land \delta(v_2, a_l)$ we know that both will be true. And if it has the form of $\delta(v, a_l)=\delta(v_1, a_l)\lor \delta(v_2, a_l)$ that at least one of them is true and that is the path added to $\mathbb{T}$. Now if we have $\delta(v, a_l) = \top$ we know that this corresponds to a dead end for $\forall$, $\bot$ is not allowed. Now if we have $\delta(v, a_l)=v'$ we are in a next level. From there on we will check for finiteness and if this path does not stop somewhere it is infinte. We also see that infinite matches correspond to infinite paths in $\rho$. Following the reasoning above we see that the max parity is even iff. So now we have shown that
\[
 \Mod_\Psf(\mathbb{G}) = \LL(\A)
\]
so $\A$ is equivalent to $\mathbb{G}$.
% \textcolor{red}{Allemaal nog iets netter, dat is echt best moeilijk zeg}
\end{proof}


\section{Translation of APA to NPA}
% \begin{definition}\label{def:APAtoNPA}
 Given an (ATPA) $\mathbb{A}=(\Sigma, Q, q_0,\delta, \Omega)$ define the (equivalent) NPA $\mathbb{A}'=(\Sigma, Q', q_0',\delta', \Omega')$ as follows:
 \begin{itemize}
  \item $Q' = \mathcal{P}(Q\times \Omega(Q)) \times \{0, 1\}$
  \item $q_0' \subseteq Q' = \begin{cases}
               \{(\{(q_0, \Omega(q_0)\}, 0)\} &= \text{if } \Omega(q_0) \text{ is even}\\
               \{(\{(q_0, \Omega(q_0))\}, 0), (\{(q_0, 0)\}, 1)\}&= \text{else}
               \end{cases}$
  \item $\delta' : Q'\times \Sigma \to \mathcal{P}(Q')$ The transition function $\delta'((X, p), a)$ has two cases:\\
  \textbf{Final} When for all $(q, p)\in X$ it holds that $p$ is even then all paths have visited an even parity (and that is also maximal) so we reset. Now set \\
  \begin{multline*}
   \delta'((X, p), a) = \biggl\{ (Y, 0) |Z\subseteq Q, \text{ such that for every } (q, p_q)\in X, \text{ holds } Z\models \delta(q, a)\land \\
   \{q|(q, p_q)\in Y\}=Z
   \land \text{ for every } (q, p_q)\in Y \text{ holds }
   p_q=\Omega(q)\biggr\}\cup\\
   \Biggl\{
   (Y, 1) |Z\subseteq Q, \text{ such that for every } (q, p_q)\in X, \text{ holds } Z\models \delta(q, a)\land \\
   \left\{q|(q, p_q)\in Y\right\}=Z
   \land \text{ for every } (q, p_q)\in Y \text{ holds }
   p_q=\begin{cases}0&\text {if }\Omega(q) \text{ odd } \\ \Omega(q)&\text{else}\end{cases}
   \Biggr\}
  \end{multline*}
  \textbf{Non-final} When for some $(q, p)\in X$ it holds that $p$ is odd then not all paths have visited a maximum even parity so we go further. Now set\\
\begin{multline*}
   \delta'((X, p), a) = \Biggl\{ (Y, 0) |Z\subseteq Q, \text{ such that for every } (q, p_q)\in X, \text{ holds } Z\models \delta(q, a)\land \\
   \{q|(q, p_q)\in Y\}=Z
   \land \text{ for every } (q, p_q)\in Y \text{ holds } \\
   p_q=\min\{\max(p_{q'}, \Omega(q))|\exists (q', p_{q'})\in X, \exists Z_{q'}\subseteq Z, \text{ With $Z_{q'}$ minimal }, q\in Z_{q'}\land Z_{q'}\models \delta(q', a)\}\Biggr\}\cup\\
   \Biggl\{ (Y, 1) |Z\subseteq Q, \text{ such that for every } (q, p_q)\in X, \text{ holds } Z\models \delta(q, a)\land \\
   \{q|(q, p_q)\in Y\}=Z\text{ for every } (q, p_q)\in Y \text{ holds } \\
   p_q=\min\Biggl\{\max\Biggl(p_{q'}, \begin{cases}0&\text {if }\Omega(q) \text{ odd } \\
   \Omega(q)&\text{else}\end{cases}\Biggr)|\exists (q', p_{q'})\in X, \exists Z_{q'}\subseteq Z\\
   \text{ With $Z_{q'}$ minimal }, q\in Z_{q'}\land Z_{q'}\models \delta(q', a)\Biggr\}\Biggr\}
  \end{multline*}
  \textcolor{red}{ do define minimal (easy but has to be done)}
  \item Now for $\Omega' : Q'\to \mathbb{N}$ define
  \[
   \Omega'(X, p) =\begin{cases}
                   3 &\text{if } p = 1\\
                   2 &\text{if for all } (q, p_q)\in X \text{ holds } p_q \text{ even }\\
                   1&\text{else}
                  \end{cases}
  \]

 \end{itemize}
\end{definition}
\begin{theorem}\label{thm:APAtoNPA}
The NPA indeed recognizes the same language as the APTA in other words: $\mathcal{L}(\mathbb{A})=\mathcal{L}(\mathbb{A}')$
\end{theorem}

\begin{proof}
First we prove \(\mathcal{L}(\mathbb{A})\subseteq \mathcal{L}(\mathbb{A}')\)\\
Suppose we have a minimal run $\rho=(V, E)$ on $w$ in the APTA $\mathbb{A}$ now we have to construct a run $\rho'=q'_0q'_1q'_2\dots$ in the NPA $\mathbb{A}$. Define
\[q'_0 = \begin{cases}
               (\{(q_0, \Omega(q_0))\}, 0) &\text{ if } \Omega(q_0)\leq \min\max \Omega \text{ of all infinite paths in }\rho\\
               (\{(q_0, 0)\}, 1) &\text{ if } \Omega(q_0)> \min\max \Omega \text{ of all infinite paths in }\rho\\
              \end{cases}
\]
For $i>0$ we have two cases. If the previous state was final (that means that every state had a even parity attached to it):
\[
 q'_i = \begin{cases}
            (\{(q, \Omega(q))|(q, i)\in V\}, 0) \text{ if all }\Omega(q) \leq \min\max \Omega \text{ of all infinite paths in }\rho\\
            \left(\left\{(q, p_q)|(q, i)\in V, \text{ where } p_q=\begin{cases}
            0 & \text{ if } \Omega(q) > \max \Omega \text{ on any infite path passing }q\\
            \Omega(q) & \text{ else }
            \end{cases}\right\}, 1\right)&\text{ else }
            \end{cases}
\]

Otherwise if the previous state is non-final define:
\[
 q'_i = \begin{cases}
            (\{(q, p_q)|(q, i)\in V, p_q=\min\{\max(\Omega(q), p_{q'})|((q', i-1), (q,i))\in E\}\}, 0)&h\\\text{ if all }\Omega(q) \leq \min\max \Omega \text{ of all infinite paths in }\rho\\
            \Biggl(\Biggl\{(q, p_q)|(q, i)\in V, \text{ where }\\
            p_q=\begin{cases}
            \min\{p_{q'}|((q', i-1), (q,i))\in E\} & \text{ if } \Omega(q) > \max \Omega \text{ on any infite path passing }q\\
            \min\{\max(\Omega(q), p_{q'})|((q', i-1), (q,i))\in E\} & \text{ else }
            \end{cases}
            \Biggr\}\\, 1\Biggr) \text{ else }

            \end{cases}
\]
\textcolor{red}{To do: prove that this is indeed a run on $w$ in $\mathbb{A}'$ (this is per definition of $\delta'$).}\\
Now if $\rho$ is an accpeting run on $w$ in $\mathbb{A}$ that means that on any infinite path the maximum parity is even. So therefore there will be an infinite amount of times that alle states have an even $p_q$ so therefore we see that the max of the $\omega$ inf of $\rho'$ is at least 2. It cannot be $3$ since then there would be an infinite number of times a state with a bigger odd parity then all it's infinite paths. So that means that there will be an infinite path without an even maximum parity which contradicts the fact that $\rho$ is accepting.
Now we see that if $\rho$ is accepting that $\rho'$ is also accepting\qed.\\
Now we prove \(\mathcal{L}(\mathbb{A}')\subseteq \mathcal{L}(\mathbb{A})\)\\
Suppose \(w\in \mathcal{L}(\mathbb{A}')\) then there is a succesfull run \(\rho=q_0q_1q_2\dots\) on $w$ in the NPA $\mathbb{A}'$. Now we are going to make a run $\rho'$ in the APTA $\mathbb{A}$, this should be a DAG.
Define \begin{itemize}
        \item \(\rho'=(V,E)\)
        \item \(V = \{(q, i)|i\in \mathbb{N}, (q, p)\in X, (X, j)\in q_i\}\)
        \item \(E=\bigcup_{l\geq 0} \{((q, l), (q', l+1))|(q, l)\in V, \exists \text{ minimal } Z_q=\{q'|(q', l+1)\in V\} \text{ such that} Z_q\models \delta(q, a_l)\land q'\in Z_q\}\)
        \item Now this fulfills the conditions R1-R4 per definition (nog uitwerken dit)
       \end{itemize}
  Since we know that $\rho$ is accepting we know that the $\max(\Omega'(\inf(\rho))$ is even. By definition of $\Omega'$ we know that this then should equal 2. Therefore there are an infinite number states where all $p_q$ are even and that means that every path that visited that state has an even maximum parity. Also there is only a finite number of times that there is a bigger odd parity so that is not in the inf set of an infinite path. That means that every infinite path in $\rho'$ has an even maximum parity and therefore $\rho'$ is accepting on $w$. \textcolor{red}{Misschien moet dit hier en daar wat rigoreuzer}
\end{proof}

\section{Final theorem}
\begin{theorem}
 Let $\phi$ be a $\mu$-TL formula. Without loss of generality assume assume $\phi$ is guarded and simple. There exists an equivalent NPA $\A$ where
 \[
  \Mod_\Psf(\phi) = \LL(\A)
 \]
\end{theorem}
% \begin{proof}
% This proof will follow a chain of theorems in this booklet. First of all transform $\phi$ to a $U/R$ free formula using theorem \ref{thm:mutlformURfree}. Then use theorem  to transform this to a strongly guarded and simple parity formula. Use theorem \ref{thm:parformtoapa} to transform this to an APA. Finally use definition to transform this to a NPA. This transformation is correct using theorem \ref{thm:APAtoNPA}. Now we have
% \[
%   \Mod_\Psf(\phi) = \LL(\A)
% \]
% \end{proof}


\chapter{Determinging the stutter-invariant fragment of $\mutl$}
In this chapter we will obtain a method to prove for an arbitrary $\phi\in\mutl$ whether it's language is stutter-invariant or not. Furthermore we give a sketch of the proof that $\mutl(U,R)$ is the stutter-invariant fragment of $\mutl$. In Section \ref{section:stutterclosedautomata} we describe a construction to determine the stutter-closure of the language of an automaton. We use this construction to check if the language of an automaton is stutter-invariant. In Section \ref{section:mutlstutinvariant} we use this check on the language of an automaton to prove that certain $\mutl$ formulas are and some are not stutter invariant and lastly we give the first ideas for a proof that $\mutl(U,R)$ is the stutter-invariant fragment of $\mutl$.
\section{Determining the stutter-closure of an automaton}\label{section:stutterclosedautomata}
As seen in the examples of last week we have a construction to obtain an automaton that recognizes the stuttter closure of a language. I informally introduced the procedures but now I will give a formal definition of the automaton $\A^s$. Later on I will prove that this indeed recognizes the stutter-closure.
\begin{definition}\label{def:stutterclosed}
Let \(\A = (Q, \Sigma, \delta, q_I, \Omega)\) be some NPA, where without loss of generality we assume that $\Sigma \cap \omega = \emptyset$. We will define it's \textbf{Stutter-closed automaton} \(\mathbb{A}^s = (Q', \Sigma, \delta', q_I', \Omega')\) as follows:\\
First define the new set of states $Q'$ as $Q' := Q\times \Omega(Q) \cup Q\times \Sigma$ and the set of initial states $q_I'$ as $q_I':=\{(q, \Omega(q))|q\in q_I\}$. To define the transition function $\delta'$ let $(q,a)\in Q'$\\
\textbf{Case 1: $a\in \Sigma$}:
\[
 \delta'((q, a), b) = \begin{cases}
                       \{(q, a), (q, \Omega(q))\}&\text{if } a = b\\
                      \emptyset & \text{else}
                      \end{cases}
\]
\textbf{Case 2: $a\in \Omega(Q)$}:
\begin{align*}
 \delta'((q, a), b) &= \{(q', \Omega(q'))\in Q \times \Omega(Q)\mid q'\in \delta(q, b)\}\\ &\text{\textcolor{red}{dit is eigenlijk overbodig want wordt ook gevangen in laatste case met $n=1$}}\\
                    &\cup \{(q', b)\in Q\times \Sigma\mid  q'\in \delta(q, b)\}\\
                    &\cup \{(q', p) \in Q\times \Omega(Q)\mid \exists n, \exists \rho (qq_1\dots q_{n-1}q') \land q\stackrel{b^n}{\twoheadrightarrow^\rho}q'\land\\
                    & p = \max\{\max(\Omega(q_i), q')\mid i\in \{1, \dots, n-1\}\} \}
\end{align*}
Lastly we define the parity map $\Omega'$ as following:
\[
 \Omega'(q) =\left\{\begin{array}{llll}
             p + 2 &\text{ if } q=(q', p)   &\text{with } p\in \omega\\
             1 &\text{ if } q = (q', l) &\text{with } l\in \Sigma
            \end{array}\right.
\]
\end{definition}

In this section I will prove that the construction presented in definition \ref{def:stutterclosed} is correct.
\begin{theorem}
Let $\A$ a NPA and $\A^s$ it's stutter closed automaton as defined in definition \ref{def:stutterclosed}. The language of $\A^s$ is the stuttterclosure of the language of $\A$:
\[
 \mathcal{L}(\mathbb{A}^s) = (\mathcal{L}(\mathbb{A}))^s
\]
\end{theorem}
\begin{proof}
First we will prove the direction \( \mathcal{L}(\mathbb{A}^s) \subseteq (\mathcal{L}(\mathbb{A}))^s\).\\
Syppose we have $w\in \LL(\A^s)$ we need to show that there exists $w'\in \LL(\A)$ such that $w\sim_sw'$. In other words we need to find a $w_b\in \Sigma^\omega$ and $f, f':\N\to\N^+$ such that $w=w_b[f]$ and $w'=w_b[f']$. Firstly I will define $w_b$ and $f$.\\
\begin{definitiont}\label{def:subsetrhow}
From $\rho_w$ and $w$ start to define $w_b$ and $f$. To do this take $i\in \omega$. First let $l := \sum_{m=0}^{i-1}f(m)$. Now we look at $\rho_w$. Since $\rho_w$ is an accepting run we know it cannot loop infinitely in a state of the form $(q,a)\in Q_\A\times\Sigma$ since $\Omega_{\A^s}(q,a)=1$ and not even. So therefore there exists a smallest $n$ such that $\rho_w(l+n)\in Q_\A\times\Omega_\A(Q_\A)$. \\ Now set $f(i):=n$ and $w_b(i) := w(l)$
\end{definitiont}
\begin{claim}\label{claim:wwb[f]}
 Let $w_b$ and $f$ from Definition \ref{def:subsetrhow} then $w=w_b[f]$.
\end{claim}
\begin{proof}
We need to prove that $w_b(k)=w(\sum_m=0^{k-1}f(m) + j)$ for $0\leq j < f(k)$ for every $k\in \omega$. We will prove this with induction on $k$.
\begin{description}
 \item[Base case $k=0$] We defined $f(k)$ as the smallest $n$ such that $\rho_w(n)\in Q_\A\times \Omega_A(Q_\A)$ so that $\rho_w(j)\in Q_\A\times \Omega_A(Q_\A)$ for $1\leq j<n$. Since since $\rho_w$ is a run on $w$ we know that $\rho_w(i+1)\in\delta_\A(\rho_w(i), w(i))$ and that leaves us only one possibility for $w(i)$ with $0\leq i < n$ namely the $w(0)$ and that is exactly how we defined $w_b(k)$. So that means $w_b(k) = w(i)$ for $0\leq i < n$.
 \item[Inductive case $k>0$] With the inductive hypothesis we know that $w_b(k-1) = w(\sum_m=0^{k-1}f(m) -1)$. Then from the definition of $f$ we again see that $w(\sum_m=0^{k-1}f(m) +i)=w(\sum_m=0^{k-1}f(m))$ for $0\leq i < n$ and since $w_b(k)=w(\sum_m=0^{k-1}f(m))$ we see that $w_b(k) = w(\sum_m=0^{k-1}f(m) +i)$ for $0\leq i< n$.
\end{description}
This proves that $w=w_b[f]$.
\end{proof}
Define $w'$ as $w_b[f']$. Nextly we will define $f'$ and $\rho_{w'}$.
\begin{definitiont}\label{def:subsetrhow'}
 To define $f'$ and $\rho_{w'}$. Let $(q, p) = \rho_w(0)$ then set $\rho_{w'} = q$. Secondly take $i\in\omega$ and distingish cases
\begin{description}
 \item[Case $f(i)=1$] Let $(q, p):= \rho_{w}(\sum_{m=0}^{i-1}f(m))$ and $(q', p'):= \rho_{w}(\sum_{m=0}^{i}f(m))=\rho_{w}(\sum_{m=0}^{i-1}f(m)+1)$. Now since $(q', p')\in \delta_{\A^s}((q,p), w(\sum_{m=0}^{i-1}f(m)))$. From Claim 1 we see that $w(\sum_{m=0}^{i-1}f(m)) = w_b(i)$ so we know that $\exists n>0$ and $\rho_s=qq_1\dots q_{n-1}q'$ such that $q\twoheadrightarrow_{\rho_s}^{(w_b(i))^n}q'$ and $p'=\max\{\Omega_\A(\rho_s(j))\mid j\in \{1,\dots,n\}\}$. Set $f'(i)=n$ and
 \[
  \rho_{w'}\left(j +\sum_{m=0}^{i-1}f'(m)\right) = \rho_s(j)\text{ for } j\in \{1,\dots,n\}
 \]
 \item[Case $f(i)\neq 1$] Now set $f'(i):=1$  let $(q, p):= \rho_{w}\left(\sum_{m=0}^if(m)\right)$ and set $\rho_{w'}\left(\sum_{m=0}^if'(m)\right)~=~q$
\end{description}
\end{definitiont}


Now Now we need to show that $\rho_{w'}$ is indeed an accepting run on $w'$, in other words:
\begin{claim}
 Let $\rho_{w'}$ from Definition \ref{def:subsetrhow'} then $\rho_{w'}(i+1)\in\delta_{\A}(\rho_{w'}(i), w(i))$ and $\rho_{w'}(0)\in q_{I,\A}$
\end{claim}
\textcolor{red}{Dit is ook een saai bewijs misschien, voor in de appendix??}
\begin{proof}
First $\rho_{w'}(0)\in q_{I,\A}$. Let $\rho_{w}(0)=(q,p)$ we know that $\rho_{w'}(0)=q$. Since $\rho_{w'}$ is a run of $\A^s$ we know that $\rho_{w'}(0)=(q,p)\in q_{I, \A^s} = \{(q, \Omega(q))\mid q\in q_{I, \A}\}$ so $\rho_{w'}=q\in q_{I,\A}$. Now we will prove that $\rho_{w'}(i+1)\in\delta_{\A}(\rho_{w'}(i), w(i))$.\\
Let $i\in \omega$. Let $i_b$ as the smallest integer such that $i+1\leq\sum_{m=0}^{i_b} f'(m)$. Let $(q,p)= \rho_w(\sum_{m=0}^{i_b}f(m)$, from the definition we know that $\rho_{w'}(\sum_{m=0}^{i_b}f'(m))=q$. Distinguish cases (the same as in the definition):
\begin{description}
 \item[Case $f(i_b)=1$] Let $j$ such that $i = \sum_{m=0}^{i_b} f'(m) + j -1$ $(j>1)$ then we see that $\rho_{w'}\left(j +\sum_{m=0}^{i-1}f'(m)\right) = \rho_s(j)$ and since $q\twoheadrightarrow_{\rho_s}^{(w_b(i_b))^n}q'$ we see that $\rho_s(j)\in \delta_\A(\rho_s(j-1), w_b(i_b))$. So that means that
 \begin{align*}
  \rho_{w'}(i+1) &= \rho_{w'}(j +\sum_{m=0}^{i-1}f'(m)) = \rho_s(j)\\
  &\in \delta_\A(\rho_s(j-1), w_b(i_b))\\
  &= \delta_\A(\rho_{w'}(\sum_{m=0}^{i_b} f'(m) + j -1), w_b(i_b)) \\
  &= \delta_\A(\rho_{w'}(i), w'(i))
 \end{align*}
 \item[Case $f(i_b)=n\neq 1$]. That means that $f'(i)=1$ (we know that from Definition \ref{def:subsetrhow'}). Let $\rho_{w}(\sum_{m=0}^{i_b-1}f(m))=(q, p)$ From Definition \ref{def:subsetrhow} we see that $\rho_{w}(l+\sum_{m=0}^{i_b-1}f(m))=(q', w_b(i_b))$ for $1\leq l <n$ and $\rho_{w}(\sum_{m=0}^{i_b}f(m))=(q', \Omega(q'))$ since $\rho_w$ is a run we know then that $q'\in \delta_\A(q, w_b(i_b))$. Since $f'(i_b)=1$  we see that $i =  \sum_{m=0}^{i_b-1}f'(m)$ and $i+1 = \sum_{m=0}^{i_b}f'(m)$ therefore we see that
 \begin{align*}
  \rho_{w'}(i+1) &= \rho_{w'}(\sum_{m=0}^{i_b}f'(m))=\rho_{w}(\sum_{m=0}^{i_b}f(m))=q'\in \delta_\A(q, w_b(i_b)) \\
  &= \delta_\A(\rho_{w}(\sum_{m=0}^{i_b-1}f(m)), w_b(i_b)) = \delta_\A(\rho_{w'}(\sum_{m=0}^{i_b-1}f'(m)), w_b(i_b))\\
  &= \delta_\A(\rho_{w'}(i+1), w_b(i_b))
 \end{align*}
\end{description}
Combinining these cases proves the claim.
\textcolor{red}{Verwijzingen naar definities en claims checken en de haakjes goed doen. }
\end{proof}
\begin{claim}
 We have that
 \[
  \max(\inf_{\Omega_\A}(\rho_{w'}))\text{ is even}
 \]
\end{claim}
\begin{proof}
We will first prove that \(\max(\inf_{\Omega_\A}(\rho_{w'}))=\max(\inf_{\Omega_{\A^s}}(\rho_{w}))-2\) via the two inequalities. First \\
\(\leq\) Let $q\in\inf(\rho_{w'})$ be a state with $\Omega_\A(q)=\max(\inf_{\Omega_\A}(\rho_{w'}))$. That means that we have infinitely many $i's$ such that $\rho_{w'}(i)=q$. Now take $i>\sum_{m=0}^{j} f'(m) > i_d$. Dan definieer nu $i_b$ dan ofwel $f(i_b)=1$ dan ligt deze $q$ op $\rho_s$ en gelijk aan max dus $p=\Omega(q)$. Als $f(i_b)=n$ dan zie je dat $q=\rho_w(\sum_{m=0}^{i_b}f(m))$ dus $\Omega_{\A^s}(\rho_w(\sum_{m=0}^{i_b}f(m)))-2=\Omega_\A(q)$ dus oneindig veel $i$tjes dus $\max(\inf_{\Omega_\A}(\rho_{w'}))\leq \max(\inf_{\Omega_{\A^s}}(\rho_{w}))-2$.\\
\(\geq\) Let $(q,p)\in\inf(\rho_{w})$ be a state with $\Omega_{\A^s}((q,p))=\max(\inf_{\Omega_{\A^s}}(\rho_{w}))$. Now take $i> i_d$. Nu neem $i_b$ dan cases $f(i_b)=n$ dan zie je dat $\rho_{w}(i) = \rho_w(\sum_{m=0}^{i_b}f(m)) = (q,\Omega(q)) = (\rho_{w'}(\sum_{m=0}^{i_b}f'(m)), \Omega(\rho_{w'}(\sum_{m=0}^{i_b}f'(m))))$ and $\Omega(\rho_{w}(i))-2=\Omega(q)$ so we have that $\max(\inf_{\Omega_\A}(\rho_{w'}))\geq \max(\inf_{\Omega_{\A^s}}(\rho_{w}))-2$ since we have infinitely many $q$ with $\Omega_\A(q)=...$.\\

Combining these two inequalities we get \(\max(\inf_{\Omega_\A}(\rho_{w'}))=\max(\inf_{\Omega_{\A^s}}(\rho_{w}))-2\) and since $\max(\inf_{\Omega_{\A^s}}(\rho_{w}))$ is even we have that  \(
  \max(\inf_{\Omega_\A}(\rho_{w'}))\) is even.
\end{proof}
Combining claims 2 and 3 we know that $w'\in \LL(\A)$ and from claim 1 and the fact that $w'=w_b[f']$ we know that $w\sim_s w'$ so that means that $w\in (\LL(\A))^s$ which proves the inclusion..\\\\
Secondly the other direction \(\mathcal{L}(\mathbb{A}^s) \supseteq (\mathcal{L}(\mathbb{A}))^s
\)

Suppose we have $w\in (\mathcal{L}(\mathbb{A}))^s$ then there exists a $w'\in\LL(\A)$ such that $w\sim_s w'$. Recall definition about language of $\A$... Following Definition \ref{def:stutequiv} we know there exists $w_b\in \Sigma$ and $f, f':\N\to\N^+$ such that $w = w_b[f]$ and $w'=w_b[f']$ \\
To prove that $w\in \LL(\A^s)$. Since $w'\in\LL(\A)$ we know that there exists an accepting run $\rho_{w'}$ in $\A$. I will first prove that $w_b\in \LL(\A^s)$ and then give an accpeting run for $w$. To show that $w_b\in\LL(\A^s)$ I will give an run and show that this is accepting.\\
\textbf{Definition} (2) To define $\rho_{w_b}:\omega\to Q_{\A^s}$ take $i\in\omega$. Since $w'\in\LL(\A)$ we know that there is an accepting run $\rho_{w'}:\omega\to Q_\A$ on $w'$. We will use this as the basis for $\rho_{w_b}$. Set $\rho_{w_b}(0) = (\rho_{w'}(0), \Omega(\rho_{w'}(0)))$. To define $\rho_{w_b}(i)$ for $i>0$ Let $j=\sum_{m=0}^{i-2}f'(m)$ and $n= f'(i)$ and let $p=\max\{\rho_{w'}(k)\mid k\in \{j+1,\dots, j+n\}\}$. Now set $\rho_{w_b}(i)=(\rho_{w'}(j+n), p)$.
\setcounter{claim}{0}
\textcolor{red}{Misschien dit bewijs in een appendix doen, is niet zo heel interessant namelijk}
\begin{claim}
Let $\rho_{w_b}$ from the definition above, then for all $i$ we have $\rho_{w_b}(i+1)\in \delta_{\A^s}(\rho_{w_b}(i), w_b(i))$ and $\rho_{w_b}(0)\in q_{I,\A^s}$.
\end{claim}
\begin{proof}
First we prove that $\rho_{w_b}(i+1)\in \delta_{\A^s}(\rho_{w_b}(i), w_b(i))$ for all $i\in\omega$.

Let $j=\sum_{m=0}^{i-1}f'(m)$ and $n=f'(i)$
 \begin{align*}
  \rho_{w_b}(i+1)&=\left(\rho_{w'}\left(j+f'(i), \max\{\rho_{w'}(k)\mid k\in \{j+1,\dots, j+n\}\}\right)\right)\text{ Definition ...}\\
  &=\left(\rho_{w'}\left(j+f'(i), \max\{\rho_{w'}(k)\mid k\in \{j+1,\dots, j+n\}\}\right)\right)\\
  &\in \{(q_n, p) \in Q\times \Omega(Q)\mid \exists n, \exists \rho= qq_1\dots q_n \land q\twoheadrightarrow_\rho^{w_b(i)}q_n\land\\
  & \phantom{\in\{}p = \max(\Omega(\{q_i\mid i\in \{1, \dots, n\}\}))\land q = \rho_{w'}(j)\}
  \intertext{To understand this inclusion we set $n=f'(i)$ and $\rho=\rho_{w'}(j)\dots\rho_{w'}(j+n)$ and since $w_b(i)=w'(j+k)$ for $k<f'(i)$ we see that $\rho$ is a partial run.}
  &\subseteq \delta_{\A^s}({\rho_{w_b}(i), w_b(i))})\text{ Since $\rho_{w_b}(i)=(\rho_{w'}(j), p)$ for a $p\in\Omega_\A(Q_\A)$}
 \end{align*}
which proves the first part of the  claim and now the second part about the initial state. We see that $\rho_{w_b}(0)=(\rho_{w'}(0),\Omega_\A(\rho_{w'}(0)))$ and since $\rho_{w'}$ is a run we know that $\rho_{w'}(0)\in q_{I,\A}$ so that means that $\rho_{w_b}(0)\in\{(q, \Omega_\A(q))\mid q\in q_{I,\A}\}=q_{I,\A^s}$.
\end{proof}
\begin{claim}
 \[
  \max\left(\inf_{\Omega_{\A^s}}(\rho_{w_b})\right) \text{ is even}
 \]
\end{claim}
\begin{proof}
I will fist prove $\max\left(\inf_{\Omega_{\A^s}}(\rho_{w_b})\right)-2 = \max\left(\inf_{\Omega_\A}(\rho_{w'})\right)$ via two inequalities.\\
First $\max\left(\inf_{\Omega_{\A^s}}(\rho_{w_b})\right)-2 \leq \max\left(\inf_{\Omega_\A}(\rho_{w'})\right)$\\
We know there exists $q\in\inf(\rho_{w_b})$ with $\Omega_{\A^s}(q) = \max\left(\inf_{\Omega_{\A^s}}(\rho_{w_b})\right)$. This occurs infinitely and we will show that there exists a state in $\rho_{w'}$ with parity $\Omega_{\A^s}(q)-2$ that also occurs infinitely. Take a $i$ such that $\rho_{w_b}(i)=q$. Let $j=\sum_{m=0}^{i-1}$ Now we see that $\Omega_{\A^s}(q)=\max\{\rho_{w'}(k)\mid k\in\{j+1,\dots,j+f'(i)\}\}+2$ so there exists a $k$ such that $\Omega_\A(\rho_{w'}(k))=\Omega_{\A^s}(q)-2$. Since there are infinitely many $i$'s such that $\rho_{w_b}(i)=q$ and every $i$ gives a unique $k$ there are infinitely many $k$'s such that $\Omega_\A(\rho_{w'}(k))=\Omega_{\A^s}(q)-2$ and since we have finitely many  states in $Q$ we now know that there is a state $q'\in\inf_{\rho_{w'}}$ with $\Omega_{\A}(q')=\max\left(\inf_{\Omega_{\A^s}}(\rho_{w_b})\right)-2$ which proves this inequality.\\
Now $\max\left(\inf_{\Omega_{\A^s}}(\rho_{w_b})\right)-2 \geq \max\left(\inf_{\Omega_\A}(\rho_{w'})\right)$\\
We know there exists $q\in\inf(\rho_{w'})$ with $\Omega_\A(q)=\max\left(\inf_{\Omega_\A}(\rho_{w'})\right)$. Now define $i_d$ as the decisive moment and take $i>\sum_{m=0}^jf'(m)\geq i_d$ that means that every state $\rho_{w'}(i)$ has smaller or equal parity then $q$. Now define $i_b$ as the base index and we see that $\rho_{w_b}(i_b+1)=\{\rho_{w'}(\sum_{m=0}^{i_b}f'(m)), \max\{\Omega_\A(\rho_{w'}(k))\mid k\in \{\sum_{m=0}^{i_b-1}f'(m)+1, \dots, \sum_{m=0}^{i_b}f'(m)\}\}\}$ where $\max\{\Omega_\A(\rho_{w'}(k))\mid k\in \{\sum_{m=0}^{i_b-1}f'(m)+1, \dots, \sum_{m=0}^{i_b}f'(m)\}\}=\Omega_\A(q)$ since $\sum_{m=0}^{i_b-1}f'(m)\geq i_d$. That means that $\Omega_{\A^s}(\rho_{w_b}(i_b+1))-2 = \Omega_\A(q)$ so that proves the inequality.\\
Combining the two inequalities we have that $\max\left(\inf_{\Omega_{\A^s}}(\rho_{w_b})\right)-2 = \max\left(\inf_{\Omega_\A}(\rho_{w'})\right)$ and since $\rho_{w'}$ is a run we have that $\max\left(\inf_{\Omega_\A}(\rho_{w'})\right)$ is even so $\max\left(\inf_{\Omega_{\A^s}}(\rho_{w_b})\right)$ as well.
\end{proof}
Following Claim 1 and 2 we know that $\rho_{w_b}$ is $w_b\in \LL(\A^s)$. Now we want to proof that $w\in\LL(\A^s)$ to do that we give an accepting run on $w$.\\
\textbf{Definition} Define $\rho_w:\omega\to Q_{\A^s}$ as following. Set $\rho_{w}(0)=\rho_{w_b}(0)$. For $i>0$ let $i_b$ as the smallest integer such that $i\leq\sum_{m=0}^{i_b} f(m)$. This $i_b$ is the index in the base word that corresponds to the index $i$ in $w'$. Now we distinguish cases.
\begin{description}
 \item[Case $f(i_b)=1$] Set $\rho_{w}(i) = \rho_{w_b}(i_b+1)$
 \item [Case $f(i_b)=n>1$] Let $(q_n, p_n):=\rho_{w_b}(i_b+1)$ and $j= i - \sum_{m=0}^{i_b-1} f(m)$ Now we again have to distinguish cases:
 \begin{description}
  \item[Case $(q_n, w(i))\in \delta((\rho_{w}(i-1), w(i-1))$] Now set
  \[
   \rho_w(i) = \begin{cases}
                   (q_n, w(i)) &\text{ if } i < \sum_{m=0}^{i_b} f(m)\\
                   (q_n,\Omega_\A(q_n)) &\text{ if } i = \sum_{m=0}^{i_b} f(m)
                  \end{cases}
  \]
  \item[Case $j < f(i_b)$] then set $\rho_w(i) = \left(\rho_{w'}\left(\sum_{m=0}^{i_b-1}f'(m)+j\right),\Omega_\A\left(\rho_{w'}\left(\sum_{m=0}^{i_b-1}f'(m)+j\right)\right)\right)$. This is the run in $w'$.
  \item[Case $j=f(i_b)$] Now since we know that $(q_n, w(i))\notin \delta((\rho_{w}(i-1), w(i))$ we see that $j=f(i_b) < f'(i_b)$. So that means we need to take a shortcut transition in the following form: Let $\text{start} = j + \sum_{m=0}^{i_b-1}f'(m)$ and $\text{end} =  \sum_{m=0}^{i_b}f'(m)$
  \[
   \rho_{w}(i) = (\rho_{w'}(\text{end}), \max\{\Omega_\A(\rho_{w'}(k))\mid k\in \{\text{start},\dots,\text{end}\}\})
  \]
 \end{description}
\end{description}
observation: We see that $\rho_w(\sum_{m=0}^{i_b}f(m))=\rho_{w_b}(i_b+1)$

\begin{claim}\label{claim:rhowbwelldefined}
Let $\rho_w$ from definition... then for all $i$ we have $\rho_{w}(i+1)\in \delta_{\A^s}(\rho_w(i), w(i))$ and $\rho_{w}(0)\in q_{I,\A^s}$
\end{claim}
Take $i\in\omega$ and we look at $\rho_{w}(i+1)$. Let $i_b$ smallest integer such that $i+1\leq \sum_{m=0}^{i_b}f(m)$. Then distinguish cases
\begin{description}
 \item[Case $f(i_b)=1$] So $\sum_{m=0}^{i_b}f(m)=\sum_{m=0}^{i_b-1}f(m)+1$ That means that $i=\sum_{m=0}^{i_b-1}f(m)$ so $\rho_{w}(i)=\rho_{w}(\sum_{m=0}^{i_b-1}f(m)) = (\rho_{w_b}(i_b))$. That easily gives
 \begin{align*}
  \rho_{w}(i+1) &= \rho_{w_b}(i_b+1)\text{ From Definition ...}\\
                &\in\delta_{\A^s}((\rho_{w_b}(i_b), w_b(i_b)))\text{Since this is a run}\\
                &=\delta_{\A^s}\left(\left(\rho_{w}\left(\sum_{m=0}^{i_b-1}f(m)\right), w\left(\sum_{m=0}^{i_b-1}f(m)\right)\right)\right)\text{ Lemma \ref{lemma:stutbaseword} and observation}\\
                &=\delta_{\A^s}(\rho_{w}(i), w(i))
 \end{align*}
 \item [Case $f(i_b)=n>1$] Let $(q_n, p_n):=\rho_{w_b}(i_b+1)$ and $j= i + 1 - \sum_{m=0}^{i_b-1} f(m)$ Now we again have to distinguish cases:
 \begin{description}
  \item[Case $(q_n, w(i))\in \delta((\rho_{w}(i-1), w(i-1))$] We immediately see $\rho_w(i+1)\in\delta(\rho_w(i), w(i)$.
  \item[Case $j=1$] That means that $\rho_{w}(i) = \rho_{w}(\sum_{m=0}^{i_b-1}f(m))=\rho_{w_b}(i_b) = (\rho_{w'}(\sum_{m=0}^{i_b-1}f'(m), p)$ for a $p\in\omega$. Since we have
  \begin{align*}
   \rho_w(i+1) &= \left(\rho_{w'}\left(\sum_{m=0}^{i_b-1}f'(m)+1\right),\Omega_\A\left(\rho_{w'}\left(\sum_{m=0}^{i_b-1}f'(m)+1\right)\right)\right)\\
   &\in \left\{(q', \Omega(q'))\mid q'\in\delta\left(\rho_{w'}\left(\sum_{m=0}^{i_b-1}f'(m)\right), w'\left(\sum_{m=0}^{i_b-1}f'(m)\right)\right)\right\}\\
   &=\left\{(q', \Omega(q'))\mid q'\in\delta\left(\rho_{w'}\left(\sum_{m=0}^{i_b-1}f'(m)\right), w\left(\sum_{m=0}^{i_b-1}f(m)\right)\right)\right\}
   \intertext{Since we have $w\left(\sum_{m=0}^{i_b-1}f(m)\right) = w_b(i_b) = w'\left(\sum_{m=0}^{i_b-1}f'(m)\right)$ with Lemma \ref{lemma:stutbaseword}}
  &\subseteq \delta_{\A^s}(\rho_w(i), w(i))
  \end{align*}
  which proves the claim
  \item[Case $1<j<f(i_b)$] We use the fact that $w\left(\sum_{m=0}^{i_b-1}f(m)+j\right) = w_b(i_b) = w'\left(\sum_{m=0}^{i_b-1}f'(m)+j\right)$, so that means that we can use the copied transitions everywhere.
  \item[Case $j=f(i_b)$] We took a shortcut transition which is there sinc $\rho=\rho_{w'}(j + \sum_{m=0}^{i_b-1}f'(m)) \dots \rho_{w'}( \sum_{m=0}^{i_b}f'(m))$ is a partial run.
 \end{description}
\end{description}
\begin{proof}
Saaaaaai moet nog eventjes checken.
\end{proof}
\begin{claim}
  \[
  \max\left(\inf_{\Omega_{\A^s}}(\rho_{w})\right) \text{ is even}
 \]
\end{claim}
\begin{proof}
We will prove that $\max\left(\inf_{\Omega_{\A^s}}(\rho_{w})\right)=\max\left(\inf_{\Omega_{\A^s}}(\rho_{w_b})\right)$ with two inequalities:\\
Firstly: $\max\left(\inf_{\Omega_{\A^s}}(\rho_{w})\right)\leq\max\left(\inf_{\Omega_{\A^s}}(\rho_{w_b})\right)$\\
Assert: it cannot be a loop state since the successor of the loop state always has bigger priority.

Now take $q\in\inf_(\rho_w)$ such that $\Omega_{\A^s}(q) = \max\left(\inf_{\Omega_{\A^s}}(\rho_{w})\right)$ that means that we have infinitely many $i$ such that $\rho_w(i)=q$. Take $i>\sum_{m=0}^jf(m)>i_d$ then we look at $i_b$ if $f(i_b)=1$ we immediately see that this state is also in $\rho_{w_b}$. Else we see that if it is a normal state then it is part of the partial run in $w'$ and in particular it is bigger so the parity is equal to max. If it is shortcut than also bigger so max of partial run.
Secondly $\max\left(\inf_{\Omega_{\A^s}}(\rho_{w})\right)\geq\max\left(\inf_{\Omega_{\A^s}}(\rho_{w_b})\right)$\\
Now if $f(i_b)=1$ then we see immediately. Now else it is the max of the partial run. If the max is before $f(i_b)$ then it there is a normal state with this parity. Else it is part of the shortcut state.
\end{proof}
Now we know that $w\in\LL(\A^s)$ which proves that \(\mathcal{L}(\mathbb{A}^s) \supseteq (\mathcal{L}(\mathbb{A}))^s
\).
\end{proof}


\section{Using the $\mutl$ to NPA translation to determine the stutter-invariant fragment }\label{section:mutlstutinvariant}
So far in this thesis we have given a translation from a $\mutl$ formula to an \npa\, in Chapter 3 which we can use for every tidy, guarded and simple formula. In Section 4.1 we have given a method to determine if the language of an automaton is stutter-invariant. We can combine these two procedures to obtain a procedure to determine if a formula is stutter-invariant.
\begin{theorem}
Let $\phi$ a tidy, guarded and modal simple $\mutl$ formula. Then there exists an effective procedure that determines wheter $\phi$ is stutter-invariant.
\end{theorem}
\begin{proof}
 Since $\phi$ is tidy, guarded and modal simple, by Theorem \ref{thm:mutltonpa}, we can obtain an equivalent \npa\, $\A_\phi$. With Theorem \ref{thm:automatonstutinvariant} we can then check if the language of $\A_\phi$ is stutter-invariant. Since $\Mod_\Psf(\phi)=\LL(\A_\phi)$ we then know if $\phi$ is stutter-invariant.
\end{proof}
\noindent Next to checking if a specific $\mutl$ formula is stutter-invariant it is however much more interesting to determine all the stutter-invariant formulas of $\mutl$. Or its stutter-invariant fragment. Recall that the goal of this thesis was to give an alternative proof of the theorem in \cite{gheerbrandt2009craiginterpolation} stating the fact that $\mutl(U,R)$ is the stutter-invariant fragment of $\mutl$. First we see that Gheerbrandt and Ten Cate first prove the following lemma.
\begin{theorem}\cite[Lemma 3]{gheerbrandt2009craiginterpolation}\label{thm:lemma3}
 For every $\phi\in\mutl$ there exists a $\phi^*\in\mutl(U,R)$ that agrees with $\phi$ on all stutter-free words:
 \[
  w, i\models \phi\iff\phi^* \text{ for all stutter-free } w\in \Sigma^\omega
 \]
\end{theorem}
Afterwards they will prove the following theorem:
\begin{theorem}\label{thm:untilstutinvariant}
 Let $\phi\in\mutl(U,R)$ a formula, then $\Mod(\phi)$ is stutter-invariant.
\end{theorem}

\noindent The goal of this thesis was to present an alternative proof with an automata based approach, we did however not succeed in this. We will explain the problems shortly
\begin{proof}[Proof (Sketch)]
 The proof of this theorem goes with induction on the complexity of $\phi$.
 The atomic cases are clear. For the case where $\phi=\psi\star\chi$ with $\star\in\{\land,\lor\}$ we will use our $\mutl$ to \npa\, translation and Propositions \ref{prop:mutltonpatranslationboolean} and \ref{prop:stutcupcap}. Let $\A_\psi$ and $\A_\chi$ the automata as in Theorem \ref{thm:mutltonpa}. Now we see that:
  \begin{align*}
   \Mod_\Psf(\psi\star\chi)&= \LL(\A_\psi)\star^s\LL(\A_\chi)
   \intertext{We use Proposition \ref{prop:mutltonpatranslationboolean} where  $\star^s$ stands for the set equivalent of $\star$}
                           &= \LL(\A_\psi)^s\star^s\LL(\A_\chi)^s \text{ (Induction hypothesis)}\\
                           &= (\LL(\A_\psi)\star^s\LL(\A_\chi))^s \text{ (Proposition \ref{prop:stutcupcap})}\\
                           &= (\Mod_\Psf(\psi\star\chi))^s \text{ (Proposition \ref{prop:mutltonpatranslationboolean})}
  \end{align*}
  Which proves that $\psi\star\chi$ is stutter-invariant.
The cases where $\phi=\eta x.\xi_x$ and $\phi=\psi T \chi$ with $T\in\{U,R\}$ are more difficult to prove and will therefore not be proven in this thesis. We will however explain some of this difficulties here. At first sight we tried to write out the the automata for $\phi$ generated by Theorem \ref{thm:mutltonpa}. This was more difficult since in our translation of alternating parity automata to nondeterministic parity automata we cite an article that uses parity progress measures to first translate the alternating parity automata into alternating Büchi automata. The deep understanding of parity progress measures required for this induction step went too far for this Bachelor project. Secondly the construction of the automaton $\A_\phi$ in Theorem \ref{thm:mutltonpa} requires $\phi$ to be guarded. We know that every $\mutl$ formula is equivalent to a guarded one but the precise understanding of this theorem was again not part of this thesis. Also since we are looking at formulas without the $\deo$ operator any formula that contains a bounded variable is clearly not guarded. In this project it remains open if this $\deo$-free formula can be transformed into a formula without fixpoint operators. If you look at the following (extremely simple) example see:
\[
 \nu x.(p\land x) U r \equiv (pUr)
\]
It is of future interest to finish this construction, and to see if this will work for every formula. If $\phi$ can be transformed into a formula $\phi^T$ that uses no fixpoint operators we see that the alternating automaton that is equivalent to $\phi^T$ actually is a Büchi automaton since without bound variables there can be no alternating chains. In this Büchi case the construction to a nondeterministic automaton can be applied immediately. It then remains open to prove that this formula $\phi^T$ is stutter-invariant.
\end{proof}
The following theorem proves that $\mutl(U,R)$ is equal to the stutter-invariant fragment of $\mutl$.
\begin{theorem}
 Let $\phi\in\mutl$ such that $\Mod_\Psf(\phi)$ is stutter-invariant. Then there exists $\phi^*\in\mutl(U,R)$ such that $\Mod(\phi)=\Mod(\phi*)$
\end{theorem}
\begin{proof}
 This follows from Theorem \ref{thm:lemma3} and Theorem \ref{thm:untilstutinvariant} \cite{gheerbrandt2009craiginterpolation}.
\end{proof}


\chapter{Formalization in LEAN}
Part of the goal of this thesis is to formalize one of its main theorems and its proof with a formal proof assistant, in this case $\lean$.  With a formal proof assistant you can formally prove that your proof is correct. When the computer accepts the proof you are absolutely sure that your proof is correct. The downside to this absolute certainty is the fact that you have to be really precise. For every small step that you make you have to show $\lean$ exactly what you mean. So that raises the question why one would want to formalize a theorem and its with $\lean$

For the first reason we need to dive into the nature of a mathematical proof. You can view the goal of a a mathematical proof as having two parts\footnote{This viewpont comes from Yde Venema, one of the supervisors of this thesis.}: on the one hand you want to convince yourself, but also your reader that the proof and ideas are correct. For this you have to be as precise as possible. Unfortunately we as mathematicians are still humans and no mathemagicians so it it is (sometimes) difficult to be completely convinced of your proof. On the other hand a mathematical proof is meant to communicate your ideas.  For this goal absolute precision is not always the best, as this pedanticity can make your text unreadable. If you also formalized your proof you can move the pure pedantic parts to your $\lean$ code and focus on the ideas of your proof in the paper version. We have noticed for example in the proof of Theorem \ref{thm:stutterclosednpa} that this can be convenient.

The second reason is that this certainty about the correctness of your proof can go wrong. For example in the case of interpolation for Propositional Dynamic Logic, there were three proof attempts that were later criticized or retracted \cite{borzechowski2025propositionaldynamiclogiccraig}. There is now a new proof, but to be absolute certain that this proof is correct a group of researchers is formalizing their proof of interpolation in $\lean$ \cite{m4lvin2025pdl}.

The third reason for formalization, and especially why we use $\lean$, is that it is hot and happening right now. There is a big communitufy of mathematicians from over the world using $\lean$. That results in a well established library \emph{mathlib} that describes and formalizes big parts of mathematics. In our case there is already an attempt formalizing the theory of $\omega$-automata in Lean \cite{ctchou2025automata}. Using $\lean$ makes it easier to collaborate and share the research in this project. Not only in the global mathematical world is there a lot of attention to $\lean$ but also locally at the Logic (ILLC) and mathematics (KdVi) institutes of the UvA. Examples include: Malvin Gattinger working on formalizing their proof of PDL interpolation, there is a $\lean$ seminar at the KdVi and fellow double bachelor students Rik Heurter and Noam Cohen have formalized part of their theses in $\lean$. Rik pointed out in his thesis that $\lean$ can help people over the world collaborating on a mathematical proof \cite{heurter2025thesis}. Since formalization is coming up in the mathematical world we thought it would be a good experience and skill to formalize part of this thesis.\\

We will not formalize all of the theorems in this thesis but will focus on Theorem \ref{thm:stutterclosednpa}. We choose this theorem for two reasons. Firstly, as already mentioned, there is already a library that implements $\omega$-automata. We only had to define the parity acceptance condition ourselves. There is not yeat a definition of $\mutl$ in the community so we would have to define this ourselves. The definition of $\mutl$ involves delicate conditions, for example that variables may only occur positively after a fixpoint binder and we only want to consider clean formulas. It would have take more time to formalize these definitions leaving less time for the proof itself. Secondly the proof of Theorem \ref{thm:stutterclosednpa} is using a lot of precise summations and indices so it is easy to make a mistake there. In order to ensure that this part of the proof is correct formalization is actually really helpful. We sadly did not succeed to fully complete the proof of this theorem in the timespan of this bachelorproject. However the left to right direction \(\LL(\A^s)\subseteq (\LL(\A))^s\) is well advanced.

The code of this project is attached to this pdf and can be found online in the GitHub repository \url{https://github.com/riemerk/BscThesis}.
\section{Definitions}\label{section:leandef}
As mentioned earlier there is already a definition of $\omega$-automata in $\lean$ in the GitHub repository \textit{Automata Theory} of Ching-Tsun Chou \cite{ctchou2025automata}. In this repository automata are defined as a class containing a state as a type, a set of initial states and a transition function.
\begin{leancode}
class NA (Alph : Type) where
    State : Type
    init : Set State
    next : State → Alph → Set State
\end{leancode}
We extend this definition to a \npa\, (Definition \ref{def:NPA}). We add a parityMap and the constraints that \verb|State| and \verb|Alph| types have to be finite and the equality of \verb|Alph| has to be decidable.
\begin{leancode}
class NPA Alph extends NA Alph where
    parityMap : State → ℕ
    FinState : Finite State
    FinAlph : Finite Alph
    DecidableAlph : DecidableEq Alph
\end{leancode}
Furthermore we define the parity acceptance condition as follows:
\begin{leancode}
def NPA.ParityAccept (A : NPA Alph) (w : Stream' Alph) :=
    ∃ ρ : Stream' A.State, A.InfRun w ρ ∧ Even (sSup ((InfOcc ρ).image A.parityMap))
\end{leancode}
We use \verb|sSup| for no particular reason. When we started formalizing in Lean our knowledge was limited and \verb|sSup| came up as the first possible solution. We could however also have used \verb|Finset.Max|.

We tried to use the same naming style in the lean code as in the text of this thesis. It is unfortunately not possible to call a type $\Sigma$ in $\lean$ since that is already used to denote \verb|Finset.sum|. So we used \verb|Alph| for this. Also we found it more appropriate to use parityMap in the $\lean$ definition of an \npa\, since this is a high level definition and parityMap is more descriptive than $\Omega$.

For the definition of stuttering we stay closely to Definition \ref{def:stutequiv}. The only difference is the use of $\mathbb{N}$ instead of $\mathbb{N}^+$. We tried to use \verb|PNat| but that quickly gave a lot of headache. We wanted to use lemmas about $\mathbb{N}$ so when using \verb|PNat| we needed a coercion to $\N$ everytime. A simple but elegant solution\footnote{Suggested by my supervisor Malvin Gattinger, to be completely honest.} was to use $\N$ and add $1$ every time we use the value of one of the functions $f$.
\begin{leancode}
def StutterEquivalent (w : Stream' Alph) (w' : Stream' Alph) : Prop :=
  ∃ wb : Stream' Alph,  ∃ f : Stream' ℕ,  ∃ f' : Stream' ℕ,
    w = (functiononword wb f) ∧ w' = (functiononword wb f')
\end{leancode}
The definition \verb|NPA.StutterClosed| follows Definition \ref{def:stutterclosedNPA}. To define $Q_{\A^s}:=Q_\A \times (\Sigma\times \Omega_\A(Q_\A))$ we use a disjoint union $\oplus$ for  $(\Sigma\times \Omega_\A(Q_\A))$ so we do not assume that $Q\cap \omega=\emptyset$ as we do in the paper version. Here we need the fact that the equality \verb|Alph| is decidable to define a case distinction in the transition function.
% Hoe referen naar papieren versie?

\section{Proof}\label{section:leanproof}
The main theorem of this project is \verb|NPA.StutterClosed.AcceptsStutterClosure| which corresponds to Theorem \ref{thm:stutterclosednpa}. For clarity we will repeat the theorem here:
\setcounter{chapter}{4}
\setcounter{definition}{1}
\begin{theorem}\label{thm:stutterclosednpa}
Let \(\A = (\Sigma, Q_\A, \delta_\A, q_{I,\A}, \Omega_\A)\) a NPA and \(\A^s = (\Sigma, Q_{\A^s} \delta_{\A^s}, q_{I,\A^s}, \Omega_{\A^s})\) its stutter closure as defined in Definition \ref{def:stutterclosedNPA}. Then we have
\[
 \LL(\A^s) = (\LL(\A))^s
\]
\end{theorem}
\setcounter{chapter}{5}
\noindent The structure of this proof in $\lean$ is similar to the structure of the proof in the paper version of this thesis. For the left to right direction we want to prove that for any $w\in\LL(\A^s)$ there exists $w'\in \LL(\A)$ such that $w\sim_s w'$. First we will define $w_b$ and $f$ in \verb|subset_wb_f_pair| (Definition \ref{def:subsetwbf}). Nextly we will define $f'$ to define $w'=w_b[f']$  and the run $\rho_{w'}$ on $w'$ in \verb|subset_f'_rhow'_pair| (Definition \ref{def:subsetrhow'_f'}). Next we will show that $\rho_{w'}$ is an accepting run on $w'$ in lemmas \verb|subset_rhow'_run| (Claim \ref{claim:subsetrhow'run}) and \verb|subset_rhow'_pareven| (Claim \ref{claim:subsetrhow'pareven}).

For the right to left direction we need to prove that for any $w\in (\LL(\A))^s$ also $w\in \LL(\A^s)$, to do this we will provide a run for $\rho_w$ in \verb|supset_rhow|(Definition \ref{def:supsetrhow}) based on a run $\rho_{w_b}$ for the base word $w_b$. We will prove that $\rho_{w_b}$ is a well defined run in \verb|supset_rhowb_run| (Claim \ref{claim:supsetrhowbrun}) and is accepting in \verb|supset_rhowb_pareven| (Claim \ref{claim:supsetrhowbpareven}). Lastly we will prove that $\rho_w$ is a run in \verb|supset_rhow_run| (Claim \ref{claim:supsetrhowrun}) and is accepting in \verb|supset_rhow_pareven| (Claim \ref{claim:supsetrhowpareven}).

At the moment only the lemmas \verb|subset_stutequiv_w_w'| and \verb|supset_rhowb_run| are completely stutter-free. Lemma \verb|subset_rhow'_pareven| is well advanced.
%
% \begin{chart}
% \fullcourse 40, 0:{}{Stutterclosure}{}
% \end{chart}
\section{Future work}
Obviously the first thing to do is finish all the lemmas in the $\lean$ project to make this theorem sorry-free. The lemmas \verb|subset_rhow'_run| and \verb|supset_rhow_run| will mostly consist of unfolding the right definitions (although this is not as easy as we let it look like this). The lemmas \verb|supset_rhowb_pareven| and \verb|supset_rhow_pareven| will follow the same structure as \verb|subset_rhow'_pareven| and their associated claims in the paper version of this thesis. Also we could determine wheter it is interesting for the $\lean$ community to publish our definitions about parity automata.

A bigger future project is to formalize $\mutl$ in $\lean$ and implement an effective translation to nondeterministic parity automata as described in Chapter 3. In this way it would be possible to build a program that checks for a given $\mutl$ formula if it is stutter-invariant. This however is a whole project on its own.



\chapter{Conclusion}
Hier ook wat reflectie. Tweestapsraket niet gelukt, niet de andere kant van NPA naar formules. Veel werk gedaan wat al gedaan was en had beter de literatuur kunnen lezen. En voor LEAN is het belangrijk om echt eerst je bewijs op papier helemaal te hebben en dan pas te gaan leanen.     :)
\clearpage%to fix page numbering in ToC
\addcontentsline{toc}{chapter}{Bibliography}
\bibliographystyle{plain}
\bibliography{sources}


\clearpage%to fix page numbering in ToC
\chapter*{Populaire samenvatting}

\addcontentsline{toc}{chapter}{Populaire samenvatting}

\appendix
\chapter{Graph theory}
\begin{definition}\label{def:DAG}

\end{definition}
\begin{definition}\label{def:topoorder}

\end{definition}
\begin{theorem}\label{thm:daghastopoorder}

\end{theorem}



\end{document}
