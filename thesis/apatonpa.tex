\begin{definition}\label{def:APAtoNPA}
 Given an Alternating Parity Automaton (APA) $\mathbb{A}=(\Sigma, Q, q_0,\delta, \Omega)$ define the equivalent Nondeterministic Parity Automata (NPA) $\mathbb{A}':=(\Sigma, Q', q_0',\delta', \Omega')$ as follows:
 \begin{itemize}
  \item Define the set of states as $Q' = \mathcal{P}(Q\times \Omega(Q)) \times \{0, 1\}$
  \item Define the set of initial states as $q_0' \subseteq Q' = \begin{cases}
               \{(\{(q_0, \Omega(q_0)\}, 0)\} &= \text{if } \Omega(q_0) \text{ is even}\\
               \{(\{(q_0, \Omega(q_0))\}, 0), (\{(q_0, 0)\}, 1)\}&= \text{else}
               \end{cases}$
  \item Define the parity map $\Omega' : Q'\to \omega$ as:
  \[
   \Omega'(X, p) =\begin{cases}
                   3 &\text{if } p = 1\\
                   2 &\text{if for all } (q, p_q)\in X \text{ we have } p_q \text{ even }\\
                   1&\text{else}
                  \end{cases}
  \]
  \end{itemize}
  First some preliminary notational abbreviations.\\
  Define $Z_{\min} : Q\times \Sigma \to \PP(Q)$ as the minimal subset such that $Z_{\min}(q,a)\models\delta(q,a)$\\
  Define $\pi : \PP(\omega)\times \omega\to \omega$.
%   For $\pi(X, p)$ distinguish two cases:
% \begin{itemize}
%  \item[Case:] If $\text{par} := \{\max(p, x)\mid x\in X\}$ contains an odd parity\\
%    Set $\pi(X, p)$ to the maximum odd parity in this set.
%
%  \item[Else:] Set $\pi(X, p)$ to the minimum (even) parity in $\{\max(p, x)\mid x\in X\}$.
% \end{itemize}
\textcolor{red}{TODO:::}
  \[
   \pi(X, p) :=\begin{cases}
                \max\{p\in \{\max(p, x)\mid x\in X\} \mid p \text{ odd }\} &\text{if } \{\max(p, x)\mid x\in X\} \text{ contains odd parity}\\
                 \min(\{\max(p, x)\mid x\in X\}) &\text{else}
               \end{cases}
  \]

  And define a Floor function $F: \{0,1\}\times \omega \to \omega$.
  \[
   F(t, x) = \begin{cases}
              0 &\text{if } t = 1 \\
              x &\text{else}
             \end{cases}
  \]
  Let $\max(\inf_\Omega(\rho)):= \min\{\max(\inf_\Omega(u))\mid u\text{ infinite path in }\rho\}$ and $\max(\inf_\Omega(\rho, (q,i)):= \min\{\max(\inf_\Omega(u))\mid u\text{ infinite path in }\rho\text{ through } (q,i)\}$


  \noindent To define the transition function $\delta' : Q'\times \Sigma \to \mathcal{P}(Q')$ take a $(X,p)\in Q'$ and $a\in \Sigma$ and distinguish two cases:
  \begin{itemize}
  \item[\textbf{Case final:}] For all $(q, p)\in X$ we have $p$ is even.\\
  In this case all paths through $X$ have an even maximum parity attached. Therefore we reset all the saved parity information and set $\delta'$ to
  \begin{align*}
   \delta'((X, p), a) &= \biggl\{ (Y, 0) \mid\exists Z\subseteq Q, \text{ such that for every } (q, p_q)\in X, \text{ holds } Z\models \delta(q, a)\land \\
                      & \phantom{=\biggl\{} \{q|(q, p_q)\in Y\}=Z \land p_q=\Omega(q) \text{ for every } (q, p_q)\in Y\biggr\}\\
                      & \cup\Biggl\{ (Y, 1) \mid\exists Z\subseteq Q, \text{ such that for every } (q, p_q)\in X, \text{ holds } Z\models \delta(q, a)\land \\
   &\phantom{\cup\Biggl\{}\left\{q|(q, p_q)\in Y\right\}=Z \land (p_q=\Omega(q)\lor p_q=0)\text{ for every } (q, p_q)\in Y
   \Biggr\}
  \end{align*}
  \item[\textbf{Case non-final:}] There is a $(q, p)\in X$  where $p$ is odd.\\
  That means that not all paths have visited a maximum even parity. Set $\delta'$ to
    \begin{align*}
    \delta'((X, p), a) &= \Biggl\{ (Y, 0) \mid \exists Z:=\{q|(q, p_q)\in Y\}\subseteq Q, \text{ such that }
    Z\models \delta(q', a)\text{ for every }(q', p_{q'})\in X\\
    &\phantom{=\Biggl\{}\land p_q=\pi(\{p_{q'}\mid (q', p_{q'})\in X\land q\in Z_{\min} (q', a)\},\Omega(q)) \text{ for every } (q, p_q)\in Y \Biggr\}\\
    &\cup \Biggl\{ (Y, 1) | \mid \exists Z:=\{q|(q, p_q)\in Y\}\subseteq Q, \text{ such that } Z\models \delta(q', a)\text{ for every }(q', p_{q'})\in X\\
    &\phantom{\cup\Biggl\{}\land (p_q=\pi\left(\{p_{q'}\mid (q', p_{q'})\in X\land q\in Z_{\min} (q', a)\},\Omega(q)\right)\lor\\ &p_q=\pi\left(\{p_{q'}\mid (q', p_{q'})\in X\land q\in Z_{\min} (q', a)\},0\right))\text{ for every } (q, p_q)\in Y\Biggr\}
    \end{align*}\textcolor{red}{Hoe werkt de case distinction hier}
\end{itemize}
\end{definition}
We have to check if this NPA indeed recognizes the same language as the APA.
\begin{theorem}\label{thm:APAtoNPA}
Let $\A$ be an APA and $\A'$ be the NPA from Definition \ref{def:APAtoNPA}. Then we have:
\[\LL(\A)=\LL(\A')\]
\end{theorem}
\begin{proof}
First we prove \(\mathcal{L}(\mathbb{A})\subseteq \mathcal{L}(\mathbb{A}')\)\\
Take $w\in \LL(\A)$, then we know that there exists an accepting run $G_{\rho_\A}=(V,E)$ on $w$ in $\A$. Without loss of generality assume $G_{\rho_\A}$ is minimal.

To show that $w\in \LL(\A')$ we have to construct an accepting run in $\A'$.\\
\textbf{Definition}: $\rho_{\A'}=q'_0q'_1q'_2\dots$. Inductively define:\\
Firstly for $i=0$
\[q'_0 = \begin{cases}
               (\{(q_0, \Omega(q_0))\}, 0) &\text{if } \Omega(q_0)\leq \max(\inf_\Omega(\rho))\\
               (\{(q_0, 0)\}, 1) &\text{else }
              \end{cases}
\]
For $i>0$ distinguish two cases:
\begin{itemize}
 \item[Final] If $q'_{i-1}=(X, p)$ was final (i.e. $p_q$ is even for every $(q,p_q)\in X$):\\
 We again distinguish two cases:
 \begin{itemize}
  \item[A)] If \(\Omega(q)\leq \max(\inf_\Omega(\rho, (q,i))) \) for all $(q,i)\in V$ then set:
  \[
   q'_i = (\{(q, \Omega(q))\mid(q, i)\in V\}, 0).
  \]
  \item[B)] Else: \(\Omega(q)> \max(\inf_\Omega(\rho, (q,i))) \) for a $(q,i)\in V$ then set:
  \[
   q'_i = \left(\left\{(q, p_q)\mid(q, i)\in V, \text{ and } p_q=F(\Omega(q) > \max(\inf_\Omega(\rho, (q, i))),
            \Omega(q)\right\}, 1\right)
  \]

 \end{itemize}
\item[Non-final]If $q'_{i-1}=(X, p)$ was non-final (i.e. $p_q$ is odd for a $q\in X$). \\
We again distinguish two cases:
\begin{itemize}
  \item[C)] If \(\Omega(q)\leq \max(\inf_\Omega(\rho, (q,i))) \) for all $(q, i)\in V$ then set:
  \[
  q'_i = \left(\left\{(q, p_q)\mid (q, i)\in V, p_q=\pi\left(\left\{p_{q'}\mid(q', p_{q'})\in X\land q\in E_{i-1}[q']f\right\}, \Omega(q)\right)\right\}, 0\right)
  \]
  \item[D)] Else: If \(\Omega(q)> \max(\inf_\Omega(\rho, (q,i))) \) for a $(q,i)\in V$ then set:
  \begin{multline*}
   q'_i = \Biggl(\Biggl\{(q, p_q)\mid (q, i)\in V, \text{ where }\\
            p_q=\pi\left(\left\{p_{q'} \mid (q', p_{q'})\in X\land q\in E_{i-1}[q']\right\}, F\left(\Omega(q)>\max(\inf_\Omega(\rho, (q,i))), \Omega(q)\right)\right)\Biggr\}, 1\Biggr)
    \end{multline*}
 \end{itemize}
\end{itemize}
\setcounter{claim}{0}
\begin{claim}\phantomsection\label{claim:rhowelldefined}
$\rho_{\A'}$ is a well-defined run on $w$ in $\A'$

\end{claim}
\begin{proof}
I will show that for every $q'_i$ indeed $q'_i\stackrel{w_i}{\longrightarrow}q'_{i+1}$. Take $q'_i = (X_{i}, p_i)$  and we disinguish the same cases as in the definition of $\rho_{\A'}$. We will use the fact that $\{q\mid (q,p_q)\in X_i\}=\{q\mid (q,i)\in V\}$
\begin{itemize}
\item[A)] If $q'_i$ is final and \(\Omega(q)\leq \max(\inf_\Omega(\rho, (q,i))) \) for all \((q,i)\in V\). \\
Then we have from the definition of $\rho$ that $q'_i = (\{(q, \Omega(q))\mid(q, i)\in V\}, 0)$. Take $Z=\{q \mid (q,i+1)\in V\}$. Now for every $(q, i)\in X_{i}$ we know from the fact that $\rho_\A$ is a run (with R3) that $E_i[q]\subseteq Z$ and $E_i[q]\models\delta(q, w(i))$ so also $Z\models\delta(q, w(i))$. That means that
\begin{align*}q'_{i+1} &= (\{(q, \Omega(q))\mid(q, i)\in V\}, 0)\in \\
                      & \biggl\{ (Y, 0) \mid\exists Z\subseteq Q, \text{ such that for every } (q, p_q)\in X_i, \text{ holds } Z\models \delta(q, w(i))\land \\
                      & \phantom{=\biggl\{} \{q|(q, p_q)\in Y\}=Z \land p_q=\Omega(q) \text{ for every } (q, p_q)\in Y\biggr\}\subseteq \delta'(q'_i, w(i))\end{align*}

  \item[B)] If $q'_i$ is final and \(\Omega(q)> \max(\inf_\Omega(\rho, (q,i))) \) for a $(q,i)\in V$\\
  Then we have from the definition of $\rho$ that
  \[q'_{i+1} = \left(\left\{(q, p_q)\mid(q, i)\in V, \text{ and } p_q=F(\Omega(q) > \max(\inf_\Omega(\rho, (q, i))),
            \Omega(q)\right\}, 1\right)\].
  Again take $Z=\{q \mid (q,i+1)\in V\}$ and we see that $Z\models\delta(q, w(i))$ for every $(q,i)\in X_i$. That means that
  \begin{align*}
   q'_{i+1} &\in \Biggl\{ (Y, 1) \mid\exists Z\subseteq Q, \text{ such that for every } (q, p_q)\in X_i, \text{ holds } Z\models \delta(q, w(i))\land \\
   &\phantom{\cup\Biggl\{}\left\{q|(q, p_q)\in Y\right\}=Z \land (p_q=\Omega(q)\lor p_q=0)\text{ for every } (q, p_q)\in Y
   \Biggr\}\\
   &\subseteq \delta'(q'_i, w(i))
  \end{align*}
  since we can choose $p_q=0$ if $\Omega(q) > \max(\inf_\Omega(\rho, (q, i)))$ and $p_q=\Omega(q)$ else.
\item[C)] If $q'_i$ is non-final and  \(\Omega(q)\leq \max(\inf_\Omega(\rho, (q,i))) \) for all $(q, i)\in V$ then we have:
  \[
  q'_{i+1} = \left(\left\{(q, p_q)\mid (q, i)\in V, p_q=\pi\left(\left\{p_{q'}\mid(q', p_{q'})\in X\land q\in E_{i-1}[q']f\right\}, \Omega(q)\right)\right\}, 0\right)
  \]
We again seet that $Z\models \delta(q,w(i)$. The crucial observation here is that for every $(q, p_q)\in X_i$ we have $E_i[q]$ is minimal since $\rho_\A$ is minimal so therefore $E_i[q]==Z_{\min}(q, w(i)$. So that gives
\begin{align*}
 q'_{i+1} &\in \Biggl\{ (Y, 0) \mid \exists Z:=\{q|(q, p_q)\in Y\}\subseteq Q, \text{ such that }
    Z\models \delta(q', w(i))\text{ for every }(q', p_{q'})\in X_i\\
    &\phantom{=\Biggl\{}\land p_q=\pi(\{p_{q'}\mid (q', p_{q'})\in X\land q\in Z_{\min} (q', w(i))\},\Omega(q)) \text{ for every } (q, p_q)\in Y \Biggr\}\\
    &\subseteq \delta'(q'_i, w(i))
\end{align*}
\item[D)] If $q'_i$ is non-final and  \(\Omega(q)> \max(\inf_\Omega(\rho, (q,i))) \) for a $(q, i)\in V$ then we have:
\begin{multline*}
   q'_{i+1} = \Biggl(\Biggl\{(q, p_q)\mid (q, i)\in V, \text{ where }
            p_q=\pi\Biggl(\left\{p_{q'} \mid (q', p_{q'})\in X_i\land q\in E_{i-1}[q']\right\},\\ F\left(\Omega(q)>\max(\inf_\Omega(\rho, (q,i))), \Omega(q)\right)\Biggr)\Biggr\}, 1\Biggr)
    \end{multline*}

  We see that
  \begin{align*}
  q'_{i+1} &\in   \Biggl\{ (Y, 1) | \mid \exists Z:=\{q|(q, p_q)\in Y\}\subseteq Q, \text{ such that } Z\models \delta(q', a)\text{ for every }(q', p_{q'})\in X_i\\
    &\phantom{\cup\Biggl\{}\land (p_q=\pi\left(\{p_{q'}\mid (q', p_{q'})\in X_i\land q\in Z_{\min} (q', a)\},\Omega(q)\right)\lor\\ &p_q=\pi\left(\{p_{q'}\mid (q', p_{q'})\in X_i\land q\in Z_{\min} (q', a)\},0\right))\text{ for every } (q, p_q)\in Y\Biggr\}\\
    &\subseteq \delta'(q'_i,w(i))
   \end{align*}
   Since we can choose $0$ if \(\Omega(q)> \max(\inf_\Omega(\rho, (q,i))) \) and $\Omega(q)$ else.
%  \item[A and B)] Take $Z=\{q \mid (q,i+1)\in V\}$. Now for every $(q, i)\in X_{i}$ we know from the fact that $\rho_\A$ is a run (with R3) that $E_{(q,i)}\subseteq Z$ and $E_i[q]\models\delta(q, w(i))$ so also $Z\models\delta(q, w(i))$. We also see that the definition of $p_q$ corresponds to the definition of $p_q$ in $\delta'$. So that means that $q'_{i+1}\in \delta'(q'_{i, a_i})$
%  \item[C and D)] The crucial observation here is that for every $(q, p_q)\in X_i$ we have $E_{(q, i)}$ is minimal since $\rho_\A$ is minimal so therefore $E_{(q, i)}=Z_{\min}(q, a_i)$. Now the fact that $q'_{i+1}\in \delta(q'_i, a_i)$ follows from the definition.
\end{itemize}
Since we now have checked every case for $q'_i$ this prooves
\end{proof}
\begin{claim}
 There is a timestep, we call this the \emph{decisive moment} $i_d$, then in $G_\rho$ we have that from level $i_d$ on every infinite path only states in $\inf(\rho, (q,i))$ occur. That means that from then on every state on an infinite path has $\Omega(q)\leq \max(\inf_\Omega(G_{\rho_\A}, (q, i)))$.x
\end{claim}
\begin{proof}
 Suppose this $i_d$ does not exists. Then we would have
\end{proof}
\begin{claim}\phantomsection\label{claim:rhoaccepting}
$\rho_{\A'}$ is accepting, i.e. $\max(\inf_\Omega(\rho_{\A'}))$ is even.
\end{claim}
\begin{proof}If $G_{\rho_\A}$ is an accpeting run on $w$ in $\A$ that means that on any infinite path $u$ we have $\max(\inf_{\Omega_\A}(u))$ is even.  the maximum parity in the inf set is even. That means that we can identify the decisive moment  $i_d$ such that after this time every infinite path only visits states with an even parity. Now we will look at $\rho_{\A'}$ again. Now I claim that the maximum priority on $\rho_{\A'}$ is even. We see that after this timestep there is no time $j$ when there occurs a state $q$ with $\Omega(q)>\max(\inf_\Omega(\rho, (q,j)))$ since we have assumed that on every infinite path the maximum parity in the inf set is the maximum parity. So that means that the parity 3 cannot occur in $\rho_{\A'}$ after i. Now we also have that every infinite path infinitely often visits its (even) maximum parity. When all paths have done this we are in a final state with parity 2 and we reset. Since every infinite path infinitely often visits this maximum parity we see that $\rho_{\A'}$ infinitely often visits a final state so that means that $\max(\inf_{\Omega_{\A'}}(\rho_{\A'}))$ is even.
Note that we do not have to consider states where no infinite paths pass through since those do not alter the acceptance of $\rho_\A$ but also not the acceptance of $\rho_{\A'}$ since in the case distinction we specifically check for this.

% So therefore there will be an infinite amount of times that alle states have an even $p_q$ so therefore we see that the max of the $\omega$ inf of $\rho'$ is at least 2. It cannot be $3$ since then there would be an infinite number of times a state with a bigger odd parity then all it's infinite paths. So that means that there will be an infinite path without an even maximum parity which contradicts the fact that $\rho$ is accepting.
% Now we see that if $\rho$ is accepting that $\rho'$ is also accepting.
\end{proof}
From the fact that $\rho$ is an accepting run on $w$ we now know that $w\in \A'$ which proves the inclusion.

Now we prove \(\mathcal{L}(\mathbb{A}')\subseteq \mathcal{L}(\mathbb{A})\)\\
Suppose \(w\in \mathcal{L}(\mathbb{A}')\) then there is a succesfull run \(\rho_{\A'}=q'_0q'_1q'_2\dots\) on $w$ in the NPA $\mathbb{A}'$. We are going to make a run $\rho_\A$ in the APA $\A$, this should be a DAG.

\textbf{Definition}: \(\rho_\A=(V,E)\) where
\begin{itemize}
        \item \(V = \{(q, i)\in Q\times \omega\mid (q, p)\in X \text{ for a } (X, j)\in q'_i\}\)
        \item \(E=\bigcup_{l\geq 0} \{((q, l), (q', l+1))\in V\times V\mid q'\in Z_{\min}(q, a_l)\}\)
\end{itemize}

\begin{claim}
$\rho_\A$ is a run in $\A$

\end{claim}

\begin{proof}
\begin{itemize}
 \item[R1)] We see that $(q_0,0)\in V$ since the set of initial states $q'_0$ for $\A'$ is defined as $q'_0=\begin{cases}
               \{(\{(q_0, \Omega(q_0)\}, 0)\} &= \text{if } \Omega(q_0) \text{ is even}\\
               \{(\{(q_0, \Omega(q_0))\}, 0), (\{(q_0, 0)\}, 1)\}&= \text{else}
               \end{cases}$. We know that the first state of $\rho_{\A'}$ should be an initial state so that means that $(q_0, p)\in X_0$ for a $p\in\omega$. Now something about that this is a graph. So that
 \item[R2)] This follows directly from the definition of $E$.
 \item[R3)]
 \item[R4)]
\end{itemize}
\end{proof}

\begin{claim}
$\rho_\A$ is accepting on $w$ in $\A$

\end{claim}
\begin{proof}Since we know that $\rho_{\A'}$ is accepting we know that the $\max(\inf_{\Omega_{\A'}}(\rho_{\A'}))$ is even. Suppose $\rho_\A$ is not accepting. That means there is an infinite path $u$ where $\max(\inf_{\Omega_\A}(u))$ is odd. Now that means that there is a $i$ such that for every $u[j]$ with $j>i$ we have $u[j]\leq \max(\inf_{\Omega_\A}(u))$ (and $u[i]=\max(\inf_{\Omega_\A}(u))$). Now look at the the run $\rho_{\A'}$ after this $i$ we see that for every $q_j\in u$ we never set $p_{q_j}$ to an even parity since it starts at the odd parity $\max(\inf_{\Omega_\A}(u))$ and on $u$ it never encounters a bigger even parity. That means that from this point $\Omega_{\A'}(q_j)=q$ for $q_j\in \rho_{\A'}$ so therefore $\max(\inf_{\Omega_{\A'}}(\rho_{\A'}))=1$ and that contradicts to the fact that $\rho_{\A'}$ is accepting.

Therefore every infinite path $u$ in $\rho_A$ should be accepting so that means $\rho_\A$ is also accepting on $w$.\end{proof}.

Following Claim \ref{claim:rho_aaccepting} we have that that $w\in \LL(\A)$ and that proves that indeed
\[
 \LL(\A)=\LL(\A'),
\]
as desired.
\end{proof}
