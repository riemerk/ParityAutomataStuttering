In this section we will define for every tidy $\mutl$ formula $\phi$ an equivalent linear parity formula $\mathbb{G}_\phi$. We will base this construction on the closure of $\phi$. Recall Defintion \ref{def:closure} that the closure is defined as follows:
 \[
  Cl(\phi):=\{\psi\mid \phi\twoheadrightarrow_C \psi\},
 \]
 where $\twoheadrightarrow_C$ is the reflexive and transitive closure of $\rightarrow_C$ as defined in Definition \ref{def:closurerel}. We will define the closure graph $\mathbb{C}_\phi:=(Cl(\phi), \rightarrow_C)$ with the closure as its vertices and the closure $\rightarrow_C$ as its edge relation. Now we will define the parity formula $\mathbb{G}_\phi$ based on this closure graph.

\begin{definition}\label{def:parformofmutl} Let $\phi\in\mutl$ be a tidy formula, we will define the structure $\mathbb{G}_\phi$ based on the closure graph. First define the labeling function by taking $\psi\in Cl(\phi)$ as follows:
    \[
     L_\phi(\psi) = \left \lbrace \begin{array}{@{}r@{\quad}l@{\quad}l@{}}
                  \psi  &\text{if }\psi\in\At(\Psf)&\\
                  \deo &\text{if }\psi =\deo \chi&\\
                  \star &\text{if }\psi = \chi \star \theta & \text{ with } \star\in\{\lor,\land\}\\
                  \epsilon &\text{if }\psi = \chi T \theta & \text{ with } T\in\{U,R\}\\
                  \epsilon &\text{if }\psi = \eta_x.\xi & \text{ with } \eta\in\{\mu,\nu\}.
                 \end{array}\right.
    \]
   Now we will define the parity map  $\Omega_\mathbb{G} : \mathcal{F}(\phi)\to \omega$. For $f\in\mathcal{F}(\phi)$ let $d = d_{\Omega(\psi_f)}(\psi_f)$.  Define the parity map as following:
 \[
  \Omega_\mathbb{G}(f) := \begin{cases}
                         d &\text{if }d \text{ has parity } \Omega(\psi_f)\\
                         d+1&\text{if }d \text{ has parity } \overline{\Omega(\psi_f)}
                        \end{cases}
 \]
Combining these definitions gives the following definition for the parity formula : $\mathbb{G}_\phi:=(\mathbb{C}_\phi, L_\phi, \Omega_\mathbb{G}, \phi)$.
\end{definition}
\begin{lemma} Let $\phi$ be a tidy $\mutl$ formula and $\G_\phi$ the structure defined in Definition \ref{def:parformofmutl}. Then $\G_\phi$ defines a parity formula.
\end{lemma}
\begin{proof}
 The only non-trivial condition to check is the condition on the size of the edge relation.
 \begin{itemize}
   \item $|E[v]=0|$ if $L(v)\in \At(\Psf)$, $|E[v]|=1$ if $L(v)\in\{\deo,\epsilon\}$ and $|E[v]|=2$ if $L(v)\in\{\lor,\land\}$.
 \end{itemize}
Recall that the $\G_\phi$ is based on the closure graph so $E[a]=\{a' \in Cl(\phi)\mid a\rightarrow_C a'\}$. To check this condition we take take $a\in Cl(\phi)$ and distinguish cases:
\begin{description}
 \item[Case $a\in \At(\Psf)$] We see that $L(a)\in\At(\Psf)$ so we want that $|E[a]=0|$. We also see that there are no formulas $\psi$ such that $a\rightarrow_C\psi$ so that means that $|E[a]=0|$ as desired.
 \item[Case $a\in \{\deo, \mu,\nu,U,R\}$] From the definition of $\rightarrow_C$ we see that there is only one formula $\psi$ such that $a\rightarrow_C \psi$ so $|E[a]|=1$.
 \item[Case $a\in \{\lor,\land\}$] In this case we have $a=\psi\star\chi$ and that gives $a\rightarrow_C \psi$ and $a\rightarrow_C \chi$ so $|E[a]|=2$ as desired.\qedhere
\end{description}
\end{proof}
Nextly we have to prove that $\G_\phi$ is indeed equivalent to $\phi$.
\begin{theorem}\label{thm:parformequivmutl}
Let $\phi$ be a tidy $\mutl$ formula and $\mathbb{G}_\phi$ the parity formula as in Definition \ref{def:parformofmutl}. Then $\phi \equiv \mathbb{G}_\phi$.
\end{theorem}
\begin{proof}
 We see that the boards of $\mathcal{E}(\phi, w)$ and $\mathcal{E}(\mathbb{G}_\phi, w)$ are identical. Also the parity map is defined in the same way so that proves that both evaluation games are identical so that proves that
\[\phi\equiv \mathbb{G}_\phi.\qedhere\]
\end{proof}


