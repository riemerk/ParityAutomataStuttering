\usepackage[english]{babel} %Remark: use the same language for uvamath and babel
\usepackage{graphicx}
\usepackage[pdfborder={0 0 0}]{hyperref}
\usepackage{lipsum}
\usepackage{amsfonts}
\usepackage{amsthm}
\usepackage{amsmath,amssymb, mathrsfs}
\usepackage{mathtools}
\usepackage{modalops}
\usepackage{hyperref}
\usepackage{stackrel}
\usepackage{enumitem}
\usepackage{prerex}

% \usepackage{logix}
% \setmainfont{STIX Two Text}
% \setmathfont{STIX Two Math}

% Required packages and libraries
\usepackage{tikz}

\usetikzlibrary{automata, arrows.meta, positioning, fit}
\usepackage{newunicodechar}
% \usepackage{fontspec}
% switch to a monospace font supporting more Unicode characters
% \setmonofont{FreeMono}
\newunicodechar{ℕ}{\ensuremath{\mathbb{N}}}
\newunicodechar{∃}{\ensuremath{\exists}}
\newunicodechar{ρ}{\ensuremath{\rho}}
\newunicodechar{∧}{\ensuremath{\land}}


\usepackage{minted}
% \newmintinline[lean]{lean4}{bgcolor=white}
\newminted[leancode]{lean4}{fontsize=\footnotesize}
\usemintedstyle{tango}  % a nice, colorful theme
\theoremstyle{plain}


\theoremstyle{definition}
\newtheorem{definition}{Definition}[chapter]
\newtheorem{example}[definition]{Example}
\newtheorem*{plain}{}
\theoremstyle{plain}
\newtheorem{theorem}[definition]{Theorem}
\newtheorem{corollary}[definition]{Corollary}

\newtheorem{claim}{Claim}[definition]

\newtheorem{lemma}[definition]{Lemma}

\newtheorem{fact}[definition]{Fact}
\newtheorem{proposition}[definition]{Proposition}
\theoremstyle{definition}
\newtheorem{definitiont}[claim]{Definition}

\counterwithin{equation}{definition}


\renewcommand{\phi}{\varphi}
\renewcommand{\emptyset}{\varnothing}

\renewcommand{\epsilon}{\varepsilon}
\DeclareMathOperator{\Mod}{Mod}
\DeclareMathOperator{\Dom}{Dom}
\DeclareMathOperator{\last}{last}
\DeclareMathOperator{\Sub}{Sub}
\DeclareMathOperator{\PM}{PM}
\DeclareMathOperator{\Win}{Win}
\DeclareMathOperator{\Ran}{Ran}


\newcommand{\Lit}{\mathtt{Lit}}
\newcommand{\lean}{\mathtt{LEAN}}
\newcommand{\At}{\mathtt{At}}




\newcommand{\A}{\mathbb{A}}
\newcommand{\G}{\mathbb{G}}
\newcommand{\E}{\mathcal{E}}

\newcommand{\N}{\mathbb{N}}
\newcommand{\mutl}{\mu\mathtt{TL}}



\newcommand{\LL}{\mathcal{L}}
\newcommand{\apa}{\textsc{apa}}
\newcommand{\npa}{\textsc{npa}}

\newcommand{\PP}{\mathcal{P}}
\newcommand{\Psf}{\mathsf{P}}


%definitions
\newcommand{\F}{\mathcal{F}}
\renewcommand{\descriptionlabel}[1]{\hspace{\labelsep}\normalfont\textit{#1}}





\title{Determining the stutter-invariant fragment of the linear time $\mu$-calculus. An automata based approach and verification in $\lean$}
\author[riemer.kerkstra@student.uva.nl, 13283529]{Riemer Kerkstra}

\supervisors{prof.\ dr.\ Yde Venema}
\supervisors{dr.\ Malvin Gattinger}
\secondgrader{dr.\ Alexi Block Gorman}
\secondgrader{dr.\ Balder ten Cate}
