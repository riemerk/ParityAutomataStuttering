In this section we will define parity games as they will form the basis of the acceptance of $\mutl$ formulas and of linear parity formulas. The definitions are, with some minor changes, adapted from  \cite[Chapter 5]{venema2024modalmucalculus}.
The games in this thesis are played by two players: Eloise whom we denote by $\exists$ and Abelard whom we denote by $\forall$. An arbitrary player is denoted by $\tau$ and the opponent by $\overline{\tau}$. In general Eloise wants to show that a structure (formula or game) is satisfied and Abelard wants to show the contrary.

\begin{definition}\label{def:pargame}
 A parity game is a quadruple $\mathcal{B}=(B_\exists, B_\forall, E, \Omega)$ such that $B_\exists$ and $B_\forall$ are disjoint sets, and $E\subseteq B^2$, where $B:=B_\exists\cup B_\forall$ and $\Omega : B\to\omega$ is the \emph{priority map}. .  We will call $\mathbb{B}=(B_0, B_1, E)$ the board or game graph of this parity game. An \emph{initialized parity game} is a parity game with a position $a$ attached to it as initial position. We denote this as $\mathcal{B}@a$.
 We will denote $E[a]$ as the set of \emph{admissible} moves for $a\in B$, that is $E[a]:=\{b\in B\mid (a,b)\in E\}$. A position $a\in B$ is a \emph{dead end} if $E[a]=\emptyset$. If $a\in B$ we let $\tau_a$ denote the (unique) player such that $a\in B_{\tau_a}$, and say that $a$ \emph{belongs to} $\tau_a$, or that it is $\tau_a$'s \emph{turn} to move at $a$.
\end{definition}
In order to explain the working of the priority map, and with it the winning condition of a parity game we first need to define matches in a parity game.
\begin{definition}\label{def:match}
 A \emph{path} through the board $\mathbb{B}=(B_\exists,B_\forall,E)$ of a parity game $\mathcal{B}=(B_\exists, B_\forall, E, \Omega)$ is a nonempty (finite or infinite) sequence $\pi\in B^\infty$ such that $E\pi_i\pi_{i+1}$ for every $i$. A \emph{full} or {complete match} through $\mathbb{B}$ is either an infinite $\mathbb{B}$-path, or a finite $\mathbb{B}$-path ending in a dead end. A \emph{partial match} is a finite path through $\mathbb{B}$ that does not end in a dead end. We let $\PM$ denote the set of all partial matches and $\PM_\tau$ denote the set of all partial matches such that it is $\tau$'s turn in the last position of the match.
\end{definition}

\noindent In order to determine which player has won the game we will look at the set of positions that occur infinitely often and the maximum parity of this set.

\begin{definition}
 Let $\alpha:\omega\to B$ be a $B$-sequence. Given an element $a\in B$ we define the \emph{frequency} of $a$ in $\alpha$ as $\#_a(\alpha):=|\{i\in\omega\mid\alpha(i)=a\}|$. Based on this frequency the set of elements that occur infinitely often as $\inf(\alpha):=\{a\in \alpha\mid \#_a(\alpha)=\omega\}$ and the set of parities that occur infinitely often as $\inf_\Omega(\alpha):=\{\Omega(a)\mid a\in \inf(\alpha)\}$
\end{definition}

\begin{definition}
 Let $\mathcal{B}=(B_\exists, B_\forall, E, \Omega)$ be a parity game and $\pi$ a match through $\mathbb{B}$. If $\pi$ is a finite match we say that that $\tau$ has won this match if $\last(\pi)$ is a dead end for $\overline{\tau}$. If $\pi$ is an infinite match we say that $\exists$ has won this match if \(\max (\inf_\Omega(\pi))\) is even and that $\forall$ has won if this is odd.
\end{definition}

\noindent In order to define the winning positions in a parity game we will define the notion of a strategy.
\begin{definition}\label{def:winning}
Give a parity game $\mathcal{B}=(B_\exists, B_\forall, E, \Omega)$ and a player $\tau$, a $\tau$-\emph{strategy}, is a map $f: \PM_\tau\to B$ that prescribes the moves $\tau$ should make. In case we are dealing with an initialized parity game $\mathcal{B}@a$ we can take the strategy as a map $f:\PM_\tau(a)\to B$. A match $\pi$ is \emph{guided by} a $\tau$-strategy if for any partial match $\pi'<\pi$ with $\last(\pi')\in B_\tau$, the next position on $\pi$ (after $\pi'$) is equal to $f(\pi')$. A $\tau$-strategy $f$ is \emph{surviving} in $\mathcal{B}@a$ if every move that it prescribes to $f$-guided partial matches in $\PM_\tau(a)$ is admissible and \emph{winning} for $\tau$ if in addition all $f$-guided full matches starting at $a$ are won by $\tau$. A position $a\in B$ is winning for $\tau$ if $\tau$ has a winning strategy for the game $\mathcal{B}@a$. The collection of all winning positions for $\tau$ in $\mathcal{B}$ is called the \emph{winning region} for $\tau$ in $\mathcal{B}$, and denoted as $\Win_\tau(\mathcal{B})$.
\end{definition}
% Note that we defined $f$ as a full map from all partial matches to ... Geen idee hier eventjes.

In general strategies can depend on all positions that lie before the last, but in some cases we want the strategy to only take into account the last position of a partial match. That is what we call a positional strategy.
\begin{definition}
 A strategy $f$ is \emph{positional} if $f(\pi)=f(\pi')$ for any $\pi,\pi'$ with $\last(\pi)=\last(\pi')$. In this case a positional $\tau$-strategy may be represented as a map $f:B_\tau\to B$.
\end{definition}

\noindent The following result will allow us to always use a positional strategy when working with parity games. It
\begin{theorem}[Positional Determinacy of Parity games]\cite[Theorem 5.28]{venema2024modalmucalculus}
 Let $\mathcal{B} = (B_\exists, B_\forall, E, \Omega)$ be a parity game. Then $B = \Win_\exists(\mathcal{B})\cup \Win_\forall(\mathcal{B})$ and there are positional strategies $f_\exists$ and $f_\forall$ such that for each player $\tau$ and every $a\in\Win_\tau(\mathcal{B})$, the strategy $f_\tau$ is winning for $\tau$ in $\mathcal{B}@a$.
 \end{theorem}
\noindent The proof of this theorem can be found in \cite{venema2024modalmucalculus}.
As last definitions in this section we will define the strategy tree and strategy graph.
\begin{definition}\label{def:strattree}
 If $f$ is a surviving player for $\tau$ in $\mathcal{B}@a$, we may represent $f$ as the pruned subtree of the gametree $\mathbb{T}_a^\mathcal{B}$ that is based on those nodes that correspond to $f$-guided matches of $\mathcal{B}@a$. In this so called \emph{strategy tree} $\mathbb{T}_a^f$ we have
 \[
  \stackrel{\rightarrow}{E}_f[\pi] := \begin{cases}
                                     \stackrel{\rightarrow}{E}[\pi] &\text{if } \pi \in \PM_{\bar{\tau}}\\
                                     \{\pi \cdot f(\pi)\} &\text{if } \pi \in \PM_{\tau}.
                                    \end{cases}
 \]
\end{definition}
\begin{definition}\label{def:stratgraph}
 If $f$ is a surviving, positional strategy for player $\tau$ in $\mathcal{B}@a$, we may represent $f$ as the pruned subgraph of the game graph (instead of the game tree) of $\mathbb{B}_a^\mathcal{B}$ that is based on those nodes that correspond to $f$-guided matches. In this so called \emph{strategy graph} $\mathbb{B}_a^f$ we have:
 \[
  E_f[b] := \begin{cases}
                                     E[b] &\text{if } b \in B_{\bar{\tau}}\\
                                     f(b) &\text{if } b \in B_\tau.
                                    \end{cases}
 \]
\end{definition}
