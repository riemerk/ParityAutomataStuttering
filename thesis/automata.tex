\subsection{Nondeterministic parity automata (NPA)}
\begin{definition} (\cite{venema2024modalmucalculus})\label{def:NPA}
A non-deterministic parity automaton (NPA) is a quintuple \(\A = (Q, \Sigma, \delta, q_I, \Omega)\) where $Q$ is a finite set of states, $\Sigma$ a finite alphabet, $\delta \subset Q\times \Sigma \to \mathcal{P}(Q)$ the transition function, $q_I\subset Q$ the set of initial states and $\Omega: Q\to \omega$ the parity map.
\end{definition}
For the definition of $\hat{\delta}$, $q\stackrel{a}{\rightarrow}q'$, $q\stackrel{w}{\twoheadrightarrow}q'$, runs and accepted languages ($\LL(\A)$) I refer to \cite{venema2024modalmucalculus}. A run is acepting iff \(\max\{\Omega(q)|q\in \inf(\rho)\}\) is even.

\begin{definition}\label{def:partialrun}
 Given a NPA \(\A = (Q, \Sigma, \delta, q_I, \Omega)\) we can extend the definition of $q\stackrel{w}{\twoheadrightarrow}q'$ to include the states visited. Let $w=a_0\dots a_{n-1}$ a $\Sigma$-word and $\rho=q_0\dots q_n$ a finite list of states. We will write $q_1\stackrel{w}{\twoheadrightarrow^\rho}q_n$ if $q_i\stackrel{a_i}{\longrightarrow}q_{i+1}$ for all $0\leq i\leq n-1$ and call $\rho$ a \textbf{partial run} in this context.
\end{definition}

\subsection{Alternating Parity Automata (APA)}

\begin{definition}(\cite[Section 14.3.1]{demri2016temporal})\label{def:posboolform}
Define $\mathbb{B}^+(X)$ as the set of \textbf{positive boolean formulas} over $X$. They respect the following grammar
\[
 \mathcal{F}::= \bot\mid\top\mid x\mid (\mathcal{F}\lor\mathcal{F})\mid (\mathcal{F}\land\mathcal{F})
\]
And we say that: $Y\models x$ iff $x\in Y$, $Y\models \mathcal{F}\lor \mathcal{F}'$ iff $Y\models \mathcal{F}$ or $Y\models \mathcal{F}'$, $Y\models \mathcal{F}\land\mathcal{F}'$ iff $Y\models \mathcal{F}$ and $Y\models \mathcal{F}'$ and we have if $Y\subset Y'$ and $Y\models \mathcal{F}$ then also $Y'\models \mathcal{F}$.
\end{definition}
\begin{definition}(\cite[Definition 14.3.2]{demri2016temporal})\label{def:APA}
 A \textbf{alternating parity tree automaton} (ATPA) $\mathcal{A}=(\Sigma, Q, q_0,\delta, \Omega)$ is defined as:
 \begin{itemize}
  \item $\Sigma$: alphabet
  \item $Q$ finite set of states
  \item $q_0\in Q$ initial state
  \item $\delta: Q\times \Sigma \to \mathbb{B}^+(Q)$ transition function
  \item $\Omega: Q\to\mathbb{N}$ the parity map
 \end{itemize}
\end{definition}
For the definition of runs and acceptance criteria I refer to \cite[Section 14.3.1]{demri2016temporal}.
\textcolor{red}{Voor de tijd stuur ik je nu even dit hoofdstuk, maar in de eindversie ga ik dit wel zelf neertypen denk ik}
