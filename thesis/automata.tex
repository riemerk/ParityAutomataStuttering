\subsection{Nondeterministic parity automata (NPA)}
\begin{definition} (\cite{venema2024modalmucalculus})\label{def:NPA}
A \emph{non-deterministic parity automaton} (NPA) is a quintuple \(\A = (\Sigma,Q, q_I, \delta, \Omega)\) where
\begin{itemize}
 \item $\Sigma$ denotes the alphabet.
 \item $Q$ is a finite set of states.
 \item $q_I\subseteq Q$ the set of initial states.
 \item $\delta: Q\times \Sigma \to \mathcal{P}(Q)$ the transition function.
 \item $\Omega: Q\to \omega$ is the parity map.
\end{itemize}
\end{definition}


For the definition of $\hat{\delta}$, $q\stackrel{a}{\rightarrow}q'$, $q\stackrel{w}{\twoheadrightarrow}q'$, runs and accepted languages ($\LL(\A)$) I refer to \cite{venema2024modalmucalculus}. A run is acepting if \(\max(\inf_\Omega(\rho))\) is even.

\begin{definition}\label{def:partialrun}
 Given a NPA \(\A = (Q, \Sigma, \delta, q_I, \Omega)\) we can extend the definition of $q\stackrel{w}{\twoheadrightarrow}q'$ to include the states visited. Let $w=a_0\dots a_{n-1}$ be a $\Sigma$-word and $\rho=q_0\dots q_n$ a finite list of states. We will write $q_1\twoheadrightarrow_w^\rho q_n$. if $q_i\stackrel{a_i}{\longrightarrow}q_{i+1}$ for all $0\leq i\leq n-1$ and call $\rho$ a \textbf{partial run} in this context.
\end{definition}

\subsection{Alternating Parity Automata (APA)}

\begin{definition}(\cite[Section 14.3.1]{demri2016temporal})\label{def:posboolform}
Define $\mathrm{BF}(X)$ as the set of \emph{positive boolean formulas} over $X$. Given by the following grammar
\[
 \mathcal{F}::= \bot\mid\top\mid x\mid (\mathcal{F}\lor\mathcal{F})\mid (\mathcal{F}\land\mathcal{F})
\]
And we say that: $Y\models x$ iff $x\in Y$, $Y\models \mathcal{F}\lor \mathcal{F}'$ iff $Y\models \mathcal{F}$ or $Y\models \mathcal{F}'$, $Y\models \mathcal{F}\land\mathcal{F}'$ iff $Y\models \mathcal{F}$ and $Y\models \mathcal{F}'$.
\end{definition}
Statement: and we have if $Y\subset Y'$ and $Y\models \mathcal{F}$ then also $Y'\models \mathcal{F}$.
\begin{definition}(\cite[Definition 14.3.2]{demri2016temporal})\label{def:APA}
 An \emph{alternating parity automaton} (APA) is a quintuple $\A=(\Sigma, Q, q_0, \delta, \Omega)$ where
 \begin{itemize}
  \item $\Sigma$ denotes the alphabet.
  \item $Q$ is a finite set of states.
  \item $q_0\in Q$ is the initial state.
  \item $\delta: Q\times \Sigma \to \mathrm{BF}(Q)$ is the transition function.
  \item $\Omega: Q\to\omega$ is the parity map.
 \end{itemize}
\end{definition}

Runs in APA's are defined as directed acyclic graphs that follow the transition function.
\begin{definition}\label{def:aparun}\cite[Definition 14.3.4]{demri2016temporal}
 Let $\A=(\Sigma, Q, q_0,\delta, \Omega)$ be an APA. A \emph{run} on the $\omega$-word $w: \omega\to\Sigma$ is a (possibly infinite) directed acyclic graph $\rho = (V, E)$ satisfying the following conditions.
 \begin{itemize}
  \item[(R1)] $V\subseteq Q\times \omega$ and $(q_0,0)\in V$.
  \item[(R2)] $E\subseteq \bigcup_{l\geq 0} (Q\times \{l\})\times (Q\times\{l+1\})$.
  \item[(R3)] For every $(q,l)\in V\setminus\{(q_0,0)\}$, there is $q'\in Q$ such that $((q',l-1), (q,l))\in E$.
  \item[(R4)] For every $(q, l)\in V$ we have $E_l[q]:=\{q'\mid ((q,l),(q', l+1))\in E\}\models\delta(q,w(l))$.
 \end{itemize}
A node $(q,l)$ is said to be of \emph{level} $l$.
 A run is accepting if for every infinite path $u$ through $\rho$ we have $\max(\inf_\Omega(u))$ is even.
 \textcolor{red}{Definieer die max verzameling}

We call $\rho$ minimal if it satisfies R4' instead of R4.
\begin{itemize}
   \item[(R4')] For every $(q, l)\in V$ we have $E_l[q]:=\{q'\mid ((q,l),(q', l+1))\in E\}\models\delta(q,a_l)$ \emph{and} for no strict subset $Y$ of $E_l[q]$ we have $Y\models\delta(q,w(l))$.
 \end{itemize}
\end{definition}
