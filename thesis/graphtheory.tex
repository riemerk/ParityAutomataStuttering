As the runs in \apa s are defined as directed acyclic graphs we will formally define what we mean by an acyclic graph. We will also introduce the notion of a topological ordering on the vertices of a graph.
\begin{definition}\label{def:DAG}
A \emph{directed graph} is a graph $G=(V,E)$ where $V$ is the set of vertices and $E\subseteq V^2$ is the set of directed edges. We call a sequence $u = v_0v_1\dots v_n$ a \emph{path} if $n\geq 1$ and for every $0\leq i <n$ we have that $(v_i,v_{i+1})\in E$. Such a path is a cycle if $v_0=v_n$. We call a graph \emph{acyclic} if it contains no cycles.
\end{definition}
\begin{definition}\label{def:topoorder}
We call an ordering $v_1,v_2,\dots,v_n$ of the vertices of a graph a \emph{acyclic} or \emph{topological} ordering if for every edge $(v_i,v_j)\in E$ we have that $i<j$.
\end{definition}
\noindent The following theorem is a well known result in graph theory.
\begin{theorem}\cite[Proposition 1.3.1]{jensen2018digraphs}\label{thm:daghastopoorder}
Every directed acyclic graph has a topological ordering of its vertices
\end{theorem}
