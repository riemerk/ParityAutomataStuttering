In this section we will kind of parity games, namely the evaluation games of linear parity formulas. Give a set $\Psf$ of proposition letters we define the sets $\Lit(\Psf)$ and $\At(\Psf)$ of \emph{literals} and \emph{atomic formulas} over $\Psf$. Define \(\Lit(\Psf) := \{p,\bar{p}\mid p\in\Psf\}\) and $\At(\Psf):=\Lit(\Psf)\cup \{\top, \bot\}$.

\begin{definition}\label{def:linparform}(Extended from \cite[Definition 6.1]{venema2024modalmucalculus})
Let $\Psf$ be a finite set of proposition letters. A \emph{linear parity formula over} $\Psf$ is a quintuple $\mathbb{G}=(V, E, L, \Omega_\mathbb{G}, v_I)$, where
\begin{itemize}
 \item $(V,E)$ is a finite, directed graph, with $|E[v]|\leq 2$ for every vertex v.
 \item $L:v\to \At(\Psf)\cup \{\land,\lor,\deo,\epsilon\}$ is a labeling function.
 \item $\Omega: V\stackrel{\deo}{\to}\omega$ is a partial map, the \emph{priority map} of $\G$.
 \item $v_I$ is a vertex in $V$, referred to as the \emph{initial node} of $\G$.
 \item $|E[v]=0|$ if $L(v)\in \At(\Psf)$, $|E[v]|=1$ if $L(v)\in\{\deo,\epsilon\}$ and  $|E[v]|=2$ if $L(v)\in\{\lor,\land\}$.
 \item Every cycle of $(V,E)$ contains at least one node im $\Dom(\Omega)$.
\end{itemize}
A node $v\in V$ is called \emph{silent} if $L(v)=\epsilon$, \emph{constant} if $L(V)\in \{\top,\bot\}$, \emph{literal} if $L(v)\in \Lit(\Psf)$, \emph{atomic} if $L(v)\in\At(\Psf)$, boolean if $L(v)\in \{\lor, \land\}$, and \emph{modal} if $L(v)=\deo$. A node $v\in V$ is called a \emph{state} if it is an element of $\Dom(\Omega)$.
\end{definition}
\textbf{Convention}: If we have a modal or silent node $v\in V$ we will use the notation $e(v)$ for the single node such that $E[v]=\{e(v)\}$. If we have a boolean node $v\in V$ we will use the notatation $e_L(v)$ and $e_R(v)$ to denote $E[v]=\{e_L(v),e_R(v)\}$.

We will define a parity game based on this linear parity formula as following.


\begin{definition}\label{def:evalgameparform}
\textcolor{red}{Maak table netjes en positions die geen speler hebben...}
 Let $w: \omega \to \PP(\Psf)$ a linear model and let $\G=(V, E, L, \Omega_\G, v_I)$ a linear parity formula. Define the \emph{evaluation game} as the parity game $\E(\G, w):=(B_\exists,B_\forall, E_\E, \Omega_\E)$ with players $\exists$ and $\forall$. Let $B:=B_\exists\cup B_\forall$ consists of the nodes $V\times \omega$ and partition them according to table .. Let $B_\tau$ consists of the positions that belong to $\tau$ and $E$ as described in the table. The priority map $\Omega_\E: V\times \omega \to \omega$ is given by
 \[
  \Omega_\E(v, i) := \begin{cases}
                    \Omega_\G(v) & \text{if }v \in \Dom (\Omega_\G)\\
                    0 &\text{else}
                   \end{cases}
 \]
We define the following game graph:\\
 \begin{tabular}{|c|c|c|}
  \hline
  Position & Player & Admissable moves\\
  \hline
  $(v, i)$ with $L(v)=\lor$& $\exists$ & $\{(e_L(v), i),(e_R(v), i)\}$\\
  $(v, i)$ with $L(v)=\land$& $\forall$ & $\{(e_L(v), i),(e_R(v), i)\}$\\
  $(v, i)$ with $L(v)=\deo$&-&$\{(e(v), i+1)\}$\\
  $(v, i)$ with $L(v)=\epsilon$&-&$\{(e(v), i)\}$\\
  $(v, i)$ with $L(v)=p$ and $p\in w(i)$& $\forall$ & $\emptyset$\\
  $(v, i)$ with $L(v)=p$ and $p\notin w(i)$& $\exists$ & $\emptyset$\\
  $(v, i)$ with $L(v)=\bar{p}$ and $p\notin w(i)$& $\forall$ & $\emptyset$\\
  $(v, i)$ with $L(v)=\bar{p}$ and $p\in w(i)$& $\exists$ & $\emptyset$\\
  $(v, i)$ with $L(v)=\top$& $\forall$ & $\emptyset$\\
  $(v, i)$ with $L(v)=\bot$& $\exists$ & $\emptyset$\\
\hline
 \end{tabular}\\
\end{definition}

\begin{definition}
 We say \(w, i\Vdash \mathbb{G}\) if the pair $(v_I, i)$ is a winning position for $\exists$ in $\mathcal{E}(\mathbb{G}, w)$.
 We define the \emph{language} of a parity formula as:
 \[
  \Mod_\Psf(\mathbb{G}) := \{w \in (\PP(\Psf))^\omega\mid w, 0 \Vdash \mathbb{G}\}
 \]
\end{definition}

\begin{definition}
 Let $\mathcal{E}(\mathbb{G}, w)=(B_\exists,B_\forall, E_\E, \Omega_\E)$ the evaluation game of a parity formula $\G=(V, E, L, \Omega_\G, v_I)$. We have $B=V\times \omega$.
 Define the \emph{level} $k$ of the strategy graph as the subgraph of $\mathbb{B}_{(q,l)}^f$ consisting of the nodes of the form $V\times \{l+k\}$ denote this with $\mathbb{B}_{(q,l),k}^f$.\\
 Define the \emph{level} $k$ of the strategy tree as the subtree of $\mathbb{T}_{(q,l)}^f$ consisting of the nodes $\pi\in \PM(a)$ where we have $\last(\pi)\in V\times \{k+l\}$ denote this with $\mathbb{T}_{(q,l),k}^f$.\\
\end{definition}

\begin{definition}\label{def:guardedparform}\cite[Definition 6.62]{venema2024modalmucalculus}
 A path $\pi=v_0v_1v_2\dots v_n$ through a parity formula is \textbf{unguarded} if $n\geq 1$, $v_0, v_n\in \Dom(\Omega)$ while there is no $i$ with $0<i\leq n$, such that $v_i$ is a modal node. A parity formula is \textbf{guarded} if it has no unguarded cycles, and \textbf{strongly guarded} if it has no unguarded paths.
\end{definition}
