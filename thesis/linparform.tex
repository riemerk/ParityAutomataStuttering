In this section we will define linear parity formulas which we will use as an intermediate step in our translation from $\mutl$ formulas to $\omega$-automata. We will again use a parity game to define the satisfaction of a linear parity formula. The definitions in this section are adapted to from \cite[Chapter 6]{venema2024modalmucalculus}.

\begin{definition}\label{def:linparform}
Let $\Psf$ be a finite set of proposition letters. A \emph{linear parity formula} over $\Psf$ is a quintuple $\mathbb{G}=(A, E, L, \Omega_\mathbb{G}, a_I)$, where
\begin{itemize}
 \item $(A,E)$ is a finite, directed graph, with $|E[a]|\leq 2$ for every vertex $a$.
 \item $L:A\to \At(\Psf)\cup \{\land,\lor,\deo,\epsilon\}$ is a \emph{labeling function}.
 \item $\Omega: A\stackrel{\deo}{\to}\omega$ is a partial map, the \emph{priority map} of $\G$.
 \item $a_I$ is a vertex in $a$, referred to as the \emph{initial node} of $\G$.
 \item $|E[a]=0|$ if $L(a)\in \At(\Psf)$, $|E[a]|=1$ if $L(a)\in\{\deo,\epsilon\}$ and  $|E[a]|=2$ if $L(a)\in\{\lor,\land\}$.
 \item Every cycle of $(A,E)$ contains at least one node im $\Dom(\Omega)$.
\end{itemize}
A node $a\in a$ is called \emph{silent} if $L(a)=\epsilon$, \emph{constant} if $L(a)\in \{\top,\bot\}$, \emph{literal} if $L(a)\in \Lit(\Psf)$, \emph{atomic} if $L(a)\in\At(\Psf)$, boolean if $L(a)\in \{\lor, \land\}$, and \emph{modal} if $L(a)=\deo$. Furthermore a node $a\in a$ is called a \emph{state} if it is an element of $\Dom(\Omega)$.
\end{definition}
\noindent\textbf{Convention}: If we describe a modal or silent node $a\in a$ we will use the notation $e(a)$ for the single node such that $E[a]=\{e(a)\}$. If we describe a boolean node $a\in A$ we will use the notation $e_L(a)$ and $e_R(a)$ to denote $E[a]=\{e_L(a),e_R(a)\}$.\\

\noindent Now to define the satisfaction of a linear parity formula we will define a parity game.
\begin{definition}\label{def:evalgameparform}
 Let $\G=(A, E, L, \Omega_\G, a_I)$ a linear parity formula and $w: \omega \to \PP(\Psf)$ a linear model. Define the \emph{evaluation game} as the parity game $\E(\G, w):=(B_\exists,B_\forall, E_\E, \Omega_\E)$.. Let $B:=B_\exists\cup B_\forall$ consists of the nodes $A\times \omega$ and partition them according to the table in Figure \ref{fig:evalgameparform}. The priority map $\Omega_\E: A\times \omega \to \omega$ is given by
 \[
  \Omega_\E(v, i) := \begin{cases}
                    \Omega_\G(v) & \text{if }v \in \Dom (\Omega_\G)\\
                    0 &\text{else}
                   \end{cases}
 \]
We will define the following game graph for this parity game.
\begin{figure}[h!]\label{fig:evalgameparform}
 \begin{tabular}{|c|c|c|}
  \hline
  Position & Player & Admissible moves\\
  \hline
  $(a, i)$ with $L(a)=\lor$& $\exists$ & $\{(e_L(a), i),(e_R(a), i)\}$\\
  $(a, i)$ with $L(a)=\land$& $\forall$ & $\{(e_L(a), i),(e_R(a), i)\}$\\
  $(a, i)$ with $L(a)=\deo$&-&$\{(e(a), i+1)\}$\\
  $(a, i)$ with $L(a)=\epsilon$&-&$\{(e(a), i)\}$\\
  $(a, i)$ with $L(a)=p$ and $p\in w(i)$& $\forall$ & $\emptyset$\\
  $(a, i)$ with $L(a)=p$ and $p\notin w(i)$& $\exists$ & $\emptyset$\\
  $(a, i)$ with $L(a)=\overline{p}$ and $p\notin w(i)$& $\forall$ & $\emptyset$\\
  $(a, i)$ with $L(a)=\overline{p}$ and $p\in w(i)$& $\exists$ & $\emptyset$\\
  $(a, i)$ with $L(a)=\top$& $\forall$ & $\emptyset$\\
  $(a, i)$ with $L(a)=\bot$& $\exists$ & $\emptyset$\\
\hline
 \end{tabular}
 \caption{Board for the evaluation game of the parity formula}
 \end{figure}
 We say $w$ satisfies $\G$ \(w, i\Vdash \mathbb{G}\) if the pair $(a_I, i)$ is a winning position for $\exists$ in $\mathcal{E}(\mathbb{G}, w)$.
 We define the \emph{language} of a linear parity formula as:
 \[
  \Mod_\Psf(\mathbb{G}) := \{w \in (\PP(\Psf))^\omega\mid w, 0 \Vdash \mathbb{G}\}
 \]
Throughout the rest of this thesis we will mostly refer to linear parity formulas as just parity formulas.
\end{definition}

\begin{definition}
Let $\G=(A, E, L, \Omega_\G, a_I)$ be a parity formula and $\mathcal{E}(\mathbb{G}, w)=(B_\exists,B_\forall, E_\E, \Omega_\E)$ the evaluation game as in Definition \ref{def:evalgameparform}. Recall Definitions \ref{def:strattree} and \ref{def:stratgraph} about the strategy tree and graph. We will define the \emph{level} $k$ of a strategy graph as the subgraph of $\mathbb{B}_{(a,l)}^f$ consisting of the nodes of the form $A\times \{l+k\}$ denote this with $\mathbb{B}_{(a,l),k}^f$.\\
 W will define the \emph{level} $k$ of a strategy tree as the subtree of $\mathbb{T}_{(a,l)}^f$ consisting of the nodes $\pi\in \PM(a)$ where we have $\last(\pi)\in A\times \{k+l\}$ denote this with $\mathbb{T}_{(a,l),k}^f$.\\
\end{definition}
\begin{definition}
 A path $\pi=a_0a_1a_2\dots a_n$ through a linear parity formula is called \emph{unguarded} if $n\geq 1$, and $a_0, a_n\in \Dom(\Omega)$ while there is no $i$ with $0<i\leq n$, such that $a_i$ is a modal node. A linear parity formula is \emph{guarded} if it has no unguarded cycles, and \emph{strongly guarded} if it has no unguarded paths.
\end{definition}

