So far in this thesis we have given a translation from a $\mutl$ formula to an \npa\, in Chapter 3 which we can use for every tidy, guarded and simple formula. In Section 4.1 we have given a method to determine if the language of an automaton is stutter-invariant. We can combine these two procedures to obtain a procedure to determine if a formula is stutter-invariant.
\begin{theorem}
Let $\phi$ a tidy, guarded and modal simple $\mutl$ formula. Then there exists an effective procedure that determines wheter $\phi$ is stutter-invariant.
\end{theorem}
\begin{proof}
 Since $\phi$ is tidy, guarded and modal simple, by Theorem \ref{thm:mutltonpa}, we can obtain an equivalent \npa\, $\A_\phi$. With Theorem \ref{thm:automatonstutinvariant} we can then check if the language of $\A_\phi$ is stutter-invariant. Since $\Mod_\Psf(\phi)=\LL(\A_\phi)$ we then know if $\phi$ is stutter-invariant.
\end{proof}
\noindent Next to checking if a specific $\mutl$ formula is stutter-invariant it is however much more interesting to determine all the stutter-invariant formulas of $\mutl$. Or its stutter-invariant fragment. Recall that the goal of this thesis was to give an alternative proof of the theorem in \cite{gheerbrandt2009craiginterpolation} stating the fact that $\mutl(U,R)$ is the stutter-invariant fragment of $\mutl$. First we see that Gheerbrandt and Ten Cate first prove the following lemma.
\begin{theorem}\cite[Lemma 3]{gheerbrandt2009craiginterpolation}\label{thm:lemma3}
 For every $\phi\in\mutl$ there exists a $\phi^*\in\mutl(U,R)$ that agrees with $\phi$ on all stutter-free words:
 \[
  w, i\models \phi\iff\phi^* \text{ for all stutter-free } w\in \Sigma^\omega
 \]
\end{theorem}
Afterwards they will prove the following theorem:
\begin{theorem}\label{thm:untilstutinvariant}
 Let $\phi\in\mutl(U,R)$ a formula, then $\Mod(\phi)$ is stutter-invariant.
\end{theorem}

\noindent The goal of this thesis was to present an alternative proof with an automata based approach, we did however not succeed in this. We will explain the problems shortly
\begin{proof}[Proof (Sketch)]
 The proof of this theorem goes with induction on the complexity of $\phi$.
 The atomic cases are clear. For the case where $\phi=\psi\star\chi$ with $\star\in\{\land,\lor\}$ we will use our $\mutl$ to \npa\, translation and Propositions \ref{prop:mutltonpatranslationboolean} and \ref{prop:stutcupcap}. Let $\A_\psi$ and $\A_\chi$ the automata as in Theorem \ref{thm:mutltonpa}. Now we see that:
  \begin{align*}
   \Mod_\Psf(\psi\star\chi)&= \LL(\A_\psi)\star^s\LL(\A_\chi)
   \intertext{We use Proposition \ref{prop:mutltonpatranslationboolean} where  $\star^s$ stands for the set equivalent of $\star$}
                           &= \LL(\A_\psi)^s\star^s\LL(\A_\chi)^s \text{ (Induction hypothesis)}\\
                           &= (\LL(\A_\psi)\star^s\LL(\A_\chi))^s \text{ (Proposition \ref{prop:stutcupcap})}\\
                           &= (\Mod_\Psf(\psi\star\chi))^s \text{ (Proposition \ref{prop:mutltonpatranslationboolean})}
  \end{align*}
  Which proves that $\psi\star\chi$ is stutter-invariant.
The cases where $\phi=\eta x.\xi_x$ and $\phi=\psi T \chi$ with $T\in\{U,R\}$ are more difficult to prove and will therefore not be proven in this thesis. We will however explain some of this difficulties here. At first sight we tried to write out the the automata for $\phi$ generated by Theorem \ref{thm:mutltonpa}. This was more difficult since in our translation of alternating parity automata to nondeterministic parity automata we cite an article that uses parity progress measures to first translate the alternating parity automata into alternating Büchi automata. The deep understanding of parity progress measures required for this induction step went too far for this Bachelor project. Secondly the construction of the automaton $\A_\phi$ in Theorem \ref{thm:mutltonpa} requires $\phi$ to be guarded. We know that every $\mutl$ formula is equivalent to a guarded one but the precise understanding of this theorem was again not part of this thesis. Also since we are looking at formulas without the $\deo$ operator any formula that contains a bounded variable is clearly not guarded. In this project it remains open if this $\deo$-free formula can be transformed into a formula without fixpoint operators. If you look at the following (extremely simple) example see:
\[
 \nu x.(p\land x) U r \equiv (pUr)
\]
It is of future interest to finish this construction, and to see if this will work for every formula. If $\phi$ can be transformed into a formula $\phi^T$ that uses no fixpoint operators we see that the alternating automaton that is equivalent to $\phi^T$ actually is a Büchi automaton since without bound variables there can be no alternating chains. In this Büchi case the construction to a nondeterministic automaton can be applied immediately. It then remains open to prove that this formula $\phi^T$ is stutter-invariant.
\end{proof}
The following theorem proves that $\mutl(U,R)$ is equal to the stutter-invariant fragment of $\mutl$.
\begin{theorem}
 Let $\phi\in\mutl$ such that $\Mod_\Psf(\phi)$ is stutter-invariant. Then there exists $\phi^*\in\mutl(U,R)$ such that $\Mod(\phi)=\Mod(\phi*)$
\end{theorem}
\begin{proof}
 This follows from Theorem \ref{thm:lemma3} and Theorem \ref{thm:untilstutinvariant} \cite{gheerbrandt2009craiginterpolation}.
\end{proof}
